\subsection{Weakly Interacting Radiation}
  \label{subsec:weakly-interacting-radiation}

  In the Cosmochrony framework, weakly interacting radiation does not correspond to
  fundamentally different particle species, but to delocalized projective regimes
  whose structural contrast and curvature are insufficient to efficiently induce
  localized reprojection when encountering matter.

  Low-frequency electromagnetic descriptions or weakly coupled projective modes
  are characterized by smooth, slowly varying relational structure at the level of
  effective projection.
  As a result, their interaction with localized $\chi$ configurations is rare:
  the probability that such descriptions trigger a stable localized energy transfer
  upon interaction is strongly suppressed.

  This explains the effective transparency of vacuum to most radiation.
  Propagation corresponds to the persistence of coherent delocalized projective
  descriptions across extended regions, while detection events occur only when local
  structural conditions allow reprojection into a localized excitation.

  In this sense, small interaction cross sections do not reflect the weakness of a
  fundamental force, but the low likelihood that a given projective configuration
  satisfies the geometric and topological conditions required for localized
  reprojection.
