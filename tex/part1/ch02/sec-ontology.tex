% ----------------------------------------------------------------------------
% Section 2.10 --- Ontological Interpretation
% Merges former §4.1--4.5, §4.4/4.11 (holography), condensed
% ----------------------------------------------------------------------------
\subsection{Ontological Interpretation}
\label{sec:chi_ontology}

\subsubsection*{The \texorpdfstring{$\chi$}{χ} Substrate as a Pre-Temporal
Structural Plan}
\label{subsec:the-chi-substrate-as-a-pre-temporal-structural-plan}

The substrate~$\chi$ admits an ontological interpretation as a pre-temporal
relational structure from which spacetime, matter, and effective physical
laws emerge.
It may be heuristically described as a \emph{structural plan}: not a
dynamical history, but a complete relational organization encoding the set
of physically admissible configurations and the constraints that relate
them.
Temporal succession is emergent, corresponding to an oriented resolution
of structural relations through irreversible relaxation.
The notion of a structural plan does not introduce teleology, determinism,
or a block-universe ontology: multiple effective histories may correspond
to the same underlying relational structure through non-injective
projection.

\subsubsection*{Relational Ontology and Conceptual Lineage}
\label{subsec:relational-ontology-and-conceptual-lineage}

The relational character of~$\chi$ bears a conceptual affinity with
relational approaches in physics, notably those of
Rovelli~\cite{Rovelli1996,Rovelli2004}, which trace part of their lineage
to Aristotelian relational ontology~\cite{AristotleCategories,Shields2016}.
Cosmochrony shares the rejection of intrinsic, observer-independent
properties but extends relationalism to a deeper ontological level:
$\chi$ configurations are not relations \emph{between} fundamental objects
but relational structures that give rise to objects only upon projection.
Relativistic causality emerges without postulating spacetime as
fundamental~\cite{Rovelli2018}.
Relationality is an intrinsic property of the pre-geometric substrate from
which spacetime and physical entities jointly emerge.

\subsubsection*{Projection, Reality, and Ontological Asymmetry}
\label{subsec:projection-reality-and-ontological-asymmetry}

The emergence of spacetime is a \emph{projection} from~$\chi$, not a dual
or bidirectional description.
The projected universe is fully real at the level of physical experience
but ontologically derivative: spacetime entities and dynamical laws do not
possess ontological primacy~\cite{Rovelli2021}.
While all physical descriptions depend on the projection of~$\chi$, the
relational structure of~$\chi$ does not admit a reformulation entirely in
geometric or field-theoretic terms.

Apparent fine-tuning is reinterpreted as a selection effect imposed by
projectability: only those configurations compatible with a stable emergent
geometry appear as physically realized universes.
Cosmochrony does not postulate a multiverse---the universe is unique at the
level of physical reality, even though its underlying description in terms
of~$\chi$ may be non-unique.

The formal developments related to projection, including its fiber-bundle
formulation and the emergence of gauge interactions, are developed in
Section~\ref{sec:projection-gauge}.

\subsubsection*{Configurational State Structure}
\label{subsec:chi_state_structure}

The substrate~$\chi$ defines a \emph{configurational structure} specifying
the set of admissible macroscopic states and the allowed transitions
between them.
Physical reality corresponds to a projected realization stabilized under
finite resolution; no causal influence from projected configurations
to~$\chi$ is postulated.
What appears as temporal evolution corresponds, at the level of~$\chi$, to
an ordering of admissible configurations under projection.
Universal bounds such as~$c_\chi$ and~$\hbar_\chi$ reflect intrinsic limits
on admissible transitions within this configurational structure.

\subsubsection*{Intrinsic Structural Indeterminacy}
\label{subsec:intrinsic-structural-indeterminacy}

A perfectly deterministic, fully symmetric relational substrate would
remain physically inert.
Cosmochrony therefore postulates an \emph{intrinsic structural
indeterminacy} at the level of~$\chi$: configurations are not exhaustively
specified by a finite, closed set of relational conditions.
This indeterminacy is ontological, not dynamical---it reflects the absence
of perfect structural closure rather than random motion or noise.

Observable variability and probabilistic behavior arise only at the level
of projected descriptions.
Because the projection is non-injective, a single underlying configuration
may correspond to multiple admissible effective realizations.
Randomness is therefore \emph{projective}: it reflects the multiplicity of
effective descriptions compatible with a given pre-geometric structure.

\subsubsection*{Relation to Holographic Descriptions}
\label{subsec:clarifying-holography}

Cosmochrony is not a holographic theory in the technical sense.
It does not posit a lower-dimensional boundary description or a dual
equivalence between bulk and boundary physics.
The limitation of physically accessible information within a given
spacetime region reflects the degeneracy of underlying $\chi$
configurations corresponding to the same effective projection---a direct
consequence of non-injectivity, not of boundary-localized degrees of
freedom.
Scaling behaviors reminiscent of holography are interpreted as emergent
signatures of projection.

Similarly, Cosmochrony differs from thermodynamic approaches to gravity
(Jacobson, Verlinde) in that it does not posit entropy or information as
primitive quantities.
Thermodynamic descriptions arise only at the effective level, as secondary
languages applicable when projected configurations admit coarse-grained
statistical interpretations.
