% ----------------------------------------------------------------------------
% Section 2.9 --- Initial Conditions and Global Structure
% From former §3.9
% ----------------------------------------------------------------------------
\subsection{Initial Conditions and Global Structure}
\label{subsec:initial-conditions-and-global-structure}

The framework does not postulate initial conditions in the conventional
temporal sense.
It assumes that~$\chi$ admits a minimal admissible ordering state,
denoted~$\chi_0$, defining a structural boundary of admissible projected
descriptions.
This state does not correspond to a distinguished moment in time but to the
earliest configurations for which an effective ordering interpretation
becomes meaningful.

In effective geometric regimes, the characteristic scale near~$\chi_0$
coincides numerically with the Planck scale.
This correspondence reflects the breakdown of projectability below this
regime, not the presence of a fundamental cutoff or underlying discreteness.

Cosmic history is interpreted as the progressive and irreversible ordering
of projected configurations away from this minimal admissible boundary.
No spacetime singularity is required at the fundamental level.
Apparent singular behavior arises only when classical notions are
extrapolated beyond the regime in which projected configurations admit a
stable geometric interpretation.

\paragraph{Ontological poverty and the growth of admissible structure.}
The minimal state~$\chi_0$ corresponds to a regime of \emph{ontological
poverty}: only a severely restricted class of simple and highly coherent
configurations can be projected.
As relaxation proceeds, the space of admissible configurations expands,
enabling the emergence of increasingly rich, localized, and hierarchical
effective structures.
The global structure of admissible descriptions is constrained by the
ordering properties of the underlying substrate.
