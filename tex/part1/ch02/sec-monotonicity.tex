% ----------------------------------------------------------------------------
% Section 2.6 --- Monotonicity and Arrow of Time
% From former §3.6, condensed
% ----------------------------------------------------------------------------
\subsection{Monotonicity and Arrow of Time}
\label{subsec:monotonicity-and-arrow-of-time}

A central structural postulate is that the relational substrate~$\chi$
admits an intrinsic, globally ordered relaxation structure.
In effective descriptions, this ordering manifests as the monotonic behavior
of the projected scalar descriptor along admissible ordering paths:
\begin{equation}
  \mathcal{D}_{\lambda} \chi_{\mathrm{eff}} \ge 0 .
\end{equation}
Here $\lambda$ denotes an ordering parameter associated with relaxation,
not a fundamental time coordinate.
The inequality is a structural constraint on admissible projected
representations.

Energy is interpreted as the remaining capacity of projected configurations
to undergo further relaxation.
Admissible ordering paths exclude any effective decrease
of~$\chi_{\mathrm{eff}}$, which would correspond to a restoration of
relaxation capacity incompatible with the underlying ordering structure.
The arrow of time is identified with this directional ordering: the
progression from configurations with greater relaxation capacity toward
configurations in which that capacity has been exhausted.
This temporal orientation arises prior to any statistical or thermodynamic
description~\cite{Prigogine1997,Rovelli1991}.
The relation to thermodynamic irreversibility is discussed further in
Section~\ref{subsec:entropy-arrow}.

\paragraph{Projectability and Kinematic Saturation.}
\label{par:projectability-kinematics}
A change of velocity corresponds to a modification of the relational
coherence constraints maintained by the projection.
As velocity increases, the informational demand on the projection grows,
progressively saturating its capacity.
The bound~$c_\chi$ is a structural limit beyond which no stable and
globally consistent projection can be maintained.
Approaching saturation, part of the relational content of~$\chi$ becomes
inaccessible to the effective description, manifesting as time dilation,
length contraction, and horizon formation.
Relativistic kinematics thus emerges as a consequence of finite projection
capacity\footnote{%
  This kinematic saturation concerns the ordering and coherence capacity of
  the projection itself and should not be conflated with thermodynamic
  entropy production.}.

\paragraph{Planck Scale and Relativistic Bounds as Projection Limits.}
\label{par:planck-c-projective-limits}
The constants~$c$ and~$h$ are interpreted as complementary manifestations of
a finite resolution of the projection from~$\chi$ to effective observables.
The bound~$c$ limits the maximal admissible rate at which relational ordering
can be projected, while~$h$ sets a lower bound on the granularity with which
the relational flux can be resolved.
Relativistic and quantum constraints thus emerge as complementary facets of
a single structural limitation: the finite capacity and resolution of the
projection.
