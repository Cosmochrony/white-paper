% ============================================================================
% Chapter 2 --- The chi Substrate: Definition, Properties, and Ontology
% Merges former sections 3 and 4 of Cosmochrony v1.13beta1
% ============================================================================
\clearpage

\section{The \texorpdfstring{$\chi$}{χ} Substrate: Definition, Properties,
and Ontology}
\label{sec:definition-and-fundamental-properties-of-the-chi-field}

This section introduces the single fundamental entity at the core of
Cosmochrony: the pre-geometric relational substrate~$\chi$.
We identify the minimal structural properties required for effective
descriptions of spacetime, dynamics, and physical observables to arise in
appropriate regimes, and we clarify the ontological status of~$\chi$ and its
relation to emergent physical quantities.

No spacetime manifold, metric structure, or background geometry is assumed.
Geometric and dynamical notions are recovered only through non-injective,
coarse-grained projections of~$\chi$ configurations once suitable stability
conditions are met.
This minimality principle, referred to as a regime of \emph{ontological
poverty}, is adopted as a foundational constraint.

\input{part1/ch02/sec-definition}
% ----------------------------------------------------------------------------
% Section 2.2 --- The Geometric Effective Description of chi Dynamics
% From former §3.2, trimmed of redundant caveats
% ----------------------------------------------------------------------------
\subsection{The Geometric Effective Description of
\texorpdfstring{$\chi$}{χ} Dynamics}
\label{subsec:geometric-action}

\subsubsection*{Effective Observables from
\texorpdfstring{$\chi$}{χ} Correlations}

Quantities conventionally described in geometric terms---such as time
intervals, spatial separation, and causal ordering---arise as effective
summaries of relational patterns within the $\chi$ substrate, accessed only
through projected, coarse-grained representations.
The configurations~$\sigma$ represent internal relational states of~$\chi$
and are defined without reference to external spacetime coordinates.
The measure~$d\mu(\sigma)$ denotes an invariant integration over
configuration space, defined intrinsically from the correlation structure
associated with~$\chi$.

\paragraph{Effective scalar descriptor.}
In regimes where projected $\chi$ configurations admit a stable geometric
interpretation, we introduce an \emph{effective scalar descriptor}
$\chi_{\mathrm{eff}}$.
This quantity is a coarse-grained, projected representation of relational
and spectral features of~$\chi$, defined only within the emergent spacetime
description.
It does not correspond to the fundamental substrate itself.

Operational time intervals are defined from the accumulated ordering of
projected configurations along paths in configuration space:
\begin{equation}
  \tau_{AB} \;\propto\;
  \int_{\gamma_{AB}}
    \mathcal{D}_{\lambda} \chi_{\mathrm{eff}} \, d\lambda ,
\end{equation}
where $\mathcal{D}_{\lambda} \chi_{\mathrm{eff}}$ is an effective
relaxation functional and $\lambda$ an ordering parameter.

Operational spatial separation is quantified by the decay of correlations
between effective descriptors:
\begin{equation}
  d(x,y) \;\propto\;
  - \log\!\left(
    \frac{\langle \chi_{\mathrm{eff}}(x)\,
      \chi_{\mathrm{eff}}(y)\rangle}
    {\langle \chi_{\mathrm{eff}}^{\,2}\rangle}
  \right),
\end{equation}
where $\langle\chi_{\text{eff}}(x)\,\chi_{\text{eff}}(y)\rangle$ denotes
an effective correlation functional encoding relational proximity between
projected configurations~$x$ and~$y$.

\begin{figure}[t]
  \centering
  \begin{tikzpicture}[
    font=\small,
    node distance=14mm,
    box/.style={draw, rounded corners, align=center, inner sep=6pt},
    arrow/.style={-Latex, thick},
    note/.style={align=left, font=\footnotesize}
  ]
    \node[box] (chi)
      {$\chi$\\\footnotesize infra-physical\\
       \footnotesize relational substrate};
    \node[box, below=of chi] (chieff)
      {$\chi_{\mathrm{eff}}$\\\footnotesize physical effective
       reality\\\footnotesize (projectable regime)};
    \node[box, below=of chieff] (obs)
      {Observables\\\footnotesize context-dependent\\
       \footnotesize operational quantities};
    \draw[arrow] (chi) -- node[right=2mm, note]
      {infra-physical\\projection $\pi$\\
       \footnotesize (generally non-injective)} (chieff);
    \draw[arrow] (chieff) -- node[right=2mm, note]
      {operational\\projection $\mathcal{O}$\\
       \footnotesize (contextual access)} (obs);
    \node[
      draw, dashed, rounded corners,
      fit=(chieff) (obs),
      inner sep=10pt,
      label={[font=\footnotesize]right:Factorizable regime}
    ] {};
  \end{tikzpicture}
  \caption{%
    Ontological pipeline in Cosmochrony.
    The infra-physical projection~$\pi$ maps the fundamental $\chi$
    substrate to an effective physical reality~$\chi_{\mathrm{eff}}$ in
    projectable regimes, generally in a non-injective manner.
    Physical observables arise from an operational
    projection~$\mathcal{O}$ that specifies how
    $\chi_{\mathrm{eff}}$ is accessed under measurement contexts.}
  \label{fig:two-stage-projection}
\end{figure}

\subsubsection*{Effective Metric as a Descriptive Tool}

In regimes where projected configurations exhibit smooth and stable
correlation patterns, the relational observables may be summarized by an
operational tensor $g_{\mu\nu}[\chi_{\mathrm{eff}}]$.
This effective metric is a derived descriptor summarizing relational
correlations, not a fundamental geometric structure.
No background $\eta_{\mu\nu}$ is assumed; Minkowski space appears only as an
effective approximation in weak-gradient regimes.

In projectable regimes admitting a low-dimensional embedding, one may write
locally
\[
  d(i,j)^2 \;\approx\;
    g_{\mu\nu}(x)\,\Delta x^\mu \Delta x^\nu .
\]
The continuum line element~$ds^2$ is therefore a derived descriptive
construct.

\subsubsection*{Consistency with General Relativity}

The effective metric reproduces the phenomenology of general relativity in
appropriate regimes: the weak-field limit yields Einstein-like dynamics; near
localized $\chi$ excitations, the metric encodes time dilation and spatial
curvature as emergent consequences of inhibited relaxation; homogeneous
relaxation yields an effective Hubble-like expansion law.
All predictive content resides in the underlying $\chi$ dynamics.

\paragraph{Operational origin of the effective metric.}
The explicit construction of $g_{\mu\nu}$ from operational distances is
given in Appendix~E, where geometric quantities are shown to arise only in
projectable regimes admitting a smooth continuum approximation.

% ----------------------------------------------------------------------------
% Section 2.3 --- Physical Interpretation
% From former §3.3, condensed
% ----------------------------------------------------------------------------
\subsection{Physical Interpretation}
\label{subsec:physical-interpretation}

In regimes where the infra-physical projection from $\chi$ to an effective
description~$\chi_{\mathrm{eff}}$ yields a factorisable structure, an
approximate decomposition into subsystems and the definition of operational
observables become possible.
This factorisation underlies the emergence of classical locality,
compatibility of measurements, and standard relativistic descriptions, as
illustrated in Fig.~\ref{fig:classical-factorisable}.

\begin{figure}[t]
  \centering
  \begin{tikzpicture}[
    font=\small,
    node distance=10mm,
    box/.style={draw, rounded corners, align=center, inner sep=6pt},
    arrow/.style={-Latex, thick},
    note/.style={align=left, font=\footnotesize},
    dashedbox/.style={draw, dashed, rounded corners, inner sep=6pt}
  ]
    \node[box] (chi)
      {$\chi$\\\footnotesize infra-physical substrate};
    \node[box, below=of chi] (chieff)
      {$\chi_{\mathrm{eff}}$\\\footnotesize factorisable regime};
    \node[dashedbox, below=of chieff, minimum width=6.6cm] (decomp) {
      \begin{tabular}{c}
        \footnotesize
        $\chi_{\mathrm{eff}} \simeq
          \chi_{\mathrm{eff}}^{(A)}
          \otimes \chi_{\mathrm{eff}}^{(B)}$\\
        \footnotesize (independent subsystems)
      \end{tabular}
    };
    \node[box, below left=10mm and 12mm of decomp] (obsA)
      {Local observables\\in subsystem $A$};
    \node[box, below right=10mm and 12mm of decomp] (obsB)
      {Local observables\\in subsystem $B$};
    \draw[arrow] (chi) -- node[right=2mm, note]
      {infra-physical\\projection $\pi$} (chieff);
    \draw[arrow] (chieff) -- (decomp);
    \draw[arrow] (decomp) -- node[left=2mm, note]
      {operational\\projection $\mathcal{O}_A$} (obsA);
    \draw[arrow] (decomp) -- node[right=2mm, note]
      {operational\\projection $\mathcal{O}_B$} (obsB);
  \end{tikzpicture}
  \caption{Classical (factorisable) regime.
    After the infra-physical projection~$\pi$, the effective description
    admits an approximate decomposition into independent subsystems.
    Operational projections then yield compatible local observables,
    recovering standard classical descriptions.
    This contrasts with the non-factorisable regimes discussed in
    Section~\ref{sec:quantum-phenomena-and-entanglement}.}
  \label{fig:classical-factorisable}
\end{figure}

Because the projection from $\chi$ to $\chi_{\mathrm{eff}}$ is generically
non-injective, distinct $\chi$ configurations may correspond to identical
effective descriptions, while a single underlying configuration may admit
multiple correlated operational realizations.
This structural asymmetry is central to the emergence of both classical and
quantum phenomenology.
The physical content of the theory resides entirely in the dynamics of the
fundamental substrate, while spacetime notions function as emergent,
context-dependent descriptive tools.

\input{part1/ch02/sec-relational-projection}
% ----------------------------------------------------------------------------
% Section 2.5 --- Structural Principles and Projective Regimes
% From former §3.5
% ----------------------------------------------------------------------------
\subsection{Structural Principles and Projective Regimes}
\label{subsec:structural-principles-and-projective-regimes}

All effective physical observables arise through a relational
projection~$\Pi$ from the underlying configuration space~$\Omega$ of
the~$\chi$ substrate.
Effective descriptions are constrained by three structural principles that
govern the admissibility of projected regimes.

\paragraph{Principle I: Substratic Locality and Bounded Relaxation.}
The fundamental relaxation dynamics of~$\chi$ is strictly local and subject
to universal bounds.
The transport of relational relaxation admits a maximal admissible flux,
constraining the rate at which structural information can be redistributed.
This bound arises from the intrinsic stability conditions of~$\chi$ itself.

\paragraph{Principle II: Non-Injective Projective Realization.}
The projection $\Pi : \Omega \rightarrow O$ from substratic configurations
to effective observables is generically non-injective.
Distinct configurations of~$\chi$ may be structurally identified at the
level of observable descriptions.
This implies that effective descriptions need not admit a factorisable or
locally complete representation, even when the underlying dynamics remains
strictly local.

\paragraph{Principle III: Projective Compensation.}
Whenever $\Pi$ fails to resolve the full relational complexity of~$\chi$,
effective descriptions compensate through the inflation of effective
parameters---temperature, curvature, horizon structure---encoding
unresolved relational structure within a reduced descriptive framework.
These quantities function as Lagrange multipliers, not as additional
fundamental degrees of freedom.

Together, these principles delineate distinct projective regimes.
When the projection is approximately injective and relaxation is far from
saturation, standard local and geometric descriptions apply.
Near structural saturation or strong non-injectivity, effective parameters
grow large, signaling the breakdown of spacetime-based descriptions.

% ----------------------------------------------------------------------------
% Section 2.6 --- Monotonicity and Arrow of Time
% From former §3.6, condensed
% ----------------------------------------------------------------------------
\subsection{Monotonicity and Arrow of Time}
\label{subsec:monotonicity-and-arrow-of-time}

A central structural postulate is that the relational substrate~$\chi$
admits an intrinsic, globally ordered relaxation structure.
In effective descriptions, this ordering manifests as the monotonic behavior
of the projected scalar descriptor along admissible ordering paths:
\begin{equation}
  \mathcal{D}_{\lambda} \chi_{\mathrm{eff}} \ge 0 .
\end{equation}
Here $\lambda$ denotes an ordering parameter associated with relaxation,
not a fundamental time coordinate.
The inequality is a structural constraint on admissible projected
representations.

Energy is interpreted as the remaining capacity of projected configurations
to undergo further relaxation.
Admissible ordering paths exclude any effective decrease
of~$\chi_{\mathrm{eff}}$, which would correspond to a restoration of
relaxation capacity incompatible with the underlying ordering structure.
The arrow of time is identified with this directional ordering: the
progression from configurations with greater relaxation capacity toward
configurations in which that capacity has been exhausted.
This temporal orientation arises prior to any statistical or thermodynamic
description~\cite{Prigogine1997,Rovelli1991}.
The relation to thermodynamic irreversibility is discussed further in
Section~\ref{subsec:entropy-arrow}.

\paragraph{Projectability and Kinematic Saturation.}
\label{par:projectability-kinematics}
A change of velocity corresponds to a modification of the relational
coherence constraints maintained by the projection.
As velocity increases, the informational demand on the projection grows,
progressively saturating its capacity.
The bound~$c_\chi$ is a structural limit beyond which no stable and
globally consistent projection can be maintained.
Approaching saturation, part of the relational content of~$\chi$ becomes
inaccessible to the effective description, manifesting as time dilation,
length contraction, and horizon formation.
Relativistic kinematics thus emerges as a consequence of finite projection
capacity\footnote{%
  This kinematic saturation concerns the ordering and coherence capacity of
  the projection itself and should not be conflated with thermodynamic
  entropy production.}.

\paragraph{Planck Scale and Relativistic Bounds as Projection Limits.}
\label{par:planck-c-projective-limits}
The constants~$c$ and~$h$ are interpreted as complementary manifestations of
a finite resolution of the projection from~$\chi$ to effective observables.
The bound~$c$ limits the maximal admissible rate at which relational ordering
can be projected, while~$h$ sets a lower bound on the granularity with which
the relational flux can be resolved.
Relativistic and quantum constraints thus emerge as complementary facets of
a single structural limitation: the finite capacity and resolution of the
projection.

% ----------------------------------------------------------------------------
% Section 2.7 --- Local Relaxation Speed
% From former §3.7
% ----------------------------------------------------------------------------
\subsection{Local Relaxation Speed}
\label{subsec:local-relaxation-speed}

The effective local ordering rate associated with projected $\chi$
configurations is bounded.
In effective geometric descriptions, this constraint takes the form
\begin{equation}
  \left| \mathcal{D}_{\mathrm{loc}} \chi_{\mathrm{eff}} \right| \le c ,
\end{equation}
where $\mathcal{D}_{\mathrm{loc}} \chi_{\mathrm{eff}}$ denotes an
effective local relaxation functional.
This inequality limits the maximal rate at which effective causal
connectivity can be established within admissible descriptions.
The quantity~$c$ characterizes the causal structure of the projected regime.

Local particle propagation, signal transmission, and field interactions are
all constrained by this bound in effective spacetime descriptions.
Apparent superluminal recession velocities at cosmological scales arise from
cumulative global effects of projected ordering and do not violate this
local causal constraint.

% ----------------------------------------------------------------------------
% Section 2.8 --- Relation to Conventional Fields
% From former §3.8
% ----------------------------------------------------------------------------
\subsection{Relation to Conventional Fields}
\label{subsec:relation-to-conventional-fields}

Effective descriptions derived from projected $\chi$ configurations may
formally resemble scalar or tensor fields used in cosmology and particle
physics.
This resemblance reflects the emergence of a spacetime-based descriptive
language, not the presence of an additional fundamental field.

Energy and quantization are not fundamental attributes of~$\chi$ but arise
at the effective level as consequences of the non-injective projection.
Matter, radiation, and interactions correspond to effective degrees of
freedom arising from structural constraints, spectral organization, and
long-lived relational patterns of projected configurations.
Standard Model fields are recovered as accurate effective descriptions
within the appropriate coarse-grained regimes.

Cosmochrony does not extend the Standard Model by introducing new
fundamental fields.
It provides an ontological explanation for the emergence, applicability, and
structural properties of effective field descriptions themselves.

% ----------------------------------------------------------------------------
% Section 2.9 --- Initial Conditions and Global Structure
% From former §3.9
% ----------------------------------------------------------------------------
\subsection{Initial Conditions and Global Structure}
\label{subsec:initial-conditions-and-global-structure}

The framework does not postulate initial conditions in the conventional
temporal sense.
It assumes that~$\chi$ admits a minimal admissible ordering state,
denoted~$\chi_0$, defining a structural boundary of admissible projected
descriptions.
This state does not correspond to a distinguished moment in time but to the
earliest configurations for which an effective ordering interpretation
becomes meaningful.

In effective geometric regimes, the characteristic scale near~$\chi_0$
coincides numerically with the Planck scale.
This correspondence reflects the breakdown of projectability below this
regime, not the presence of a fundamental cutoff or underlying discreteness.

Cosmic history is interpreted as the progressive and irreversible ordering
of projected configurations away from this minimal admissible boundary.
No spacetime singularity is required at the fundamental level.
Apparent singular behavior arises only when classical notions are
extrapolated beyond the regime in which projected configurations admit a
stable geometric interpretation.

\paragraph{Ontological poverty and the growth of admissible structure.}
The minimal state~$\chi_0$ corresponds to a regime of \emph{ontological
poverty}: only a severely restricted class of simple and highly coherent
configurations can be projected.
As relaxation proceeds, the space of admissible configurations expands,
enabling the emergence of increasingly rich, localized, and hierarchical
effective structures.
The global structure of admissible descriptions is constrained by the
ordering properties of the underlying substrate.

% ----------------------------------------------------------------------------
% Section 2.10 --- Ontological Interpretation
% Merges former §4.1--4.5, §4.4/4.11 (holography), condensed
% ----------------------------------------------------------------------------
\subsection{Ontological Interpretation}
\label{sec:chi_ontology}

\subsubsection*{The \texorpdfstring{$\chi$}{χ} Substrate as a Pre-Temporal
Structural Plan}
\label{subsec:the-chi-substrate-as-a-pre-temporal-structural-plan}

The substrate~$\chi$ admits an ontological interpretation as a pre-temporal
relational structure from which spacetime, matter, and effective physical
laws emerge.
It may be heuristically described as a \emph{structural plan}: not a
dynamical history, but a complete relational organization encoding the set
of physically admissible configurations and the constraints that relate
them.
Temporal succession is emergent, corresponding to an oriented resolution
of structural relations through irreversible relaxation.
The notion of a structural plan does not introduce teleology, determinism,
or a block-universe ontology: multiple effective histories may correspond
to the same underlying relational structure through non-injective
projection.

\subsubsection*{Relational Ontology and Conceptual Lineage}
\label{subsec:relational-ontology-and-conceptual-lineage}

The relational character of~$\chi$ bears a conceptual affinity with
relational approaches in physics, notably those of
Rovelli~\cite{Rovelli1996,Rovelli2004}, which trace part of their lineage
to Aristotelian relational ontology~\cite{AristotleCategories,Shields2016}.
Cosmochrony shares the rejection of intrinsic, observer-independent
properties but extends relationalism to a deeper ontological level:
$\chi$ configurations are not relations \emph{between} fundamental objects
but relational structures that give rise to objects only upon projection.
Relativistic causality emerges without postulating spacetime as
fundamental~\cite{Rovelli2018}.
Relationality is an intrinsic property of the pre-geometric substrate from
which spacetime and physical entities jointly emerge.

\subsubsection*{Projection, Reality, and Ontological Asymmetry}
\label{subsec:projection-reality-and-ontological-asymmetry}

The emergence of spacetime is a \emph{projection} from~$\chi$, not a dual
or bidirectional description.
The projected universe is fully real at the level of physical experience
but ontologically derivative: spacetime entities and dynamical laws do not
possess ontological primacy~\cite{Rovelli2021}.
While all physical descriptions depend on the projection of~$\chi$, the
relational structure of~$\chi$ does not admit a reformulation entirely in
geometric or field-theoretic terms.

Apparent fine-tuning is reinterpreted as a selection effect imposed by
projectability: only those configurations compatible with a stable emergent
geometry appear as physically realized universes.
Cosmochrony does not postulate a multiverse---the universe is unique at the
level of physical reality, even though its underlying description in terms
of~$\chi$ may be non-unique.

The formal developments related to projection, including its fiber-bundle
formulation and the emergence of gauge interactions, are developed in
Section~\ref{sec:projection-gauge}.

\subsubsection*{Configurational State Structure}
\label{subsec:chi_state_structure}

The substrate~$\chi$ defines a \emph{configurational structure} specifying
the set of admissible macroscopic states and the allowed transitions
between them.
Physical reality corresponds to a projected realization stabilized under
finite resolution; no causal influence from projected configurations
to~$\chi$ is postulated.
What appears as temporal evolution corresponds, at the level of~$\chi$, to
an ordering of admissible configurations under projection.
Universal bounds such as~$c_\chi$ and~$\hbar_\chi$ reflect intrinsic limits
on admissible transitions within this configurational structure.

\subsubsection*{Intrinsic Structural Indeterminacy}
\label{subsec:intrinsic-structural-indeterminacy}

A perfectly deterministic, fully symmetric relational substrate would
remain physically inert.
Cosmochrony therefore postulates an \emph{intrinsic structural
indeterminacy} at the level of~$\chi$: configurations are not exhaustively
specified by a finite, closed set of relational conditions.
This indeterminacy is ontological, not dynamical---it reflects the absence
of perfect structural closure rather than random motion or noise.

Observable variability and probabilistic behavior arise only at the level
of projected descriptions.
Because the projection is non-injective, a single underlying configuration
may correspond to multiple admissible effective realizations.
Randomness is therefore \emph{projective}: it reflects the multiplicity of
effective descriptions compatible with a given pre-geometric structure.

\subsubsection*{Relation to Holographic Descriptions}
\label{subsec:clarifying-holography}

Cosmochrony is not a holographic theory in the technical sense.
It does not posit a lower-dimensional boundary description or a dual
equivalence between bulk and boundary physics.
The limitation of physically accessible information within a given
spacetime region reflects the degeneracy of underlying $\chi$
configurations corresponding to the same effective projection---a direct
consequence of non-injectivity, not of boundary-localized degrees of
freedom.
Scaling behaviors reminiscent of holography are interpreted as emergent
signatures of projection.

Similarly, Cosmochrony differs from thermodynamic approaches to gravity
(Jacobson, Verlinde) in that it does not posit entropy or information as
primitive quantities.
Thermodynamic descriptions arise only at the effective level, as secondary
languages applicable when projected configurations admit coarse-grained
statistical interpretations.

\input{part1/ch02/sec-energy-mass-constants}
