% ----------------------------------------------------------------------------
% Section 2.2 --- The Geometric Effective Description of chi Dynamics
% From former §3.2, trimmed of redundant caveats
% ----------------------------------------------------------------------------
\subsection{The Geometric Effective Description of
\texorpdfstring{$\chi$}{χ} Dynamics}
\label{subsec:geometric-action}

\subsubsection*{Effective Observables from
\texorpdfstring{$\chi$}{χ} Correlations}

Quantities conventionally described in geometric terms---such as time
intervals, spatial separation, and causal ordering---arise as effective
summaries of relational patterns within the $\chi$ substrate, accessed only
through projected, coarse-grained representations.
The configurations~$\sigma$ represent internal relational states of~$\chi$
and are defined without reference to external spacetime coordinates.
The measure~$d\mu(\sigma)$ denotes an invariant integration over
configuration space, defined intrinsically from the correlation structure
associated with~$\chi$.

\paragraph{Effective scalar descriptor.}
In regimes where projected $\chi$ configurations admit a stable geometric
interpretation, we introduce an \emph{effective scalar descriptor}
$\chi_{\mathrm{eff}}$.
This quantity is a coarse-grained, projected representation of relational
and spectral features of~$\chi$, defined only within the emergent spacetime
description.
It does not correspond to the fundamental substrate itself.

Operational time intervals are defined from the accumulated ordering of
projected configurations along paths in configuration space:
\begin{equation}
  \tau_{AB} \;\propto\;
  \int_{\gamma_{AB}}
    \mathcal{D}_{\lambda} \chi_{\mathrm{eff}} \, d\lambda ,
\end{equation}
where $\mathcal{D}_{\lambda} \chi_{\mathrm{eff}}$ is an effective
relaxation functional and $\lambda$ an ordering parameter.

Operational spatial separation is quantified by the decay of correlations
between effective descriptors:
\begin{equation}
  d(x,y) \;\propto\;
  - \log\!\left(
    \frac{\langle \chi_{\mathrm{eff}}(x)\,
      \chi_{\mathrm{eff}}(y)\rangle}
    {\langle \chi_{\mathrm{eff}}^{\,2}\rangle}
  \right),
\end{equation}
where $\langle\chi_{\text{eff}}(x)\,\chi_{\text{eff}}(y)\rangle$ denotes
an effective correlation functional encoding relational proximity between
projected configurations~$x$ and~$y$.

\begin{figure}[t]
  \centering
  \begin{tikzpicture}[
    font=\small,
    node distance=14mm,
    box/.style={draw, rounded corners, align=center, inner sep=6pt},
    arrow/.style={-Latex, thick},
    note/.style={align=left, font=\footnotesize}
  ]
    \node[box] (chi)
      {$\chi$\\\footnotesize infra-physical\\
       \footnotesize relational substrate};
    \node[box, below=of chi] (chieff)
      {$\chi_{\mathrm{eff}}$\\\footnotesize physical effective
       reality\\\footnotesize (projectable regime)};
    \node[box, below=of chieff] (obs)
      {Observables\\\footnotesize context-dependent\\
       \footnotesize operational quantities};
    \draw[arrow] (chi) -- node[right=2mm, note]
      {infra-physical\\projection $\pi$\\
       \footnotesize (generally non-injective)} (chieff);
    \draw[arrow] (chieff) -- node[right=2mm, note]
      {operational\\projection $\mathcal{O}$\\
       \footnotesize (contextual access)} (obs);
    \node[
      draw, dashed, rounded corners,
      fit=(chieff) (obs),
      inner sep=10pt,
      label={[font=\footnotesize]right:Factorizable regime}
    ] {};
  \end{tikzpicture}
  \caption{%
    Ontological pipeline in Cosmochrony.
    The infra-physical projection~$\pi$ maps the fundamental $\chi$
    substrate to an effective physical reality~$\chi_{\mathrm{eff}}$ in
    projectable regimes, generally in a non-injective manner.
    Physical observables arise from an operational
    projection~$\mathcal{O}$ that specifies how
    $\chi_{\mathrm{eff}}$ is accessed under measurement contexts.}
  \label{fig:two-stage-projection}
\end{figure}

\subsubsection*{Effective Metric as a Descriptive Tool}

In regimes where projected configurations exhibit smooth and stable
correlation patterns, the relational observables may be summarized by an
operational tensor $g_{\mu\nu}[\chi_{\mathrm{eff}}]$.
This effective metric is a derived descriptor summarizing relational
correlations, not a fundamental geometric structure.
No background $\eta_{\mu\nu}$ is assumed; Minkowski space appears only as an
effective approximation in weak-gradient regimes.

In projectable regimes admitting a low-dimensional embedding, one may write
locally
\[
  d(i,j)^2 \;\approx\;
    g_{\mu\nu}(x)\,\Delta x^\mu \Delta x^\nu .
\]
The continuum line element~$ds^2$ is therefore a derived descriptive
construct.

\subsubsection*{Consistency with General Relativity}

The effective metric reproduces the phenomenology of general relativity in
appropriate regimes: the weak-field limit yields Einstein-like dynamics; near
localized $\chi$ excitations, the metric encodes time dilation and spatial
curvature as emergent consequences of inhibited relaxation; homogeneous
relaxation yields an effective Hubble-like expansion law.
All predictive content resides in the underlying $\chi$ dynamics.

\paragraph{Operational origin of the effective metric.}
The explicit construction of $g_{\mu\nu}$ from operational distances is
given in Appendix~E, where geometric quantities are shown to arise only in
projectable regimes admitting a smooth continuum approximation.
