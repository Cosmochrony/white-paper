% ----------------------------------------------------------------------------
% Section 2.3 --- Physical Interpretation
% From former §3.3, condensed
% ----------------------------------------------------------------------------
\subsection{Physical Interpretation}
\label{subsec:physical-interpretation}

In regimes where the infra-physical projection from $\chi$ to an effective
description~$\chi_{\mathrm{eff}}$ yields a factorisable structure, an
approximate decomposition into subsystems and the definition of operational
observables become possible.
This factorisation underlies the emergence of classical locality,
compatibility of measurements, and standard relativistic descriptions, as
illustrated in Fig.~\ref{fig:classical-factorisable}.

\begin{figure}[t]
  \centering
  \begin{tikzpicture}[
    font=\small,
    node distance=10mm,
    box/.style={draw, rounded corners, align=center, inner sep=6pt},
    arrow/.style={-Latex, thick},
    note/.style={align=left, font=\footnotesize},
    dashedbox/.style={draw, dashed, rounded corners, inner sep=6pt}
  ]
    \node[box] (chi)
      {$\chi$\\\footnotesize infra-physical substrate};
    \node[box, below=of chi] (chieff)
      {$\chi_{\mathrm{eff}}$\\\footnotesize factorisable regime};
    \node[dashedbox, below=of chieff, minimum width=6.6cm] (decomp) {
      \begin{tabular}{c}
        \footnotesize
        $\chi_{\mathrm{eff}} \simeq
          \chi_{\mathrm{eff}}^{(A)}
          \otimes \chi_{\mathrm{eff}}^{(B)}$\\
        \footnotesize (independent subsystems)
      \end{tabular}
    };
    \node[box, below left=10mm and 12mm of decomp] (obsA)
      {Local observables\\in subsystem $A$};
    \node[box, below right=10mm and 12mm of decomp] (obsB)
      {Local observables\\in subsystem $B$};
    \draw[arrow] (chi) -- node[right=2mm, note]
      {infra-physical\\projection $\pi$} (chieff);
    \draw[arrow] (chieff) -- (decomp);
    \draw[arrow] (decomp) -- node[left=2mm, note]
      {operational\\projection $\mathcal{O}_A$} (obsA);
    \draw[arrow] (decomp) -- node[right=2mm, note]
      {operational\\projection $\mathcal{O}_B$} (obsB);
  \end{tikzpicture}
  \caption{Classical (factorisable) regime.
    After the infra-physical projection~$\pi$, the effective description
    admits an approximate decomposition into independent subsystems.
    Operational projections then yield compatible local observables,
    recovering standard classical descriptions.
    This contrasts with the non-factorisable regimes discussed in
    Section~\ref{sec:quantum-phenomena-and-entanglement}.}
  \label{fig:classical-factorisable}
\end{figure}

Because the projection from $\chi$ to $\chi_{\mathrm{eff}}$ is generically
non-injective, distinct $\chi$ configurations may correspond to identical
effective descriptions, while a single underlying configuration may admit
multiple correlated operational realizations.
This structural asymmetry is central to the emergence of both classical and
quantum phenomenology.
The physical content of the theory resides entirely in the dynamics of the
fundamental substrate, while spacetime notions function as emergent,
context-dependent descriptive tools.
