\clearpage
\part{Foundations}
  \label{part:foundations}

  This part establishes the conceptual and ontological basis of the Cosmochrony framework.
  Its purpose is to introduce the minimal assumptions required to describe physical reality
  without presupposing spacetime, geometry, or quantization as fundamental primitives.

  We define the pre-geometric relational substrate $\chi$, clarify its ontological status,
  and identify the structural principles governing its admissible configurations.
  Special emphasis is placed on the role of non-injective projection, bounded relaxation,
  and intrinsic ordering as the common origin of time, geometry, and physical observables.

  Rather than postulating spacetime or fields a priori, this part shows how effective
  geometric and dynamical descriptions arise only in specific projective regimes.
  All subsequent developments rely on the definitions, conventions, and limitations
  introduced here, which will not be reiterated in later parts.

  \clearpage
  \clearpage

\section{Radiation and Quantization}
\label{sec:radiation-and-quantization}

% ----------------------------------------------------------------------------
% Section 9.1 --- Radiation as chi--Matter Interaction
% From former §13.1, condensed
% ----------------------------------------------------------------------------
\subsection{Radiation as
\texorpdfstring{$\chi$}{χ}--Matter Interaction}
\label{subsec:radiation-as-chimatter-interaction}

Radiation arises when a localized, relaxation-resistant configuration
undergoes a transition toward a less constrained state.
The excess relational content ceases to admit a particle-like projected
description and becomes expressible only through delocalized projected
modes.
In effective spacetime descriptions, this redistribution appears as
radiative emission---a transfer of descriptive weight from particle-like
configurations to propagating field-like descriptions, without invoking
discrete objects or stochastic processes.

% ----------------------------------------------------------------------------
% Section 9.2 --- Emergence of Photons
% From former §13.2, condensed
% ----------------------------------------------------------------------------
\subsection{Emergence of Photons}
\label{subsec:emergence-of-photons}

Photons are not fundamental entities.
Prior to emission or detection, no photon exists as an independent
object.
A reconfiguration of relational structure within~$\chi$ ceases to admit
a localized projection and becomes expressible only through extended,
delocalized effective modes---represented in spacetime as electromagnetic
waves.

Photon-like events emerge only at interaction: when a delocalized
projective mode becomes locally constrained by interaction with a
localized excitation, the projection resolves into a discrete transfer of
relaxation capacity.
Quantization is a property of interaction and local reprojection, not of
propagation.
Wave--particle duality reflects a duality of description rather than of
underlying ontology.
Interference phenomena arise from the coherence of delocalized projective
modes; individual detection events correspond to localized reprojections.

% ----------------------------------------------------------------------------
% Section 4.5 --- Energy--Frequency Relation
% From former §6.5
% ----------------------------------------------------------------------------
\subsection{Energy--Frequency Relation}
\label{subsec:energy-frequency-solitons}

The energy associated with a particle-like excitation is linked to a
characteristic internal spectral scale of the corresponding projected
configuration.
Configurations associated with higher characteristic frequencies
correspond to more tightly constrained structures and encode greater
effective resistance to relaxation.
This yields
\begin{equation}
  E \propto \nu ,
\end{equation}
where $\nu$ characterizes the spectral scale of internal organization.
Planck's constant appears as an effective proportionality factor whose
universality reflects the robustness of spectral scales in the current
relaxation epoch.
A more explicit realization in the context of radiation is presented in
Section~\ref{subsec:energy-frequency-radiation}.

% ----------------------------------------------------------------------------
% Section 9.4 --- Vacuum Fluctuations and the Casimir Effect
% From former §13.4, condensed
% ----------------------------------------------------------------------------
\subsection{Vacuum Fluctuations and the Casimir Effect}
\label{subsec:vacuum-fluctuations-and-the-casimir-effect}

Vacuum fluctuations reflect the intrinsic structural indeterminacy
of~$\chi$ in regimes where no stable localized excitations are present.
The relaxation admits a wide range of locally compatible projective
descriptions; these fluctuations represent variability of effective
descriptions rather than physical energy stored in the vacuum.

When material boundaries impose structural constraints on local
projectability, certain effective descriptions become incompatible with
the boundary conditions, reducing the set of admissible projective
configurations between the boundaries.
The Casimir effect arises from this asymmetry: a difference in the
density of admissible effective reprojections, manifesting as a pressure
on the confining surfaces.
No fundamental vacuum energy density or propagating vacuum modes are
required.

% ----------------------------------------------------------------------------
% Section 9.5 --- Weakly Interacting Radiation
% From former §13.5, condensed
% ----------------------------------------------------------------------------
\subsection{Weakly Interacting Radiation}
\label{subsec:weakly-interacting-radiation}

Weakly interacting radiation corresponds to delocalized projective
regimes whose structural contrast is insufficient to efficiently induce
localized reprojection upon encountering matter.
Low-frequency or weakly coupled modes are characterized by smooth, slowly
varying relational structure, strongly suppressing the probability of
stable localized energy transfer.
Small interaction cross sections reflect the low likelihood that a given
projective configuration satisfies the geometric and topological
conditions required for localized reprojection.

\input{5-predictions_discussion_and_conclusion/09-radiation-and-quantization/sec-summary-radiation}

  \clearpage

\section{Radiation and Quantization}
\label{sec:radiation-and-quantization}

% ----------------------------------------------------------------------------
% Section 9.1 --- Radiation as chi--Matter Interaction
% From former §13.1, condensed
% ----------------------------------------------------------------------------
\subsection{Radiation as
\texorpdfstring{$\chi$}{χ}--Matter Interaction}
\label{subsec:radiation-as-chimatter-interaction}

Radiation arises when a localized, relaxation-resistant configuration
undergoes a transition toward a less constrained state.
The excess relational content ceases to admit a particle-like projected
description and becomes expressible only through delocalized projected
modes.
In effective spacetime descriptions, this redistribution appears as
radiative emission---a transfer of descriptive weight from particle-like
configurations to propagating field-like descriptions, without invoking
discrete objects or stochastic processes.

% ----------------------------------------------------------------------------
% Section 9.2 --- Emergence of Photons
% From former §13.2, condensed
% ----------------------------------------------------------------------------
\subsection{Emergence of Photons}
\label{subsec:emergence-of-photons}

Photons are not fundamental entities.
Prior to emission or detection, no photon exists as an independent
object.
A reconfiguration of relational structure within~$\chi$ ceases to admit
a localized projection and becomes expressible only through extended,
delocalized effective modes---represented in spacetime as electromagnetic
waves.

Photon-like events emerge only at interaction: when a delocalized
projective mode becomes locally constrained by interaction with a
localized excitation, the projection resolves into a discrete transfer of
relaxation capacity.
Quantization is a property of interaction and local reprojection, not of
propagation.
Wave--particle duality reflects a duality of description rather than of
underlying ontology.
Interference phenomena arise from the coherence of delocalized projective
modes; individual detection events correspond to localized reprojections.

% ----------------------------------------------------------------------------
% Section 4.5 --- Energy--Frequency Relation
% From former §6.5
% ----------------------------------------------------------------------------
\subsection{Energy--Frequency Relation}
\label{subsec:energy-frequency-solitons}

The energy associated with a particle-like excitation is linked to a
characteristic internal spectral scale of the corresponding projected
configuration.
Configurations associated with higher characteristic frequencies
correspond to more tightly constrained structures and encode greater
effective resistance to relaxation.
This yields
\begin{equation}
  E \propto \nu ,
\end{equation}
where $\nu$ characterizes the spectral scale of internal organization.
Planck's constant appears as an effective proportionality factor whose
universality reflects the robustness of spectral scales in the current
relaxation epoch.
A more explicit realization in the context of radiation is presented in
Section~\ref{subsec:energy-frequency-radiation}.

% ----------------------------------------------------------------------------
% Section 9.4 --- Vacuum Fluctuations and the Casimir Effect
% From former §13.4, condensed
% ----------------------------------------------------------------------------
\subsection{Vacuum Fluctuations and the Casimir Effect}
\label{subsec:vacuum-fluctuations-and-the-casimir-effect}

Vacuum fluctuations reflect the intrinsic structural indeterminacy
of~$\chi$ in regimes where no stable localized excitations are present.
The relaxation admits a wide range of locally compatible projective
descriptions; these fluctuations represent variability of effective
descriptions rather than physical energy stored in the vacuum.

When material boundaries impose structural constraints on local
projectability, certain effective descriptions become incompatible with
the boundary conditions, reducing the set of admissible projective
configurations between the boundaries.
The Casimir effect arises from this asymmetry: a difference in the
density of admissible effective reprojections, manifesting as a pressure
on the confining surfaces.
No fundamental vacuum energy density or propagating vacuum modes are
required.

% ----------------------------------------------------------------------------
% Section 9.5 --- Weakly Interacting Radiation
% From former §13.5, condensed
% ----------------------------------------------------------------------------
\subsection{Weakly Interacting Radiation}
\label{subsec:weakly-interacting-radiation}

Weakly interacting radiation corresponds to delocalized projective
regimes whose structural contrast is insufficient to efficiently induce
localized reprojection upon encountering matter.
Low-frequency or weakly coupled modes are characterized by smooth, slowly
varying relational structure, strongly suppressing the probability of
stable localized energy transfer.
Small interaction cross sections reflect the low likelihood that a given
projective configuration satisfies the geometric and topological
conditions required for localized reprojection.

\input{5-predictions_discussion_and_conclusion/09-radiation-and-quantization/sec-summary-radiation}

