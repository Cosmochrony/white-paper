% ============================================================================
% Chapter 1 --- Introduction and Motivation
% Merges former sections 1 and 2 of Cosmochrony v1.13beta1
% ============================================================================
\clearpage

\section{Introduction and Motivation}
\label{sec:introduction}

Modern fundamental physics is built upon two highly successful yet conceptually
distinct frameworks: quantum mechanics and general
relativity~\cite{Dirac1930,Einstein1915}.
Quantum theory accurately describes microscopic phenomena, while general
relativity provides a geometric account of gravitation and spacetime dynamics
at macroscopic and cosmological scales.
Despite their empirical success, these theories rely on incompatible
foundational assumptions and resist unification within a single coherent
conceptual framework~\cite{MisnerThorneWheeler1973,Weinberg1972,rovelli2004quantum}.

In this work, we introduce \emph{Cosmochrony}\footnote{%
  From \textit{$\kappa\acute{o}\sigma\mu o\varsigma$} and
  \textit{$\chi\rho\acute{o}\nu o\varsigma$}, denoting a framework in which
  cosmic structure and temporal ordering emerge from a common pre-geometric
  substrate.},
a deliberately minimalist framework whose guiding hypothesis is that spacetime
geometry, gravitation, and quantum phenomena emerge from the dynamics of a
single continuous underlying entity, denoted~$\chi$, whose effective
descriptions arise through a constrained projection process.
This projection is generically non-injective, allowing distinct underlying
$\chi$-configurations to correspond to identical effective observables and,
conversely, allowing a single underlying configuration to admit multiple
correlated effective realizations.
A detailed and formal treatment of this projection asymmetry is given in
Section~\ref{subsec:projection-reality-and-ontological-asymmetry}.

The substrate~$\chi$ is not defined on a pre-existing spacetime manifold, nor
is it interpreted as a conventional physical field propagating within
spacetime.
Instead, spacetime notions themselves arise as effective and relational
descriptions, applicable only once suitable stability and projection conditions
are satisfied.
The precise ontological status of~$\chi$ and the minimal assumptions governing
its dynamics are introduced systematically in
Section~\ref{sec:definition-and-fundamental-properties-of-the-chi-field}.

The fundamental dynamical postulate of Cosmochrony is that $\chi$ undergoes an
irreversible relaxation process, locally bounded by an invariant structural
propagation speed.
The effective projection of this bound defines the observed causal limit~$c$
and induces an intrinsic ordering of physical processes, identified with
physical time.
Spatial relations emerge relationally from differences, gradients, and
correlations of~$\chi$ once a stable geometric regime is reached.
Within this perspective, spacetime expansion, gravitation, particle-like
excitations, radiation processes, and quantum correlations are emergent
phenomena associated with specific configurations or interactions of the
underlying substrate.
In particular, discreteness, inertial mass, and quantum indeterminacy arise
from structural constraints on projection and relaxation.

\begin{conventionbox}{Convention on spacetime language}
  Throughout this article, spacetime coordinates, metric quantities,
  variational principles, and differential geometry are employed strictly as
  effective descriptive tools, valid only in regimes where $\chi$-configurations
  admit a stable geometric interpretation.
  They are not treated as fundamental postulates of the theory.
  Technical reconstructions and mathematical details are confined to the
  appropriate effective regimes and collected in the appendices.
  This point will not be reiterated in subsequent sections.
\end{conventionbox}

Cosmochrony does not aim to replace the Standard Model or general relativity
in their empirically validated domains, nor does it claim to provide a final
unification of quantum theory and gravitation.
It offers an exploratory and internally coherent framework designed to clarify
the physical origin of time, geometry, gravitation, and quantum correlations
within a single relational dynamics.
The unifying thread is the idea that apparent multiplicity, indeterminacy, and
nonlocality reflect structural features of projection.

% ----------------------------------------------------------------------------
% Section 1.1 --- The Unification Problem
% Merges former §2.1 and §2.2
% ----------------------------------------------------------------------------
\subsection{The Unification Problem}
\label{subsec:conceptual-tension-between-quantum-theory-and-gravitation}

Quantum mechanics and general relativity differ not only in their mathematical
formalisms but also in their foundational concepts.
Quantum theory is intrinsically probabilistic, relies on a fixed causal
structure, and treats time as an external
parameter~\cite{Dirac1930,Born1926}.
General relativity describes gravitation as the dynamics of spacetime geometry
itself, with time acquiring a coordinate-dependent and observer-relative
status~\cite{Einstein1915,MisnerThorneWheeler1973}.

This conceptual mismatch becomes acute in regimes where both quantum effects
and strong gravitational fields are relevant, such as near spacetime
singularities or in the early
universe~\cite{penrose1989emperors,Prigogine1997}.
Direct attempts to quantize gravity encounter persistent difficulties,
including the problem of time, non-renormalizability, and the absence of a
preferred background structure.
These difficulties suggest that the tension may reflect a deeper
incompatibility in the assumed ontological status of time and geometry.

Several major research programs have sought to address these challenges.
Quantum field theory in curved spacetime accounts for particle creation and
vacuum effects but retains a classical spacetime
background~\cite{weinberg1972gravitation}.
Canonical and covariant approaches to quantum gravity attempt to quantize
spacetime geometry itself, often at the cost of substantial mathematical
complexity and interpretational ambiguity.
String theory introduces extended fundamental objects and higher-dimensional
structures, offering deep mathematical unification but leading to a large
space of possible low-energy
realizations~\cite{rovelli2004quantum}.
While internally rich, these approaches face ongoing challenges concerning
empirical testability and the physical interpretation of their fundamental
degrees of freedom.

These limitations motivate the exploration of alternative perspectives in
which spacetime geometry, matter, and quantum behavior emerge from a common
underlying mechanism operating at a pre-geometric level.

\input{part1/ch01/sec-minimalism}
\input{part1/ch01/sec-time}
% ----------------------------------------------------------------------------
% Section 1.4 --- Conceptual Context and Related Approaches
% From former §1.1
% ----------------------------------------------------------------------------
\subsection{Conceptual Context and Related Approaches}
\label{subsec:conceptual-context-and-related-approaches}

The idea that spacetime geometry and gravitation may be emergent has been
explored in a variety of contemporary theoretical frameworks.
Several approaches interpret the spacetime metric as an effective description
arising from deeper geometric, informational, or dynamical structures, and
recast gravitation as a collective
phenomenon~\cite{Nye2024,Singh2025}.
Cosmochrony belongs to this broad conceptual lineage while adopting a
deliberately minimalist ontological stance: a single pre-geometric relational
substrate~$\chi$ whose irreversible relaxation governs the emergence of
physical observables.

Like Loop Quantum Gravity (LQG), Cosmochrony holds that spacetime geometry is
not fundamental~\cite{Rovelli2004}.
However, the two frameworks operate at distinct conceptual levels.
LQG provides a quantized description of geometry once a spacetime structure is
already assumed, encoding areas and volumes through spin networks and
holonomies.
Related ideas have also appeared in modern holographic approaches formulated in
asymptotically flat spacetimes, such as celestial holography, where scattering
amplitudes are reorganized as conformal correlators on the celestial
sphere~\cite{StromingerInfraredLectures}.

Cosmochrony addresses an earlier and more primitive level.
It does not quantize geometry but seeks to explain how geometric notions
themselves arise as effective, coarse-grained descriptions of underlying
$\chi$-configurations.
The emergence of spacetime is mediated by a non-injective projection from the
pre-geometric substrate to effective observables, allowing geometric,
dynamical, and quantum features to appear only once specific relational and
spectral conditions are met.
From this perspective, Cosmochrony does not compete with LQG but conceptually
precedes it: it aims to account for the physical origin of the geometric
degrees of freedom that may subsequently be quantized within such approaches,
while remaining agnostic about the detailed form of their quantization at the
effective level.

\input{part1/ch01/sec-scope}
% ----------------------------------------------------------------------------
% Section 1.6 --- Structure of the Paper
% New section replacing the scattered plan paragraphs
% ----------------------------------------------------------------------------
\subsection{Structure of the Paper}
\label{subsec:structure-of-the-paper}

The paper is organized in five parts.
Part~I introduces the $\chi$ substrate, its minimal structural properties, and
its ontological status
(Section~\ref{sec:definition-and-fundamental-properties-of-the-chi-field}).
Part~II develops the effective dynamics of~$\chi$
(Section~\ref{sec:effective-dynamics-of-the-chi-substrate}), the emergence of
particle-like excitations
(Section~\ref{sec:particles-as-localized-excitations-of-the-chi-field}), and
the projection fiber with gauge emergence
(Section~\ref{sec:projection-fiber-and-gauge-emergence}).
Part~III addresses gravity as a collective effect
(Section~\ref{sec:gravity-as-a-collective-effect-of-particle-excitations}) and
cosmological implications
(Section~\ref{sec:cosmological-implications}).
Part~IV presents the unified treatment of quantum phenomena, entanglement, and
the relation to quantum formalism
(Section~\ref{sec:quantum-phenomena-and-entanglement}).
Part~V collects radiation and quantization
(Section~\ref{sec:radiation-and-quantization}), the spectral mass spectrum
(Section~\ref{sec:spectral-mass-spectrum-and-hierarchy}), testable predictions
(Section~\ref{sec:testable-predictions-and-observational-signatures}), and the
discussion with conclusion
(Section~\ref{sec:discussion-and-comparison-with-existing-frameworks}).
Appendices~\ref{appendix:math}--\ref{appendix:relational-glossary} provide
mathematical foundations, conceptual extensions, cosmological and numerical
supplements, and a glossary of notation.


For convenience, a glossary summarizing the main quantities and operators used
throughout the article is provided in Appendix~\ref{appendix:glossary}.
