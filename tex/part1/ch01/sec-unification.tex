% ----------------------------------------------------------------------------
% Section 1.1 --- The Unification Problem
% Merges former §2.1 and §2.2
% ----------------------------------------------------------------------------
\subsection{The Unification Problem}
\label{subsec:conceptual-tension-between-quantum-theory-and-gravitation}

Quantum mechanics and general relativity differ not only in their mathematical
formalisms but also in their foundational concepts.
Quantum theory is intrinsically probabilistic, relies on a fixed causal
structure, and treats time as an external
parameter~\cite{Dirac1930,Born1926}.
General relativity describes gravitation as the dynamics of spacetime geometry
itself, with time acquiring a coordinate-dependent and observer-relative
status~\cite{Einstein1915,MisnerThorneWheeler1973}.

This conceptual mismatch becomes acute in regimes where both quantum effects
and strong gravitational fields are relevant, such as near spacetime
singularities or in the early
universe~\cite{penrose1989emperors,Prigogine1997}.
Direct attempts to quantize gravity encounter persistent difficulties,
including the problem of time, non-renormalizability, and the absence of a
preferred background structure.
These difficulties suggest that the tension may reflect a deeper
incompatibility in the assumed ontological status of time and geometry.

Several major research programs have sought to address these challenges.
Quantum field theory in curved spacetime accounts for particle creation and
vacuum effects but retains a classical spacetime
background~\cite{weinberg1972gravitation}.
Canonical and covariant approaches to quantum gravity attempt to quantize
spacetime geometry itself, often at the cost of substantial mathematical
complexity and interpretational ambiguity.
String theory introduces extended fundamental objects and higher-dimensional
structures, offering deep mathematical unification but leading to a large
space of possible low-energy
realizations~\cite{rovelli2004quantum}.
While internally rich, these approaches face ongoing challenges concerning
empirical testability and the physical interpretation of their fundamental
degrees of freedom.

These limitations motivate the exploration of alternative perspectives in
which spacetime geometry, matter, and quantum behavior emerge from a common
underlying mechanism operating at a pre-geometric level.
