% ----------------------------------------------------------------------------
% Section 1.4 --- Conceptual Context and Related Approaches
% From former §1.1
% ----------------------------------------------------------------------------
\subsection{Conceptual Context and Related Approaches}
\label{subsec:conceptual-context-and-related-approaches}

The idea that spacetime geometry and gravitation may be emergent has been
explored in a variety of contemporary theoretical frameworks.
Several approaches interpret the spacetime metric as an effective description
arising from deeper geometric, informational, or dynamical structures, and
recast gravitation as a collective
phenomenon~\cite{Nye2024,Singh2025}.
Cosmochrony belongs to this broad conceptual lineage while adopting a
deliberately minimalist ontological stance: a single pre-geometric relational
substrate~$\chi$ whose irreversible relaxation governs the emergence of
physical observables.

Like Loop Quantum Gravity (LQG), Cosmochrony holds that spacetime geometry is
not fundamental~\cite{Rovelli2004}.
However, the two frameworks operate at distinct conceptual levels.
LQG provides a quantized description of geometry once a spacetime structure is
already assumed, encoding areas and volumes through spin networks and
holonomies.
Related ideas have also appeared in modern holographic approaches formulated in
asymptotically flat spacetimes, such as celestial holography, where scattering
amplitudes are reorganized as conformal correlators on the celestial
sphere~\cite{StromingerInfraredLectures}.

Cosmochrony addresses an earlier and more primitive level.
It does not quantize geometry but seeks to explain how geometric notions
themselves arise as effective, coarse-grained descriptions of underlying
$\chi$-configurations.
The emergence of spacetime is mediated by a non-injective projection from the
pre-geometric substrate to effective observables, allowing geometric,
dynamical, and quantum features to appear only once specific relational and
spectral conditions are met.
From this perspective, Cosmochrony does not compete with LQG but conceptually
precedes it: it aims to account for the physical origin of the geometric
degrees of freedom that may subsequently be quantized within such approaches,
while remaining agnostic about the detailed form of their quantization at the
effective level.
