\subsection{Topological Stability}
  \label{subsec:topological-stability}

  The stability of particle-like excitations in Cosmochrony does not rely on
  fundamental conserved charges postulated \emph{a priori}.
  It arises instead at the level of effective descriptions, from intrinsic
  structural constraints on admissible projected $\chi$ configurations.
  Certain projected configurations exhibit non-trivial internal organization
  that prevents them from being continuously deformed into homogeneous effective
  descriptions without violating admissibility conditions.

  This form of stability is topological in character.
  It reflects the existence of inequivalent classes of admissible projected
  configurations that cannot be smoothly connected through continuous
  reconfiguration while preserving monotonic relaxation ordering.
  As a result, particle-like excitations appear discrete and robust under
  perturbations, without invoking externally imposed symmetries or fundamental
  conservation laws.

  Importantly, these topological distinctions are not defined with respect to a
  pre-existing spacetime geometry.
  They are properties of the space of admissible projected descriptions itself
  and remain meaningful even in regimes where no effective geometric
  interpretation applies.
  When geometric representations are introduced, they serve solely as
  descriptive tools valid in projectable regimes.

  The long-lived character of solitonic structures therefore follows from the
  incompatibility between distinct classes of admissible projected
  configurations, rather than from a dynamical balance of forces or nonlinear
  self-interactions.
  Particle stability is thus understood as a structural consequence of
  projection and admissibility, fully consistent with the pre-geometric and
  relational foundations of the Cosmochrony framework.

  Detailed geometric constructions of topological solitons, including vortex,
  skyrmion, and knotted configurations, are provided in
  Appendix~\ref{app:topological_solitons}.
  The fully relational formulation of topological stability, independent of
  geometric representations, is developed in
  Appendix~\ref{app:relational_topological_stability}.

  \paragraph{Topological Stability.}

    The stability of particle-like excitations arises from the
    \textbf{non-trivial topology of their internal configuration space},
    defined by the admissibility constraints of the projection.
    Such configurations cannot be continuously deformed into homogeneous
    relaxation states without violating these constraints.

    This form of topological protection does not rely on spacetime geometry,
    nor on the existence of an underlying manifold or metric structure.
    It is a purely structural property of the relational $\chi$ substrate
    and therefore persists even in regimes where a geometric or projectable
    description breaks down.
