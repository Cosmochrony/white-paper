\subsection{Implications for quantum computation}
  \label{subsec:implications-quantum-computation}

  The Cosmochrony framework also bears nontrivial implications for the conceptual
  foundations of quantum computation.
  While no specific computational model is postulated in this work, several structural
  features of the framework suggest a reinterpretation of what is commonly referred to
  as \emph{quantum computational advantage}.

  In standard quantum information theory, computational gains are typically attributed
  to intrinsically quantum resources such as superposition, entanglement, and unitary
  interference.
  Within Cosmochrony, however, quantum correlations are not fundamental dynamical
  phenomena, but emerge from the generally non-injective projection between the
  underlying relational substrate $\chi$ and effective observable descriptions.
  As a consequence, non-factorizable correlations arise at the level of observables
  without requiring fundamentally nonlocal dynamics or intrinsically quantum degrees
  of freedom.

  From this perspective, the effective resources exploited in quantum computation may
  be reinterpreted as manifestations of structural non-injectivity and global consistency
  constraints, rather than as exclusively quantum dynamical effects.
  Multiple underlying relational configurations may correspond to the same effective
  observable state, allowing correlated outcomes to emerge without explicit information
  exchange or stepwise algorithmic coordination.
  This mechanism provides a natural origin for forms of effective parallelism and global
  constraint satisfaction that are usually ascribed to quantum superposition and
  entanglement.

  An important consequence of this viewpoint is that part of the computational advantage
  associated with quantum systems may not depend uniquely on the implementation of
  fully coherent quantum hardware.
  Instead, it may reflect more general properties of non-factorizable descriptive
  mappings and collective relaxation processes.
  In such a setting, computation is more naturally understood as the selection of
  structurally admissible configurations under global constraints, rather than as the
  execution of a sequence of reversible logical operations.

  This observation does not invalidate the relevance of quantum computing architectures,
  which remain effective physical realizations of non-injective and non-factorizable
  descriptions.
  However, it suggests that the source of their advantage may be broader than usually
  assumed, and potentially accessible to alternative architectures exploiting analogous
  structural principles, such as constraint-based optimization, spectral relaxation, or
  collective analog computation.

  Within Cosmochrony, computation itself may therefore be regarded as an emergent
  process, grounded in irreversible relaxation toward globally consistent configurations
  of the relational substrate.
  From this standpoint, quantum computation appears as a particular instantiation of a
  more general class of non-algorithmic, non-symbolic computational processes, rather
  than as a fundamentally separate computational paradigm.
