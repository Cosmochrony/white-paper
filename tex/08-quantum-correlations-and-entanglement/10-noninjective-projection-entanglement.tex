\subsection{Structural Stability of Projected Descriptions}
  \label{subsec:noninjective-projection-entanglement}

  Quantum entanglement, measurement, and decay correspond to distinct stability regimes
  of projected descriptions in Cosmochrony.
  Building on the non-injectivity of the projection $\Pi$ established in
  Section~\ref{sec:relational_projection}, we analyze how non-factorizable projected
  descriptions may remain stable, become unstable, or fragment into localized
  factorizable descriptions under admissible interactions and fluctuations.

  A single relational configuration of \(\chi\) may admit multiple effective
  descriptions that cannot be decomposed into independent subsystems without violating
  admissibility constraints.

  Quantum entanglement corresponds to the regime in which such a non-factorizable
  projected description remains structurally stable.
  Although effective observables may be associated with spatially separated regions, the
  projected configuration retains its unity and cannot be expressed as a product of
  independent components.
  Nonlocal correlations therefore reflect the relational character of the underlying
  \(\chi\)-configuration rather than any superluminal influence.

  Measurement and decoherence correspond to a selective stabilization within the space of
  admissible projected descriptions.
  Interaction with an environment amplifies certain relational features while rendering
  alternative projected descriptions inadmissible.
  The apparent collapse of the wavefunction thus reflects a loss of admissibility of
  non-selected descriptions, rather than a fundamental discontinuity at the level of the
  \(\chi\)-substrate.

  Particle decay represents a distinct but closely related regime.
  Here, the non-factorizable projected description becomes unstable under admissible
  fluctuations.
  No single projected configuration remains admissible, and stability is recovered only
  through factorization into several localized projected configurations.
  In this sense, decay may be understood as a structural transition from non-factorizable
  to factorizable projected descriptions.

  Entanglement, measurement, and decay therefore arise from a common structural origin:
  the stability properties of non-factorizable projected descriptions under admissible
  interactions and fluctuations.
  They differ not in the nature of the underlying $\chi$ substrate, but in the dynamical
  response of projected descriptions to perturbations within the admissible spectrum.

  \begin{figure}[t]
    \centering
    \begin{tikzpicture}[
      node distance=10mm and 18mm,
      box/.style={draw, rounded corners, align=center, inner sep=6pt},
      arr/.style={->, thick},
      lab/.style={font=\small, align=center},
      faint/.style={font=\small}
    ]

% Top chain
      \node[box] (chi) {$\chi$\\[-2pt]\footnotesize unique relational configuration};
      \node[box, below=of chi] (proj) {non-injective\\[-2pt]\footnotesize projection $\Pi$};
      \node[box, below=of proj] (chieff) {$\chi_{\mathrm{eff}}$\\[-2pt]\footnotesize global projected description};

% Split
      \node[box, below left=16mm and 24mm of chieff] (stable) {stable\\[-2pt]\footnotesize (non-factorizable)};
      \node[box, below right=16mm and 24mm of chieff] (unstable) {unstable\\[-2pt]\footnotesize (non-factorizable)};

% Outcomes
      \node[box, below=of stable] (ent) {entanglement\\[-2pt]\footnotesize unity preserved};
      \node[box, below=of unstable] (frag) {fragmentation\\[-2pt]\footnotesize (decay)};

      \node[box, below=of frag] (sum){$\chi_{\mathrm{eff},1}\ \oplus\ \chi_{\mathrm{eff},2}\ \oplus\ \cdots$
        \\[-2pt]\footnotesize factorizable localized descriptions};

% Arrows
      \draw[arr] (chi) -- (proj);
      \draw[arr] (proj) -- (chieff);

      \draw[arr] (chieff) -- (stable);
      \draw[arr] (chieff) -- (unstable);

      \draw[arr] (stable) -- (ent);
      \draw[arr] (unstable) -- (frag);
      \draw[arr] (frag) -- (sum);

% Bracket / grouping
      \draw[thick] ($(stable.north west)+(-4mm,3mm)$) -- ($(unstable.north east)+(4mm,3mm)$);
      \draw[thick] ($(stable.north west)+(-4mm,3mm)$) -- ($(stable.south west)+(-4mm,-3mm)$);
      \draw[thick] ($(unstable.north east)+(4mm,3mm)$) -- ($(unstable.south east)+(4mm,-3mm)$);

% Optional label above bracket
      \node[lab, above=3mm of chieff] {projected description branches by stability};
    \end{tikzpicture}
    \caption{Conceptual branching induced by non-injective projection in Cosmochrony.
    A single relational configuration of \(\chi\) may admit a global non-factorizable
    projected description. If this description is stable, it manifests as entanglement.
    If it is unstable, admissibility is recovered through fragmentation into multiple
    localized factorizable descriptions (particle decay).}
    \label{fig:noninjective-branching-entanglement-decay}
  \end{figure}

  A technical analysis of non-injective projection, admissibility conditions, and
  structural factorization is provided in
  Appendix~\ref{subsec:non-injective-projection-and-structural-factorization}.
