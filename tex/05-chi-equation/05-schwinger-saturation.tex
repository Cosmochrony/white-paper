\subsection{Schwinger Effect as a Saturation Threshold of Relaxation Flux}
  \label{subsec:schwinger-saturation}

  The Schwinger effect occupies a singular conceptual position at the interface between
  quantum electrodynamics and non-linear field theories.
  Historically, it is one of the rare phenomena where the perturbative framework of QED
  breaks down and an intrinsically non-linear, non-perturbative description becomes unavoidable.

  In standard quantum field theory, the Schwinger effect is interpreted as a vacuum
  instability: under an electric field exceeding a critical threshold,
  virtual electron--positron pairs become real.
  While quantitatively successful, this interpretation relies on the formal machinery
  of second quantization and on the notion of vacuum fluctuations as physical precursors.

  In Cosmochrony, the emergence of a Born--Infeld--type effective dynamics provides a
  structurally different interpretation.
  As shown in Section~\ref{subsec:variational-formulation}, the Born--Infeld Lagrangian
  arises as the unique effective description compatible with the causal saturation of
  relaxation fluxes within the projection fiber.
  The maximal field scale is therefore not an imposed bound, but the manifestation of a
  transport limit of the underlying $\chi$ relaxation.

  From this perspective, the Schwinger effect corresponds to a \emph{threshold of flux
saturation}.
  When the imposed electric field drives the effective relaxation flux beyond what can
  be transported smoothly through the projection fiber, the homogeneous relaxation
  regime becomes unstable.
  The system resolves this instability by activating additional admissible modes of the
  projection, corresponding to the nucleation of conjugate torsional excitations.
  These excitations are naturally identified with electron--positron pairs,
  appearing as \emph{conjugate torsional modes} within the projection fiber.
  As discussed in Section~\ref{subsec:charge-as-a-topological-and-relaxational-property-of-chi}, such pairs necessarily
  preserve global torsional neutrality, ensuring conservation of the total
  twist and explaining the unavoidable charge conjugation symmetry of the process.

  In this interpretation, pair production does not originate from vacuum fluctuations,
  nor from a breakdown of the notion of emptiness, but from a topological reconfiguration
  of the projection fiber under maximal relaxation stress.
  The creation of matter acts as a dissipation channel that restores admissibility
  by redistributing the excess relaxation flux into stable, projectable vortical modes.
  In thermodynamic terms, this process enlarges the space of admissible configurations,
  effectively increasing the relaxation bandwidth of the system in accordance with the
  principle of monotonic growth of admissible states introduced in the Abstract.

  This re-interpretation explains why the Schwinger effect is intrinsically linked to
  non-linear electrodynamics and why its characteristic threshold coincides with the
  Born--Infeld saturation scale.
  It also clarifies why Schwinger pair production is insensitive to the microscopic
  details of the field representation: it is a structural consequence of flux saturation,
  not a quantum fluctuation effect.

  In this sense, Cosmochrony predicts the existence of a Schwinger-like threshold
  \emph{ab initio}, without invoking the Dirac sea or field quantization.
  The effect emerges as a universal transition between smooth relaxation and
  topologically mediated dissipation, marking the onset of matter production as a
  necessary response of the substrate to excessive directional stress.

  Matter creation thus appears not as an anomaly of the vacuum, but as a
  thermodynamically necessary extension of the system's admissible configuration
  space when relaxation fluxes approach their maximal transport capacity.

  \paragraph{Dissipation by structure creation.}
    The Schwinger threshold illustrates a more general principle of Cosmochrony:
    when directional relaxation fluxes approach their maximal transport capacity,
    admissibility can be restored not by further smooth transport, but by creating
    new stable, projectable structures that enlarge the space of admissible configurations.
    This mechanism suggests a unified interpretation of several high-energy regimes,
    including pair-loaded astrophysical jets near compact objects, primordial matter
    production during early relaxation, and potentially local ultra-high-energy
    excitations in regions of extreme curvature or flux density
    (Section~\ref{sec:testable-predictions-and-observational-signatures}).
