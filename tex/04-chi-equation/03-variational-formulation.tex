\subsection{Variational Formulation and Born-Infeld Action}
  \label{subsec:variational-formulation}

  To extend this kinematic constraint to a full dynamical theory including matter, we propose an effective Lagrangian
  density of the Born-Infeld type:
  \begin{equation}
    \mathcal{L} = -c^2 \sqrt{1 - \frac{|\nabla \chi|^2}{c^2}} + \partial_t \chi - \frac{4\pi G}{c^2} \rho \chi ,
  \end{equation}
  where $\rho$ represents the matter density. The presence of the term $\partial_t \chi$ linear in the first-order
  temporal derivative is crucial: it ensures that the momentum conjugate to $\chi$, defined as
  $\Pi_\chi = \frac{\partial \mathcal{L}}{\partial (\partial_t \chi)}$, is a non-vanishing constant ($\Pi_\chi = 1$).

  In the Hamiltonian formalism, this constant momentum acts as a primary constraint that effectively enforces the
  unit-velocity evolution of the field.
  This structure ensures that the field dynamics remain locked onto the Hamiltonian
  constraint~\eqref{eq:hamiltonian_constraint} while the square-root term acts as a non-linear regularizer for spatial
  gradients.
  The variation with respect to $\chi$ yields a non-linear Poisson equation:
  \begin{equation}
    \nabla \cdot \left( \frac{\nabla \chi}{\sqrt{1 - |\nabla \chi|^2/c^2}} \right) = \frac{4\pi G}{c^2} \rho .
    \label{eq:nonlinear_poisson}
  \end{equation}
  This formulation naturally recovers the Newtonian limit for weak gradients ($|\nabla \chi| \ll c$) while preventing
  gravitational singularities as the gradient magnitude is bounded by $c$.
