\subsection{Homogeneous Cosmological Limit}
  \label{subsec:homogeneous-cosmological-limit}

  In a spatially homogeneous and isotropic configuration, $\nabla \chi = 0$
  , and the evolution equation simplifies to:
  \begin{equation}
    \partial_t \chi = c .
  \end{equation}

  This implies a linear growth\cite{Friedmann1922,Lemaitre1927}:
  \begin{equation}
    \chi(t) = \chi_0 + c t ,
  \end{equation}

  where $\chi_0$ denotes the initial value of $\chi$.

  This simple relation already reproduces a Hubble-like expansion law when distances are identified with accumulated
  $\chi$ increments, as discussed in Section~\ref{sec:cosmology}.

  As shown in Appendix~\ref{sec:mond_derivation}, the requirement $\partial_t \chi \geq 0$ in an expanding background
  ($H_0$) implies a minimal residual gradient $\nabla \chi_{\min} \propto \sqrt{H_0}$.
  This ``acceleration floor'' provides a first-principles derivation for MOND-like phenomenology, explaining galactic
  rotation curves without invoking dark matter particles.
