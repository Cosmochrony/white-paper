% ----------------------------------------------------------------------------
% Section 6.7 --- Strong Gravity and Black Holes
% From former §7.7, heavily condensed
% ----------------------------------------------------------------------------
\subsection{Strong Gravity and Black Holes}
\label{subsec:strong-gravity-and-black-holes}

\begin{figure}[h]
  \centering
  \begin{tabular}{c c c c c}
    \textbf{Projectable} &
    $\rightarrow$ &
    \textbf{Horizon} &
    $\rightarrow$ &
    \textbf{Non-projectable} \\[6pt]
    Gravitational && Boundary of && Deprojected \\
    Waves && Projection && Regime \\[6pt]
    Effective && $\Pi$ non-injective && No spacetime \\
    Modulations &&&& Representation
  \end{tabular}
  \caption{Conceptual regimes of projection in Cosmochrony.
    Black holes mark the boundary beyond which spacetime
    representations cease to be injective.}
  \label{fig:chi_projection_regimes}
\end{figure}

In regions of sufficiently high density of localized projected
configurations, admissible relaxation ordering becomes strongly
constrained, defining an effective horizon.
Black holes correspond to domains where physical processes become
asymptotically inaccessible due to the loss of injectivity of spacetime
projection.
This naturally accounts for extreme time dilation without requiring
divergent curvature invariants.

Because admissible ordering is bounded, configurations corresponding to
infinite curvature or density cannot be physically realized.
Apparent singularities signal the breakdown of spacetime
representability, not genuine divergences of the underlying relational
structure.

\subsubsection*{Black Holes, Deprojection, and Vacuum Reprojection}
\label{subsec:black-hole-deprojection-cycle}

In strong-gravity regimes, the projection
$\Pi : \mathcal{C}_{\mathrm{rel}} \longrightarrow \mathcal{M}$
loses injectivity: multiple inequivalent relational configurations
correspond to the same effective spacetime event.
This \emph{deprojection} does not destroy information.
Relational information ceases to be expressible in spatiotemporal form
but remains encoded structurally and is in principle reprojectable once
projectability is restored.
Reprojection manifests as radiation-like excitations or
particle--antiparticle pairs.

\subsubsection*{Black Hole Entropy as Relaxation Saturation}
\label{subsec:bh-entropy-relaxation}

Black hole entropy measures the saturation of relaxation capacity at the
horizon.
Near the boundary where the effective temporal ordering ceases to be
resolvable, a large multiplicity of distinct $\chi$ micro-configurations
become indistinguishable at the level of the effective metric:
\begin{equation}
  S_{\mathrm{BH}} \;\sim\;
    \log \left| \Pi^{-1}(g_H) \right|,
\end{equation}
where $g_H$ denotes the saturated horizon geometry.
The area law follows from the fact that saturation occurs at the
boundary between projectable and non-projectable regimes.
Hawking thermality arises from the coarse-grained statistics of discrete
reprojection events at the saturation interface.

The numerical factor $1/4$ in $S = A/4$ arises from the fourfold
degeneracy in the stability spectrum of $\chi$ excitations, linked to
the intrinsic $4\pi$ periodicity discussed in
Section~\ref{subsec:topological-stability}.
