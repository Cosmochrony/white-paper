% ----------------------------------------------------------------------------
% Section 7.1 --- The Big Bang as a Maximal Constraint Regime
% From former §12.1, condensed
% ----------------------------------------------------------------------------
\subsection{The Big Bang as a Maximal Constraint Regime of the
\texorpdfstring{$\chi$}{χ} Substrate}
\label{subsec:big-bang-maximal-constraint}

The Big Bang is interpreted as the boundary of applicability of effective
spacetime descriptions.
In this regime, the density of structural and topological constraints
within~$\chi$ exceeds the threshold required for stable geometric
projection: effective notions of spatial distance, temporal duration, and
causal ordering cease to be well-defined.
The apparent singular behavior of standard cosmological models reflects
the extrapolation of geometric descriptions beyond their domain of
validity.

Cosmological evolution is therefore described as the progressive
relaxation of this maximal constraint regime.
The Big Bang marks not the origin of spacetime, but the transition beyond
which spacetime becomes an appropriate effective framework.
The arrow of time arises from the intrinsic monotonic ordering of~$\chi$
configurations
(Section~\ref{subsec:monotonicity-and-arrow-of-time}).
