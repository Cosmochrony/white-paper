% ----------------------------------------------------------------------------
% Section 7.9 --- The Hubble Tension
% From former §12.10, condensed
% ----------------------------------------------------------------------------
\subsection{The Hubble Tension}
\label{subsec:hubble-tension}

The discrepancy between early- and late-universe determinations of~$H_0$
is well
established~\cite{Riess2019,Riess2022,Planck2020,HubbleTensionReview}.
In Cosmochrony, different observational probes access different regimes
of effective projectability.
Early-universe measurements probe a regime close to the transition from
maximal constraint to geometric projectability; late-time measurements
probe a more weakly constrained regime where effective spacetime
descriptions are more fully developed.
The tension arises from using a single spacetime-based parametrization to
describe observations sampling distinct stages of relational relaxation.

\subsubsection*{The Hubble Tension as a Diagnostic of Topological
Decoherence}
\label{subsec:hubble_tension_tau}

The effective Hubble parameter may be expressed as
\begin{equation}
  H_{\mathrm{eff}}(\mathbf{x}, t)
    \;\sim\; \frac{1}{\tau_\chi(\mathbf{x}, t)} ,
\end{equation}
where $\tau_\chi$ is an effective relaxation timescale.
Regions of high structural complexity (clusters, filaments) correspond to
topologically frustrated configurations in which the relaxation timescale
acquires a spatial dependence:
\begin{equation}
  \tau_\chi(\mathbf{x})
  = \tau_\chi^{(0)}
    \left[1 + \epsilon\, \mathcal{T}(\mathbf{x})\right] ,
\end{equation}
where $\mathcal{T}(\mathbf{x})$ encodes local topological density.
The tension reflects a non-commutativity between cosmological averaging
and local projection.

Cosmochrony predicts that locally inferred values of~$H_0$ should exhibit
weak but systematic correlations with the surrounding topological
environment.
Quantitative estimates are provided in
Section~\ref{subsec:hubble-constant-from-chi-dynamics} and
Appendix~\ref{app:hubble_tension}.
