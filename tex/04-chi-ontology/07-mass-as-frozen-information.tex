\subsection{Mass as Frozen Information}
  \label{subsec:mass-as-frozen-information}

  Localized and long-lived configurations of the $\chi$ substrate—describable in
  effective regimes as particle-like excitations—correspond to regions in which further
  relaxation is strongly inhibited.
  Such configurations trap a fixed amount of unresolved structural information,
  preventing it from participating freely in the global relaxation process.

  In this interpretation, mass represents \emph{frozen energy}: structural information
  whose capacity for further relaxation has been locally suppressed.
  The term ``information'' is used here in a structural sense, referring to persistent
  relational constraints within projected $\chi$ configurations, and should not be
  understood in terms of Shannon entropy or symbolic encoding.
  Mass therefore quantifies the degree to which relaxation capacity has been
  immobilized into stable relational patterns\footnote{This notion of frozen structural information should be distinguished from
  projection-induced entropy, which quantifies loss of distinguishability under
  non-injective projection rather than inhibited relaxation within stable
  configurations}.

  The quantitative formulation of this relationship, including the operational
  definition $m = E/c^2$ and its derivation from resistance to relaxation ordering,
  is developed in Section~\ref{subsec:mass_as_resistance}.

  From this perspective, processes such as particle annihilation, decay, or radiation
  emission correspond to the partial or complete release of frozen structural
  information, restoring its ability to participate in relaxation.
  Mass is thus not a fundamental attribute of the $\chi$ substrate, but an emergent
  measure of inhibited relaxation arising from stable, spectrally isolated
  configurations of the underlying relational structure.
