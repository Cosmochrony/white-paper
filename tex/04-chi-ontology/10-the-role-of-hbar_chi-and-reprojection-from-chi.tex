\subsection{The Role of $\hbar_{\chi}$ and Reprojection from $\chi$}
  \label{subsec:the-role-of-hbar_chi-and-reprojection-from-chi}

  In Cosmochrony, the parameter $\hbar_{\chi}$ is not identified with the quantum
  constant $\hbar$, but emerges from the fundamental structural scales $K_0$, $\chi_c$,
  and $c$.
  Its numerical coincidence with $\hbar$ in quantum regimes reflects the universality
  of action quantization across effective physical theories.

  $\hbar_{\chi}$ does not represent a quantum of action evolving in time, but a
  fundamental quantum of reprojection.
  Intrinsic indeterminacy of $\chi$ does not give rise to continuous emergence;
  rather, any reprojection of structural information into spacetime occurs in discrete
  units set by $\hbar_{\chi}$.

  As spacetime structure stabilizes, reprojection becomes increasingly localized,
  manifesting phenomenologically as vacuum fluctuations within otherwise stable regions
  of spacetime.

\subsection{The Origin of Planck's Constant: $\hbar_\chi$ vs $\hbar_{\text{eff}}$}
  \label{subsec:h-origin}

  A potential conceptual tension arises regarding the fundamental nature of $\hbar$. We clarify here that $\hbar$
  is not an independent postulate but a \textbf{spectral invariant} of the relaxation process.

  \begin{itemize}
    \item \textbf{Fundamental Substrate Constant ($\hbar_\chi$):} At the level of the $\chi$
    substrate, the quantum of action is uniquely determined by the ratio of the propagation speed $c$
    , the coupling density $K_0$, and the correlation scale $\chi_c$:
    \begin{equation}
      \hbar_\chi = \frac{c^3}{K_0 \chi_c}
    \end{equation}
    This indicates that the ``graininess'' of quantum reality is a direct consequence of the \textbf{spectral rigidity}
    of the substrate.
    \item \textbf{Emergent Scaling ($\hbar_{\text{eff}}$):} In the effective spacetime description, $\hbar_{\text{eff}}$
    appears as a scaling parameter.
    The perceived universality of $\hbar$ stems from the fact that $K_0$ and $\chi_c$
    are global invariants of the current relaxation epoch.
  \end{itemize}

  In this view, the transition from $\hbar_\chi$ to $\hbar_{\text{eff}}$
  is not a change in value, but a change in representation: from a relational constraint in the substrate to a
  dynamical constant in the projected Hilbert space.

\subsection{Spectral Invariance of Planck's Constant and $\alpha$}
  \label{subsec:quantum-invariants}

  The dependency of $\hbar_\chi$ on $K_0$ and $\chi_c$ (Eq. 113) is not a contradiction but a
  \textbf{requirement for spectral unification}. By defining $\hbar_\chi \equiv c^3 / (K_0 \chi_c)$,
  we establish that the quantum of action is a manifestation of the substrate's relational density.

  \begin{itemize}
    \item \textbf{Resolution of the $\hbar$ tension:} $\hbar_{\text{eff}}$
    is the projection of the fundamental relational graininess $\hbar_\chi$.
    \item Its apparent constancy across spacetime is due to the homogeneity of the relaxation flux $\Phi_\chi$
    in the current epoch.
    \item \textbf{Geometric Origin of $\alpha$:} The fine-structure constant $\alpha$ then emerges as a
    \textbf{dimensionless spectral ratio}:
    \begin{equation}
      \alpha = \mathcal{F}\left( \frac{\text{topology of } \Pi}{\text{spectral rigidity } K_0} \right)
    \end{equation}
    where $\mathcal{F}$ is a functional determined by the geometry of the projection fiber.
  \end{itemize}

  This derivation shows that if $K_0$ were to evolve (e.g., in the primordial high-constraint regime), $\hbar$ and
  $\alpha$ would scale accordingly, preserving the \textbf{structural coherence}
  of the theory while allowing for non-stationary quantum laws at cosmological singularities.
