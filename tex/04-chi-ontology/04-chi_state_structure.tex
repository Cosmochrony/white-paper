\subsection{The $\chi$ Substrate as a Configurational State Structure}
  \label{subsec:chi_state_structure}

  The preceding discussion has established a fundamental ontological asymmetry
  between the pre-geometric substrate $\chi$ and its spacetime-based projections.
  We now clarify the positive ontological status of $\chi$ itself.

  In the Cosmochrony framework, $\chi$ is not interpreted as a dynamical field
  evolving in time, nor as a collection of localized degrees of freedom embedded
  in a background geometry.
  Rather, $\chi$ defines a \emph{configurational structure} specifying the set of
  admissible macroscopic states and the allowed transitions between them.

  From this perspective, physical reality as described in spacetime corresponds
  to a particular projected realization of $\chi$, stabilized under a finite
  resolution of description.
  The observable Universe represents an effective state maintained by projection,
  not a dynamical agent acting back on the substrate.
  No causal influence from projected configurations to $\chi$ is postulated or
  required.

  The notion of evolution must therefore be understood carefully.
  What appears phenomenologically as temporal evolution in spacetime corresponds,
  at the level of $\chi$, to an ordering of admissible configurations under
  projection.
  This ordering is not itself temporal in the conventional sense, but reflects
  the structural organization of the configurational space defined by $\chi$.

  In this sense, $\chi$ plays the role of a pre-temporal transition structure:
  it specifies which macroscopic configurations are mutually compatible and
  which transitions are admissible, without invoking an external time parameter.
  Temporal succession, causality, and dynamical laws emerge only after projection,
  as effective descriptors of how stabilized configurations relate within the
  projected spacetime picture.

  Crucially, this interpretation avoids any form of retroaction.
  Projected observables do not influence the structure of $\chi$.
  Constraints flow unidirectionally: invariant structural properties defined at
  the level of $\chi$ determine the space of possible projections, while the
  projected Universe merely instantiates one such admissible configuration.

  This configurational interpretation provides a unifying ontological basis for
  several key features of the framework.
  The existence of universal bounds, such as $c_\chi$ and $\hbar_\chi$, reflects
  intrinsic limits on admissible transitions within the configurational structure.
  Likewise, apparent dynamical saturation effects arise when the projection
  exhausts the available configurational resolution, rather than from modifications
  of underlying laws.

  Viewed in this way, $\chi$ defines a state-based ontology in which physical
  quantities do not correspond to fundamental substances or fields, but to effective
  rates, constraints, and capacities associated with transitions between
  projected configurations.
  The familiar notions of motion, force, and expansion are thus emergent descriptors
  of how the projected state is updated under finite structural resolution.

  This interpretation will be essential in the following sections, where
  indeterminacy, energy, mass, and quantization are reinterpreted as manifestations
  of configurational constraints and projection limits rather than as primitive
  dynamical inputs.
