% ----------------------------------------------------------------------------
% Section 12.11 --- Bounds on Projective Resolvability Across Scales
% From former §15.11, heavily condensed
% ----------------------------------------------------------------------------
\subsection{Bounds on Projective Resolvability Across Scales}
\label{subsec:projective-resolvability-bound}

Saturation phenomena across scales---gravitational response in
low-density environments, accessibility of quantum correlations---reflect
a common limitation on the projective resolvability of relational
structure: the rate at which global relational structure encoded in
$\chi$ can be rendered operationally accessible within emergent spacetime
descriptions.
Attosecond chronoscopy experiments illustrate this notion: measured
delays reflect the minimal temporal resolution required for a
non-factorizable projected description to become resolvable, not a
dynamical buildup of
correlations~\cite{PhysRevLett.133.163201}.

The bound manifests as an effective inequality:
\[
  \left| \frac{\partial \mathcal{O}_{\mathrm{eff}}}
    {\partial \tau} \right|
  \;\leq\;
  b \, \mathcal{S}_{\Pi},
\]
where $b$ is the fundamental resolvability bound and
$\mathcal{S}_{\Pi}$ characterizes the structural complexity of the
non-injective projection.

Its phenomenological manifestations are:
\begin{equation}
  a_{\star} \approx \eta_{G} \, b \, c,
  \label{eq:a_star_b_relation}
\end{equation}
for gravitational saturation, and
\begin{equation}
  B_{\Pi}(\mathcal{M})
  = \eta_{\mathcal{M}} \, b \, \mathcal{S}_{\Pi},
  \label{eq:bandwidth_b_relation}
\end{equation}
for the operational projective bandwidth associated with a measurement
protocol~$\mathcal{M}$.
These are distinct dimensional expressions of the same underlying bound.
A modification of the fundamental resolvability of the
$\chi \rightarrow \Pi$ mapping would induce correlated shifts in both
galactic saturation scales and quantum chronoscopy timescales.
