% ----------------------------------------------------------------------------
% Section 12.12 --- Projective Non-Termination and Time
% From former §15.12, condensed
% ----------------------------------------------------------------------------
\subsection{Projective Non-Termination and the Condition of Temporal
Ordering}
\label{subsec:projective-non-termination}

Time is identified with the ordering of distinguishable projected
states: $\Delta \tau \sim \mathrm{dist}(U_n, U_{n+1})$.
A hypothetical terminal state~$U_0$ satisfying
$\Pi(\chi + \delta\chi) = U_0$ for all admissible~$\delta\chi$ would
imply $\Delta\tau \rightarrow 0$ for all internal observers.

The inaccessibility of absolute zero emerges as a projective necessity:
zero-point fluctuations are reinterpreted as the minimal projective
bandwidth required to prevent collapse of the observable description into
a stationary state.
Time ceases only when the projection becomes stationary---a regime
structurally forbidden by the existence of a finite bound on projective
resolvability.
