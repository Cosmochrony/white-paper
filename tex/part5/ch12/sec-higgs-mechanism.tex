% ----------------------------------------------------------------------------
% Section 12.9 --- Relation to the Higgs Mechanism
% From former §15.9, heavily condensed
% ----------------------------------------------------------------------------
\subsection{Relation to the Higgs Mechanism: Emergence from
\texorpdfstring{$\chi$}{χ} Dynamics}
\label{subsec:relation-to-the-higgs-mechanism}

The Higgs field and its VEV are reinterpreted as effective low-energy
descriptors of a specific projective regime of~$\chi$.

\subsubsection*{Structural Transition}
\label{subsec:emergence-higgs-vev}
\label{subsec:chi_c-electroweak-scale}
Below a critical scale~$\chi_c$ (homogeneous regime), only massless
globally coherent configurations are admissible.
Above~$\chi_c$ (structured regime), localized relaxation-resistant
configurations stabilize as massive excitations.
The electroweak scale is related through
\begin{equation}
  \langle \phi_H \rangle
  \;\propto\;
  \frac{\hbar_{\mathrm{eff}}\, c}{\chi_c},
\end{equation}
naturally recovering the observed scale for
$\chi_c \sim 10^{-18}\,\mathrm{m}$.

\subsubsection*{Mass Generation as Solitonic Stabilization}
\label{subsec:mass-generation-solitons}
Fermion masses scale as
$m_f \propto y_f\,\hbar_{\mathrm{eff}}/\chi_c$;
gauge boson masses as
$m_W \propto g\,\hbar_{\mathrm{eff}}/\chi_c$.
These relations reproduce standard Higgs-generated mass terms at the
effective level.

\subsubsection*{Phenomenological Status}
\label{subsec:phenomenological-implications}
\label{subsec:higgs-summary}
No deviation from established collider results is implied at accessible
energies.
Departures may arise only in extreme regimes.
Open challenges include deriving the detailed mapping between~$\chi$
soliton spectra and the full Standard Model mass spectrum.
