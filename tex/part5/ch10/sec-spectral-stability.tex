% ----------------------------------------------------------------------------
% Section 10.1 --- Spectral Stability and the Unit of Mass
% From former §11.1, condensed
% ----------------------------------------------------------------------------
\subsection{Spectral Stability and the Unit of Mass}
\label{sec:spectral-stability}

Rest mass is a spectral invariant associated with the stability of
admissible projected configurations.
It is identified with the fundamental eigenmode of the scalar
Laplacian~$\Delta^{(0)}_G$ acting on the projection fiber~$\Pi$, subject
to topological constraints~$\mathcal{T}$:
\begin{equation}
  m^2 c^2 \;=\; \lambda_{\mathcal{T}}
  \;\equiv\;
  \mathrm{Eig}\!\left(\Delta^{(0)}_G\right)
    \Big|_{\mathcal{T}} .
  \label{eq:mass_spectral_def}
\end{equation}
Here $\lambda_{\mathcal{T}}$ encodes the minimal energetic cost required
to maintain the configuration against the global relaxation of~$\chi$.
Mass therefore quantifies resistance to relaxation.

The electron mass~$m_e$ corresponds to the lowest non-trivial admissible
eigenmode~$\lambda_1$, representing the fundamental resonance of~$\chi$
within the finite-volume geometry $\Pi \cong S^3$.
The conversion to physical mass units is
\begin{equation}
  m \;=\; m_e
    \sqrt{\frac{\lambda_{\mathcal{T}}}{\lambda_1}} .
\end{equation}
