% ----------------------------------------------------------------------------
% Section 11.7 --- CMB Polarization Signatures
% From former §14.7, condensed
% ----------------------------------------------------------------------------
\subsection{CMB Polarization Signatures (Outlook)}
\label{subsec:cmb-polarization-signatures}

Residual large-scale projective correlations inherited from the
pre-geometric relaxation of~$\chi$ are expected to imprint
scale-dependent signatures on the CMB
(Section~\ref{subsec:cmb}).
The observed $\sim 10\%$ suppression of the CMB quadrupole power relative
to $\Lambda$CDM arises naturally from these relaxation
dynamics~\cite{Aghanim2020}.
Appendix~\ref{app:lowell_attenuation} provides quantitative estimates.

\paragraph{The \texorpdfstring{$8/3$}{8/3} Scaling in CMB Polarization.}
\label{subsec:cmb_8_3_scaling}
The fundamental ratio $\lambda_2/\lambda_1 = 8/3$ is expected to leave a
structural signature in the polarization sector.
The bare geometric tensor-to-scalar ratio is
\begin{equation}
  r_0
  = \frac{\Delta_t^2}{\Delta_s^2}
  = \frac{\lambda_{\text{base}}}{\lambda_{\text{fiber}}}
  = \frac{3}{8} \simeq 0.375 .
\end{equation}
The observed ratio undergoes topological decoherence:
\begin{equation}
  r_{\text{obs}}(t)
  = r_0 \cdot \exp\!\left(
    -\zeta \frac{\tau_\chi}{t} \right),
\end{equation}
providing a structural explanation for the low observed $r < 0.036$
without invoking slow-roll dynamics.
