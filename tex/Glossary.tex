\section{Glossary of Core Quantities and Notation}
  \label{appendix:glossary}

  This appendix summarizes the meaning and status of the main quantities used
  throughout the Cosmochrony framework. It is intended as a reference guide and
  does not introduce new assumptions or definitions.

  \subsection{Fundamental and Effective Quantities}

    \paragraph{$\chi$ (Chi field).}
      The fundamental scalar quantity of the Cosmochrony framework.
      $\chi$ is not defined on a pre-existing spacetime manifold but constitutes a
      pre-geometric substrate whose monotonic relaxation provides an intrinsic ordering
      of physical processes.
      Localized, topologically stable configurations of $\chi$ correspond to particle-like
      excitations.

    \paragraph{$V(\chi)$ (Effective potential).}
      An effective, coarse-grained description used to model localization and stability
      properties of $\chi$ configurations.
      $V(\chi)$ is not assumed to be fundamental; its form may emerge from underlying
      discrete relaxation dynamics and is secondary to the spectral description of mass.

    \paragraph{$K_{ij}$ (Relaxation coupling).}
      Edge-dependent coupling coefficients defined on the relaxation network $G(V,E)$.
      $K_{ij}$ quantify the local resistance to relative variations of $\chi$ between
      neighboring nodes and encode geometric and topological information of the network.
      They may depend on the local configuration of $\chi$ in effective descriptions.
      
\subsection{Derived Operators and Dimensionless Parameters}

  \paragraph{$G(V,E)$ (Relaxation network).}
    A discrete graph representing the underlying relational structure on which the
    $\chi$ field is defined at the fundamental level.
    Vertices correspond to elementary degrees of freedom, and edges encode relaxation
    couplings.

  \paragraph{$\Delta_G$ (Graph Laplacian / relaxation operator).}
    The discrete Laplace--Beltrami operator associated with the network $G(V,E)$ and
    the couplings $K_{ij}$.
    It governs the stability and mode structure of localized $\chi$ configurations.
    Its spectral properties play a central role in the quantitative characterization
    of inertial mass.

  \paragraph{$S$ (Gradient saturation parameter).}
    A dimensionless quantity defined as
    \begin{equation}
      S \equiv \frac{1}{c^2}\sum_{j\sim i} K_{ij}(\chi_i-\chi_j)^2,
    \end{equation}
    measuring the local density of $\chi$ gradients.
    The condition $S \leq 1$ ensures causal consistency and bounds the local relaxation
    rate of $\chi$.

  \paragraph{$\lambda_n$ (Spectral eigenvalues).}
    The eigenvalues of the linearized relaxation or stability operator acting on
    small fluctuations around a localized configuration.
    In effective wave descriptions, $\sqrt{\lambda_n}$ determines the inertial mass
    scale of particle-like excitations.

  \paragraph{$\Omega_\chi$ (Relaxation budget parameter).}
    A dimensionless global quantity characterizing the fraction of the total $\chi$
    relaxation budget stored in spatial gradients.
    In cosmological regimes, $\Omega_\chi$ plays a role analogous to the matter
    density parameter in standard cosmology.
