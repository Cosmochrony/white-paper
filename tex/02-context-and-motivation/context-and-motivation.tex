\clearpage
\section{Theoretical Context and Motivation}
  \label{sec:theoretical-context-and-motivation}

  \subsection{Conceptual Tension Between Quantum Theory and Gravitation}
    \label{subsec:conceptual-tension-between-quantum-theory-and-gravitation}

    Quantum mechanics and general relativity differ not only in their mathematical
    formalisms, but also in their foundational concepts.
    Quantum theory is intrinsically probabilistic, relies on a fixed causal structure,
    and treats time as an external parameter~\cite{Dirac1930,Born1926}.
    General relativity, by contrast, describes gravitation as the dynamics of spacetime
    geometry itself, with time acquiring a coordinate-dependent and observer-relative
    status~\cite{Einstein1915,MisnerThorneWheeler1973}.

    This conceptual mismatch becomes particularly acute in regimes where both quantum
    effects and strong gravitational fields are expected to be relevant, such as near
    spacetime singularities or in the early universe~\cite{penrose1989emperors,Prigogine1997}.
    Direct attempts to quantize gravity encounter persistent difficulties, including
    the problem of time, non-renormalizability, and the absence of a preferred background
    structure.
    These difficulties suggest that the tension may reflect not merely technical
    obstacles, but a deeper incompatibility in the assumed ontological status of time
    and geometry.

  \subsection{Limitations of Existing Unification Approaches}
    \label{subsec:limitations-of-existing-unification-approaches}

    Several major research programs have sought to address these challenges.
    Quantum field theory in curved spacetime successfully accounts for particle creation
    and vacuum effects, but retains a classical spacetime background~\cite{weinberg1972gravitation}.
    Canonical and covariant approaches to quantum gravity attempt to quantize spacetime
    geometry itself, often at the cost of substantial mathematical complexity and
    interpretational ambiguity.

    String theory and related frameworks introduce extended fundamental objects and
    higher-dimensional structures, offering deep mathematical unification but leading
    to a large space of possible low-energy realizations~\cite{rovelli2004quantum}.
    While internally rich, these approaches face ongoing challenges concerning empirical
    testability and the physical interpretation of their fundamental degrees of freedom.

    Collectively, these limitations motivate the exploration of alternative perspectives
    in which spacetime geometry, matter, and quantum behavior are not independently
    postulated, but emerge from a common underlying mechanism operating at a more
    primitive, pre-geometric level.

  \subsection{Minimalism as a Guiding Principle}
    \label{subsec:minimalism-as-a-guiding-principle}

    The framework developed in this work adopts minimalism as a guiding principle.
    Rather than introducing multiple fundamental fields, additional dimensions, or
    independent quantization rules, we explore whether a single continuous fundamental
    entity can account for temporal ordering, spatial relations, and quantum features
    within a unified relational dynamics.

    The scalar quantity $\chi$ is not interpreted as a conventional matter field, nor
    as a component of spacetime geometry.
    Instead, it represents a pre-geometric substrate whose irreversible relaxation
    underlies the emergence of both duration and separation.
    In this view, time and space are not independent primitives, but complementary
    aspects of a single dynamical process.
    Effective geometric and quantum descriptions arise only through coarse-grained,
    generally non-injective projections of the underlying $\chi$-configurations.

  \subsection{Time, Irreversibility, and Cosmological Expansion}
    \label{subsec:time-irreversibility-and-cosmological-expansion}

    A central motivation for the Cosmochrony framework is the close connection between
    time, irreversibility, and cosmological expansion.
    In standard cosmology, expansion is described kinematically through the scale factor,
    while the arrow of time is typically attributed to boundary conditions or entropy
    growth~\cite{peebles1993principles,Prigogine1997,penrose1989emperors}.

    In Cosmochrony, the monotonic relaxation of $\chi$ provides a unified origin for both
    phenomena.
    Irreversibility follows directly from the intrinsic directionality of the relaxation
    process, while cosmological expansion is interpreted as its large-scale geometric
    manifestation in the effective, projected description.
    From this perspective, expansion does not require an externally imposed energy
    component, but arises as an emergent consequence of the underlying pre-geometric
    dynamics.

  \subsection{Scope and Limitations}
    \label{subsec:scope-and-limitations2}

    The aim of this work is exploratory rather than definitive.
    Cosmochrony does not seek to replace established theories within their empirically
    validated domains, but to offer a coherent reinterpretation that may clarify
    persistent conceptual difficulties concerning time, geometry, and quantization.

    Throughout the paper, emphasis is placed on internal consistency, conceptual clarity,
    and qualitative contact with observable phenomena, while openly acknowledging open
    questions and limitations.
    In the following section, we introduce the $\chi$ substrate formally and specify the
    minimal assumptions underlying its relational and dynamical structure.
