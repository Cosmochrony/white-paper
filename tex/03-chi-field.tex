Having outlined the ontological and conceptual principles underlying Cosmochrony, we now
introduce the fundamental quantity at the core of the framework.
This section is devoted to defining the scalar entity $\chi$ and clarifying its role as a
pre-geometric substrate from which spacetime notions, dynamical laws, and physical observables
emerge.

The purpose of this section is not to assume a pre-existing spacetime structure, but to
establish the minimal properties required of $\chi$ in order to recover, in appropriate
regimes, effective notions of time, space, metric geometry, and field dynamics.
Accordingly, $\chi$ is introduced independently of any spacetime coordinates or metric,
and only later related to effective geometric descriptions once a stable regime is reached.

In this sense, the use of variational, Lagrangian, or metric-based formulations later in this
section does not imply that spacetime or a four-dimensional manifold is fundamental.
These formalisms are employed as effective tools to describe the dynamics of $\chi$ in regimes where
a spacetime interpretation becomes meaningful, and should be understood as emergent,
coarse-grained representations of the underlying pre-geometric dynamics.

We begin by providing a unified conceptual definition of the $\chi$ field and its physical
interpretation.
Subsequent subsections introduce effective dynamical descriptions---including Lagrangian and
metric formulations---that are intended as coarse-grained representations valid when the
underlying $\chi$ configurations admit a spacetime interpretation.

\subsection{Definition of the $\chi$ Field}
  \label{subsec:definition-of-the-$chi$-field}

  We postulate the existence of a single fundamental scalar quantity, denoted $\chi$, which constitutes the primitive substrate of physical reality.
  The field $\chi$
  is not defined on a pre-existing spacetime manifold and does not presuppose any metric, causal, or geometric
  structure.
  Instead, spacetime itself emerges as an effective description of the relational and dynamical properties of
  $\chi$ configurations.

  Ontologically, $\chi$ is a real scalar order parameter whose value characterizes the local geometric state of the
  underlying substrate.
  It carries dimensions of length and encodes a characteristic wavelength associated with physical processes.
  This wavelength does not evolve *in* time; rather, its monotonic and irreversible relaxation defines what is
  operationally perceived as the flow of time.

  Temporal ordering emerges from the global, monotonic evolution of $\chi$ across physical processes,
  establishing an intrinsic arrow of time without reference to an external temporal coordinate.
  Spatial separation, in turn, arises from relational differences between $\chi$ values across configurations,
  giving rise to an effective notion of distance once a stable geometric regime is established.
  In this sense, time corresponds to ordering, while space corresponds to relational structure.

  At no stage is $\chi$ interpreted as a spacetime coordinate or as a material field propagating on spacetime.
  Rather, spacetime coordinates and metric structure appear as secondary, coarse-grained constructs that become
  meaningful only when $\chi$ configurations admit a quasi-stable geometric interpretation.
  The spacetime metric thus functions as an emergent, effective descriptor of resistance to $\chi$
  relaxation and of the propagation of perturbations within the field.

  This role of $\chi$ is analogous to that of thermodynamic order parameters such as temperature:
  it encodes collective geometric information about an underlying substrate without being itself a fundamental
  spacetime entity.
  In the Cosmochrony framework, $\chi$ therefore provides the minimal ontological basis from which time, space,
  gravitation, and quantum phenomena jointly emerge.

  In the following sections, spacetime coordinates and metric quantities will be introduced as effective tools,
  valid in regimes where $\chi$ admits a stable geometric interpretation.

  \paragraph{On the use of spacetime language.}
    Throughout this work, phrases such as ``spacetime coordinates,'' ``metric tensor,'' and ``four-dimensional manifold'' appear frequently. These should be understood as \emph{emergent effective descriptions} valid in regimes where $\chi$ has relaxed into a quasi-stable geometric configuration. They are not fundamental ingredients of the theory.

    At the deepest level, only $\chi$ and its local variation structure exist. The appearance of familiar geometric language reflects the effectiveness of spacetime as a coarse-grained description of collective $\chi$ behavior, analogous to how thermodynamic variables (temperature, pressure) emerge from molecular dynamics without those variables being fundamental.

    This interpretational stance is essential for distinguishing Cosmochrony from approaches that merely reformulate existing geometric theories in different variables.

\subsection{The Geometric Effective Action and Lagrangians of Cosmochrony ($\mathcal{L}_{\text{CC}}$)}

\paragraph{Interpretational caution.}
  The action principle presented below employs conventional field-theoretic notation, including a metric tensor $g_{\mu\nu}$ and a four-dimensional integration measure. \emph{This should not be interpreted as assuming pre-existing spacetime structure.}

  The formalism serves two purposes:
  \begin{enumerate}
    \item To provide a compact representation of $\chi$ dynamics in regimes where an effective spacetime description is valid.
    \item To establish a bootstrap framework where the metric appearing in the action is determined self-consistently from $\chi$ configurations.
  \end{enumerate}

  The fundamental content of the theory is the field $\chi$ and its relaxation dynamics. The metric $g_{\mu\nu}$ appearing in the action is an \emph{emergent effective structure}, not an input.

\paragraph{Effective action formulation.}
  In regimes where $\chi$ admits a quasi-stable geometric interpretation, the dynamics may be encoded in an effective action:
  \begin{equation}
    S_{\text{CC}} = \int \mathcal{L}_{\text{CC}} \sqrt{-g} \, d^4x
  \end{equation}
  where the Lagrangian density decomposes as:
  \begin{equation}
    \mathcal{L}_{\text{CC}} = \mathcal{L}_{\text{Gravity/Time}} + \mathcal{L}_{\chi/\text{Soliton}} + \mathcal{L}_{\text{Forces/Matter}}
  \end{equation}

  The symbol $\sqrt{-g}$ represents the invariant volume element.
  In regimes where no spacetime interpretation yet exists, this should be understood as an abstract integration measure
  $d\mu$ on the configuration space of $\chi$.\footnote{This is analogous to path integrals in quantum mechanics, where
  $\int \mathcal{D}\phi$ defines a measure on field configurations before any notion of ``paths in time'' is introduced.}

\paragraph{Status of $g_{\mu\nu}$ in this formulation.}
  The metric $g_{\mu\nu}$ appearing here is determined by the condition:
  \begin{equation}
    g_{\mu\nu}[\chi] = \text{arg min}_g \left\{ S_{\text{CC}}[g,\chi] \right\}
  \end{equation}
  subject to consistency with $\chi$-induced causal structure. This is a \emph{functional} of $\chi$, not an independent degree of freedom.

\subsection{Physical Interpretation}
\label{subsec:physical-interpretation}

The central interpretative assumption of Cosmochrony is that spacetime is not a static background but the
macroscopic manifestation of the continuous relaxation of $\chi$.
An increase in $\chi$ corresponds simultaneously to:
\begin{itemize}
  \item the passage of local proper time,
  \item the emergence of spatial distance between causally connected events,
  \item the global expansion of the universe when integrated over large scales.
\end{itemize}

In this framework, distance may be interpreted as ``frozen time,'' while time corresponds to ``locally thawed
distance.''
This dual interpretation is not imposed ad hoc but follows from the identification of both quantities with the
same underlying field.

\subsection{Monotonicity and Arrow of Time}
\label{subsec:monotonicity-and-arrow-of-time}

A fundamental postulate of the theory is that $\chi$ evolves monotonically\cite{Prigogine1997,Penrose1989Weyl}:
\begin{equation}
  \frac{\partial \chi}{\partial t} \ge 0 .
\end{equation}

This monotonicity is not derived from statistical considerations but is taken as a primitive geometric property.
It provides a natural origin for the arrow of time and ensures global causal ordering without invoking special
boundary conditions.

Irreversibility arises because any decrease of $\chi$ would correspond to a contraction of both temporal and spatial
structure, which is dynamically forbidden within the theory.

\subsection{Local Relaxation Speed}
\label{subsec:local-relaxation-speed}

The local rate of change of $\chi$ is bounded by a universal constant:
\begin{equation}
  \left| \nabla_\mu \chi \right| \le c ,
\end{equation}
where $c$ coincides with the observed speed of light.

This bound does not represent the propagation speed of particles or signals, but the maximal rate at which
spacetime itself can locally unfold.
Superluminal recession velocities at cosmological scales arise naturally through cumulative effects and do not
violate local causality.

\subsection{Relation to Conventional Fields}
\label{subsec:relation-to-conventional-fields}

Although $\chi$ shares mathematical similarities with scalar fields used in cosmology (e.g., inflaton-like fields),
its role is fundamentally different.
It does not carry energy in the conventional sense, nor does it require quantization at the fundamental level.

Matter, radiation, and interactions emerge as localized excitations, constraints, or topological features of $\chi$,
rather than as independent entities coupled to it.

\subsection{Initial Conditions and Global Structure}
\label{subsec:initial-conditions-and-global-structure}

The theory assumes an initial condition characterized by a minimal value $\chi_0$, naturally associated with the Planck
scale.
Cosmic evolution corresponds to the progressive relaxation of $\chi$ from this initial state.

Importantly, the framework does not require a spacetime singularity in the traditional sense.
Instead, the apparent singular behavior arises from extrapolating classical notions of time and distance beyond the
domain where $\chi$ is well-defined.

In the next section, we derive a minimal dynamical equation governing the evolution of $\chi$ and explore its immediate
consequences.
