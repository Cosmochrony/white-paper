% ----------------------------------------------------------------------------
% Section 5.3 --- Topological Constraints and Invariants
% From former §10.3
% ----------------------------------------------------------------------------
\subsection{Topological Constraints and Invariants}
\label{sec:topological-constraints}

The stability of localized physical descriptions within~$\Pi$ is governed
by the conservation of topological invariants.
When an excitation of~$\chi$ admits a closed, non-contractible
configuration within~$\Pi$, it forms a persistent topological obstruction
to complete relaxation.
Such configurations cannot be eliminated by smooth deformation without
violating admissibility constraints.

These obstructions are described in terms of knot-like (non-contractible
loop and self-linking) topological structures within the projection fiber.
Once formed, they impose global constraints on the relaxation process and
give rise to long-lived, localized projected configurations.
Particles correspond to stable topological defects of the projection.

The winding number~$w$ constitutes the primary invariant characterizing
these obstructions.
It labels distinct equivalence classes of admissible projected
configurations and plays a central role in the organization of the mass
spectrum (Section~\ref{sec:spectral-mass-hierarchy}).
The energetic cost required to maintain a non-trivial winding against the
global pressure of relaxation is perceived, at the effective level, as
rest energy~($mc^2$).
Mass emerges as a spectral and topological consequence of persistent
non-contractible structures within the projection fiber.
