\subsection{Collective Coherent Phases of the \texorpdfstring{$U(1)$}{U(1)} Fiber}
  \label{subsec:collective-U1-coherence}

  In the previous sections, electric charge and gauge structure were derived from the
  non-injective projection of the $\chi$ substrate onto an effective $U(1)$ fiber.
  We now consider the possibility that this fiber sector admits collective coherent phases.

  The analysis below does not introduce new fundamental fields.
  It studies a specific collective regime of already established degrees of freedom.
  Superconductivity is interpreted as a macroscopic coherent phase of the $U(1)$ fiber,
  arising from the stabilization and phase locking of composite topological excitations.

  \subsubsection{Stability of \texorpdfstring{$w=2$}{w=2} Composite Classes}
    \label{subsubsec:w2-stability}

    Section~4 established that charged excitations correspond to topological configurations
    of the $\chi$ field projected onto the $U(1)$ fiber.
    A single excitation carries winding number $w=1$ and effective charge $e$.

    We now consider composite configurations of winding number $w=2$.
    Such configurations are not trivial superpositions of two independent $w=1$ modes.
    They define a distinct topological class characterized by a continuous fiber raccordement.

    Let $L_{\Pi}$ denote the effective relaxation operator restricted to the fiber sector.
    We define the projective stability gap
    \begin{equation}
      \Delta_{\Pi} \equiv \lambda_{\min}\!\left(L_{\Pi}\big|_{w=2}\right),
    \end{equation}
    as the lowest positive eigenvalue around the $w=2$ configuration.

    A finite $\Delta_{\Pi}$ implies that the composite class is metastable against
    separation into two $w=1$ excitations or local phase defects.
    The existence of this gap is a structural consequence of bounded projective
    transport capacity established in Section~5.6.

  \subsubsection{Phase Coherence and the London Structure}
    \label{subsubsec:london-structure}

    If a macroscopic density of $w=2$ composites occupies a common fiber phase sector,
    we may introduce the order parameter
    \begin{equation}
      \Psi = |\Psi| e^{i\theta},
    \end{equation}
    where $|\Psi|^2$ measures the density of coherent composites.

    Gauge covariance fixes the unique admissible covariant gradient
    \begin{equation}
      D\theta = \nabla\theta - \frac{2e}{\hbar} A.
    \end{equation}

    The leading contribution to the free energy must then take the form
    \begin{equation}
      F_{\text{grad}} =
      \frac{\hbar^2 \rho_s}{2 m^*}
      \left(\nabla\theta - \frac{2e}{\hbar} A\right)^2,
    \end{equation}
    where $\rho_s$ is the phase stiffness and $m^*$ an effective inertia.

    This term is not postulated as a symmetry-breaking potential.
    It represents the energetic cost of breaking global fiber coherence.
    The London equation and flux quantization follow directly from this structure.

  \subsubsection{Frustration-Induced Stabilization in Strongly Correlated Regimes}
    \label{subsubsec:frustration-stabilization}

    In conventional low-$T_c$ materials, mediator-assisted interactions lower
    $\Delta_{\Pi}$ and stabilize the $w=2$ class.

    In strongly correlated systems, stabilization may arise instead from
    real-space frustration in admissibility constraints of the fiber.
    Let $F$ denote a frustration density proxy, operationally linked to
    staggered magnetic correlations.

    In this regime, isolated $w=1$ excitations increase local mismatch,
    whereas a $w=2$ composite redistributes phase incompatibility
    over an extended region.
    Pair formation thus minimizes a geometric cost functional rather than
    resulting from a specific momentum-space interaction kernel.

    The amplitude scale for composite formation is set by the dominant
    frustration energy scale $J$,
    \begin{equation}
      \Delta_{\Pi}^{\text{ampl}} \sim J.
    \end{equation}

  \subsubsection{Phase Stiffness and BKT-Type Transition}
    \label{subsubsec:phase-stiffness}

    Global superconductivity requires phase locking across the system.
    The relevant control parameter is the phase stiffness $\rho_s$,
    which measures the relaxation capacity of the projected substrate.

    In doped quasi-two-dimensional systems, the density of active composites
    scales with carrier concentration $\delta$.
    To leading order,
    \begin{equation}
      \rho_s(0) \propto \delta J.
    \end{equation}

    In strongly anisotropic layered systems, the transition temperature
    is governed by vortex unbinding and satisfies approximately
    \begin{equation}
      k_B T_c \simeq \frac{\pi}{2} \rho_s(T_c^-).
    \end{equation}

    This naturally allows separation between amplitude formation
    ($T^* \sim \Delta_{\Pi}^{\text{ampl}}/k_B$)
    and global phase coherence ($T_c$).

  \subsubsection{Superconductivity as Collective Saturation}
    \label{subsubsec:sc-collective-saturation}

    The emergence of superconductivity can thus be interpreted as a collective
    saturation phase of the $U(1)$ fiber.

    At the microscopic level, bounded projective transport constrains
    admissible field gradients.
    At the macroscopic level, coherent occupation of a single fiber phase
    sector minimizes relaxation cost and expels magnetic flux from the bulk.

    In this perspective, superconductivity is structurally analogous to
    other saturation phenomena derived in this work.
    Mass corresponds to isotropic inhibition of $\chi$ relaxation.
    Schwinger pair production corresponds to saturation of directed flux.
    Gravitational curvature corresponds to collective slowdown of relaxation ordering.

    Superconductivity represents the coherent phase of the same bounded
    projective dynamics in the electromagnetic sector.

    No modification of Standard Model interactions is required.
    The phenomenon emerges as a specific collective regime of the
    already established $U(1)$ fiber structure.
