\subsection{Schwinger Effect as a Saturation Threshold of Relaxation Flux}
\label{subsec:schwinger-saturation}

In standard quantum field theory, the Schwinger effect is interpreted as a
vacuum instability under an electric field exceeding a critical threshold.
In Cosmochrony, the Born--Infeld-type effective dynamics provides a
structurally different interpretation.

The Schwinger effect corresponds to a \emph{threshold of flux saturation}.
When the imposed electric field drives the effective relaxation flux beyond
what can be transported smoothly through the projection fiber, the
homogeneous relaxation regime becomes unstable.
The system resolves this instability by activating additional admissible
modes of the projection, corresponding to the nucleation of conjugate
torsional excitations---naturally identified with electron--positron pairs.
As discussed in
Section~\ref{subsec:charge-as-a-topological-and-relaxational-property-of-chi},
such pairs preserve global torsional neutrality, ensuring charge
conjugation symmetry.

Pair production thus originates from a topological reconfiguration of the
projection fiber under maximal relaxation stress, not from vacuum
fluctuations.
Matter creation acts as a dissipation channel that restores admissibility
by redistributing excess relaxation flux into stable, projectable vortical
modes.
This enlarges the space of admissible configurations, in accordance with
the principle of monotonic growth of admissible states.

Cosmochrony therefore predicts the existence of a Schwinger-like threshold
\emph{ab initio}, without invoking the Dirac sea or field quantization.
The effect emerges as a universal transition between smooth relaxation and
topologically mediated dissipation.

\paragraph{Dissipation by structure creation.}
This mechanism illustrates a more general principle: when directional
relaxation fluxes approach their maximal transport capacity, admissibility
is restored by creating new stable structures that enlarge the space of
admissible configurations.
This suggests a unified interpretation of pair-loaded astrophysical jets,
primordial matter production, and ultra-high-energy excitations in extreme
curvature regimes
(Section~\ref{sec:testable-predictions-and-observational-signatures}).
