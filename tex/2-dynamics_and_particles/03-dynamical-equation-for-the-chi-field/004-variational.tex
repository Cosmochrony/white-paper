\subsection{Variational Formulation and Born--Infeld Action}
\label{subsec:variational-formulation}

In regimes admitting a stable geometric interpretation, the effective
relaxation constraints may be summarized in a compact variational form.
Motivated by Born--Infeld-type non-linear
actions~\cite{BornInfeld1934,DeserGibbons1998}, we consider the effective
Lagrangian density
\begin{equation}
  \mathcal{L}_{\mathrm{eff}}
  = -c^2 \sqrt{1
      - \frac{|\nabla \chi_{\mathrm{eff}}|^2}{c^2}}
  + \mathcal{D}_{\mathrm{loc}}\chi_{\mathrm{eff}}
  - \frac{4\pi G_{\mathrm{eff}}}{c^2}\,\rho\,\chi_{\mathrm{eff}} ,
  \label{eq:leff_equation}
\end{equation}
where $\mathcal{D}_{\mathrm{loc}}\chi_{\mathrm{eff}}$ is the effective
local relaxation ordering (Section~\ref{subsec:parameter-independent-relaxation})
and $\rho$ the effective density of localized relaxation-resistant
configurations.

The linear dependence on
$\mathcal{D}_{\mathrm{loc}}\chi_{\mathrm{eff}}$ enforces monotonicity
without additional propagating degrees of freedom.
The square-root structure acts as a non-linear regulator ensuring that
effective spatial variations remain bounded by~$c$.
The Euler--Lagrange equation reproduces the non-linear elliptic relation
\begin{equation}
  \nabla \cdot
  \left(
    \frac{\nabla \chi_{\mathrm{eff}}}
    {\sqrt{1 - |\nabla \chi_{\mathrm{eff}}|^2 / c^2}}
  \right)
  = \frac{4\pi G_{\mathrm{eff}}}{c^2}\,\rho ,
  \label{eq:nonlinear_poisson}
\end{equation}
coinciding with the effective relation obtained in
Section~\ref{subsec:microscopic-origin-of-the-coupling-tensor-and-the-poisson-equation}.

This Born--Infeld-like action is an \emph{auxiliary variational
representation}.
It does not define a fundamental action principle, nor equations of motion
for the $\chi$ substrate.
Its purpose is to regularize the effective description, enforce universal
structural bounds, and facilitate comparison with standard gravitational
phenomenology.

The physical interpretation is discussed in
Appendix~\ref{subsec:hydrodynamic-limit}, and the mathematical consistency
with the relational dynamics is established in
Appendix~\ref{sec:born-lagrangian_derivation}.

\paragraph{Connection to Emergent Geometry.}
In regimes where a geometric description is applicable, the Hessian of
$\mathcal{L}_{\text{eff}}$ defines an emergent metric tensor~$g_{\mu\nu}$,
summarizing the effective geometric regularities of projected
configurations.

\paragraph{Why This Is Not a Scalar--Tensor Theory.}
\label{subsec:not-scalar-tensor}
The effective scalar descriptor~$\chi_{\mathrm{eff}}$ is not a fundamental
dynamical field: it does not possess intrinsic values or conjugate momenta,
and does not propagate independently.
No modification of the gravitational sector is postulated, no
scalar--tensor coupling of the form $f(\chi_{\mathrm{eff}}) R$ is
introduced, and no additional propagating modes arise.
Gravitation emerges from the local inhibition of relaxation ordering, not
from the exchange of a scalar mediator.

\subsubsection{Operational saturation scale and acceleration proxy}
  \label{subsubsec:operational_saturation_scale_wp}

  The invariant scale $b_{\chi}$ bounds the maximal rate of admissible
  relaxation transport in the substrate~$\chi$.
  The kinematic bound $c$ appearing in the square-root structure is the
  effective projection of this invariant relaxation limit in weak-field
  geometric regimes.

  In projectable regimes, the effective constitutive response therefore
  takes a Born--Infeld-like form.
  Projected gradients cannot grow without bound while remaining within a
  smooth quasi-injective update.

  In the macroscopic limit, one may introduce an operational acceleration
  proxy $a_\star$ marking the onset of the saturated response.
  Dimensional consistency and bounded transport jointly imply that this
  threshold must scale inversely with the local descriptive resolution~$\ell$
  of the projected state~$U$.
  We therefore parametrize
  \begin{equation}
    a_\star \sim \kappa \frac{b_{\chi}}{\ell},
  \end{equation}
  where $\kappa$ is a dimensionless normalization factor encoding the
  convention adopted for the projective mapping~$\Pi$.
  The role of $\ell$ and its scale dependence are discussed in
  Section~\ref{subsec:projective-resolvability-bound}.
