\subsection{Mass as Resistance to \texorpdfstring{$\chi$}{χ} Relaxation}
\label{subsec:mass_as_resistance}

Building on the ontological interpretation of mass as frozen structural
information (Section~\ref{subsec:mass-as-frozen-information}), we develop
the quantitative formulation.
Mass emerges at the effective level as a measure of how strongly a
localized projected configuration resists admissible relaxation ordering.

The effective structural energy associated with a projected solitonic
configuration $\chi_{\mathrm{eff},s}$ is
\begin{equation}
  E[\chi_{\mathrm{eff},s}] \;\equiv\;
  \int_{\Sigma}
  \left(
    \frac{1}{\sqrt{1
      - |\nabla \chi_{\mathrm{eff},s}|^2 / c^2}} - 1
  \right)
  \, d\Sigma ,
  \label{eq:chi_soliton_energy}
\end{equation}
where $\Sigma$ denotes a hypersurface of constant effective ordering
parameter and $|\nabla \chi_{\mathrm{eff},s}|$ quantifies effective
structural deformation.
The inertial mass is then defined operationally as
\begin{equation}
  m \;\equiv\; \frac{E[\chi_{\mathrm{eff},s}]}{c^2}.
  \label{eq:mass_definition}
\end{equation}
For example, the electron mass~$m_e$ reflects its topological stability as
a $4\pi$-periodic soliton
(Section~\ref{subsec:4pi_soliton}), while the proton mass arises from a
composite 3-soliton configuration.

The relation $E = mc^2$ is interpreted as a kinematic identity: mass
quantifies relaxation resistance, while energy expresses the same quantity
in relaxation units.
The question of how distinct particle masses arise from different classes
of projected configurations is addressed in
Appendix~\ref{app:topological_solitons}.
