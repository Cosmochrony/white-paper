\subsection{Summary}
  \label{subsec:summary-particles}

  Particles arise as stable, localized projected configurations that resist
  admissible relaxation ordering.
  Mass is identified with the degree of effective resistance to $\chi$
  relaxation, naturally yielding $E = mc^2$ as a kinematic identity.
  Spin and statistical behavior originate from topological obstructions:
  fermionic configurations exhibit $4\pi$ periodicity, providing a unified
  origin for spin-$\tfrac{1}{2}$ behavior, fermionic antisymmetry, and the
  Pauli exclusion
  principle~\cite{Pauli1925,Dirac1928}.
  Electric charge is interpreted as a chiral--torsional invariant of the
  relaxation flux.
  Residual spectral splittings---such as the Lamb shift and hyperfine
  structure---arise as finite corrections induced by non-linear saturation
  and projectability constraints.

  Localized excitations of the χ field may also admit collective coherent regimes in the $U(1)$
      fiber sector, to be analyzed in Section~\ref{subsec:collective-U1-coherence}.
