% ----------------------------------------------------------------------------
% Section 4.7 --- Charge as a Topological and Relaxational Property of chi
% From former §6.8, condensed
% ----------------------------------------------------------------------------
\subsection{Charge as a Topological and Relaxational Property of
\texorpdfstring{$\chi$}{χ}}
\label{subsec:charge-as-a-topological-and-relaxational-property-of-chi}

\subsubsection*{Unified Relaxation Budget for Mass and Charge}
\label{subsec:unified_budget_mass_charge}

Both inertial mass and electric charge draw on the same finite relaxation
capacity of~$\chi$.
A localized excitation mobilizes capacity in two structurally distinct
channels: a scalar inhibition channel associated with inertial response
(mass), and an oriented or chiral channel associated with a net
topological flux (charge).
Admissible projected configurations satisfy a single combined bound
\begin{equation}
  \label{eq:combined_budget_bound}
  \mathcal{B}_m[\chi] + \mathcal{B}_e[\chi]
    \le \mathcal{B}_{\max},
\end{equation}
where $\mathcal{B}_{\max}$ is fixed by the universal relaxation capacity
constraint (Section~\ref{subsec:variational-formulation}).
Effective mass and charge are therefore competing manifestations of a
common substrate resource, and the observed discreteness of particle
masses may be approached as a classification of admissible stable
configurations under a single saturation constraint.

\subsubsection*{Charge as a Directed and Conjugate Relaxation Mode}
\label{subsec:charge-as-directed-relaxation}

Charges arise only after projection, as stable invariants characterizing
admissible projected configurations.
Charge conjugation is reinterpreted as a transformation between
\emph{relationally conjugate} projected configurations: paired
topological classes related by an internal reversal of relational
organization, reflecting the internal duality of the projection fiber.

At the discrete level, the $\chi$ dynamics induces a bounded relaxation
flux $\vec{J}_\chi \sim \nabla \chi$.
Charge corresponds to the non-integrability of this flux around localized
excitations: the resulting winding number defines an integer-valued
topological invariant whose sign encodes chirality.
Electric attraction and repulsion arise from the geometric compatibility
or frustration between torsional flux patterns of neighboring
excitations.
The bounded nature of $\vec{J}_\chi$ implies a maximal admissible charge
density and naturally removes short-distance singularities.

The robustness of this chiral--torsional invariant is tested in
Appendix~\ref{subsec:cp-asymmetry-and-chiral-selection}.

\paragraph{CP Symmetry as Projective Chirality.}
\label{subsec:cp-as-projective-chirality}
CP symmetry corresponds to a combined transformation acting on the
orientation of the projection itself.
When the projection is achiral, conjugate configurations are mapped
symmetrically and CP is preserved.
When the projection is chiral, CP is violated as a direct consequence of
projective asymmetry, without introducing explicit CP-violating terms at
the level of~$\chi$.
