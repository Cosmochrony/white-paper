% ----------------------------------------------------------------------------
% A.10 --- Level Sets, Projections, and Apparent Orbital Geometry
% From former A10, condensed
% ----------------------------------------------------------------------------
\subsection{Level Sets, Projections, and Apparent Orbital Geometry}
\label{app:level_sets_orbitals}

For a continuous scalar field
$\phi : \mathbb{R}^3 \rightarrow \mathbb{R}$, the level set at
constant~$c$ is
\begin{equation}
  \mathcal{L}_c
  = \{ \mathbf{x} \in \mathbb{R}^3
       \mid \phi(\mathbf{x}) = c \}.
\end{equation}
Even when~$\phi$ is continuous everywhere, the projected set
\begin{equation}
  P_c
  = \{ z \in \mathbb{R} \mid \exists (x,y)
       \text{ s.t.\ } \phi(x,y,z) \ge c \}
\end{equation}
typically consists of disjoint intervals---a purely geometric
consequence of thresholding.
Using the envelope function
$f(z) = \max_{x,y} \phi(x,y,z)$, one writes
$P_c = \{z \mid f(z) \ge c\}$.

Threshold-based visualizations of continuous scalar fields generically
produce apparently disjoint structures that do not correspond to
independent physical entities.
This provides a natural framework for understanding orbital-like
patterns as emergent manifestations of an underlying continuous field:
apparent discreteness reflects projection and detection criteria rather
than fundamental discontinuity.
