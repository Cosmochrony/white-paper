% ----------------------------------------------------------------------------
% A.6 --- Minimal Kinematic Constraint
% From former A06, condensed
% ----------------------------------------------------------------------------
\subsection{Minimal Kinematic Constraint}
\label{subsec:minimal-kinematic-constraint}

In its saturated form:
\begin{equation}
  (\partial_t \chi)^2 + |\nabla \chi|^2 = c^2,
  \label{eq:minimal-kinematic-constraint}
\end{equation}
where $\partial_t$ and $\nabla$ are effective operators introduced only
once a projectable geometric regime is established.
More generally, admissible configurations satisfy the corresponding
inequality.

This constraint does not presuppose a spacetime metric or Lorentzian
structure.
It expresses a purely kinematic saturation condition: admissible
projected descriptions must lie on the boundary of a fixed relaxation
budget.
The constant~$c$ is the effective manifestation of the invariant
structural bound~$c_\chi$
(Section~\ref{subsec:role-of-cchi}).
Only after projection does this bound acquire an operational
interpretation as a maximal local ordering rate.

In homogeneous regimes, the constraint enforces
$\partial_t\chi = c$, leading to effective cosmic expansion.
In inhomogeneous regimes, partial saturation manifests as local slowdown
underlying gravitational time dilation.
Lorentz symmetry arises \emph{a~posteriori} as a property of saturated
relaxation.

At scales comparable to the fundamental relational spacing, the
continuum approximation breaks down and the constraint must be
rederived from purely discrete considerations.
