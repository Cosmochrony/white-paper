% ----------------------------------------------------------------------------
% A.2 --- Stability Analysis
% From former A02, condensed
% ----------------------------------------------------------------------------
\subsection{Stability Analysis of the
\texorpdfstring{$\chi$}{χ}-Field Dynamics}
\label{subsec:stability-analysis}

The effective relaxation dynamics may be written as
\begin{equation}
  \partial_t \chi
  = c \sqrt{1 - \frac{|\nabla \chi|^2}{c^2}}.
\end{equation}

\paragraph{Marginal linear stability.}
Around the homogeneous background
$\chi_0(t) = c\,t + \chi_{0,0}$, a perturbation
$\delta\chi(x,t)$ satisfies
\begin{equation}
  \partial_t \delta \chi
  = -\frac{1}{2c}\,|\nabla \delta \chi|^2
    + \mathcal{O}(|\nabla \delta \chi|^4).
\end{equation}
No linear term appears: the homogeneous solution is marginally stable
at linear order.

\paragraph{Nonlinear stability.}
The leading nonlinear correction is strictly negative, so spatial
inhomogeneities reduce the local relaxation rate and are dynamically
suppressed.
The functional
\begin{equation}
  E[\delta \chi]
  = \frac{1}{2}
    \int |\nabla \delta \chi|^2 \, d^3x
\end{equation}
is non-increasing.
The effective dynamics is dissipative, contractive, and nonlinearly
stable.
This stability is inseparable from the monotonic character of the
relaxation process.
