% ----------------------------------------------------------------------------
% A.11 --- Emergent Electrodynamics from chi Dynamics
% From former A11, condensed
% ----------------------------------------------------------------------------
\subsection{Emergent Electrodynamics from
\texorpdfstring{$\chi$}{χ} Dynamics}
\label{app:emergent-electrodynamics}

In the weak-field limit, the nonlinear field equation linearizes to
\begin{equation}
  \nabla^2 \chi
  - \frac{1}{c^2}
    \frac{\partial^2 \chi}{\partial t^2}
  = 4 \pi G_{\mathrm{eff}} \rho_\chi,
\end{equation}
admitting propagating collective modes of the projected description.

The spatial gradient decomposes as
\begin{equation}
  \nabla \chi = - \nabla \phi
    + \mathbf{A}_{\mathrm{T}},
  \qquad
  \nabla \cdot \mathbf{A}_{\mathrm{T}} = 0,
\end{equation}
an effective Helmholtz projection induced by the topology of
localized~$\chi$ excitations.

\subsubsection*{Charge as Transverse Torsion}
\label{subsec:charge-as-torsion}

The bounded canonical current
\begin{equation}
  \mathbf{J}_\chi
  \;\propto\;
  \frac{\nabla \chi}
    {\sqrt{1 - |\nabla \chi|^2 / c^2}}
\end{equation}
saturates as $|\nabla\chi| \to c$.
An effective charge invariant is defined as
\begin{equation}
  q \;\equiv\;
  \kappa \oint_{\gamma}
    \mathbf{A}_T \cdot d\boldsymbol{\ell}
  \;=\;
  \kappa \int_{S}
    (\nabla \times \mathbf{A}_T)
    \cdot d\mathbf{S}.
\end{equation}

\subsubsection*{Emergent Electromagnetic Fields}

\begin{equation}
  \mathbf{E}
  = -\nabla \phi
    - \frac{1}{c}
      \frac{\partial \mathbf{A}_{\mathrm{T}}}
           {\partial t},
  \qquad
  \mathbf{B}
  = \nabla \times \mathbf{A}_{\mathrm{T}}.
\end{equation}
These satisfy Maxwell-like relations:
\begin{align}
  \nabla \cdot \mathbf{E}
    &= 4\pi G_{\mathrm{eff}}
       \rho_{\mathrm{em}}, \\
  \nabla \times \mathbf{E}
    + \frac{1}{c}
      \frac{\partial \mathbf{B}}{\partial t}
    &= 0, \\
  \nabla \cdot \mathbf{B} &= 0, \\
  \nabla \times \mathbf{B}
    - \frac{1}{c}
      \frac{\partial \mathbf{E}}{\partial t}
    &= \frac{4\pi G_{\mathrm{eff}}}{c}
       \mathbf{J}_{\mathrm{em}}.
\end{align}
These are compatibility conditions of the projected description, not
fundamental field equations.
Emergent $U(1)$ gauge invariance reflects the relational nature
of~$\chi$: only gradient differences have operational meaning.
