% ----------------------------------------------------------------------------
% A.9 --- Energy and Curvature
% From former A09, condensed
% ----------------------------------------------------------------------------
\subsection{Energy and Curvature}
\label{subsec:energy-and-curvature}

Energy emerges as an effective measure of how strongly a configuration
resists global relaxation.
At the phenomenological level:
\begin{equation}
  \mathcal{E}_\chi^{\mathrm{eff}}
  \;=\;
  \frac{1}{2}
  \left[
    (\partial_t \chi)^2 + (\nabla \chi)^2
  \right],
\end{equation}
a diagnostic functional with no Hamiltonian or variational status.

Regions of large~$\mathcal{E}_\chi^{\mathrm{eff}}$ correspond to
configurations with strong internal gradients, identified in the
effective description with particle-like excitations carrying inertial
and gravitational properties.
The energetic ordering of atomic orbitals reflects the structural cost
of sustaining increasingly extended relaxation-resistant patterns, not
spatial distance to a nucleus.

Effective spacetime curvature arises as a macroscopic descriptor of how
constrained configurations modulate admissible ordering and correlation
structures.
No independent geometric degree of freedom is introduced.
Stable solitonic configurations arise when nonlinear self-interaction
balances dispersive gradients in the projected description.
