% ----------------------------------------------------------------------------
% A.1 --- Effective Lagrangian Description as a Hydrodynamic Limit
% From former A01, condensed
% ----------------------------------------------------------------------------
\subsection{Effective Lagrangian Description as a Hydrodynamic Limit}
\label{subsec:hydrodynamic-limit}

In regimes where~$\chi$ varies smoothly, the discrete relational
couplings encoded in~$K_{ij}$ can be summarized by effective continuum
quantities.
Distances are defined through the resistance encountered by relaxation
propagation:
\[
  g_{\mu\nu}\,dx^\mu dx^\nu
  \;\sim\;
  \sum_{(u,v)\in\text{path}} \frac{1}{K_{uv}},
\]
understood as a diagnostic illustration of how effective distance
emerges as cumulative resistance to~$\chi$ relaxation.

To reproduce the continuum evolution equations
(Equation~\ref{eq:discrete-dynamics}), one introduces an effective
Lagrangian density:
\[
  \mathcal{L}_{\text{eff}}
  = \frac{1}{16\pi G_{\text{eff}}}\,F(\chi)\,R
    - \Lambda_{\text{flow}}^{4}\,\chi + \cdots
\]
where~$R$ is the Ricci scalar of the effective metric.
This Lagrangian is purely representational, does not define fundamental
dynamics, and has no ontological status.
Singularities of the effective metric signal limits of the hydrodynamic
approximation, not failures of the underlying~$\chi$ dynamics.
