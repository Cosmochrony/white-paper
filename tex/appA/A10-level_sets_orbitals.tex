% A10 --- Level Sets, Projections, and Apparent Orbital Geometry
\subsection{Level Sets, Projections, and Apparent Orbital Geometry}
\label{app:level_sets_orbitals}

For a continuous scalar field
$\phi : \mathbb{R}^3 \rightarrow \mathbb{R}$, the level set
\begin{equation}
  \mathcal{L}_c
  = \{ \mathbf{x} \in \mathbb{R}^3 \mid
       \phi(\mathbf{x}) = c \}.
\end{equation}
may consist of disconnected components without implying
discontinuity of~$\phi$.
The projected set
\begin{equation}
  P_c = \{ z \in \mathbb{R} \mid \exists (x,y)\
    \text{s.t.}\ \phi(x,y,z) \ge c \}
\end{equation}
typically consists of disjoint intervals, and the envelope function
\begin{equation}
  f(z) = \max_{x,y} \phi(x,y,z)
\end{equation}
gives $P_c = \{ z \mid f(z) \ge c \}$.
Fragmentation is a geometric consequence of thresholding, not
fundamental discontinuity.
Orbital-like patterns, nodal structures, and visibility regions are
emergent manifestations of an underlying continuous field.
