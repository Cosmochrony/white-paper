\subsection{Spectral and Inertial Quantities}
  \label{subsec:spectral-and-inertial-quantities}

  \paragraph[$\Delta_{\Pi}$]{Projective stability gap of a composite fiber class.}
    $\Delta_{\Pi}$ is defined as the lowest positive eigenvalue of the effective relaxation operator restricted to
    the fiber sector around a composite winding configuration (notably $w=2$),
    $\Delta_{\Pi} \equiv \lambda_{\min}(L_{\Pi}|_{w=2})$.
    Operationally, it measures the minimal excitation cost required to destabilize the composite class,
    either by separation into lower-winding excitations or by inducing a local phase defect.
    $\Delta_{\Pi}$ is a spectral stability quantity associated with admissibility of coherent composite sectors,
    and it may admit distinct amplitude and phase-controlled manifestations depending on the regime.

  \paragraph[$\Delta_{\Pi}^{\mathrm{ampl}}$]{Amplitude (formation) scale associated with composite stability.}
    $\Delta_{\Pi}^{\mathrm{ampl}}$
        denotes the amplitude scale governing local composite formation in strongly constrained regimes.
        In the strongly correlated case discussed in Section~\ref{subsec:collective-U1-coherence},
    it is controlled by the dominant frustration or exchange scale, written schematically as
    $\Delta_{\Pi}^{\mathrm{ampl}} \sim J$.
    This quantity should be distinguished from the phase ordering scale controlled by $\rho_s$.

  \paragraph{\texorpdfstring{$\lambda_n$}{λn} (Spectral eigenvalues).}
    Eigenvalues of the linearized relaxation or stability operator acting on small
    perturbations of a localized $\chi$ configuration.
    They determine inertial mass scales in the effective description.

  \paragraph{\texorpdfstring{$\psi_n$}{ψn} (Spectral modes).}
    Eigenmodes associated with the operator $\Delta_G$.
    They encode the internal structure and stability of particle-like configurations.

  \paragraph{\texorpdfstring{$m_{\mathrm{eff}}$}{meff} (Effective mass).}
    An emergent invariant determined by the spectral stability of localized
    $\chi$ configurations under projection.
    Operationally, effective mass measures the self-consistency overhead
    associated with maintaining a coherent projected description under
    local updates.
    Mass is not a fundamental parameter nor a coupling constant.

  \paragraph{\texorpdfstring{$Q$}{Q} (Topological charge).}
    An integer-valued invariant characterizing the topology of a stable
    $\chi$ configuration.
    Different values of $Q$ correspond to distinct particle families.

  \paragraph{\texorpdfstring{$\Omega^\pm$}{Ω±} (Chiral topological sectors).}
    Opposite chiral configurations of topological $\chi$ structures.
    They are related by orientation reversal and need not be energetically equivalent.
