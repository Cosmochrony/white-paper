\subsection{Key Conceptual Terms}
  \label{subsec:key-conceptual-terms}

  \paragraph{Energy.}
    Energy is an effective quantity measuring the operational cost associated
    with modifying a projected configuration while preserving coherence under
    admissible updates.
    This cost reflects the mobilization of the underlying relaxation capacity
    of the $\chi$ substrate under projection, but energy itself is not a
    fundamental property of $\chi$.
    Standard conservation laws remain valid at the effective level.

  \paragraph{Fluctuations.}
    Local stochastic modulations of $\chi$ configurations that affect event timing
    and localization without altering underlying topological constraints.

  \paragraph{Matter.}
    Stable topological configurations of $\chi$ whose persistence gives rise to
    particle-like behavior and inertial properties.

  \paragraph{Measurement.}
    A localized interaction that selects a specific effective realization
    within a non-injective projection.
    Measurement does not induce a fundamental collapse, but resolves an
    equivalence class of projected configurations into a stable operational
    outcome.

  \paragraph{Probability.}
    An emergent descriptor reflecting the multiplicity of projected realizations
    compatible with a given relational configuration and effective state.
    Probability is projective rather than fundamental, and does not correspond
    to intrinsic randomness of the substrate.

  \paragraph{Relaxation (of the \texorpdfstring{$\chi$}{χ} field).}
    The intrinsic irreversible reorganization of relational configurations of
    $\chi$ under internal coupling constraints.
    Relaxation supplies the flux of candidate configurations required for
    successive projected updates.
    It is pre-thermodynamic and does not correspond to dissipation.

  \paragraph{Spacetime.}
    An emergent relational structure arising from large-scale configurations of
    the $\chi$ substrate.
    Its metric description remains valid within its domain of applicability.

  \paragraph{Time.}
    An effective parameter associated with the local rate of $\chi$ relaxation.
    Operational and relativistic notions of time are recovered without modification.

  \paragraph{Relaxation transmittance (gauge interpretation).}
    An effective, context-dependent measure of how efficiently relaxation flux (or
    ordering capacity) is transmitted through a given projected configuration.
    In Cosmochrony, gauge structure can be interpreted as a parametrization of
    transmittance variations across the projection fiber, rather than as a fundamental
    interaction field.

  \paragraph{Wavefunction.}
    An effective statistical representation of the dynamics and topology of the
    $\chi$ substrate.
    It has no fundamental ontological status.

  \paragraph{Wave--Particle Duality.}
    A manifestation of interaction-induced changes in the local configuration of
    $\chi$, producing localized particle-like behavior from an underlying
    wave-like substrate.
