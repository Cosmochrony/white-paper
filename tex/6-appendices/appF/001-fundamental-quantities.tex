\subsection{Fundamental Quantities}
  \label{subsec:fundamental-quantities}

  \paragraph{\texorpdfstring{$\chi$}{χ} (Chi substrate).}
    The unique fundamental entity of the Cosmochrony framework.
    $\chi$ is a pre-geometric, relational substrate not defined on a pre-existing
    spacetime manifold.
    Its irreversible relaxation supplies a continuous flux of relational
    configurations.
    $\chi$ does not encode a prescriptive set of admissible physical states;
    admissibility arises only through projection under finite resolution and
    compatibility constraints.
    Localized, topologically stable configurations of $\chi$ correspond to
    particle-like excitations upon projection.

  \paragraph{\texorpdfstring{$\chi_i$}{χᵢ} (Local configuration).}
    Discrete local degrees of freedom of the $\chi$ substrate, associated with
    vertices of the relaxation network.
    They encode the microscopic relational state prior to any geometric projection.

  \paragraph{\texorpdfstring{$\chi_c$}{χc} (Critical relaxation threshold).}
    A fundamental structural bound limiting local variations of $\chi$.
    It constrains the projection and update process by enforcing causal
    consistency, and underlies all effective speed and action bounds.
    This threshold does not prescribe physical laws, but limits the capacity
    of admissible projected updates.

  \paragraph{\texorpdfstring{$\tau$}{τ} (Operational time).}
    An effective ordering parameter defined from the monotonic succession of
    distinguishable projected configurations, constructed from the effective
    descriptor $\chi_{\mathrm{eff}}$.
    Operational time $\tau$ is not a fundamental temporal coordinate and does
    not exist at the level of the $\chi$ substrate.
    It emerges only in projectable regimes as an operational measure of duration,
    defined through the ordered differentiation of admissible effective states.
    A perfectly stationary projection would correspond to $\Delta\tau=0$ and
    is therefore excluded by the requirement of projective continuity.

  \paragraph{\texorpdfstring{$c_\chi$}{cχ} (Fundamental relaxation speed).}
    The maximal propagation speed of relaxation disturbances within the $\chi$ substrate.
    It represents the fundamental causal bound of the theory, from which the effective speed of light emerges.
    The maximal propagation speed of relaxation disturbances within the $\chi$ substrate.
    It represents the kinematic manifestation of the invariant relaxation
    bound $b_{\chi}$ in transport-limited regimes.
    The effective speed of light $c$ emerges as its geometric projection.
