\subsection{Dimensionless Parameters}
  \label{subsec:dimensionless-parameters}

  \paragraph{\texorpdfstring{$S$}{S} (Gradient saturation parameter).}
    A dimensionless quantity defined as
    \begin{equation}
      S \equiv \frac{1}{c^2}\sum_{j\sim i} K_{ij}(\chi_i-\chi_j)^2 ,
      \label{eq:gradient-saturation-parameter}
    \end{equation}
    measuring the local density of projected $\chi$ gradients.
    The bound $S \leq 1$ enforces causal consistency in the emergent
    spacetime description.
    The constant $c$ appearing here is the effective geometric projection
    of the invariant transport bound $b_{\chi}$.

  \paragraph{\texorpdfstring{$\Omega_\chi$}{Ωχ} (Relaxation budget parameter).}
    A dimensionless global quantity characterizing the fraction of total $\chi$
    relaxation stored in spatial gradients.
    In cosmological regimes, it plays a role analogous to a density parameter.

  \paragraph[$F$] \textit{Frustration density (operational proxy).}
    $F$ denotes a dimensionless proxy for the density of incompatible local raccordement constraints in the fiber
    sector.
    It is introduced as an operational quantity, grounded in independently measurable correlators in the relevant
    materials.
    In the strongly correlated regime, $F$ parametrizes the strength of staggered frustration and enters the qualitative
    selection of the low-cost composite coherence channel.
    $F$ is not an additional fundamental field or a new ontological variable.
    It summarizes environmental and material constraints within an effective description.

  \paragraph[$r_F$] \textit{Relative frustration ratio.}
    $r_F$is a dimensionless ratio encoding the relative amplitude of the dominant frustration sector across material
    families.
    It is used to compare coherence channel selection across systems with different lattice constraints and correlation
    structure.
    In particular, $r_F$
        organizes the expected transition between distinct symmetry channels in the superconducting regime.
