\subsection{Effective and Projected Quantities}
  \label{subsec:effective-and-projected-quantities}

  \paragraph{\texorpdfstring{$\chi_{\mathrm{eff}}$}{χeff} (Effective projected field).}
    A coarse-grained scalar field arising from the non-injective projection of $\chi$
    onto an emergent spacetime description.
    $\chi_{\mathrm{eff}}$ provides an effective field-theoretic representation
    without fundamental status.

  \paragraph{Fiber (of the projection).}
    For a given effective configuration $\chi_{\mathrm{eff}}$, the fiber is the set of
    underlying $\chi$ configurations mapped to it by the projection $\pi$.
    Elements of a fiber are operationally indistinguishable at the spacetime level.
    Non-trivial fibers reflect the structural non-injectivity of the projection.

  \paragraph{\texorpdfstring{$O$} Operational projection (contextual access).}
    A measurement-context-dependent map that specifies how the effective description
    $\chi_{\mathrm{eff}}$ is accessed and turned into operational observables.
    It introduces no additional ontology: it formalizes the contextual readout of
    $\chi_{\mathrm{eff}}$ (e.g., choice of apparatus, coarse-graining, relational query).

  \paragraph{\texorpdfstring{$\pi$}{π} (Projection).}
    An operational projection correspondence relating configurations of $\chi$
    to effective descriptions $\chi_{\mathrm{eff}}$ in projectable regimes.
    The projection is generally non-injective and does not define a single-valued
    function: distinct underlying configurations of $\chi$ may correspond to
    the same effective state, defining equivalence classes (fibers).

  \paragraph{\texorpdfstring{$\pi^{-1}$}{π⁻¹} (Deprojection).}
    The inverse reconstruction problem of identifying classes of $\chi$
    configurations compatible with a given effective state.
    Deprojection is not unique and does not destroy structural information.

  \paragraph{\texorpdfstring{$V(\chi)$}{V(χ)} (Effective potential).}
    An effective, coarse-grained description used to model localization and
    stability properties of $\chi$ configurations.
    $V(\chi)$ is not fundamental and is secondary to the spectral characterization
    of mass and inertia.

  \paragraph{\texorpdfstring{$\ell$}{ℓ} (Descriptive resolution scale).}
    A scale characterizing the local spectral resolution of the projected
    state $U$.
    It measures the minimal relational differentiation that can be
    operationally resolved within the effective description.
    Variations of $\ell$ across scales modulate the apparent saturation
    threshold $a_{\star}$ without altering the underlying bound $b_{\chi}$.

  \paragraph{Observable.}
    A stable, projectable quantity defined on $\chi_{\mathrm{eff}}$ through an
    interpretative framework.
    Observables do not correspond to additional ontological entities, but to
    operational readings of the same effective physical reality.

  \paragraph{Physical reality.}
    In the Cosmochrony framework, physical reality is identified with the effective
    level $\chi_{\mathrm{eff}}$.
    The substrate $\chi$ is ontologically real but does not constitute a physical
    universe until projected into a projectable regime.

  \paragraph{\texorpdfstring{$t_{\mathrm{proj}}$}{tproj} (Projected time).}
    Operational time measured within the emergent spacetime description.
    It arises from the local rate of $\chi$ relaxation and reproduces relativistic
    time dilation effects.

  \paragraph{Universe.}
    The physically real domain described at the $\chi_{\mathrm{eff}}$ level, where
    spacetime structure, causality, and physical observables are well-defined.
    The Universe does not refer to the fundamental substrate $\chi$, which is
    ontologically prior to any notion of universe.
