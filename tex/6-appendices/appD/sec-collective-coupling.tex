% ----------------------------------------------------------------------------
% D.1 --- Collective Gravitational Coupling and Operational Geometry
% From former D01, condensed
% ----------------------------------------------------------------------------
\subsection{Collective Gravitational Coupling and Operational
Geometry}
\label{subsec:collective-coupling}

Localized excitations act as persistent resistances to global
relaxation.
Their collective influence is summarized by a response operator
$K_{ij}$, the dressed counterpart of the bare connectivity
$K_{0,\mathrm{bare}}$
(Section~\ref{subsec:renormalization-hbar}).

In the weak-structure regime, the effective potential satisfies
\begin{equation}
  \nabla^2 \Phi_{\text{eff}}
  \approx 4\pi G_{\text{eff}}\,\rho,
  \label{eq:poisson-emergent}
\end{equation}
with
\begin{equation}
  G_{\text{eff}}
  \approx
  \frac{c^4}
       {K_{0,\text{eff}}\,\chi_{c,\text{eff}}^2}.
  \label{eq:G-emergent}
\end{equation}
Spatial geometry is defined operationally: configurations are close if
perturbations propagate efficiently between them.
Gravity is recovered as a macroscopic manifestation of relaxation
resistance.
