% ----------------------------------------------------------------------------
% D.4 --- Simulation Algorithms for chi-Field Dynamics
% From former D04, condensed
% ----------------------------------------------------------------------------
\subsection{Simulation Algorithms for
\texorpdfstring{$\chi$}{χ}-Field Dynamics}
\label{subsec:simulation-algorithms}

Numerical simulations implement finite-dimensional approximations of
the continuous relaxation dynamics.
Any apparent graph-like structure reflects the choice of numerical
basis, not a physical discretization.

\paragraph{Relaxation update rule.}
\begin{equation}
  \frac{d\chi_i}{d\lambda}
  = c \sqrt{
      1 - \frac{1}{c^2}
        \sum_j K_{ij}(\chi_i - \chi_j)^2
    }.
  \label{eq:discrete-dynamics}
\end{equation}
This enforces strict monotonicity, a universal upper bound on local
relaxation rate, and suppression of gradient-driven instabilities.

\paragraph{Reference pseudocode.}
\begin{algorithm}[t]
\caption{Bounded $\chi$-relaxation with diagnostics}
\label{alg:bounded-chi-relaxation}
\begin{algorithmic}[1]
  \Require $\chi^{(0)}_i$, $K_{ij}$, bound $c$,
    tolerances $\varepsilon_\chi,\varepsilon_S$
  \Ensure Relaxed $\chi^\star$ and diagnostics
  \State $n \gets 0$, $\chi \gets \chi^{(0)}$
  \While{$n < N_{\max}$}
    \For{each $i$}
      \State $S_i \gets
        \frac{1}{c^2}\sum_j K_{ij}(\chi_i-\chi_j)^2$
      \State $S_i \gets \min(S_i,1)$
      \State $v_i \gets c\sqrt{1 - S_i}$
    \EndFor
    \State Choose $\Delta\lambda$ adaptively
    \State $\chi_i \gets \chi_i + \Delta\lambda\,v_i$
    \If{$\max_i |\Delta\chi_i| < \varepsilon_\chi$}
      \textbf{break}
    \EndIf
    \State $n \gets n+1$
  \EndWhile
\end{algorithmic}
\end{algorithm}

\paragraph{Chiral--torsional charge invariant.}
For closed loops $\gamma$ in the numerical representation:
\begin{equation}
  Q_\gamma
  \equiv
  \frac{1}{2\pi}
    \sum_{(i,j)\in\gamma} \Delta\theta_{ij}.
  \label{eq:charge_winding_pseudocode}
\end{equation}

\paragraph{Localized configurations.}
Simulations robustly exhibit spontaneous emergence of stable localized
configurations that resist global relaxation, interpreted as numerical
counterparts of solitonic excitations
(Section~\ref{sec:particles-as-localized-excitations-of-the-chi-field}).

\paragraph{Effective entanglement observable.}
\begin{equation}
  E_{\mathrm{eff}}(\mathcal{C})
  \equiv
  \Delta_{\Pi}(\mathcal{C})\,
  \bigl(1 - \mathcal{C}^{\nu}\bigr),
  \quad \nu > 0,
  \label{eq:effective-entanglement}
\end{equation}
where $\Delta_{\Pi}$ quantifies spectral non-injectivity and
$(1 - \mathcal{C}^\nu)$ encodes the loss of projective mobility near
Born--Infeld saturation.
Entanglement emerges only during discrete spectral reorganization
events---a structural prediction of Cosmochrony.

\begin{figure}[t]
  \centering
  \includegraphics[width=0.95\linewidth]
    {6-appendices/appD/D04_entanglement_intermittence_b}
  \caption{Effective entanglement and chiral bias vs.\
    compression $\mathcal{C}$.
    Top: spectral non-injectivity $\Delta_\Pi$.
    Middle: $E_{\mathrm{eff}}$ showing intermittent activation.
    Bottom: chiral (CP) bias, monotonic and robust.}
  \label{fig:D4-entanglement-intermittence}
\end{figure}

\paragraph{Projectability threshold $\Theta_p$.}
Monitored through spectral gap stability
($(\lambda_2 - \lambda_1)/\mathrm{Tr}(L) > \epsilon_p$) and
topological coherence
($\mathcal{E}_{\mathrm{proj}} < \mathcal{E}_{\max}$).
Crossing $\Theta_p$ marks the transition from ontological poverty to
complex solitonic structures.
The natural order of magnitude of $\epsilon_p$ is set by the emergence
scale of $\hbar$; in continuum terms the Planck length is understood as
the effective manifestation of a nonzero projection threshold.
