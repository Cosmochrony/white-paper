\subsection{Relational Configurations of $\chi$}
  \label{subsec:relational-configurations}

  The substrate $\chi$ is defined as a relational structure whose admissible
  configurations are characterized by adjacency relations rather than by
  embedding in a pre-existing spacetime manifold.
  Two configurations are considered adjacent if they can be connected
  through an admissible elementary relaxation update.

  A minimal structural requirement of the framework is that the relational
  adjacency structure of $\chi$ be connected.
  That is, for any two admissible configurations, there exists a finite
  relational chain of admissible updates linking them at the $\chi$ level.
  This connectedness condition applies to the pre-geometric relational
  network itself (or to its continuum limit), not to the emergent spacetime
  description.

  Connectedness of $\chi$ does not imply persistent effective coupling
  between all projected regions.
  Because the projection $\Pi$ is generally non-injective and subject
  to saturation, its image $\Pi(\Omega)$ may decompose into multiple
  disjoint effective domains.
  Such disjointness reflects a loss of mutual projectability rather
  than the existence of ontologically independent substrates.

  This distinction is essential.
  While effective domains may become mutually inaccessible within the
  emergent geometric description, the underlying $\chi$-structure remains
  a single connected system supporting a global relaxation ordering.
  The uniqueness of this ordering ensures the universality of structural
  invariants such as the effective bounds associated with $c$ and $\hbar$.

  The connectedness condition can also be formulated spectrally.
  Let $\Delta_\chi$ denote the relational Laplacian associated with the
  adjacency structure of $\chi$.
  Connectedness requires that $\Delta_\chi$ admit a single zero mode on
  the admissible sector.
  Multiple zero modes would correspond to independent relational components,
  which are excluded by the assumption of a single substrate.
  Effective ``islands'' may nonetheless arise as spectrally weakly coupled
  sectors under projection, without implying ontological multiplicity.
