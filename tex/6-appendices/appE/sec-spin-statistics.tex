% ----------------------------------------------------------------------------
% E.7 --- Topological Origin of Fermionic and Bosonic Statistics
% From former E.9, condensed
% ----------------------------------------------------------------------------
\subsection{Topological Origin of Fermionic and Bosonic Statistics}
\label{app:relational_spin_statistics}

The distinction between fermionic and bosonic behavior originates from
the internal topological structure of~$\chi$ configurations in
configuration space.

\paragraph{Fermionic behavior.}
Configurations with intrinsic $4\pi$-periodicity are double-valued under
$2\pi$ reorientation.
When projected, they exhibit fermion-like behavior: sign change under
$2\pi$ rotations, restoration only after $4\pi$, and
spin-$\tfrac{1}{2}$ transformation properties---without introducing
fundamental spinors.

\paragraph{Bosonic behavior.}
Topologically orientable configurations return to an equivalent state
after $2\pi$ reorientation, yielding integer-spin transformation
properties.

\paragraph{Topological mass ratios (heuristic).}
The mass ratios between particles emerge from knot-like configurations:
an electron corresponds to a twisted unknot ($Q_e = 1$) with fiber
volume $\propto \chi_c$; a proton to a trefoil knot ($Q_p = 3$) with
fiber volume $\propto \chi_c^3$.
The observed ratio is
\[
  \frac{m_p}{m_e}
  = \frac{\mathrm{Vol}(\Pi^{-1}(\text{proton}))}
         {\mathrm{Vol}(\Pi^{-1}(\text{electron}))}
  \approx 27\,\chi_c^2.
\]
For $\chi_c \approx 8.3$, this reproduces $m_p/m_e \approx 1836$,
providing a topological explanation independent of ad hoc parameters.

\paragraph{Spin--statistics connection.}
The relational-topological distinction between $4\pi$- and
$2\pi$-periodic configurations provides a natural qualitative
explanation of the spin--statistics connection, demonstrating that the
observed dichotomy can arise from internal organization of~$\chi$ prior
to any effective quantum description.
