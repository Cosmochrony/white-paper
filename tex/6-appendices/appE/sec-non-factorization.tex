% ----------------------------------------------------------------------------
% E.2 --- Non-Factorization and Entanglement
% From former E.2, condensed
% ----------------------------------------------------------------------------
\subsection{Non-Factorization and Entanglement}
\label{subsec:non-factorization-entanglement}

Factorization---decomposition preserving internal relaxation structure
while isolating disjoint subsets of relations---is not fundamental but
emerges only in restricted regimes.
Quantum entanglement arises as a direct manifestation of persistent
non-factorization: when a non-factorizable relational configuration
admits an effective projection onto spatially separated degrees of
freedom, its components remain relationally inseparable.

\paragraph{Projection-induced non-factorization.}
Because the projection
$\Pi:\mathcal{C}_\chi \rightarrow \mathcal{C}_{\mathrm{eff}}$ is
generically non-injective, a single effective configuration~$y$
corresponds to an equivalence class
\begin{equation}
  \Pi^{-1}(y) \subset \mathcal{C}_\chi .
\end{equation}
Entanglement arises when this fiber contains globally constrained
configurations that do not admit decomposition into independent
substructures compatible with the effective subsystem decomposition.
No conditioning on underlying relational degrees of freedom can restore
a product structure for joint outcome statistics.

\paragraph{Compression and limits of entanglement.}
Entanglement persists only in an intermediate regime where projection
preserves sufficient global relational structure to prevent
factorization, while still allowing stable decomposition into effective
subsystems.
If projection is effectively injective, fibers collapse to single
elements and factorization is recovered; if excessively
coarse-grained, relational constraints are erased and descriptions
become fully factorized.
