% ----------------------------------------------------------------------------
% E.8 --- Vacuum Energy versus Relaxation Capacity
% From former E.10, condensed
% ----------------------------------------------------------------------------
\subsection{Vacuum Energy versus Relaxation Capacity of the
\texorpdfstring{$\chi$}{χ} Field}
\label{subsec:vacuum-energy-relaxation-capacity}

No fundamental vacuum energy density is postulated.
Phenomena commonly attributed to vacuum energy are reinterpreted as
manifestations of the \emph{relaxation capacity} of~$\chi$: the ability
of a relational configuration to undergo further structural
reorganization under bounded constraints.
Relaxation capacity is contextual and non-extensive; it cannot be
meaningfully assigned to spacetime points.

Observable vacuum effects arise only when relational constraints restrict
admissible configurations.
The Casimir effect, for instance, results from a differential in
relaxation capacity between constrained and unconstrained configurations,
not from an absolute vacuum energy density.

Because relaxation capacity is non-extensive, it does not enter
gravitational dynamics as a uniform source term.
This provides a natural conceptual resolution of the cosmological
constant problem: the enormous vacuum energy inferred from zero-point
counting is not a physically meaningful quantity in the Cosmochrony
ontology.
