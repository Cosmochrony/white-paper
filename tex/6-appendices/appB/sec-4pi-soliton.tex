% ----------------------------------------------------------------------------
% B.4 --- 4pi-Periodic Soliton and Spinorial Behavior
% From former B04, condensed
% ----------------------------------------------------------------------------
\subsection{Example:
\texorpdfstring{$4\pi$}{4π}-Periodic Soliton and Spinorial
Behavior}
\label{subsec:4pi_soliton}

An illustrative construction supporting the topological interpretation
of spin
(Section~\ref{subsec:spin-statistics}).

In the projectable regime, certain localized excitations admit an
effective internal phase $\theta$.
A complex-valued proxy (the underlying field remains real):
$\chi_{\mathrm{eff}}(x)
  = \eta\tanh(\kappa x)\,e^{i\theta(x)}$.
Choosing $\theta(\alpha) = \alpha/2$:
\begin{equation}
  \psi(\alpha)
  \equiv \psi_0\,e^{i\alpha/2}.
  \label{eq:B4-half-angle-map}
\end{equation}
A $2\pi$ cycle inverts the sign:
\begin{equation}
  \psi(\alpha + 2\pi) = -\psi(\alpha).
  \label{eq:B4-2pi-sign}
\end{equation}
A $4\pi$ cycle restores identity:
\begin{equation}
  \psi(\alpha + 4\pi) = \psi(\alpha).
  \label{eq:B4-4pi-identity}
\end{equation}
Schematically:
\begin{equation}
  \begin{aligned}
    \alpha: 0 \to 2\pi
    &\Rightarrow \text{nontrivial loop }
      (\mathbb{Z}_2)
    \Rightarrow \psi \mapsto -\psi, \\
    \alpha: 0 \to 4\pi
    &\Rightarrow \text{trivial loop}
    \Rightarrow \psi \mapsto \psi.
  \end{aligned}
  \label{eq:B4-schematic-doublecover}
\end{equation}

This $4\pi$-periodicity mirrors the double-cover
$\mathrm{SU}(2) \to \mathrm{SO}(3)$ and provides a geometric basis
for fermion-like transformation behavior without fundamental spinor
fields.
The sign change is an effective encoding of the $\mathbb{Z}_2$
obstruction
(Section~\ref{subsec:status-formulation}:
$\pi_1(\mathcal{C}_{\mathrm{eff}}) = \mathbb{Z}_2$).
