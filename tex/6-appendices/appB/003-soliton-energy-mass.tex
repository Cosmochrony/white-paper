% ----------------------------------------------------------------------------
% B.3 --- Soliton Energy and Structural Mass Scaling
% From former B03, condensed
% ----------------------------------------------------------------------------
\subsection{Soliton Energy and Structural Mass Scaling}
\label{subsec:soliton_energy_mass}

Mass is interpreted as total resistance to relaxation.
In the effective geometric regime:
\[
  M_{\mathrm{eff}}
  \propto
  \int_{\mathcal{V}}
    \bigl[
      \mathcal{T}(\nabla\chi_{\mathrm{eff}})
      + \mathcal{U}(\chi_{\mathrm{eff}})
    \bigr]\,d^3x.
\]
For kink-like configurations, a balance between gradient resistance and
nonlinear stabilization fixes the soliton width~$\xi$, yielding
$M_{\mathrm{eff}}
  \sim \sqrt{\lambda_{\mathrm{eff}}}\,\xi\,\chi_c^2$.
Configurations with higher winding or linking indices involve
increased gradients, so $M_{n+1} > M_n$, establishing mass hierarchies
structurally.
From a spectral viewpoint,
$M_{\mathrm{eff}} \sim \lambda_{\min}^{-1}$, where
$\lambda_{\min}$ is the smallest positive eigenvalue of the linearized
relaxation operator.
