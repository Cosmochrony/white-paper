% ----------------------------------------------------------------------------
% B.11 --- Emergence of hbar_eff
% From former B11, condensed
% ----------------------------------------------------------------------------
\subsection{Spectral Stability and the Emergence of
\texorpdfstring{$\hbar_{\mathrm{eff}}$}{ℏeff}}
\label{sec:hbar_eff_derivation}

From the fundamental scales $K_0$ (stiffness), $\chi_c$ (correlation
length), and $c$ (maximal speed), a natural action unit is
$\hbar_\chi \equiv c^3/(K_0\chi_c)$.
Here $K_0$ and $\chi_c$ denote bare substrate parameters.

For a solitonic mode with eigenvalue $\lambda_n$, identifying the rest
energy $E_n = \hbar_{\mathrm{eff}}\,\nu_n$ yields
$\hbar_{\mathrm{eff}}$ as a geometric and spectral quantity.
At microscopic scales ($\ell \sim \chi_c$),
$\hbar_{\mathrm{eff}} \approx \hbar_\chi \approx \hbar$.
At macroscopic scales,
$\hbar_{\mathrm{eff}}
  \approx \hbar_\chi
    (\chi_c/\ell_{\mathrm{spacetime}})^2$,
suppressing quantum effects through reduced spectral accessibility
rather than decoherence.
Reproducing particle-scale behavior requires
$K_0\chi_c^2 \sim \hbar$.
