\subsection{CP Asymmetry and Chiral Selection}
  \label{subsec:cp-asymmetry-and-chiral-selection}

  \subsubsection*{CPT versus CP as Admissibility Symmetries}

    Let projected configurations carry a set of signed structural invariants
    \(\{Q_i\}\), associated with orientation, chirality, or phase winding.
    The admissibility conditions are invariant under the combined transformation
    \[
      (Q_i,\;\tau,\;\mathbf{x}) \rightarrow (-Q_i,\;-\tau,\;-\mathbf{x}),
    \]
    which defines an effective CPT symmetry at the level of the projected description.

    In contrast, CP acts only on a subset of the invariants \(\{Q_i\}\)
    and does not reverse the effective ordering parameter.
    CP is therefore not, in general, an invariance of admissibility.
    Effective CP violation may arise without violating CPT invariance,
    reflecting a structural asymmetry of the projection
    rather than a breakdown of the relational substrate.

  \subsubsection*{Structural Bias and Statistical Asymmetry}

    Assume that admissible projected configurations exhibit a slight asymmetry in
    relaxation efficiency with respect to the sign of a structural invariant \(Q\).
    Let \(\Gamma(Q)\) denote the effective stabilization rate.

    If
    \[
      \Gamma(Q) \neq \Gamma(-Q),
    \]
    then configurations carrying one orientation are statistically favored during
    relaxation.
    Matter--antimatter asymmetry may therefore emerge as a dynamical bias in the
    projective selection process, without requiring explicit symmetry breaking
    at the fundamental level.

    This statistical asymmetry provides the ontological background
    within which the more specific spectral structure of flavor mixing,
    reconstructed in Section~\ref{subsec:flavor-structure-cp},
    can operate.

  \subsubsection*{Spectral Structure and the Jarlskog Invariant}

    In the Standard Model, CP violation in the quark sector is characterized
    by the Jarlskog invariant~\cite{Jarlskog1985}, which vanishes if any two
    fermion masses within a given sector are degenerate or if the mixing
    matrix is reducible to a two-generation form.

    Within Cosmochrony, this structure is interpreted in terms of the
    spectral properties of the projection fiber~\(\Pi\).
    Let \(\{\lambda_{u,i}\}\) and \(\{\lambda_{d,i}\}\) denote the effective
    spectral eigenvalues associated with the up-type and down-type relaxation
    modes, respectively.
    These eigenvalues determine the inertial masses via
    Eq.~(\ref{eq:mass_definition}) and encode the relaxation resistance of
    the corresponding projected configurations.

    Degeneracy of eigenvalues within either sector,
    \[
      \lambda_{u,i} = \lambda_{u,j}
      \quad \text{or} \quad
      \lambda_{d,i} = \lambda_{d,j},
    \]
    restores rotational freedom within the corresponding subspace,
    allowing phase redefinitions that eliminate any CP-odd invariant.

    The presence of three spectrally distinct generations
    (\(n \geq 3\)) is therefore the minimal condition under which an
    irreducible phase can arise.
    This threshold signals that the projective holonomy of the flavor
    manifold cannot be embedded in a two-dimensional spectral subspace.

  \subsubsection*{Sectorial Misalignment and Cubic Commutator Structure}

    The full CP-violating structure depends not only on spectral
    non-degeneracy within a single sector, but on the incompatibility
    between the up-type and down-type spectral bases.

    Let \(\Pi_u\) and \(\Pi_d\) denote the sectorial restrictions of the
    global non-injective mapping \(\Pi\) associated with these two
    relaxation channels.
    Because the projected state \(U\) has finite descriptive capacity
    (bounded by the projective bandwidth \(b_\chi\)),
    these operators cannot in general be simultaneously diagonalized.
    Their non-commutativity reflects an irreducible misalignment of
    spectral anchors within the same fiber.

    Following the algebraic structure identified by
    Jarlskog~\cite{Jarlskog1985}, the CP-odd invariant can be expressed
    in terms of the imaginary part of the trace
