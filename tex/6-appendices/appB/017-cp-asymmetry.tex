\subsection{CP Asymmetry and Chiral Selection}
  \label{subsec:cp-asymmetry-and-chiral-selection}

  \subsubsection*{CPT versus CP as Admissibility Symmetries}

    Let projected configurations carry a set of signed structural invariants
    \(\{Q_i\}\), associated with orientation, chirality, or phase winding.
    The admissibility conditions are invariant under the combined transformation
    \[
      (Q_i,\;\tau,\;\mathbf{x}) \rightarrow (-Q_i,\;-\tau,\;-\mathbf{x}),
    \]
    which defines an effective CPT symmetry at the level of the projected description.

    In contrast, CP acts only on a subset of the invariants \(\{Q_i\}\) and does not
    reverse the effective ordering parameter.
    CP is therefore not, in general, an invariance of the admissibility conditions.
    Effective CP violation may arise without violating CPT invariance,
    reflecting a structural asymmetry of the projection rather than a breakdown
    of the underlying relational substrate.

  \subsubsection*{Structural Bias and Statistical Asymmetry}

    Assume that admissible projected configurations exhibit a slight asymmetry in
    relaxation efficiency with respect to the sign of a structural invariant \(Q\).
    Let \(\Gamma(Q)\) denote the effective stabilization rate.

    If
    \[
      \Gamma(Q) \neq \Gamma(-Q),
    \]
    then configurations carrying one orientation are statistically favored during
    relaxation.
    Matter–antimatter asymmetry may therefore emerge as a dynamical bias in the
    projective selection process, without requiring explicit symmetry breaking
    at the fundamental level.

    Within this perspective, CP violation is interpreted as a differential
    relaxation effect induced by the non-injective character of the projection
    \(\Pi\).

  \subsubsection*{Spectral Generations and the $n=3$ Threshold}

    The Standard Model exhibits CP violation only when at least three generations
    of fermions are present.
    This structural constraint can be reformulated in spectral terms.

    Let \(\{\lambda_i\}\) denote the spectral eigenvalues associated with
    three effective relaxation modes of the fiber \(\Pi\),
    interpreted as the projected mass parameters of the flavor generations.
    These eigenvalues determine the relaxation energy of the corresponding projected configurations,
    and are therefore directly related to the inertial masses defined in Eq.~(\ref{eq:mass_definition}).
    If two eigenvalues become degenerate,
    \[
      \lambda_i = \lambda_j,
    \]
    the corresponding subspace becomes rotationally free,
    and any accumulated phase can be reabsorbed by a redefinition of the basis.
    No irreducible CP-odd invariant remains.

    The emergence of CP violation therefore requires
    \(n \geq 3\) spectrally distinct modes.
    This threshold signals that the projective holonomy of the flavor sector
    can no longer be embedded in a two-dimensional manifold.
    The appearance of an irreducible phase is thus linked to the minimal
    spectral complexity of the fiber.

  \subsubsection*{Dual Spectral Hierarchies and Sectoral Misalignment}

    In the Standard Model, CP violation depends not only on spectral
    non-degeneracy within a single sector, but on the incompatibility
    between the up-type and down-type quark sectors.

    Within Cosmochrony, this structure is interpreted as the
    non-alignment of two sectorial restrictions of the global
    non-injective mapping \(\Pi\).
    Let \(\Pi_u\) and \(\Pi_d\) denote the effective operators encoding
    the extraction of up-type and down-type relaxation modes from
    the same relational substrate \(\chi\).

    Because the descriptive capacity of the projected state \(U\) is finite
    (the bandwidth constraint \(b_\chi\)),
    these two extraction channels cannot in general be simultaneously
    diagonalized.
    They compete for the same relational anchors.
    This operational competition induces a non-vanishing commutator
    \[
      [\Pi_u,\Pi_d] \neq 0,
    \]
    representing the irreducible mismatch between the two flavor sectors.

  \subsubsection*{Cubic Commutator Invariant and CP Violation}

    Following the algebraic structure of the CP-violating invariant
    in the Standard Model, the physically relevant quantity is identified
    with the imaginary part of the trace of the cubic commutator,
    \begin{equation}
      \mathcal{J}
      \propto
      \mathrm{Im}\,\mathrm{Tr}\!\left(
                                  [\Pi_u,\Pi_d]^3
      \right).
    \end{equation}

    The cubic order is the minimal one that vanishes identically for
    two-dimensional sectors and becomes non-trivial only when
    three independent generations are present.
    If the two sectorial operators commute,
    \(
    [\Pi_u,\Pi_d]=0,
    \)
    then \(\mathcal{J}=0\) and CP symmetry is effectively restored,
    even in the presence of mass hierarchies.

    When combined with the spectral separations
    \(
    \lambda_{u,i}, \lambda_{d,i},
    \)
    the CP-odd curvature scales structurally as
    \[
      \mathcal{J}
      \propto
      \mathrm{Im}\,\mathrm{Tr}([\Pi_u,\Pi_d]^3)
      \prod_{i<j}
      (\lambda_{u,i}^2-\lambda_{u,j}^2)
      \prod_{k<l}
      (\lambda_{d,k}^2-\lambda_{d,l}^2).
    \]

    Degeneracy in either sector suppresses the invariant,
    recovering the structural constraint known from the
    Jarlskog formulation.

  \subsubsection*{Status and Open Structural Question}

    The present construction provides a structural reconstruction
    of the CP-violating sector rather than a full derivation
    from first principles.
    The non-vanishing of the commutator
    \(
    [\Pi_u,\Pi_d]
    \)
    is linked to the finite projective capacity \(b_\chi\),
    but a complete proof that this saturation necessarily enforces
    sectoral misalignment requires an explicit characterization
    of the operator space of the fiber.

    Within Cosmochrony, CP violation is therefore not treated
    as an arbitrary complex parameter,
    but as a geometric residue of incompatible spectral extractions
    from a single relational substrate.
    It represents the first observable signal of the topological
    constraints induced by finite projective bandwidth.

    These operators act within the same admissible fiber
    \(\Pi \cong S^3\) introduced in Section~\ref{sec:geometry-pi},
    but correspond to distinct spectral restrictions of its
    projectable subspace.
