% ----------------------------------------------------------------------------
% A.9 --- Energy and Curvature
% From former A09, condensed
% ----------------------------------------------------------------------------
\subsection{Energy and Curvature}
\label{subsec:energy-and-curvature}

Energy emerges as a diagnostic of constrained relaxation.
At the effective level:
\[
  \mathcal{E}_\chi^{\mathrm{eff}}
  = \tfrac{1}{2}
    \bigl[
      (\partial_t\chi)^2 + (\nabla\chi)^2
    \bigr].
\]
This functional has no Hamiltonian or variational status.
Regions of large $\mathcal{E}_\chi^{\mathrm{eff}}$ correspond to
strong internal gradients and are identified with particle-like
excitations.

The apparent ordering of atomic orbitals by increasing energy reflects
the structural cost of sustaining extended oscillatory configurations.
An outer orbital constrains relaxation over a wider region, increasing
total resistance independently of the geometric visibility discussed
in Appendix~\ref{app:level_sets_orbitals}.

Curvature characterizes internal deformation of projected $\chi$
configurations and how they modulate the propagation of relaxation.
Effective spacetime curvature arises secondarily as a macroscopic
descriptor.
Stable solitonic configurations emerge when nonlinear self-interaction
balances dispersive gradients, providing a dynamical origin for
long-lived particle-like excitations.
