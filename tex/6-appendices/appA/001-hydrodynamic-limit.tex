\subsection{Effective Lagrangian Description as a Hydrodynamic Limit}
  \label{subsec:hydrodynamic-limit}

  In regimes where $\chi$ varies smoothly, discrete relational couplings
  $K_{ij}$ can be summarized by effective continuum quantities.
  Distances emerge as cumulative resistance to relaxation.
  The effective metric $g_{\mu\nu}$ encodes the coarse-grained density of correlations.

  To reproduce the continuum evolution from the discrete rule
  (Eq.~\ref{eq:discrete-dynamics}), one introduces a purely
  representational effective Lagrangian:
  \[
    \mathcal{L}_{\text{eff}}
    = \frac{1}{16\pi G_{\text{eff}}}\,F(\chi)\,R
    - \Lambda_{\text{flow}}^4\,\chi + \cdots
  \]
  where $R$ is the Ricci scalar of the effective metric and $F(\chi)$
  parametrizes how relaxation dynamics is encoded in the geometric description.
  Einstein-like field equations reflect the universality of geometric
  descriptions for slowly varying collective phenomena.
  Singularities signal only the limits of the hydrodynamic approximation.
  The effective Lagrangian has no ontological status and should not be quantized.

  In projective regimes where the mapping $\Pi$ is effectively injective,
  the existence of $S_{\text{eff}}=\int \mathcal{L}_{\text{eff}}\,\mathrm{d}^4x$
  implies additional structural constraints.
  Admissible variations of underlying $\chi$ configurations that preserve
  projectability map to variations of the effective fields that leave
  $S_{\text{eff}}$ invariant up to boundary terms.
  As a consequence, the effective description admits local continuity identities of Noether type.
  In particular, when $\mathcal{L}_{\text{eff}}$ is generally covariant
  with respect to the emergent metric $g_{\mu\nu}$, the associated effective stress tensor
  \[
    T^{\mu\nu}_{\text{eff}}
    := \frac{2}{\sqrt{-g}}\,
    \frac{\delta S_{\text{eff}}}{\delta g_{\mu\nu}}
  \]
  satisfies the local identity
  \[
    \nabla_\mu T^{\mu\nu}_{\text{eff}} = 0
  \]
  within the hydrodynamic domain.

  From this perspective, local conservation is not a primitive postulate
  of the $\chi$ description but a stability condition of the projective regime.
  If relaxation approaches saturation and effective injectivity fails,
  the variational formulation may cease to be well-posed,
  and the above local identities need not remain strictly defined
  at the boundary of the saturated region.
  Global admissibility of the underlying $\chi$ dynamics is nonetheless preserved,
  and apparent non-local redistribution in the effective description
  is interpreted as deprojection followed by reprojection
  once projectability conditions are restored.
