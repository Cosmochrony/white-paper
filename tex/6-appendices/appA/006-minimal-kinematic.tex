% ----------------------------------------------------------------------------
% A.6 --- Minimal Kinematic Constraint
% From former A06, condensed
% ----------------------------------------------------------------------------
\subsection{Minimal Kinematic Constraint}
\label{subsec:minimal-kinematic-constraint}

In its saturated form:
\begin{equation}
  (\partial_t\chi)^2 + |\nabla\chi|^2 = c^2,
  \label{eq:minimal-kinematic-constraint}
\end{equation}
where $\partial_t$ and $\nabla$ denote effective variations introduced
only once a projectable geometric regime is established.
More generally, admissible configurations satisfy the corresponding
inequality.

This constraint is imposed ab initio at the pre-geometric level and
does not presuppose a spacetime metric, light cones, or Lorentzian
structure.
The constant $c$ is the effective manifestation of the invariant
structural bound $c_\chi$
(Section~\ref{subsec:role-of-cchi}).
Lorentz symmetry arises a~posteriori as a property of saturated
relaxation.

In homogeneous regimes, the constraint enforces
$\partial_t\chi = c$, leading to linear growth and effective cosmic
expansion.
In inhomogeneous regimes, partial saturation manifests as local
slowdown underlying gravitational time dilation.
At scales near the Planck length, the continuum approximation breaks
down and a fully discrete formulation is required.
