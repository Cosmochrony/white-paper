% ----------------------------------------------------------------------------
% A.2 --- Stability Analysis of the chi-Field Dynamics
% From former A02, condensed
% ----------------------------------------------------------------------------
\subsection{Stability Analysis of the
\texorpdfstring{$\chi$}{χ}-Field Dynamics}
\label{subsec:stability-analysis}

In the hydrodynamic regime, the effective relaxation dynamics reads
$\partial_t\chi
  = c\sqrt{1 - |\nabla\chi|^2/c^2}$.

Around a homogeneous background $\chi_0(t) = ct + \chi_{0,0}$, a
perturbation $\delta\chi$ satisfies
$\partial_t\delta\chi
  = -(2c)^{-1}|\nabla\delta\chi|^2
    + \mathcal{O}(|\nabla\delta\chi|^4)$.
No linear term appears: the homogeneous solution is marginally stable
at linear order.
The leading nonlinear correction is strictly negative, so any spatial
inhomogeneity reduces the local relaxation rate and is dynamically
suppressed.

The diagnostic functional
$E[\delta\chi]
  = \tfrac{1}{2}\int|\nabla\delta\chi|^2\,d^3x$
is non-increasing.
Planar perturbations are progressively flattened; spherical ones decay
monotonically.
The effective dynamics is dissipative and contractive, establishing
nonlinear stability.
This stability is inseparable from the monotonic character of
relaxation: the same constraint defining the arrow of time precludes
dynamical instabilities.
