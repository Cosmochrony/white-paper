\section{Step A: Admissible Injective Projections Have Trivial Holonomy}
  \label{sec:stepA-injective-holonomy}

  This appendix formalizes Step~A announced in Section~\ref{subsec:necessity-open}.
  We classify admissible projection maps $\Pi : \mathcal{C}_\chi \to \mathcal{M}$ by their fiber topology.
  We show that injective maps in this class have trivial holonomy.
  Since non-trivial holonomy is the minimal condition for a re-identification symmetry to arise as a descriptive invariance,
  injectivity structurally precludes emergent gravity under Hypothesis~H and ontological uniqueness.

  \subsection{Setup: admissible projections as fibered structures}
    \label{subsec:stepA-setup}

    We work in the regime where $\mathcal{C}_\chi$ and $\mathcal{M}$ admit smooth structures
    (the continuum limit of Section~E.1 and Appendix~D.5).
    In this regime, the projection $\Pi : \mathcal{C}_\chi \to \mathcal{M}$ is required to be \emph{admissible}.

    \begin{definition}[Admissible projection]
      \label{def:admissible-projection}
      A smooth surjective map $\Pi : \mathcal{C}_\chi \to \mathcal{M}$ is called admissible if:
      \begin{enumerate}[label=(A\arabic*)]
  \item \label{A1}
  $\Pi$ is a fiber bundle map.
  For each $y \in \mathcal{M}$, the preimage $\Pi^{-1}(y) \subset \mathcal{C}_\chi$ is a smooth submanifold.
  Fibers are mutually diffeomorphic.
  \item \label{A2}
  $\Pi$ is compatible with the relaxation rule $F$.
  If $x, x' \in \mathcal{C}_\chi$ satisfy $\Pi(x) = \Pi(x')$, then $\Pi(F(x)) = \Pi(F(x'))$.
  Effective dynamics is therefore well-defined on $\mathcal{M}$.
  \item \label{A3}
  $\Pi$ supports a connection.
  There exists a smooth horizontal distribution $H \subset T\mathcal{C}_\chi$ complementary to the vertical,
  with the usual equivariance properties whenever a fiber group action is present.
      \end{enumerate}
    \end{definition}

    Condition~\ref{A1} places $\Pi$ in the category of fiber bundles over $\mathcal{M}$.
    When $\Pi$ is a \emph{principal} $G_\Pi$-bundle, $G_\Pi$ acts freely and transitively on each fiber,
    and fibers are diffeomorphic to $G_\Pi$.
    In that principal case, non-injectivity is equivalent to $G_\Pi$ being non-trivial.
    We will use this principal specialization whenever holonomy is interpreted as a fiber re-identification.

  \subsection{Holonomy as the obstruction to global re-identification}
    \label{subsec:stepA-holonomy}

    Let $\Pi$ be an admissible projection equipped with a connection.
    In the principal case, the connection is a $\mathfrak{g}_\Pi$-valued one-form $\omega$ on $\mathcal{C}_\chi$.
    The holonomy group $\mathrm{Hol}(\omega, x_0)$ at $x_0 \in \mathcal{C}_\chi$ is the subgroup of $G_\Pi$
    generated by parallel transport along loops $\gamma$ in $\mathcal{M}$ based at $y_0 = \Pi(x_0)$:
    \begin{equation}
      \mathrm{Hol}(\omega, x_0)
      =
      \bigl\{ P_\gamma \in G_\Pi \;\big|\;
      \gamma : [0,1] \to \mathcal{M},
      \;
      \gamma(0) = \gamma(1) = y_0
      \bigr\}
      \subseteq G_\Pi.
    \end{equation}

    Holonomy measures the obstruction to globally consistent re-identification of fibers.
    It detects the failure of horizontal lifts of loops in $\mathcal{M}$ to close in $\mathcal{C}_\chi$.

    \begin{definition}[Re-identification symmetry]
      \label{def:re-id-symmetry}
      A \emph{re-identification symmetry} of the projection $\Pi$ is a transformation
      $\phi : \mathcal{C}_\chi \to \mathcal{C}_\chi$ that:
      \begin{enumerate}[label=(\roman*)]
  \item preserves effective states: $\Pi(\phi(x)) = \Pi(x)$ for all $x$,
  \item is non-trivial: $\phi \neq \mathrm{id}$ on a non-empty open set,
  \item is locally implementable: $\phi$ acts fiberwise and continuously.
      \end{enumerate}
    \end{definition}

    In the principal case, re-identification symmetries correspond to bundle automorphisms acting along fibers.
    Their global consistency is constrained by holonomy.

    \begin{proposition}[Holonomy controls global re-identification consistency]
      \label{prop:holonomy-gauge}
      Let $\Pi$ be an admissible projection with connection $\omega$.
      If $\mathrm{Hol}(\omega, x_0) = \{e\}$, the bundle admits a global flat section and
      re-identification symmetries are globally trivializable.
      If $\mathrm{Hol}(\omega, x_0)$ is a positive-dimensional Lie subgroup of $G_\Pi$,
      re-identification symmetries are intrinsic and cannot be globally trivialized.
      The converse, flat connection implies trivial holonomy, holds when $\mathcal{M}$ is simply connected,
      but not in general.
      The argument below does not rely on this converse.
    \end{proposition}

    \begin{proof}
      Standard result in differential geometry (Ambrose\textendash Singer theorem and its converse):
      the holonomy algebra $\mathfrak{hol}(\omega, x_0)$ equals the Lie algebra generated by
      curvature values $\Omega(u,v)$ for all horizontal vectors $u, v$ at points reachable
      from $x_0$.

      A flat connection ($\Omega = 0$) has holonomy group contained in a discrete subgroup of $G_\Pi$
      when $\mathcal{M}$ is simply connected.
      In the general case, the holonomy of a flat connection is a representation of
      $\pi_1(\mathcal{M})$ into $G_\Pi$, which may be non-trivial.

      However, under injectivity $G_\Pi = \{e\}$ (Theorem~\ref{thm:injective-trivial-holonomy}),
      so the holonomy group is trivial regardless of the topology of $\mathcal{M}$.
      The flat connection argument is therefore not the primary route here.
      It is superseded by the triviality of $G_\Pi$ established directly from injectivity.
    \end{proof}

  \subsection{The triviality theorem for injective admissible projections}
    \label{subsec:stepA-triviality}

    \begin{theorem}[Injective admissible projections have trivial holonomy]
      \label{thm:injective-trivial-holonomy}
      Let $\Pi : \mathcal{C}_\chi \to \mathcal{M}$ be an admissible projection in the sense of
      Definition~\ref{def:admissible-projection}.
      In the principal bundle case, where $G_\Pi$ acts freely and transitively on fibers,
      non-injectivity of $\Pi$ corresponds to $G_\Pi$ being non-trivial.
      For general fiber bundles where the fiber is not identified with the structure group,
      non-injectivity is the weaker condition $|\Pi^{-1}(y)| > 1$.
      The structure group may still be non-trivial even when fibers are discrete.
      The arguments below apply to the principal bundle case, which is the relevant one
      for Cosmochrony (Section~5.1, $\Pi \cong S^3$ with
      $G_\Pi \supseteq \mathrm{SU}(2) \times \mathrm{U}(1)$).
      If $\Pi$ is injective, then:
      \begin{enumerate}[label=(\alph*)]
  \item \label{thm-a}
  the fiber over every $y \in \mathcal{M}$ is a single point: $\Pi^{-1}(y) = \{x_y\}$,
  \item \label{thm-b}
  in the principal specialization, the structure group is trivial: $G_\Pi = \{e\}$,
  \item \label{thm-c}
  in that principal case, the holonomy group is trivial:
  $\mathrm{Hol}(\omega, x_0) = \{e\}$ for any connection $\omega$ and any basepoint $x_0$,
  \item \label{thm-d}
  no non-trivial re-identification symmetry exists in the sense of Definition~\ref{def:re-id-symmetry}.
      \end{enumerate}
    \end{theorem}

    \begin{proof}
      Part~\ref{thm-a} is immediate from injectivity.
      If $\Pi(x) = \Pi(x') = y$, then $x = x'$, hence $|\Pi^{-1}(y)| = 1$ for all $y$.

      Part~\ref{thm-b} is a principal-bundle statement.
      If fibers are singletons, the only group acting freely and transitively on a one-element set is $\{e\}$.

      Part~\ref{thm-c} follows from $\mathrm{Hol}(\omega, x_0) \subseteq G_\Pi$.
      If $G_\Pi = \{e\}$, then $\mathrm{Hol}(\omega, x_0) = \{e\}$.

      Part~\ref{thm-d} follows directly from injectivity.
      If $\Pi(\phi(x)) = \Pi(x)$ for all $x$, injectivity implies $\phi(x) = x$ for all $x$.
      Hence $\phi = \mathrm{id}$, contradicting non-triviality.
    \end{proof}

  \subsection{From trivial holonomy to absence of emergent gauge structure}
    \label{subsec:stepA-gauge}

    \begin{corollary}[No emergent gauge field from injective projection]
      \label{cor:no-gauge}
      Under the hypotheses of Theorem~\ref{thm:injective-trivial-holonomy} in the principal specialization,
      any connection is trivial in the sense that its curvature vanishes.
      No non-trivial gauge field can emerge from the projection structure alone.
    \end{corollary}

    \begin{proof}
      With $G_\Pi = \{e\}$, the Lie algebra is $\{0\}$.
      Any connection is therefore $\omega = 0$, and $\Omega = d\omega + \omega \wedge \omega = 0$.
      A non-trivial gauge field would require an independent postulate at the level of $\mathcal{M}$,
      violating ontological uniqueness.
    \end{proof}

    \begin{corollary}[No emergent diffeomorphism invariance from injective projection]
      \label{cor:no-diffeo}
      Under the hypotheses of Theorem~\ref{thm:injective-trivial-holonomy} and Hypothesis~H,
      no diffeomorphism invariance as a descriptive invariance can arise from $\Pi$ alone.
    \end{corollary}

    \begin{proof}
      Diffeomorphism invariance as a descriptive invariance (D2) requires multiple representatives of the same effective state.
      That is, it requires $|\Pi^{-1}(y)|>1$ on some region of $\mathcal{M}$.
      Under injectivity, $|\Pi^{-1}(y)| = 1$ everywhere.
      Therefore, the diffeomorphism group of $\mathcal{M}$ cannot arise as a re-identification symmetry induced by $\Pi$.
      Any such invariance would have to be postulated directly at the level of $\mathcal{M}$,
      contradicting emergence and ontological uniqueness.
    \end{proof}

  \subsection{Closing the gap: from holonomy to gravitational curvature}
    \label{subsec:stepA-curvature}

    The preceding results establish that injective projections cannot produce the structural prerequisites of emergent gravity,
    namely re-identification symmetry and descriptive invariance as consequences of the projection.
    We now connect fiber curvature to effective base curvature in the standard bundle-induced manner.

    In the Cosmochrony framework, the effective metric on $\mathcal{M}$ is induced by the Hessian of the effective Lagrangian
    (Section~A.12, Eq.~(117)):
    \begin{equation}
      g^\mathrm{eff}_{\mu\nu}
      \propto
      \frac{\partial^2 \mathcal{L}_\mathrm{eff}}
      {\partial(\partial_\mu \chi)\,\partial(\partial_\nu \chi)},
      \label{eq:eff-metric-hessian}
    \end{equation}
    valid in projectable regimes.

    \begin{proposition}[Fiber curvature contributes to base curvature]
      \label{prop:fiber-base-curvature}
      Let $\Pi : \mathcal{C}_\chi \to \mathcal{M}$ be an admissible principal $G_\Pi$-bundle with connection $\omega$
      and curvature $\Omega$.
      If $g^\mathrm{eff}$ is induced from the bundle geometry via Eq.~\eqref{eq:eff-metric-hessian},
      then the projection-induced contribution to the effective Ricci tensor contains a term of the form
      \begin{equation}
        R^\mathrm{eff}_{\mu\nu}
        \supset
        \mathrm{tr}_{\mathfrak{g}_\Pi}\!\bigl(\Omega_{\mu\alpha}\,\Omega_\nu{}^\alpha\bigr) + \ldots,
        \label{eq:fiber-to-base-curvature}
      \end{equation}
      where the trace runs over $\mathfrak{g}_\Pi$ and the ellipsis denotes additional terms fixed by the same induction scheme.
      In particular, if $G_\Pi = \{e\}$, then $\Omega = 0$ and the contribution~\eqref{eq:fiber-to-base-curvature} vanishes.
    \end{proposition}

    \begin{proof}
      In a principal bundle with non-flat connection, $\Omega$ is a horizontal adjoint-valued 2-form.
      Bundle-induced reductions yield quadratic curvature contributions of Yang\textendash Mills type,
      which appear in the base Ricci tensor in the standard Kaluza\textendash Klein reduction.
      If $G_\Pi = \{e\}$, then $\mathfrak{g}_\Pi = \{0\}$ and $\Omega = 0$, so the term vanishes identically.
    \end{proof}

    \begin{remark}
      Equation~\eqref{eq:fiber-to-base-curvature} shows that a non-zero projection-induced curvature term requires $\Omega \neq 0$.
      In the principal specialization,
      \begin{equation}
        \Pi \text{ injective}
        \Rightarrow
        G_\Pi = \{e\}
        \Rightarrow
        \Omega = 0
        \Rightarrow
        \text{no fiber-induced curvature on } \mathcal{M}
        \Rightarrow
        R^\mathrm{eff}_{\mu\nu\rho\sigma}
        \text{ receives no projective sourcing.}
        \label{eq:injective-chain}
      \end{equation}
      This is the precise content needed for Step~A in Theorem~\ref{thm:necessity}.
    \end{remark}

  \subsection{The minimal non-injective case: $G_\Pi = \mathrm{U}(1)$}
    \label{subsec:stepA-minimal}

    As a consistency check and to connect to Section~5.1, we consider the minimal non-trivial case $G_\Pi = \mathrm{U}(1)$.
    A principal $\mathrm{U}(1)$-bundle over $\mathcal{M}$ with connection $\omega$ has:
    \begin{align}
      \mathrm{Hol}(\omega, x_0) &\subseteq \mathrm{U}(1), \\
      \Omega &= d\omega \in \Omega^2(\mathcal{M}, i\mathbb{R}),
    \end{align}
    where $\Omega$ is interpreted as the electromagnetic field strength in the Cosmochrony setting (Section~A.11).
    The first Chern class $c_1(\Pi) = [\Omega / 2\pi i] \in H^2(\mathcal{M}, \mathbb{Z})$ is non-trivial if and only if the
    bundle is topologically non-trivial.

    This matches the structure of Section~5.2.
    The electromagnetic $\mathrm{U}(1)$ gauge symmetry emerges because the projection fiber supports a non-trivial
    Hopf fibration $S^1 \hookrightarrow S^3 \to S^2$.

    In the injective limit, where the fiber degenerates to a point, the bundle trivializes and the $\mathrm{U}(1)$ gauge
    structure disappears, consistent with Eq.~\eqref{eq:injective-chain}.

  \subsection{Summary of Step A}
    \label{subsec:stepA-summary}

    The classification of admissible injective projections establishes the following hierarchy,
    completing Step~A of Theorem~\ref{thm:necessity}.

    \begin{center}
      \begin{tabular}{lll}
        \hline
        Fiber type & Structure group $G_\Pi$ & Emergent structure \\
        \hline
        Point ($|\Pi^{-1}(y)| = 1$, injective)
        & $G_\Pi = \{e\}$
        & No gauge, no projection-induced curvature, no re-identification symmetry \\
        Circle ($S^1$, minimal non-injective)
        & $G_\Pi = \mathrm{U}(1)$
        & $\mathrm{U}(1)$ gauge field, electromagnetic curvature \\
        Three-sphere ($S^3$, Cosmochrony setting)
        & $G_\Pi \supseteq \mathrm{SU}(2) \times \mathrm{U}(1)$
        & Gauge structure and projection-induced curvature via Prop.~\ref{prop:fiber-base-curvature} \\
        \hline
      \end{tabular}
    \end{center}

    The injective case is the unique case in which no re-identification symmetry and no projection-induced curvature can arise
    without an independent postulate.
    Non-injectivity is therefore not merely sufficient but necessary for emergent gauge and gravitational structure under
    Hypothesis~H and ontological uniqueness.
