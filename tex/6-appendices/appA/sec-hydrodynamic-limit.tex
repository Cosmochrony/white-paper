% ----------------------------------------------------------------------------
% A.1 --- Effective Lagrangian Description as a Hydrodynamic Limit
% From former A01, condensed
% ----------------------------------------------------------------------------
\subsection{Effective Lagrangian Description as a Hydrodynamic
Limit}
\label{subsec:hydrodynamic-limit}

In regimes where $\chi$ varies smoothly, discrete relational couplings
$K_{ij}$ can be summarized by effective continuum quantities.
Distances emerge as cumulative resistance to relaxation.
The effective metric $g_{\mu\nu}$ encodes the coarse-grained density
of correlations.

To reproduce the continuum evolution from the discrete rule
(Eq.~\ref{eq:discrete-dynamics}), one introduces a purely
representational effective Lagrangian:
\[
  \mathcal{L}_{\text{eff}}
  = \frac{1}{16\pi G_{\text{eff}}}\,F(\chi)\,R
    - \Lambda_{\text{flow}}^4\,\chi + \cdots
\]
where $R$ is the Ricci scalar of the effective metric and $F(\chi)$
parametrizes how relaxation dynamics is encoded in the geometric
description.
Einstein-like field equations reflect the universality of geometric
descriptions for slowly varying collective phenomena.
Singularities signal only the limits of the hydrodynamic
approximation.
The effective Lagrangian has no ontological status and should not be
quantized.
