% ----------------------------------------------------------------------------
% A.12 --- Relational Consistency of the Effective Lagrangian
% From former A12, condensed
% ----------------------------------------------------------------------------
\subsection{Relational Consistency of the Effective Lagrangian}
\label{sec:born-lagrangian_derivation}

\subsubsection*{Step 1: Relational Constraint}
\label{subsec:A12-relational-constraint_v171}

\begin{equation}
  \mathcal{C}_i[\chi]
  \equiv
  \sum_j K_{ij}(\chi_i - \chi_j)^2
  \le \chi_c^2.
  \label{eq:A12-relational-constraint_v171}
\end{equation}

\subsubsection*{Step 2: Variational Formulation}
\label{subsec:A12-variational-structure_v171}

Constrained action with KKT conditions:
\begin{equation}
  S[\{\chi_i\},\{\mu_i\}]
  = \int d\lambda
    \left[
      \sum_i \tfrac{1}{2}
        (d\chi_i/d\lambda)^2
      - U[\{\chi_i\}]
      - \sum_i \mu_i
        (\mathcal{C}_i - \chi_c^2)
    \right].
  \label{eq:A12-action_v171}
\end{equation}

\subsubsection*{Step 3: Continuum Limit and Canonical Form}
\label{subsec:A12-continuum-limit_v171}

In projectable regimes:
\begin{equation}
  |\nabla\chi|^2 \le c^2,
  \label{eq:A12-continuum-bound_v171}
\end{equation}
with $c^2 \equiv a^2\chi_c^2/K_0$.
The canonical representation satisfying boundedness, monotonicity, and
regularity:
\begin{equation}
  f(x) = -c^2\sqrt{1 - x},
  \label{eq:A12-born-infeld-derivation_v171}
\end{equation}
yielding the Born--Infeld-like Lagrangian:
\begin{equation}
  \mathcal{L}_{\text{eff}}
  = -c^2\sqrt{
      1 - \frac{|\nabla\chi|^2}{c^2}
    }
    + \partial_t\chi.
  \label{eq:A12-effective-lagrangian_v171}
\end{equation}

\subsubsection*{Step 4: Role of $U[\{\chi_i\}]$}
\label{subsec:A12-potential-role_v171}

In the continuum limit,
$U[\{\chi_i\}] \to \int d^3x\,V(\chi)$,
connected to the main text's $V(\chi)$ via coarse-graining.

\subsubsection*{Step 5: Connection to Emergent Geometry}
\label{subsec:A12-emergent-geometry_v171}

The effective metric is defined via the Hessian:
\begin{equation}
  g_{\mu\nu}^{\text{eff}}
  \propto
  \frac{\partial^2\mathcal{L}_{\text{eff}}}
       {\partial(\partial_\mu\chi)\,
        \partial(\partial_\nu\chi)},
  \label{eq:A12-emergent-metric_v171}
\end{equation}
valid in projectable regimes
(Section~\ref{subsec:numerical-validation-of-the-chi-rightarrow-chi_eff-transition}).

\subsubsection*{Continuum Limit: Laplace--Beltrami Operator}
\label{subsec:A12-continuum-limit}

In the dense limit ($N \to \infty$), the discrete graph Laplacian
converges to the Laplace--Beltrami operator on an emergent Riemannian
manifold.
No background geometry is assumed: $g_{ab}$ arises as the continuum
encoding of microscopic connectivity.

\subsubsection*{Necessity of the Born--Infeld Structure}
\label{subsec:A12-born-infeld-necessity}

A quadratic action permits unbounded gradients and instantaneous
propagation, contradicting the maximal relaxation speed $c_\chi$.
The Born--Infeld form is the minimal non-polynomial functional
enforcing strict saturation, causal consistency, and
self-regularization.
The saturation constant $b$ represents the substrate's upper
relaxation bound; the speed of light $c$ is a derived parameter with
$c \le b$.
