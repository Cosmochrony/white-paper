% ----------------------------------------------------------------------------
% C.5 --- Phenomenological Implications
% From former C05, condensed
% ----------------------------------------------------------------------------
\subsection{Phenomenological Implications}
\label{subsec:phenomenology}

\paragraph{Gravitational perturbation speed.}
Linearizing the kinematic constraint around
$\chi_0(t) = ct$ yields
\begin{equation}
  \left(
    \frac{1}{c^2}\partial_t^2 - \nabla^2
  \right)\delta\chi = 0,
  \label{eq:gw_wave}
\end{equation}
so gravitational perturbations propagate at exactly~$c$, consistent
with GW170817.

\paragraph{Emergent acceleration scale.}
The kinematic constraint
$({\partial_t \chi})^2 + |\nabla\chi|^2 = c^2$ enforces a residual
spatial gradient even in the absence of matter.
This defines an effective acceleration scale
$a_0(t) \sim c\,H(t)$.
At large radii the effective acceleration approaches
$g_{\mathrm{eff}} \simeq \sqrt{g_N\,a_0}$, recovering deep-MOND
scaling without interpolation functions or dark matter particles.

\paragraph{Gravitational lensing.}
Near a localized mass, an effective refractive index
$n(r) \simeq 1 + GM/(c^2 r)$ yields a deflection angle
$\alpha = 4GM/(bc^2)$, reproducing the GR prediction from the
nonlinear structure of~$\chi$ relaxation dynamics.
