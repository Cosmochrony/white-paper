% ----------------------------------------------------------------------------
% C.10 --- Cosmic Voids as Observational Tests
% From former C11, condensed
% ----------------------------------------------------------------------------
\subsection{Cosmic Voids as Observational Tests of Maximal Substrate
Relaxation}
\label{subsec:observational-tests-voids}

Voids probe the regime of near-maximal relaxation, where departures
from $\Lambda$CDM are most pronounced.
Observables are modeled as a $\Lambda$CDM baseline plus a
Born--Infeld-like saturating correction:
\begin{align}
  \kappa_{\mathrm{obs}}(R)
  &= \kappa_{\Lambda\mathrm{CDM}}(R)
     \left[
       1 + \beta_{\mathrm{void}}\,
         \mathcal{S}\!\big(\mathcal{A}(R)\big)
     \right], \\
  v_{\mathrm{obs}}(r)
  &= v_{\Lambda\mathrm{CDM}}(r)
     \left[
       1 + \beta_{\mathrm{void}}\,
         \mathcal{S}\!\big(\mathcal{B}(r)\big)
     \right],
\end{align}
with saturation function
$\mathcal{S}(x) = x/\sqrt{1+x^2}$.
Key predictions: more negative void lensing than $\Lambda$CDM with
non-linear saturation, enhanced peculiar velocity outflows at void
boundaries, and cross-consistency between lensing and velocities
through a single $\beta_{\mathrm{void}}$.
Enhanced void outflows predict a correlation between negative
void-lensing strength, boundary outflows, and elevated local~$H_0$
estimates.
