% ----------------------------------------------------------------------------
% Section 11.3 --- Gravitational Wave Propagation
% From former §14.3, condensed
% ----------------------------------------------------------------------------
\subsection{Gravitational Wave Propagation}
\label{subsec:gravitational-wave-propagation}

In regions of high excitation density near compact objects, local
suppression of~$\chi$ relaxation modifies the persistence and coherence
of gravitational-wave projective descriptions, inducing
frequency-dependent phase shifts or dispersion-like behavior rather than
dissipative losses.

\textbf{LISA signature}: a $\sim 10\%$ effective reduction of coherent
gravitational-wave amplitudes near black holes, distinct from general
relativity's purely propagative behavior.

The relative amplitude reduction scales as
\[
  \frac{\Delta A}{A}
    \sim \left(\frac{r_s}{r}\right)^2,
\]
yielding $\Delta A / A \sim 10^{-2}$ for propagation at
$r \approx 10\,r_s$.
This effect should manifest most clearly during the late-time ringdown
phase of binary black hole mergers, appearing as frequency-dependent
deviations from general relativistic ringdown templates.
Future space-based observatories (LISA) with signal-to-noise ratios
exceeding~$\sim 100$ may provide sufficient sensitivity.
