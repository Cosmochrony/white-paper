\subsection{Flavor Structure and CP Violation}
  \label{subsec:flavor-structure-cp}

  The spectral interpretation of mass developed in the preceding sections
  naturally extends to the structure of flavor.
  In the present framework, distinct fermionic generations correspond
  to spectrally separated relaxation modes of the same projection fiber~\(\Pi\),
  whose geometry was introduced in Section~\ref{sec:geometry-pi}.
  The associated eigenvalues determine the relaxation energy of the
  projected configurations and are therefore directly related to the
  inertial masses defined in Eq.~(\ref{eq:mass_definition}).

  \paragraph{Spectral non-degeneracy and the $n=3$ threshold.}
    Let \(\{\lambda_i\}\) denote the spectral eigenvalues governing
    three distinct relaxation modes.
    If two eigenvalues become degenerate,
    \[
      \lambda_i = \lambda_j,
    \]
    the corresponding subspace admits a rotational freedom,
    allowing any accumulated phase to be reabsorbed by a redefinition
    of the projected basis.
    No irreducible CP-odd invariant remains.

    The existence of a non-trivial CP-violating sector therefore requires
    at least three spectrally distinct modes.
    The threshold \(n \ge 3\) signals that the projective holonomy of
    the flavor sector can no longer be embedded in a two-dimensional
    manifold.
    An irreducible geometric phase becomes unavoidable once the
    spectral complexity of the fiber exceeds this minimal bound.

  \paragraph{Dual spectral hierarchies.}
    In the Standard Model, CP violation depends on the incompatibility
    between the up-type and down-type quark sectors.
    Within Cosmochrony, this structure is interpreted as the non-alignment
    of two sectorial spectral restrictions of the same projection fiber.

    Let \(\Pi_u\) and \(\Pi_d\) denote the effective operators
    selecting the up-type and down-type relaxation modes from
    the admissible spectral subspace of \(\Pi \cong S^3\).
    These operators do not define independent geometries;
    they act within the same projection fiber but correspond to
    distinct spectral extractions.

    Because the projected state~\(U\) has finite descriptive capacity,
    characterized by the bandwidth constraint~\(b_\chi\),
    the two extraction channels cannot in general be simultaneously
    diagonalized.
    They compete for the same relational degrees of freedom.
    This induces a structural non-commutativity,
    \[
      [\Pi_u,\Pi_d] \neq 0,
    \]
    reflecting the operational mismatch between the two flavor sectors.

  \paragraph{Cubic commutator invariant.}
    The physically relevant CP-odd quantity is identified with
    the imaginary part of the trace of the cubic commutator,
    \begin{equation}
      \mathcal{J}
      \propto
      \mathrm{Im}\,\mathrm{Tr}
      \bigl(
      [\Pi_u,\Pi_d]^3
      \bigr).
    \end{equation}

    The cubic order is minimal for which the invariant vanishes
    identically in a two-generation system and becomes non-trivial
    only when three independent generations are present.
    If the sectorial operators commute,
    \(
    [\Pi_u,\Pi_d]=0,
    \)
    then \(\mathcal{J}=0\) and CP symmetry is effectively restored,
    even in the presence of mass hierarchies.

    When combined with the spectral separations
    \(\lambda_{u,i}\) and \(\lambda_{d,i}\),
    the CP-odd invariant scales structurally as
    \[
      \mathcal{J}
      \propto
      \mathrm{Im}\,\mathrm{Tr}([\Pi_u,\Pi_d]^3)
      \prod_{i<j}
      (\lambda_{u,i}^2-\lambda_{u,j}^2)
      \prod_{k<l}
      (\lambda_{d,k}^2-\lambda_{d,l}^2).
    \]
    Degeneracy in either sector suppresses the invariant,
    recovering the structural constraint known from the
    Jarlskog formulation of the Standard Model.

  \paragraph{Structural interpretation.}
    Within Cosmochrony, CP violation is not introduced as an
    independent complex phase.
    It emerges as a geometric residue of incompatible spectral
    extractions from a single relational substrate.
    The non-commutativity of the sectorial operators reflects
    the finite projective capacity of the fiber and the
    impossibility of simultaneously synchronizing distinct
    relaxation hierarchies within one global ordering.

    This construction provides a structural reconstruction
    of the CP-violating sector.
    A full derivation of the numerical value of the invariant
    would require an explicit characterization of the operator
    space of the projection fiber and of the constraints imposed
    by the bandwidth parameter~\(b_\chi\).
    Nevertheless, the emergence of a non-trivial cubic invariant
    at the three-generation threshold offers a geometric rationale
    for the observed structure of flavor mixing.

    The non-commutativity considered here differs from the
    torsion–relaxation mismatch discussed in
    Section~\ref{sec:noncommutativity-mass}.
    There, the commutator operates within a single
    topological excitation and manifests as inertial mass.
    Here, it arises between distinct sectorial spectral
    restrictions of the same projection fiber and
    manifests as a CP-odd geometric phase.
    Both effects reflect different levels of the same
    finite projective capacity constraint.
