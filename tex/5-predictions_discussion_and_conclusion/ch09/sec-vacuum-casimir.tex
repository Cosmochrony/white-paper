% ----------------------------------------------------------------------------
% Section 9.4 --- Vacuum Fluctuations and the Casimir Effect
% From former §13.4, condensed
% ----------------------------------------------------------------------------
\subsection{Vacuum Fluctuations and the Casimir Effect}
\label{subsec:vacuum-fluctuations-and-the-casimir-effect}

Vacuum fluctuations reflect the intrinsic structural indeterminacy
of~$\chi$ in regimes where no stable localized excitations are present.
The relaxation admits a wide range of locally compatible projective
descriptions; these fluctuations represent variability of effective
descriptions rather than physical energy stored in the vacuum.

When material boundaries impose structural constraints on local
projectability, certain effective descriptions become incompatible with
the boundary conditions, reducing the set of admissible projective
configurations between the boundaries.
The Casimir effect arises from this asymmetry: a difference in the
density of admissible effective reprojections, manifesting as a pressure
on the confining surfaces.
No fundamental vacuum energy density or propagating vacuum modes are
required.
