\subsection{Neutrino-Mediated Relaxation and Decay Signatures}
\label{subsec:neutrino-decay-signatures}

Particle decay and neutrino emission are manifestations of structural
reorganization
(Sections~\ref{subsec:metastability-and-decay},~\ref{subsec:neutrinos-partially-projectable-modes}).
Neutrino-like excitations act as non-local relaxation channels,
contributing to irreversible smoothing of admissible configurations.

\subsubsection*{Environmental Modulation of Particle Stability}
\label{subsec:environmental-decay-modulation}

\subsection{Environmental Modulation of Particle Lifetimes}
\label{subsec:environmental-modulation-lifetimes}

Particle stability may exhibit a weak environmental dependence.
At leading order:
\begin{equation}
  \frac{\delta \Gamma}{\Gamma}
  \;\simeq\;
  \beta \, \frac{\Delta U}{c^{2}},
  \label{eq:lifetime-modulation}
\end{equation}
where $\beta \lesssim 10^{-6}$ (from local position invariance
constraints) and
$\Delta U / c^{2} \sim 10^{-7}$--$10^{-6}$ for typical galactic
environments, yielding
\begin{equation}
  \frac{\delta \tau}{\tau}
  \;\sim\;
  10^{-13} \text{--} 10^{-12},
  \label{eq:lifetime-order-of-magnitude}
\end{equation}
well below current sensitivities but conceptually distinct from standard
effects.
The cleanest experimental target is leptonic weak decay; the most
amplified interferometric target is neutral-meson mixing.
Near compact objects
($\Delta U/c^2 \sim 10^{-4}$), the effect rises to
$\delta\tau/\tau \sim 10^{-10}$, manifesting as environment-correlated
spectral biases in hadronic cascades.
