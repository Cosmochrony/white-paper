\subsection{Perspectives: Towards a Derivation of the Mass Spectrum}
\label{subsec:perspectives_mass_spectrum}

While the identification of particles as topological solitons (Skyrmions, vortices) provides a qualitative mechanism for mass generation via the configuration energy of the $\chi$ field, the explicit derivation of the Standard Model mass spectrum—specifically the hierarchy of lepton and quark flavors—remains an open challenge. In the Cosmochrony framework, this spectrum should not be tuned by arbitrary coupling constants but should emerge from the intrinsic geometry of the network $G(V,E)$.

\subsubsection{The Geometric Resonance Hypothesis}
  We conjecture that observed masses correspond to the eigenvalues of a transfer operator on the discrete network, where the mass $m_n$ of a configuration $n$ follows a scaling law linked to the local curvature induced by the soliton:
  \begin{equation}
    m_n c^2 \approx E_{\text{fund}} \cdot \Lambda(Q_n, \mathcal{K})
  \end{equation}
  where $E_{\text{fund}}$ is a fundamental energy scale (potentially linked to the Planck scale or the global relaxation density), $Q_n$ is the topological charge (winding number), and $\mathcal{K}$ represents a curvature invariant of the network.

\subsubsection{Future Research Program}
  Transitioning toward a predictive theory of the mass spectrum requires:
  \begin{enumerate}
    \item \textbf{Discretization of the Potential $V(\chi)$}: Demonstrating how the minima of the potential on the network favor specific mass scales over others.
    \item \textbf{Network Eigenmode Analysis}: Investigating whether the flavor hierarchy (the three generations of particles) can be interpreted as higher-order harmonic modes of a single fundamental topological structure.
    \item \textbf{Numerical Simulations}: Implementing relaxation algorithms on large-scale graphs to verify if stable configurations spontaneously emerge with mass ratios corresponding to physical constants (e.g., the proton-to-electron mass ratio).
  \end{enumerate}

  This approach aims to transform the ``magic numbers'' of the Standard Model into geometric properties derivable from
  the first principles of relaxation dynamics.

\subsubsection{Spectral Relaxation Operators and Multi-Level Structure}
\label{subsec:spectral-relaxation}

    Following the conceptual separation introduced in Section~\ref{subsec:conceptual-implications-and-open-challenges},
    this appendix outlines an explicit but exploratory specification of the discrete relaxation operator underlying the
    spectral interpretation of particle masses in Cosmochrony.
    Rather than proposing a finalized microscopic model, the constructions presented here are intended
    to clarify how this separation may be implemented in a concrete and computationally tractable way,
    while preserving the non-circular distinction between fundamental geometry, emergent spacetime, and
    dynamical interactions.

\subsubsubsection{*Fundamental spectral operator}

  Let $G=(V,E)$ be a discrete network encoding the intrinsic connectivity of the pre-geometric substrate.
  The fundamental relaxation operator is defined as a weighted graph Laplacian
  \begin{equation}
  (\Delta_G^{(0)} \psi)_i = \sum_{j\sim i} w_{ij}^{(0)} (\psi_i - \psi_j),
  \end{equation}
  where the weights $w_{ij}^{(0)} = 1/K_{ij}^{(0)}$ encode the intrinsic compliance of the relaxation network.
  At this level, $K_{ij}^{(0)}$ are fixed coefficients determined by network topology and symmetry
  constraints, and do not depend on the dynamical state of the $\chi$ field.

  The eigenvalue problem
  \begin{equation}
    \Delta_G^{(0)} \psi_n = \lambda_n \psi_n
  \end{equation}
  defines a discrete spectrum.
  Within the Cosmochrony framework, particle masses are conjectured to scale as
  \begin{equation}
    m_n c^2 \propto \sqrt{\lambda_n},
  \end{equation}
  so that mass hierarchies emerge as geometric properties of the relaxation network rather than as
  parameters encoded in a fundamental potential.

\subsubsubsection{*Emergent geometric level}

  On larger scales, coarse-grained configurations of the $\chi$ field define an effective geometric
  description.
  This level governs gravitational phenomena, time dilation, and cosmological expansion through the
  local and global rates of $\chi$ relaxation.
  Importantly, while this emergent geometry influences the propagation of excitations, it does not
  redefine the fundamental spectral operator $\Delta_G^{(0)}$.

\subsubsubsection{*Dynamical and interaction level}

  Fast processes such as radiation, scattering, and decoherence correspond to interaction-induced
  redistributions of relaxation potential within the $\chi$ field.
  At this effective level, it is natural to consider corrections to the relaxation dynamics encoded
  by coefficients $K_{ij}(\chi)$ that depend smoothly on coarse-grained field variations.
  Such dependencies modify local dynamics and observable interaction rates but do not enter the
  definition of the fundamental spectral problem associated with mass generation.

\subsubsubsection{*Boundary conditions and numerical implementation}

  For numerical studies, the graph $G$ may be taken as large but finite, with periodic boundary
  conditions to minimize edge effects, or with Dirichlet or Neumann conditions to model confined
  regions.
  Standard sparse eigensolvers may then be used to compute the low-lying spectrum of $\Delta_G^{(0)}$
  and to test whether simple network connectivities can reproduce observed mass hierarchies.

  These constructions illustrate how Cosmochrony may be developed into a calculable spectral program,
  while preserving the non-circular separation between fundamental geometry, emergent spacetime, and
  dynamical interactions.
