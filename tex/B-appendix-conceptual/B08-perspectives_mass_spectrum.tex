\subsection{Perspectives: Towards a Derivation of the Mass Spectrum}
\label{subsec:perspectives_mass_spectrum}
  While the identification of particles as topological solitons provides a qualitative
  mechanism for inertial mass—understood as the resistance of localized $\chi$
  configurations to field relaxation—the explicit derivation of the Standard Model
  mass spectrum, in particular the hierarchy of lepton and quark flavors, remains an
  open challenge.

  In the Cosmochrony framework, this spectrum should not be fixed by arbitrary coupling
  constants, but should emerge from the intrinsic geometry of the relaxation network
  $G(V,E)$. The following sections outline a programmatic approach toward such a
  derivation.

  The conceptual interpretation of inertial mass underlying this programmatic
  approach is discussed in~\ref{subsec:spectral_mass}.

  \subsubsection{The Geometric Resonance Hypothesis}
  We conjecture that observed masses correspond to the eigenvalues of a transfer operator on the discrete network, where the mass $m_n$ of a configuration $n$ follows a scaling law linked to the local curvature induced by the soliton:
  \begin{equation}
    m_n c^2 \approx E_{\text{fund}} \cdot \Lambda(Q_n, \mathcal{K})
  \end{equation}
  where $E_{\text{fund}}$ is a fundamental energy scale (potentially linked to the Planck scale or the global relaxation density), $Q_n$ is the topological charge (winding number), and $\mathcal{K}$ represents a curvature invariant of the network.

\subsubsection{Future Research Program}
  Transitioning toward a predictive theory of the mass spectrum requires:
  \begin{enumerate}
    \item \textbf{Discretization of the Potential $V(\chi)$}: Demonstrating how the minima of the potential on the network favor specific mass scales over others.
    \item \textbf{Network Eigenmode Analysis}: Investigating whether the flavor hierarchy (the three generations of particles) can be interpreted as higher-order harmonic modes of a single fundamental topological structure.
    \item \textbf{Numerical Simulations}: Implementing relaxation algorithms on large-scale graphs to verify if stable configurations spontaneously emerge with mass ratios corresponding to physical constants (e.g., the proton-to-electron mass ratio).
  \end{enumerate}

  This approach aims to transform the ``magic numbers'' of the Standard Model into geometric properties derivable from
  the first principles of relaxation dynamics.

\subsubsection{Spectral Relaxation Operators and Multi-Level Structure}
\label{subsec:spectral-relaxation}

    This subsection introduces a spectral relaxation operator as a technical tool
    for quantifying the stability properties of localized configurations within the
    relaxation network.
    It does not redefine the physical origin of inertial mass, which remains the
    resistance of solitonic configurations to the relaxation of the $\chi$ field,
    but provides a mathematical framework for analyzing its spectral manifestation.

  Following the conceptual separation introduced in Section~\ref{subsec:conceptual-implications-and-open-challenges},
    this appendix outlines an explicit but exploratory specification of the discrete relaxation operator underlying the
    spectral interpretation of particle masses in Cosmochrony.
    Rather than proposing a finalized microscopic model, the constructions presented here are intended
    to clarify how this separation may be implemented in a concrete and computationally tractable way,
    while preserving the non-circular distinction between fundamental geometry, emergent spacetime, and
    dynamical interactions.

    \paragraph{Spectral origin of the mass scale.}
      While the fundamental relaxation equation of Cosmochrony is first order in time,
      it fixes the geometric structure of the underlying relaxation network through the
      graph Laplacian $\Delta_G^{(0)}$.
      The inertial mass of localized excitations instead emerges from the stability
      properties of small fluctuations around a stationary solitonic configuration.

      To make this explicit, we introduce an effective localization functional,
      defined at the coarse-grained level,
      \begin{equation}
        E[\chi] \;=\; \frac{1}{2}\sum_{(i,j)\in E} w_{ij}^{(0)}(\chi_i-\chi_j)^2 \;+\; E_{\mathrm{loc}}[\chi],
      \end{equation}
      where $w_{ij}^{(0)}$ are fixed geometric weights encoding the relaxation network,
      and $E_{\mathrm{loc}}$ denotes an effective stabilizing contribution ensuring the
      existence of localized stationary configurations.
      This functional is not assumed to be fundamental but serves to characterize
      the stability of solitonic excitations.

      Let $\chi_{\mathrm{sol}}$ be a stationary localized configuration, and consider
      small fluctuations $\chi = \chi_{\mathrm{sol}} + \delta\chi$.
      The second variation of $E[\chi]$ around $\chi_{\mathrm{sol}}$ defines a linear operator
      \begin{equation}
        \delta^2 E \;=\; \langle \delta\chi,\, \mathcal{L}_{\mathrm{sol}}\,\delta\chi \rangle,
        \qquad
        \mathcal{L}_{\mathrm{sol}} \equiv \Delta_G^{(0)} + U_{\mathrm{sol}},
      \end{equation}
      where $U_{\mathrm{sol}}$ is a local restoring operator determined by the background
      configuration.

      In the regime where an effective wave description applies, small fluctuations
      are governed by a Klein--Gordon-type equation,
      \begin{equation}
        \left(\frac{1}{c^2}\partial_t^2 + \mathcal{L}_{\mathrm{sol}}\right)\delta\chi = 0.
      \end{equation}
      For normal modes $\delta\chi(t)=e^{-i\omega t}\psi_n$, one obtains the spectral problem
      \begin{equation}
        \mathcal{L}_{\mathrm{sol}}\psi_n = \lambda_n \psi_n,
        \qquad
        \omega_n^2 = c^2 \lambda_n .
      \end{equation}
      Identifying the rest frequency $\omega_0$ with the inertial mass via
      $\omega_0 = m c^2/\hbar$ yields
      \begin{equation}
        m_n = \frac{\hbar}{c}\sqrt{\lambda_n}.
      \end{equation}

      Hence, particle masses in Cosmochrony arise as square roots of the eigenvalues
      of the linearized relaxation operator governing the stability of localized
      configurations.

  \subsubsubsection{*Fundamental spectral operator}

  Let $G=(V,E)$ be a discrete network encoding the intrinsic connectivity of the pre-geometric substrate.
  The fundamental relaxation operator is defined as a weighted graph Laplacian
  \begin{equation}
  (\Delta_G^{(0)} \psi)_i = \sum_{j\sim i} w_{ij}^{(0)} (\psi_i - \psi_j),
  \end{equation}
  where the weights $w_{ij}^{(0)} = 1/K_{ij}^{(0)}$ encode the intrinsic compliance of the relaxation network.
  At this level, $K_{ij}^{(0)}$ are fixed coefficients determined by network topology and symmetry
  constraints, and do not depend on the dynamical state of the $\chi$ field.

  The eigenvalue problem
  \begin{equation}
    \Delta_G^{(0)} \psi_n = \lambda_n \psi_n
  \end{equation}
  defines a discrete spectrum.
  Within the Cosmochrony framework, particle masses are conjectured to scale as
  \begin{equation}
    m_n c^2 \propto \sqrt{\lambda_n},
  \end{equation}
  so that mass hierarchies emerge as geometric properties of the relaxation network rather than as
  parameters encoded in a fundamental potential.

\subsubsubsection{*Emergent geometric level}

  On larger scales, coarse-grained configurations of the $\chi$ field define an effective geometric
  description.
  This level governs gravitational phenomena, time dilation, and cosmological expansion through the
  local and global rates of $\chi$ relaxation.
  Importantly, while this emergent geometry influences the propagation of excitations, it does not
  redefine the fundamental spectral operator $\Delta_G^{(0)}$.

\subsubsubsection{*Dynamical and interaction level}

  Fast processes such as radiation, scattering, and decoherence correspond to interaction-induced
  redistributions of relaxation potential within the $\chi$ field.
  At this effective level, it is natural to consider corrections to the relaxation dynamics encoded
  by coefficients $K_{ij}(\chi)$ that depend smoothly on coarse-grained field variations.
  Such dependencies modify local dynamics and observable interaction rates but do not enter the
  definition of the fundamental spectral problem associated with mass generation.

\subsubsubsection{*Boundary conditions and numerical implementation}

  For numerical studies, the graph $G$ may be taken as large but finite, with periodic boundary
  conditions to minimize edge effects, or with Dirichlet or Neumann conditions to model confined
  regions.
  Standard sparse eigensolvers may then be used to compute the low-lying spectrum of $\Delta_G^{(0)}$
  and to test whether simple network connectivities can reproduce observed mass hierarchies.

  These constructions illustrate how Cosmochrony may be developed into a calculable spectral program,
  while preserving the non-circular separation between fundamental geometry, emergent spacetime, and
  dynamical interactions.
