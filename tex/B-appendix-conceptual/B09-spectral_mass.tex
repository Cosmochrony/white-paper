\subsection{Spectral Characterization of Mass and the Secondary Role of $V(\chi)$}
  \label{subsec:spectral_mass}

  This appendix clarifies the conceptual status of inertial mass in Cosmochrony.
  While mass originates from the resistance of solitonic configurations to the
  relaxation of the $\chi$ field, this resistance can be quantitatively characterized
  through the spectral properties of the corresponding stability operator.

  In this context, the spectral analysis does not redefine the physical origin of inertial mass,
which remains the resistance of solitonic configurations to the relaxation of the $\chi$ field,
but provides a quantitative characterization of this resistance.

  A key conjecture of the Cosmochrony framework is that particle masses are not fundamental parameters encoded in the
  nonlinear potential $V(\chi)$, but instead emerge as spectral properties of a relaxation operator defined on the underlying discrete substrate.

  In this perspective, the role of $V(\chi)$ is secondary and effective: it serves as a convenient coarse-grained description of localization and stability, but does not fundamentally determine the mass spectrum.

  \paragraph{Mass spectrum as eigenmodes of a relaxation operator.}
    Localized particle-like excitations are identified with normal modes of a discrete Laplace--Beltrami operator acting on the graph $G(V,E)$,
    \begin{equation}
      \Delta_G \psi_n = -\lambda_n \psi_n ,
    \end{equation}
    where $\psi_n$ are eigenmodes of the relaxation dynamics.
    The associated particle masses are conjectured to scale as
    \begin{equation}
      m_n c^2 \propto \sqrt{\lambda_n}.
    \end{equation}

    This mechanism is directly analogous to the emergence of discrete acoustic frequencies in bounded elastic systems, where the spectrum is entirely fixed by geometry and boundary conditions rather than by adjustable material parameters.
    Within Cosmochrony, mass hierarchies are thus interpreted as geometric properties of the underlying network topology and connectivity.

    A decisive test of this conjecture consists in computing the low-lying spectrum of $\Delta_G$ on large but finite graphs with physically motivated connectivity rules.
    If even approximate agreement with observed mass ratios were obtained, this would strongly suggest that $V(\chi)$ is not a fundamental ingredient of the theory.

  \paragraph{Spectral structure and level separation.}

    To avoid any circular dependence between geometry, dynamics, and emergent particle properties,
    the Cosmochrony framework distinguishes three conceptual levels.

    At the fundamental level, particle masses are associated with the spectral properties of a fixed,
    background-independent relaxation operator $\Delta_G^{(0)}$, defined solely by the intrinsic
    connectivity of the underlying discrete network.
    This operator does not depend on the instantaneous configuration of the $\chi$ field and provides
    a stable spectral structure whose eigenvalues $\lambda_n$ define mass scales through
    $m_n c^2 \propto \sqrt{\lambda_n}$.

    At a secondary level, coarse-grained configurations of $\chi$ give rise to an effective geometric
    description, including gravitational time dilation and cosmological expansion.
    This emergent geometry influences the propagation and interaction of excitations but does not
    modify the underlying spectral operator responsible for mass generation.

    Finally, fast dynamical processes such as radiation, scattering, and decoherence occur as
    interaction-induced redistributions of relaxation potential within the $\chi$ field.
    These processes affect observables without redefining the fundamental spectral structure.

    This separation ensures that particle masses are universal spectral properties of the relaxation
    network, while geometric and dynamical effects remain emergent and non-circular.

  \paragraph{Residual role of the potential $V(\chi)$.}
    Within this spectral picture, the nonlinear potential $V(\chi)$ may be understood as an effective description of localization mechanisms arising after coarse-graining.
    Its form is constrained by the requirement that it admit stable solitonic solutions corresponding to the low-lying eigenmodes of the relaxation operator, but it does not independently fix their masses.

  \paragraph{Supporting perspectives.}
    Discrete symmetry constraints and information-theoretic considerations may further restrict admissible network structures or provide complementary interpretations of the emergent dynamics.
    However, these directions are secondary to the central spectral hypothesis and are not required for its internal consistency.

    Taken together, these considerations suggest that a substantial part of the
    explanatory burden for mass generation in Cosmochrony may lie in the spectral
    properties of the underlying discrete relaxation dynamics, with $V(\chi)$ playing
    a derived and non-fundamental role in effective descriptions, and with the
    spectral structure understood as a quantitative characterization of solitonic
    inertial resistance.

  Extending this spectral characterization toward a concrete prediction of particle
    mass hierarchies requires specifying the underlying relaxation operator and its
    boundary conditions, as discussed in~\ref{subsec:perspectives_mass_spectrum}.
