\subsection{Normalization of the $\chi$ Field}
  \label{subsec:normalization-of-the-$chi$-field}

  To connect the $\chi$ field to observable quantities, a normalization must be specified.
  We identify the present-day value $\chi(t_0)$
  with the characteristic cosmological length scale governing expansion.

  Operationally, $\chi(t_0)$
  may be interpreted as the proper wavelength accumulated since the epoch at which coherent propagation of
  radiation became possible, approximately recombination.

\subsection{Hubble Constant}
  \label{subsec:hubble-constant}

  From the fundamental relation
  \begin{equation}
    H(t) = \frac{\dot{\chi}}{\chi},
  \end{equation}
  and assuming maximal relaxation speed $\dot{\chi} \simeq c$, the present Hubble constant follows as
  \begin{equation}
    H_0 \simeq \frac{c}{\chi(t_0)}.
  \end{equation}

  Using the observed value $H_0 \sim 70~\mathrm{km\,s^{-1}\,Mpc^{-1}}$, one infers
  \begin{equation}
    \chi(t_0) \sim 4 \times 10^{26}~\mathrm{m},
  \end{equation}
  consistent with the current Hubble radius.

  This correspondence arises without introducing free cosmological parameters.

  The soliton mass scale $m \propto \sqrt{\lambda}$ then requires $\lambda \sim 10^{-116} \, \text{m}^{-2}$
  to reproduce the electron mass $m_e \approx 9.11 \times 10^{-31} \, \text{kg}$. This tiny value suggests that
  $\lambda$ may be dynamically generated rather than fundamental.

\subsection{Age of the Universe}
  \label{subsec:age-of-the-universe}

  Integrating the relation $\dot{\chi} \simeq c$ yields
  \begin{equation}
    \chi(t) \simeq c t + \chi_{\mathrm{init}},
  \end{equation}
  where $\chi_{\mathrm{init}}$ denotes the effective value at the onset of coherent $\chi$ relaxation.

  Neglecting $\chi_{\mathrm{init}}$ compared to present values gives
  \begin{equation}
    t_0 \simeq \frac{\chi(t_0)}{c} \sim 4 \times 10^{17}~\mathrm{s},
  \end{equation}
  corresponding to approximately 13.8 billion years, in agreement with standard cosmological estimates.

\subsection{Redshift Interpretation}\label{subsec:redshift-interpretation}

  Cosmological redshift arises from the increase of $\chi$ between emission and observation:
  \begin{equation}
    1 + z = \frac{\chi(t_{\mathrm{obs}})}{\chi(t_{\mathrm{emit}})}.
  \end{equation}

  This interpretation reproduces standard redshift relations while attributing them to geometric scaling rather than
  recessional motion through spacetime.

\subsection{Cosmic Microwave Background Scale}\label{subsec:cosmic-microwave-background-scale}

  At recombination ($z_{\mathrm{rec}} \simeq 1100$), the characteristic scale of $\chi$
  was smaller by the same factor:
  \begin{equation}
    \chi(t_{\mathrm{rec}}) \simeq \frac{\chi(t_0)}{1 + z_{\mathrm{rec}}}.
  \end{equation}

  Fluctuations imprinted at that epoch are stretched by subsequent $\chi$
  growth, explaining the observed angular power spectrum of the CMB\@.

\subsection{Orders of Magnitude and Robustness}\label{subsec:orders-of-magnitude-and-robustness}

  All numerical estimates presented here rely solely on observed cosmological quantities and the assumption of
  bounded $\chi$ relaxation.
  No fine-tuning of parameters is required.

  While precise numerical modeling remains to be developed, these estimates demonstrate that cosmochrony naturally
  reproduces the correct orders of magnitude for key cosmological observables.

\subsection{Summary}\label{subsec:summary}

  The $\chi$ framework connects directly to measured cosmological quantities through simple scaling relations.
  The Hubble constant, cosmic age, redshift, and CMB scales emerge consistently from the same underlying dynamics.


  \bibliographystyle{plain} % We choose the "plain" reference style
  \bibliography{refs} % Entries are in the refs.bib file
