\subsection{Phenomenological Implications}
  \label{subsec:phenomenology}

  This section discusses the principal phenomenological consequences of Cosmochrony
  that are accessible to observation.
  The emphasis is placed on results that follow robustly from the kinematic and dynamical structure of the $\chi$ field,
  without introducing additional assumptions or tunable parameters.

  \paragraph{Speed of gravitational perturbations.}
    To determine the propagation speed of gravitational information in Cosmochrony, we
    consider small perturbations $\delta\chi$ around a homogeneous background solution
    \begin{equation}
      \chi_0(t) = c t ,
    \end{equation}
    such that
    \begin{equation}
      \chi(\mathbf{x},t) = c t + \delta\chi(\mathbf{x},t),
      \qquad |\nabla \delta\chi| \ll c .
    \end{equation}

    Substituting into the fundamental evolution equation
    (Eq.~\ref{eq:chi_dynamics}) gives
    \begin{equation}
      c + \partial_t \delta\chi
      = c \sqrt{1 - \frac{|\nabla \delta\chi|^2}{c^2}} .
    \end{equation}

    Expanding to leading order in spatial gradients yields
    \begin{equation}
      \partial_t \delta\chi \simeq -\frac{|\nabla \delta\chi|^2}{2c} .
    \end{equation}

    While this first-order equation reflects the irreversible relaxation character of
    the dynamics, the propagation of perturbations is more transparently captured by
    considering the second-order operator obtained from the squared Hamiltonian
    constraint (Eq.~\ref{eq:chi_dynamics}). Linearizing this operator leads to
    the effective wave equation
    \begin{equation}
      \left( \frac{1}{c^2}\partial_t^2 - \nabla^2 \right)\delta\chi = 0 .
      \label{eq:gw_wave}
    \end{equation}

    The characteristic propagation speed of disturbances in the $\chi$ field is
    therefore
    \begin{equation}
      v_{\mathrm{prop}} = c .
    \end{equation}

    This result ensures strict consistency with multi-messenger observations,
    including the near-simultaneous arrival of gravitational and electromagnetic
    signals in GW170817.
    In Cosmochrony, this equality is not imposed but follows directly from the fundamental kinematic bound on $\chi$
    relaxation.

  \paragraph{Emergent acceleration scale and MOND-like phenomenology.}
    In Cosmochrony, the arrow of time is encoded in the monotonic evolution of the
    fundamental field $\chi$, with $\partial_t \chi \ge 0$.
    At late cosmic times and on large scales, where $\chi$ admits an approximately homogeneous and isotropic
    description, this evolution may be coarse-grained into an effective cosmological clock.

    In this regime, and only as an effective description, the temporal evolution of $\chi$ may be written as
    \begin{equation}
      \partial_t \chi \simeq H(t)\,\chi ,
    \end{equation}
    where $H(t)$ denotes the emergent Hubble parameter associated with the global relaxation of the field.

    The local kinematic constraint
    \begin{equation}
    (\partial_t \chi)^2 + |\nabla \chi|^2 = c^2
    \end{equation}
    then implies that even in the absence of localized matter excitations, the
    cosmological evolution of $\chi$ induces a non-vanishing residual spatial gradient.
    In the homogeneous limit, this minimal gradient is
    \begin{equation}
      |\nabla \chi|_{\min} = \sqrt{c^2 - (H\chi)^2} .
    \end{equation}

    This residual gradient defines a background kinematic scale that constrains how
    additional, locally induced gradients contribute to the effective dynamics.
    It may be expressed operationally as an effective acceleration floor
    \begin{equation}
      a_0(t) \sim c\,H(t) .
    \end{equation}

    When localized matter excitations are present, they induce additional gradients
    $\nabla\chi_N$ that reproduce the Newtonian scaling $|\nabla\chi_N| \propto M/r^2$ at short distances.
    Due to the non-linear nature of the kinematic constraint, the total gradient does not superpose linearly.
    At sufficiently large radii, the effective acceleration asymptotically approaches
    \begin{equation}
      g_{\mathrm{eff}} \simeq \sqrt{g_N\,a_0(t)} ,
    \end{equation}
    recovering the characteristic deep-MOND scaling without interpolation functions or
    additional fields~\cite{Milgrom2002}.

    Importantly, Cosmochrony predicts that the acceleration scale $a_0$ is not
    fundamental but slowly evolves with cosmic time through its dependence on $H(t)$,
    providing a potential observational discriminator at high redshift.

  \paragraph{Gravitational lensing.}
    In Cosmochrony, light propagation follows wave fronts of constant $\chi$.
    The effective refractive index of the vacuum is defined operationally as
    \begin{equation}
      n(r) = \frac{c}{\partial_t \chi}
      = \frac{1}{\sqrt{1 - |\nabla \chi|^2/c^2}} .
    \end{equation}

    Near a localized mass $M$, where
    $|\nabla \chi| \simeq GM/(c^2 r)$, the weak-field expansion gives
    \begin{equation}
      n(r) \simeq 1 + \frac{GM}{c^2 r} .
    \end{equation}

    Integrating the transverse gradient of $n$ along a photon trajectory yields the
    deflection angle
    \begin{equation}
      \alpha = \frac{4GM}{b c^2},
    \end{equation}
    with $b$ the impact parameter.
    This reproduces the general-relativistic prediction, with the enhancement relative to the Newtonian result arising from the non-linear
    structure of the $\chi$ dynamics rather than from fundamental spacetime curvature.

  \paragraph{Summary.}
    The phenomenology of Cosmochrony reproduces key observational signatures of gravity
    and cosmology while relying on a single scalar degree of freedom.
    Gravitational perturbations propagate at exactly the invariant speed $c$, a MOND-like acceleration
    scale emerges naturally from cosmological relaxation, and gravitational lensing is
    recovered without postulating a fundamental metric.
    These results illustrate how classical gravitational phenomena arise as coarse-grained manifestations of the
    underlying $\chi$ dynamics.
