\subsection{Cosmic Voids as Observational Tests of Maximal Substrate Relaxation}
  \label{subsec:observational-tests-voids}

  Cosmic voids constitute a particularly clean observational laboratory for the
  Cosmochrony framework.
  In contrast with overdense environments, voids correspond to regions where the
  relaxation of the relational substrate $\chi$ is only weakly frustrated by
  localized excitations.
  They therefore probe the regime of near-maximal relaxation, in which departures
  from standard $\Lambda$CDM phenomenology are expected to be most pronounced.

  \paragraph{Born--Infeld parametrization of void observables.}
    We model observable signals associated with cosmic voids as a $\Lambda$CDM
    baseline supplemented by a saturating correction inspired by the Born--Infeld
    structure of the effective $\chi$ dynamics.
    For the weak-lensing convergence and the radial peculiar velocity field, we write
    \begin{align}
      \kappa_{\mathrm{obs}}(R)
      &=
      \kappa_{\Lambda\mathrm{CDM}}(R)
      \left[
        1+\beta_{\mathrm{void}}\,
        \mathcal{S}\!\big(\mathcal{A}(R)\big)
      \right],
      \\
      v_{\mathrm{obs}}(r)
      &=
      v_{\Lambda\mathrm{CDM}}(r)
      \left[
        1+\beta_{\mathrm{void}}\,
        \mathcal{S}\!\big(\mathcal{B}(r)\big)
      \right],
    \end{align}
    where $\beta_{\mathrm{void}}$ controls the amplitude of the cosmochronic
    correction and
    \begin{equation}
      \mathcal{S}(x)=\frac{x}{\sqrt{1+x^2}}
    \end{equation}
    is a Born--Infeld--like saturation function.
    The dimensionless activities $\mathcal{A}$ and $\mathcal{B}$ quantify the local
    degree of relaxation and may be defined using observational proxies, e.g.
    \begin{equation}
      \mathcal{A}(R)=\frac{|\Delta(R)|}{s_\star},
      \qquad
      \mathcal{B}(r)=\frac{|\delta(r)|}{s_\star}
    \end{equation}
    with $\Delta(R)$ the projected density contrast and $\delta(r)$ the three-dimensional
    density contrast.
    The parameter $s_\star$ sets the saturation threshold.

  \paragraph{Key observational signatures.}
    This parametrization leads to three distinctive and falsifiable predictions:
    \begin{enumerate}
      \item \textbf{Negative void lensing enhancement.}
      Cosmic voids are expected to exhibit a more negative weak-lensing signal than in
      $\Lambda$CDM, with a characteristic non-linear saturation for large and deep
      voids.
      The lensing profile is predicted to peak near the void boundary and to approach a
      plateau as the relaxation activity increases.

      \item \textbf{Enhanced peculiar velocity outflows.}
      Galaxies at void boundaries should display radial outflows exceeding standard
      $\Lambda$CDM expectations by a fraction controlled by $\beta_{\mathrm{void}}$.
      In the saturated regime, the excess velocity approaches
      $\Delta v_r \simeq \beta_{\mathrm{void}}\,v_{\Lambda\mathrm{CDM}}$.

      \item \textbf{Cross-consistency between lensing and velocities.}
      Both effects originate from the same relaxation mechanism and must therefore be
      described by a single value of $\beta_{\mathrm{void}}$.
      The simultaneous fitting of void lensing profiles and peculiar velocity data thus
      provides a stringent internal consistency test of the framework.
    \end{enumerate}

  \paragraph{Connection to local expansion measurements.}
    Because enhanced void outflows bias low-redshift distance--redshift relations,
    regions dominated by large voids are predicted to yield locally inferred values of
    the Hubble parameter exceeding the global average.
    Cosmochrony therefore predicts a correlation between negative void-lensing
    strength, enhanced boundary outflows, and elevated local $H_0$ estimates, offering
    a unified explanation testable with upcoming weak-lensing and redshift surveys.
