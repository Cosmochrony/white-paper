\subsection{Cosmological Role of Neutrino-Like Excitations}
  \label{subsec:cosmological-role-of-neutrino-like-excitations}

\subsubsection*{Neutrino-Like Excitations in the Structural Spectrum}

  Neutrino-like excitations correspond to projected configurations with minimal
  structural energy and weak localization.
  Their effective mass reflects a very low resistance to relaxation, while their weak
  interaction indicates a limited coupling to strongly localized configurations.

  These excitations occupy an intermediate regime between particle-like solitons and
  radiative modes.

\subsubsection*{Free Streaming as Non-Local Relaxation}

  Let \(\chi_{\nu}\) denote a neutrino-like projected excitation.
  Due to weak localization, \(\chi_{\nu}\) propagates across large effective distances
  without significant interaction.

  This free streaming implements a non-local relaxation channel that transports residual
  structural mismatch generated during localized structure formation.
  Unlike photons, this channel does not thermalize efficiently and therefore preserves
  memory of early relaxation phases.

\subsubsection*{Impact on Structure Formation}

  The presence of a background population of neutrino-like excitations suppresses the
  growth of small-scale projected structures by continuously redistributing mismatch
  away from overdense regions.

  This effect is non-dissipative in the thermodynamic sense and does not correspond to
  ordinary pressure.
  It reflects a geometric constraint imposed by non-local relaxation channels.

\subsubsection*{Contribution to Temporal Ordering}

  Once emitted, neutrino-like excitations are unlikely to be reincorporated into
  localized projected configurations.
  This asymmetry induces an effective irreversibility in the space of admissible
  descriptions.

  Neutrino free streaming therefore contributes to the emergence of a cosmological arrow
  of time as an ordering of admissible projected configurations rather than as a
  fundamental temporal flow.
