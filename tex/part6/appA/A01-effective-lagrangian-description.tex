% A01 --- Effective Lagrangian Description as a Hydrodynamic Limit
\subsection{Effective Lagrangian Description as a Hydrodynamic Limit}
\label{subsec:hydrodynamic-limit}

In regimes where~$\chi$ varies smoothly, a continuum approximation
provides contact with standard geometric formulations.
Distances are defined through the resistance to relaxation propagation,
leading to an effective line element
\[
  g_{\mu\nu}\,dx^\mu dx^\nu
  \;\sim\;
  \sum_{(u,v)\in\text{path}} \frac{1}{K_{uv}}.
\]
To reproduce the continuum evolution equations
(Equation~\ref{eq:discrete-dynamics}), one introduces an effective
Lagrangian density:
\[
  \mathcal{L}_{\text{eff}}
  = \frac{1}{16\pi G_{\text{eff}}}\,F(\chi)\,R
    - \Lambda_{\text{flow}}^{4}\,\chi + \cdots
\]
where $R$ is the Ricci scalar of the effective metric and $F(\chi)$
parametrizes how relaxation dynamics maps onto the geometric description.
This Lagrangian is purely representational; it does not define the
fundamental dynamics, has no ontological status, and should not be
quantized.
Einstein-like field equations emerge as universal geometric descriptions
of slowly varying collective phenomena.
