% ----------------------------------------------------------------------------
% A.12 --- Relational Consistency of the Effective Lagrangian
% From former A12, condensed
% ----------------------------------------------------------------------------
\subsection{Relational Consistency of the Effective Lagrangian}
\label{sec:born-lagrangian_derivation}

The Born--Infeld-like form emerges from fundamental principles, not
arbitrary postulation.

\subsubsection*{Step 1: Relational Constraint}
\label{subsec:A12-relational-constraint_v171}

The discrete relational constraint:
\begin{equation}
  \mathcal{C}_i[\chi]
  \equiv \sum_j K_{ij} (\chi_i - \chi_j)^2
  \le \chi_c^2,
  \label{eq:A12-relational-constraint_v171}
\end{equation}
enforces bounded relative variations without assuming pre-existing
spacetime.

\subsubsection*{Step 2: Variational Formulation}
\label{subsec:A12-variational-structure_v171}

Constrained action with KKT conditions:
\begin{equation}
  S[\{\chi_i\}, \{\mu_i\}]
  = \int d\lambda \left[
      \sum_i \frac{1}{2}
        \left(\frac{d\chi_i}{d\lambda}\right)^2
      - U[\{\chi_i\}]
      - \sum_i \mu_i(\lambda)
        \left(\mathcal{C}_i[\chi]
          - \chi_c^2\right)
    \right].
  \label{eq:A12-action_v171}
\end{equation}

\subsubsection*{Step 3: Continuum Limit}
\label{subsec:A12-continuum-limit_v171}

In projectable regimes:
$|\nabla \chi|^2 \le c^2$.
\label{eq:A12-continuum-bound_v171}
The canonical selection criteria (free theory normalization, saturation,
monotonicity, regularity) yield:
\begin{equation}
  f(x) = -c^2 \sqrt{1 - x},
  \label{eq:A12-born-infeld-derivation_v171}
\end{equation}
giving the Born--Infeld-like Lagrangian:
\begin{equation}
  \mathcal{L}_{\text{eff}}
  = -c^2 \sqrt{1 - \frac{|\nabla \chi|^2}{c^2}}
    + \partial_t \chi.
  \label{eq:A12-effective-lagrangian_v171}
\end{equation}

\subsubsection*{Step 4: Role of the Potential}
\label{subsec:A12-potential-role_v171}

In the continuum limit,
$U[\{\chi_i\}] \to \int d^3x \, V(\chi)$.

\subsubsection*{Step 5: Emergent Geometry}
\label{subsec:A12-emergent-geometry_v171}

The effective metric is defined via the Hessian:
\begin{equation}
  g_{\mu\nu}^{\text{eff}} \propto
  \frac{\partial^2 \mathcal{L}_{\text{eff}}}
    {\partial (\partial_\mu \chi)
     \partial (\partial_\nu \chi)}.
  \label{eq:A12-emergent-metric_v171}
\end{equation}

\subsubsection*{Continuum Limit and Laplace--Beltrami Operator}
\label{subsec:A12-continuum-limit}

With $K_{ij} = V_i^{-1}\,w(d_{ij}/\varepsilon)$, the dense limit
gives convergence to the Laplace--Beltrami operator on an emergent
Riemannian manifold defined by the relational density.

\subsubsection*{Necessity of the Born--Infeld Structure}
\label{subsec:A12-born-infeld-necessity}

A quadratic Lagrangian permits unbounded gradients, contradicting the
maximal relaxation speed~$c_\chi$.
The minimal functional satisfying boundedness, smooth saturation, and
finite characteristic speeds is:
\begin{equation}
  \mathcal{L}_{\mathrm{BI}}
  = b^2\!\left(
      1 - \sqrt{1 - \frac{1}{b^2}\,
        \partial_\mu \chi\,\partial^\mu \chi}
    \right).
\end{equation}
The saturation constant~$b$ represents the upper bound on the
relaxation speed of the~$\chi$ substrate itself, while the speed of
light~$c$ emerges as the group velocity of projected perturbative modes:
\begin{equation}
  c \leq b.
\end{equation}
