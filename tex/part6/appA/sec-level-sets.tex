% ----------------------------------------------------------------------------
% A.10 --- Level Sets, Projections, and Apparent Orbital Geometry
% From former A10, condensed
% ----------------------------------------------------------------------------
\subsection{Level Sets, Projections, and Apparent Orbital
Geometry}
\label{app:level_sets_orbitals}

For a continuous scalar field
$\phi : \mathbb{R}^3 \to \mathbb{R}$, the level set
$\mathcal{L}_c
  = \{\mathbf{x} \mid \phi(\mathbf{x}) = c\}$
is generically a two-dimensional surface, possibly with disconnected
components.

Projecting the thresholded set
$P_c
  = \{z \mid \exists(x,y)\;\phi(x,y,z) \ge c\}$
generically produces disjoint intervals even when $\phi$ is
continuous.
This fragmentation is a purely geometric consequence of thresholding
followed by projection: no discontinuity of $\phi$ is involved.

The envelope $f(z) = \max_{x,y}\phi(x,y,z)$ is continuous, but the
condition $f(z) \ge c$ selects disconnected regions whose appearance
and disappearance as $c$ varies reflect visibility changes, not
structural changes.
The inverse reconstruction of $\phi$ from $P_c$ is underdetermined.

Orbital-like patterns, nodal structures, and probabilistic visibility
regions are therefore emergent manifestations of an underlying
continuous field.
Their apparent discreteness reflects projection and detection
criteria.
