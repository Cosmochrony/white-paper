% A06 --- Minimal Kinematic Constraint
\subsection{Minimal Kinematic Constraint}
\label{subsec:minimal-kinematic-constraint}

In its saturated form, the universal kinematic bound reads:
\begin{equation}
  (\partial_t \chi)^2 + |\nabla \chi|^2 = c^2 ,
  \label{eq:minimal-kinematic-constraint}
\end{equation}
with inequality for unsaturated configurations.
This does not presuppose a spacetime metric or Lorentzian structure; it
expresses a purely kinematic admissibility condition at the pre-geometric
level.
The constant~$c$ is the effective manifestation of the invariant bound
$c_\chi$ (Section~\ref{subsec:role-of-cchi}).

Lorentz symmetry arises \emph{a posteriori} as a property of saturated
relaxation.
In homogeneous regimes, the constraint enforces $\partial_t\chi = c$,
yielding linear growth of $\chi_{\mathrm{eff}}$ and effective cosmic
expansion.
In inhomogeneous regimes, partial saturation manifests as gravitational
time dilation and curvature.

At Planck-scale resolution, the continuum gradient must be replaced by
finite differences on the underlying relational graph.
A fully discrete formulation is deferred to future work.
