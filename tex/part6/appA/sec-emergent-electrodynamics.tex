% ----------------------------------------------------------------------------
% A.11 --- Emergent Electrodynamics from chi Dynamics
% From former A11, condensed
% ----------------------------------------------------------------------------
\subsection{Emergent Electrodynamics from
\texorpdfstring{$\chi$}{χ} Dynamics}
\label{app:emergent-electrodynamics}

In the weak-gradient regime, $\nabla\chi$ decomposes as
$\nabla\chi = -\nabla\phi + \mathbf{A}_\mathrm{T}$ with
$\nabla \cdot \mathbf{A}_\mathrm{T} = 0$, induced by the topology of
localized excitations.

\subsubsection*{Charge as Transverse Torsion}
\label{subsec:charge-as-torsion}

The bounded canonical current
$\mathbf{J}_\chi
  \propto \nabla\chi/\sqrt{1 - |\nabla\chi|^2/c^2}$
saturates as $|\nabla\chi| \to c$.
An effective charge invariant is
$q = \kappa \oint_\gamma
    \mathbf{A}_\mathrm{T} \cdot d\boldsymbol{\ell}$,
whose sign reflects the chirality of the transverse torsion.
The magnetic field is
$\mathbf{B} = \nabla \times \mathbf{A}_\mathrm{T}$.

\subsubsection*{Emergent Electromagnetic Fields}

Electric and magnetic fields:
\[
  \mathbf{E}
  = -\nabla\phi
    - \frac{1}{c}\frac{\partial\mathbf{A}_\mathrm{T}}
                      {\partial t},
  \quad
  \mathbf{B}
  = \nabla \times \mathbf{A}_\mathrm{T}.
\]
These satisfy a closed system of Maxwell-like relations.
Gauge invariance under
$\phi \to \phi - c^{-1}\partial_t\Lambda$,
$\mathbf{A}_\mathrm{T}
  \to \mathbf{A}_\mathrm{T} + \nabla\Lambda$
reflects the relational nature of $\chi$
(Section~\ref{subsec:relational-foundation-pointer}).

Electromagnetism appears as a geometric and topological manifestation
of the $\chi$ substrate, not an independent interaction mediated by
elementary gauge fields.
