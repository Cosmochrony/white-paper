% ----------------------------------------------------------------------------
% E.1 --- Relational Configurations of chi
% From former E.1, condensed
% ----------------------------------------------------------------------------
\subsection{Relational Configurations of
\texorpdfstring{$\chi$}{χ}}
\label{subsec:relational-configurations-of-chi}

A relational configuration of~$\chi$ is defined by its internal pattern
of mutual relaxation constraints, without coordinates, distances, or
background geometry
(Section~\ref{subsec:definition-of-the-chi-field}).
Physical states correspond to equivalence classes under
relaxation-preserving transformations.
Configurations do not generally factorize into independent subsystems;
this structural non-factorization underlies nonlocal correlations
(Section~\ref{subsec:non-factorization-entanglement}).
Only under conditions of approximate homogeneity and stable relaxation
does a relational configuration admit an effective geometric projection,
via a many-to-one, inherently approximate mapping.
