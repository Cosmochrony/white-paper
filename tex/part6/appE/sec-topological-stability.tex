% ----------------------------------------------------------------------------
% E.6 --- Topological Stability of Relational chi Configurations
% From former E.8, condensed
% ----------------------------------------------------------------------------
\subsection{Topological Stability of Relational
\texorpdfstring{$\chi$}{χ} Configurations}
\label{app:relational_topological_stability}

Particle-like excitations are identified with nontrivial relational
patterns whose stability is guaranteed by intrinsic topological
constraints in configuration space
(Section~\ref{subsec:topological-stability}).
Topology here characterizes inequivalent classes of~$\chi$
configurations that cannot be continuously transformed into one another
without violating relaxation constraints.

Stability arises from topological obstructions to global relaxation:
certain configurations cannot relax continuously to the homogeneous
vacuum without passing through forbidden regions.
This does not rely on conserved charges imposed by symmetry principles.

Geometric metaphors (knots, twists, vortices) provide intuition when
configurations admit effective geometric projection but should not be
taken literally at the relational level.
A paradigmatic example: configurations exhibiting intrinsic
$4\pi$-periodic internal structure cannot be unwound and, when projected,
exhibit spinorial transformation properties.
Distinct particle species correspond to inequivalent topological sectors;
the energetic cost of deformation provides a unified origin for mass,
stability, and spectral separation.
