% ----------------------------------------------------------------------------
% B.5 --- Relation to Classical Limits
% From former B05, condensed
% ----------------------------------------------------------------------------
\subsection{Relation to Classical Limits}
\label{subsec:classical-limits}

Classical behavior corresponds to a dynamical regime where
$\chi_{\mathrm{eff}}$ varies slowly, localized excitations are dilute,
and topological constraints are suppressed.
Perturbations propagate as weak disturbances on an effectively flat
background, recovering superposition and approximate locality.
In the nonlinear regime, large gradients induce effective curvature and
horizon-like behavior.
The classical limit is defined not by $\hbar \to 0$ but by the
stability of a particular relaxation regime of the underlying~$\chi$
field.
