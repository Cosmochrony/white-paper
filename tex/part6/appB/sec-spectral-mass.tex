% ----------------------------------------------------------------------------
% B.10 --- Spectral Characterization of Mass
% From former B10, condensed
% ----------------------------------------------------------------------------
\subsection{Spectral Characterization of Mass and the Secondary
Role of \texorpdfstring{$V(\chi)$}{V(χ)}}
\label{subsec:spectral_mass}

Particle masses emerge as eigenmodes of an effective relaxation
operator $\Delta_G^{(0)}\psi_n = -\lambda_n\psi_n$, with
$m_n c^2 \propto \sqrt{\lambda_n}\,\chi_c$.
This is analogous to bounded elastic systems where vibrational
frequencies arise from geometry and connectivity.

Three conceptual levels are distinguished: the background-independent
spectral operator (fundamental), coarse-grained
$\chi_{\mathrm{eff}}$ geometry (emergent), and interaction-induced
redistributions (dynamical).
$V(\chi)$ plays a secondary role, providing local stabilization and
fine splittings without setting the overall mass scale.
Its admissible form is constrained by compatibility with the
pre-existing spectral structure.
Potential-induced corrections shift eigenvalues as
$\lambda_n \to \lambda_n^{(0)} + \Delta\lambda_n^{(V)}$, affecting
small splittings (e.g.\ neutron--proton) while leaving topologically
dominated ratios largely invariant.
