% ----------------------------------------------------------------------------
% B.14 --- Metastability, Decay Channels, Exponential Lifetimes
% From former B14, condensed
% ----------------------------------------------------------------------------
\subsection{Metastability, Decay Channels, and Exponential
Lifetimes}
\label{subsec:metastability-decay-channels-and-exponential-lifetimes}

\subsubsection*{Diagnostic Structural Functional}
\label{subsec:diagnostic-structural-functional}

$E_{\mathrm{struct}}[\chi_{\mathrm{eff}}]
  \sim \int_\mathcal{V}
    (|\nabla\chi_{\mathrm{eff}}|^2
     + \mu^2|\chi_{\mathrm{eff}}|^2)\,d^3x$
quantifies resistance to relaxation.

\subsubsection*{Admissible Factorization Channels}
\label{subsec:admissible-factorization-channels}

A decay channel is an admissible factorization preserving topological
invariants:
$Q(\chi_{\mathrm{eff},A})
  = \sum_i Q(\chi_{\mathrm{eff},i})$.
Kinematic accessibility requires
$\Delta E_{\mathrm{struct}} > 0$.
The survival probability is
$P(\tau) = \exp(-\Gamma\tau)$ with
$\Gamma = \sum_c \Gamma_c$.

Different interaction classes correspond to different degrees of
constraint: strong decays involve direct topological reorganization,
electromagnetic decays preserve core topology, and weak decays require
deep reconfiguration through rarer paths.

\subsubsection*{Non-Injective Projection and Structural
Factorization}
\label{subsec:non-injective-projection-and-structural-factorization}

Entanglement: a single $\chi_0$ admits a non-factorizable projection.
Decay: the projection becomes unstable and admissibility is recovered
through factorization
$\Pi(\chi_0) \to
  \Pi(\chi_1) \oplus \Pi(\chi_2) \oplus \cdots$.
The distinction is not at the substrate level but in the stability
properties of the projected description.
