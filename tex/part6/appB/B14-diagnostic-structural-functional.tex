\subsection{Metastability, Decay Channels, and Exponential Lifetimes}
  \label{subsec:metastability-decay-channels-and-exponential-lifetimes}

\subsubsection*{Diagnostic Structural Functional}
  \label{subsec:diagnostic-structural-functional}

  We introduce a diagnostic functional \(E_{\mathrm{struct}}[\chi_{\mathrm{eff}}]\),
  which quantifies the degree of structural constraint associated with a localized
  projected configuration.
  This functional should be understood as an effective measure of resistance to
  relaxation, consistent with the interpretation of mass developed in
  Section~6.3.

  The explicit form of \(E_{\mathrm{struct}}\) is not unique.
  For stability analysis, it may be constructed from quadratic variations of
  \(\chi_{\mathrm{eff}}\), for instance
  \begin{equation}
    E_{\mathrm{struct}}[\chi_{\mathrm{eff}}]
    \sim
    \int_{\mathcal{V}}
    \left(
      |\nabla \chi_{\mathrm{eff}}|^2
    +
      \mu^2 |\chi_{\mathrm{eff}}|^2
    \right)\, d^3x,
  \end{equation}
  where \(\mathcal{V}\) denotes the effective localization region.

\subsubsection*{Admissible Factorization Channels}
  \label{subsec:admissible-factorization-channels}

  A decay channel is defined as an admissible factorization of a localized projected
  configuration into several localized configurations plus weakly structured modes,
  \begin{equation}
    \chi_{\mathrm{eff},A}
    \;\rightarrow\;
    \bigoplus_{i=1}^{N} \chi_{\mathrm{eff},i}
    \;\oplus\;
    \chi_{\mathrm{eff,rad}}.
  \end{equation}

  Admissibility requires the preservation of global structural invariants,
  \begin{equation}
    Q(\chi_{\mathrm{eff},A})
    =
    \sum_{i=1}^{N} Q(\chi_{\mathrm{eff},i}),
  \end{equation}
  where \(Q\) denotes any topological or relational invariant associated with the
  projected description.


  A channel is kinematically accessible if the total diagnostic structural functional
  satisfies
  \begin{equation}
    \Delta E_{\mathrm{struct}}
    =
    E_{\mathrm{struct}}[\chi_{\mathrm{eff},A}]
    -
    \sum_i E_{\mathrm{struct}}[\chi_{\mathrm{eff},i}]
    -
    E_{\mathrm{struct}}[\chi_{\mathrm{eff,rad}}]
    > 0.
  \end{equation}

\subsubsection*{Exponential Lifetimes}

  Projected configurations explore nearby admissible micro-rearrangements due to
  intrinsic projective variability.
  Let \(\Gamma\) denote the effective rate at which such fluctuations reach an admissible
  factorization threshold.

  If \(\Gamma\) is approximately constant over the relevant range of the effective
  ordering parameter \(\tau\), the survival probability satisfies
  \begin{equation}
    P(\tau) = \exp(-\Gamma \tau).
  \end{equation}

  The decay width \(\Gamma\) decomposes into partial widths associated with distinct
  admissible channels,
  \begin{equation}
    \Gamma = \sum_c \Gamma_c.
  \end{equation}

  This statistical description reproduces the phenomenology of quantum decay without
  postulating fundamental randomness or microscopic time evolution.

\subsubsection*{Structural Interpretation of Interaction Classes}

  Within this framework, different decay classes correspond to different degrees of
  constraint on admissible factorization paths.
  Strong decays involve direct and local reorganization of projected topology.
  Electromagnetic decays correspond to rearrangements preserving the core topological
  structure.
  Weak decays require deeper internal reconfiguration and therefore proceed through
  rarer admissible paths, resulting in longer lifetimes.

\subsubsection*{Non-Injective Projection and Structural Factorization}
  \label{subsec:non-injective-projection-and-structural-factorization}

  Let \(\Pi\) denote the projection operator from the \(\chi\)-substrate to effective
  observable descriptions.
  This projection is generically non-injective: distinct relational configurations of
  \(\chi\) may correspond to identical or indistinguishable effective observables, and
  conversely a single \(\chi\)-configuration may give rise to multiple correlated
  effective observables.

  Quantum entanglement corresponds to the case in which a single underlying
  \(\chi\)-configuration \(\chi_0\) admits a non-factorizable projected description
  \(\Pi(\chi_0)\).
  Although effective observables may be associated with spatially separated regions,
  the projected configuration cannot be written as a product of independent
  sub-configurations without violating admissibility.

  Particle decay corresponds to a different regime of the same projection structure.
  In this case, the projected configuration \(\Pi(\chi_0)\) becomes unstable under
  admissible fluctuations.
  No single projected description remains admissible.
  Admissibility is recovered only through factorization,
  \begin{equation}
    \Pi(\chi_0)
    \;\longrightarrow\;
    \Pi(\chi_1) \oplus \Pi(\chi_2) \oplus \cdots,
  \end{equation}
  where the \(\chi_i\) are distinct relational configurations whose projections are
  individually admissible and localized.

  The distinction between entanglement and decay is therefore not a distinction at the
  level of the \(\chi\)-substrate, but a distinction in the stability properties of the
  projected description under non-injective projection.
