\subsection{CP Asymmetry and Chiral Selection}
  \label{subsec:cp-asymmetry-and-chiral-selection}

\subsubsection*{CPT versus CP as Admissibility Symmetries}

  Let projected configurations carry a set of signed structural invariants
  \(\{Q_i\}\), associated with orientation, chirality, or phase winding.
  The admissibility conditions are invariant under the combined transformation
  \[
    (Q_i,\;\tau,\;\mathbf{x}) \rightarrow (-Q_i,\;-\tau,\;-\mathbf{x}),
  \]
  which defines an effective CPT symmetry.

  In contrast, CP acts only on a subset of the invariants \(\{Q_i\}\) and does not
  reverse the effective ordering parameter.
  As a result, CP is not, in general, an invariance of the admissibility conditions.
  Effective CP violation may therefore arise without violating CPT invariance.

\subsubsection*{Structural Bias and Matter--Antimatter Asymmetry}

  Assume that admissible projected configurations exhibit a slight asymmetry in
  relaxation efficiency with respect to the sign of a structural invariant \(Q\).
  Let \(\Gamma(Q)\) denote the effective stabilization rate.

  If
  \[
    \Gamma(Q) \neq \Gamma(-Q),
  \]
  then configurations carrying one orientation will be statistically favored during
  relaxation, leading to an emergent matter--antimatter asymmetry without requiring
  explicit symmetry breaking at the fundamental level.

\subsubsection*{Chiral Filtering and Neutrino-Like Excitations}

  Consider weakly localized projected configurations carrying a chiral invariant
  \(\chi_L = \pm 1\).
  Admissibility constraints may select only one sign of \(\chi_L\) as compatible with
  stable relaxation.

  Configurations of opposite chirality either fail to localize or decay rapidly.
  Neutrino-like excitations correspond to such minimally constrained configurations,
  which remain admissible only in one chiral sector and interact weakly with more
  structured excitations.
