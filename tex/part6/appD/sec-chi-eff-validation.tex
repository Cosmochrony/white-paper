% ----------------------------------------------------------------------------
% D.5 --- Numerical Validation of chi -> chi_eff
% From former D05, condensed
% ----------------------------------------------------------------------------
\subsection{Numerical Validation of the
\texorpdfstring{$\chi \to \chi_{\mathrm{eff}}$}%
{χ→χeff} Transition}
\label{subsec:numerical-validation-of-the-chi-rightarrow-chi_eff-transition}

\paragraph{Discrete model.}
On a cubic lattice with periodic boundaries, the local slope and
bounded relaxation rate are
\begin{equation}
  S_i(\chi)
  \equiv
  \frac{1}{c^2}
    \sum_{j\in\mathcal{N}(i)}
      K_{ij}(\chi_i - \chi_j)^2,
  \quad
  K_{ij}
  = \frac{K_0}
         {1 + (\chi_i - \chi_j)^2/\chi_c^2},
  \label{eq:D4_Si_Kij}
\end{equation}
\begin{equation}
  R_i \equiv c\,\sqrt{\max(0,\,1-S_i)}.
  \label{eq:D4_Ri_def}
\end{equation}
The explicit update is
\begin{equation}
  \chi_i(t+\Delta t)
  = \chi_i(t)
    + \Delta t\bigl(
        R_i(t) + \kappa\,(\Delta_G\chi)_i(t)
      \bigr).
  \label{eq:D4_update}
\end{equation}

\paragraph{Coarse-graining.}
\begin{equation}
  \chi_{\mathrm{eff}}(t)
  \equiv \mathrm{CG}\bigl(\chi(t)\bigr).
  \label{eq:D4_chi_eff_def}
\end{equation}

\paragraph{Validation target.}
Because coarse-graining does not commute with the nonlinear dynamics,
the target is the coarse-grained micro-dynamics:
\begin{equation}
  \partial_t \chi_{\mathrm{eff}}
  \approx
  \mathrm{CG}\!\Big(
    c\sqrt{\max(0,1-S(\chi))}
    + \kappa\,\Delta_G\chi
  \Big).
  \label{eq:D4_target_cg_dynamics}
\end{equation}

\paragraph{Residual metric.}
\begin{equation}
  \varepsilon(t)
  \equiv
  \frac{\bigl\|
    \partial_t\chi_{\mathrm{eff}}
    - \mathcal{R}_{\mathrm{eff}}
  \bigr\|}
  {\bigl\|\partial_t\chi_{\mathrm{eff}}\bigr\|}.
  \label{eq:D4_epsilon_def}
\end{equation}

\paragraph{Results.}
For $N=32$, $\ell_0=4$, $\Delta t = 0.03$: smooth run
$\varepsilon_{L^2} \approx 9.3 \times 10^{-3}$; rough run
$\varepsilon_{L^2} \approx 1.45 \times 10^{-2}$.
The same order of magnitude persists at $N=48$, confirming the
small-residual regime is not a resolution artifact.

\begin{figure}[t]
  \centering
  \includegraphics[width=0.78\linewidth]
    {part6/appD/D04_epsilon_vs_time_compare}
  \caption{Residual $\varepsilon(t)$ for smooth and rough runs,
    converging to $\sim 10^{-2}$.}
  \label{fig:D4-epsilon-time}
\end{figure}

\begin{figure}[t]
  \centering
  \includegraphics[width=0.78\linewidth]
    {part6/appD/%
D04_saturation_fraction_compare}
  \caption{Saturation fraction vs.\ time.
    The nonprojectable run rapidly reaches
    $f_{\mathrm{sat}} \simeq 1$.}
  \label{fig:D4-satfrac-compare}
\end{figure}

\begin{figure}[t]
  \centering
  \includegraphics[width=0.72\linewidth]
    {part6/appD/D04_residual_hist_smooth}
  \caption{Pointwise residual distribution (smooth run),
    centered around zero.}
  \label{fig:D4-residual-hist}
\end{figure}

\begin{figure}[t]
  \centering
  \includegraphics[width=0.47\linewidth]
    {part6/appD/D04_chi_eff_slice_smooth}
  \hfill
  \includegraphics[width=0.47\linewidth]
    {part6/appD/D04_residual_slice_smooth}
  \caption{Spatial slices: $\chi_{\mathrm{eff}}$ (left,
    smooth) and residual (right, no coherent
    long-wavelength structure).}
  \label{fig:D4-spatial-slices}
\end{figure}

\paragraph{Limitations.}
Numerical stability ($\varepsilon \ll 1$) does not imply
projectability.
Fully saturated configurations ($S_i > 1$ everywhere) yield trivially
small residuals without faithful geometric interpretation.
