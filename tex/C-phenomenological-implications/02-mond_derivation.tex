\subsection{Emergence and Evolution of an Effective MOND-like Acceleration Scale}
  \label{sec:mond_derivation}

  In Cosmochrony, the intrinsic arrow of time is encoded in the monotonic evolution of the fundamental
  field $\chi$, with $\partial_t \chi \geq 0$. At large scales and late cosmic times, when $\chi$ admits
  a quasi-homogeneous and isotropic description, this monotonic relaxation can be coarse-grained
  into an effective cosmological clock, analogous to the role played by cosmic time in standard
  FLRW cosmology~\cite{Ellis1971,Weinberg2008}.

  In such regimes, and only as an effective description, the temporal evolution of $\chi$ may be
  approximated by
  \begin{equation}
    \partial_t \chi \simeq H(t)\,\chi,
  \end{equation}
  where $H(t)$ denotes the emergent Hubble parameter associated with the global relaxation of the
  field. This relation should not be interpreted as a fundamental equation of motion, but as a
  phenomenological correspondence valid in the FLRW-like limit of the theory, in the same spirit as
  coarse-grained descriptions commonly employed in emergent gravity and cosmological averaging
  approaches~\cite{Buchert2000,Padmanabhan2010}.

  The local kinematic constraint governing the $\chi$ field,
  \begin{equation}
  (\partial_t \chi)^2 + |\nabla \chi|^2 = c^2,
  \end{equation}
  then implies that even in the absence of localized matter excitations, the cosmological evolution
  of $\chi$ generically induces a non-vanishing residual spatial gradient. In the homogeneous limit,
  this minimal gradient magnitude is given by
  \begin{equation}
    |\nabla \chi|_{\mathrm{\min}} = \sqrt{c^2 - (H(t)\chi)^2}.
  \end{equation}

  This residual gradient does not correspond to a directional force acting on test particles.
  Rather, it defines a background kinematic scale that constrains how additional, locally induced
  gradients can contribute to the effective dynamics. For a local observer, the associated scale may
  be expressed as an effective acceleration floor,
  \begin{equation}
    a_0(t) \sim c\,H(t),
  \end{equation}
  a relation that has long been noted empirically in the context of galactic dynamics and MOND-like
  phenomenology~\cite{Milgrom1983a,Milgrom2002,FamaeyMcGaugh2012}, but which here arises dynamically
  from the global relaxation of the $\chi$ field. Unlike Milgromian dynamics, where $a_0$ is postulated
  as a universal constant, Cosmochrony predicts that this scale evolves slowly with cosmic time,
  tracking the evolution of $H(t)$.

  When localized matter excitations are present, they induce additional spatial gradients
  $\nabla \chi_N$ that, in the weak-field and short-distance limit, reproduce the Newtonian scaling
  $|\nabla \chi_N| \propto M/r^2$.
  ue to the non-linear nature of the kinematic constraint, the total
  effective gradient is not a linear superposition of the cosmological and local contributions.
  Instead, the field dynamics enforces a saturation behavior: at sufficiently large radii, where the
  Newtonian contribution would otherwise vanish, the total gradient asymptotically approaches the
  cosmological floor set by $|\nabla \chi|_{\mathrm{\min}}$.

  In this regime, the resulting effective gravitational acceleration naturally approaches
  \begin{equation}
    g_{\mathrm{eff}} \simeq \sqrt{g_N\,a_0(t)},
  \end{equation}
  recovering the characteristic deep-MOND scaling originally identified by Milgrom~\cite{Milgrom1983a}
  without introducing an ad hoc interpolation function or modifying the underlying gravitational law.

  This interpretation offers two conceptual advantages over phenomenological MOND formulations:
  \begin{enumerate}
    \item \textbf{Cosmological scaling.} The acceleration scale $a_0$ is not fundamental but emerges
    from the cosmological state of the $\chi$ field. As a result, $a_0(t)$ is expected to be larger at
    earlier cosmic epochs, providing a potential observational handle through high-redshift galaxy
    kinematics, now becoming accessible with modern surveys~\cite{McGaugh2015}.
    \item \textbf{Environmental dependence.} Because the relaxation of $\chi$ is sensitive to the
    global and local distribution of gradients, the effective acceleration scale can be weakly modulated
    by environmental factors. This naturally incorporates effects analogous to the external field effect
    (EFE) discussed in MOND literature~\cite{BekensteinMilgrom1984,FamaeyMcGaugh2012} and allows for
    deviations from perfectly flat rotation curves in the far field or in strongly inhomogeneous
    environments.
  \end{enumerate}

  In this framework, flat galactic rotation curves do not signal a breakdown of Newtonian gravity
  nor the presence of unseen matter components. Instead, they arise as a kinematic projection of the
  global cosmological relaxation of $\chi$ onto local gravitational dynamics, with MOND-like behavior
  emerging as an effective, scale-dependent regime rather than as a fundamental modification of
  gravity.
