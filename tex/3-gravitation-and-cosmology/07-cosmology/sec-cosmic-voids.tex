% ----------------------------------------------------------------------------
% Section 7.8 --- Cosmic Voids as Maximal Relaxation Probes
% From former §12.9, condensed
% ----------------------------------------------------------------------------
\subsection{Cosmic Voids as Maximal Relaxation Probes}
\label{subsec:cosmic-voids}

Cosmic voids are regions where the relaxation of~$\chi$ is least
frustrated by localized excitations.
Within the effective description
(Section~\ref{subsec:variational-formulation}), near-maximal substrate
relaxation produces enhanced geodesic defocusing, leading to a negative
gravitational lensing signal and non-linear peculiar velocity outflows at
void boundaries---effects absent or suppressed in $\Lambda$CDM.

\paragraph{Connection to local $H_0$ determinations.}
Enhanced outward peculiar velocities at void boundaries can bias
low-redshift distance--redshift inferences toward higher locally inferred
expansion rates.
A decisive test is the cross-correlation between void lensing profiles
and locally inferred $H_0$ maps.

\paragraph{Phenomenological void parametrization.}
Observable void signals are modeled as a $\Lambda$CDM baseline plus a
saturating correction controlled by a single dimensionless amplitude
$\beta_{\textrm{void}}$:
\begin{align}
  \kappa_{\textrm{obs}}(R) &=
    \kappa_{\Lambda{\textrm{CDM}}}(R)
    \left[1+\beta_{\textrm{void}}\,
      \mathcal{S}\!\big(\mathcal{A}(R)\big)\right],\\
  v_{\textrm{obs}}(r) &=
    v_{\Lambda{\textrm{CDM}}}(r)
    \left[1+\beta_{\textrm{void}}\,
      \mathcal{S}\!\big(\mathcal{B}(r)\big)\right],
\end{align}
with $\mathcal{S}(x)=x/\sqrt{1+x^2}$ interpolating between linear and
saturated regimes.
