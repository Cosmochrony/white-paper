% ----------------------------------------------------------------------------
% Section 7.6 --- Dark Matter as Residual Relaxation Effects
% From former §12.7, condensed
% ----------------------------------------------------------------------------
\subsection{Dark Matter as Residual Relaxation Effects}
\label{subsec:dark-matter-phenomenology}

Dark matter phenomena correspond to configurations of~$\chi$ that resist
relaxation while failing the projectability conditions required for
Standard Model interactions.
These \textbf{non-projected spectral modes} possess inertial mass and
contribute to gravitational curvature while remaining invisible to
electromagnetic or electroweak probes.

\paragraph{Galactic Rotation and Effective Spectral Stiffness.}
The flattening of rotation curves is interpreted as a spatial variation of
$G_{\mathrm{eff}}$ induced by the local relaxation state.
At large radii, a transition in the effective spectral stiffness leads to
a logarithmic gravitational potential, reproducing MOND-like behavior
without invoking a modification of gravity or a universal acceleration
scale.
Unlike the universal constant~$a_0$ in MOND, the transition threshold
$\mathcal{K}_c$ is local and environment-dependent, naturally explaining
the observed variation of the apparent dark matter fraction among
galaxies.

\paragraph{Gravitational Lensing and Substrate Memory.}
Lensing phenomena (e.g.\ the Bullet Cluster) are manifestations of
\textbf{relaxation lag}: the projective geometry associated with
mass-solitons persists after baryonic gas has lost coherence.
Light deflection is treated as effective refraction within the spectral
gradient of the projected~$\chi$ geometry.

\paragraph{Predictive Distinction from Particulate Dark Matter.}
Cosmochrony predicts non-local correlations between gravitational mass
discrepancies and the global spectral age of a system, the absence of
sharp central cusps, a minimum smoothing scale imposed by the spectral
response, and \textbf{spectral echoes}---faint gravitational signatures
in regions where matter was previously present.
