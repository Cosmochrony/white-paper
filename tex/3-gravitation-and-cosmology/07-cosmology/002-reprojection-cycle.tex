% ----------------------------------------------------------------------------
% Section 7.2 --- Cosmological Cycles of Constraint and Reprojection
% From former §12.2, condensed
% ----------------------------------------------------------------------------
\subsection{Cosmological Cycles of Constraint and Reprojection}
\label{subsec:cosmic-reprojection-cycle}

The maximally constrained regime is not confined to the early universe.
It may be locally reapproached whenever structural constraints saturate,
most notably in black holes.
Deprojection does not destroy information but renders it inaccessible to
spacetime descriptions.
Reprojection is the restoration of descriptive projectability once
relational consistency conditions are satisfied.

Phenomena commonly associated with the quantum vacuum reflect the
persistent presence of reprojectable relational structures within~$\chi$.
The vacuum represents a regime of minimal yet non-vanishing
projectability.
Cosmological evolution involves a continuous interplay between global
relaxation, local reconfinement, deprojection, and reprojection across
scales.
