% ----------------------------------------------------------------------------
% Section 7.11 --- Effective Potential for Galactic Dynamics from
%                  chi-Relaxation Saturation
% From former §12.12, condensed while preserving all equations
% ----------------------------------------------------------------------------
\subsection{Effective Potential for Galactic Dynamics from
$\chi$-Relaxation Saturation}
\label{subsec:phi-eff-galaxies}

\paragraph{Operational status.}
The effective potential is defined only through observable kinematics:
\begin{equation}
  g_{\mathrm{eff}}(r) \equiv -\frac{d\Phi_{\mathrm{eff}}}{dr},
  \qquad
  v^2(r) = r\,\frac{d\Phi_{\mathrm{eff}}}{dr}.
\end{equation}

\paragraph{Emergent acceleration scale.}
The nonlinear $\chi$-relaxation constraint induces an effective
background kinematic scale
\begin{equation}
  a_0(t) \sim c\,H(t),
\end{equation}
weakly time-dependent, with a projection efficiency factor
$\eta = O(0.1)$ for galactic dynamics.

\paragraph{Asymptotic regimes and flat rotation curves.}
Let $g_N(r) = GM_b(r)/r^2$.
In the high-acceleration regime ($g_N \gg a_0$),
$g_{\mathrm{eff}} \simeq g_N$.
In the low-acceleration regime ($g_N \ll a_0$), saturation yields
\begin{equation}
  g_{\mathrm{eff}}(r)
    \simeq \sqrt{g_N(r)\,a_0(t)},
\end{equation}
implying
\begin{equation}
  \Phi_{\mathrm{eff}}(r)
  \simeq \sqrt{G M_b\,a_0(t)}\,
    \ln\!\left(\frac{r}{r_s}\right) + \mathrm{const.},
  \label{eq:phi-log}
\end{equation}
a logarithmic potential producing asymptotically flat rotation curves.
The transition radius is
$r_s(t) = \sqrt{G M_b / a_0(t)}$.

\paragraph{Minimal smooth interpolation.}
A parsimonious operational fit function is
\begin{equation}
  g_{\mathrm{eff}}(r)
  = \sqrt{g_N(r)^2 + a_0(t)\,g_N(r)}.
  \label{eq:geff-interp}
\end{equation}

\paragraph{Baryonic Tully--Fisher scaling.}
In the deep-saturation regime:
\begin{equation}
  v_\infty^4 \simeq G\,M_b\,a_0(t),
\end{equation}
with a mild redshift dependence through $a_0(t) \sim cH(t)$.

\begin{table}[t]
  \centering\small
  \begin{tabular}{p{2.8cm} p{3.8cm} p{3.8cm} p{3.8cm}}
    \hline
    \textbf{Aspect}
      & \textbf{$\Lambda$CDM}
      & \textbf{MOND}
      & \textbf{Cosmochrony} \\
    \hline
    Ontology
      & Non-baryonic halo
      & Modified dynamics
      & Relaxation properties of~$\chi$ \\
    \hline
    Flat rotation curves
      & Invisible mass
      & $g \simeq \sqrt{g_N a_0}$
      & Saturation:
        $g \simeq \sqrt{g_N a_0(t)}$ \\
    \hline
    Key scale
      & Halo profile parameters
      & Universal $a_0$
      & Emergent $a_0(t) \sim cH(t)$ \\
    \hline
    Tully--Fisher
      & Formation models
      & $v^4 \propto G M_b a_0$
      & $v^4 \propto G M_b a_0(t)$ \\
    \hline
    Discriminating signature
      & Cusps vs.\ cores
      & Strict universality
      & Slow evolution of $a_0(t)$ \\
    \hline
  \end{tabular}
  \caption{Conceptual comparison of flat-rotation-curve
    explanations.}
  \label{tab:dm-mond-cosmochrony}
\end{table}

\subsubsection*{Observed rotation curves: multi-galaxy test}

The Cosmochrony prediction is confronted with observed rotation curves
of NGC~3198 (flat), NGC~2403 (rising), and NGC~5055 (mildly declining).
No dark matter halo is introduced.
Baryonic contributions are taken from the literature; the only fitted
parameter is the stellar mass-to-light ratio~$\Upsilon_\star$.
The acceleration scale is fixed by $a_0(t_0) \sim cH(t_0)$.

\begin{figure}[t]
  \centering
  \includegraphics[width=\linewidth]
    {3-gravitation-and-cosmology/07-cosmology/galaxy_rotcurves_3panel}
  \caption{Observed rotation curves compared with the Cosmochrony
    saturation prediction.
    Left: NGC~3198. Center: NGC~2403. Right: NGC~5055.
    See Appendix~\ref{subsec:rotation-curve-fits} for data sources.}
  \label{fig:rotation-curves-comparison}
\end{figure}
