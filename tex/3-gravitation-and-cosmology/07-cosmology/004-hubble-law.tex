\subsection{Emergent Hubble Law}
  \label{subsec:emergent-hubble-law}

  In homogeneous regimes, the relaxation ordering is uniform and admits the
  linear representation
  \begin{equation}
    \chi(t) = \chi_0 + c\, t .
  \end{equation}
  Identifying effective spatial scales with accumulated relational
  differentiation yields a Hubble-like
  law~\cite{Hubble1929,Hogg1999} with
  \begin{equation}
    H(t) \equiv \frac{1}{\chi}\,\frac{d\chi}{dt},
  \end{equation}
  where $H_0$ quantifies the current state of global relaxation.

  \textbf{Observational discriminant}: Cosmochrony predicts a $\sim 5\%$
  higher $H(z)$ at $z\sim 1$ compared to $\Lambda$CDM, testable with
  DESI/Euclid
  (Section~\ref{subsec:hubble-constant-from-chi-dynamics}).

  \paragraph{Cosmic Acceleration Without Dark Energy.}
    \label{par:acceleration-without-dark-energy}
    The observed late-time acceleration does not require a cosmological
    constant.
    As structure formation proceeds, localized configurations increasingly
    constrain local relaxation, introducing growing spatial inhomogeneities.
    When interpreted within homogeneous and isotropic models, these
    inhomogeneities manifest as apparent acceleration.
    Cosmic acceleration is an emergent interpretative effect arising from
    progressively uneven relaxation across cosmic scales.
