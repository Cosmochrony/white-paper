\subsection{Collective Gravitational Coupling and Operational Geometry}
  \label{subsec:collective-gravitational-coupling-and-operational-geometry}

  The collective reduction of admissible relaxation ordering modulates how
  efficiently structural variations can be correlated between different
  effective locations.
  In regions where projected descriptions are nearly homogeneous, the
  effective coupling approaches a uniform value; localized configurations
  weaken it by introducing structural constraints.

  From an operational perspective, this modulation can be interpreted as a
  \emph{self-consistency overhead}: maintaining mutual compatibility of
  projected descriptions across separated regions requires an increasing
  consistency cost as local constraints accumulate.
  Inertial and gravitational responses arise precisely as manifestations of
  this overhead in collective updates of the effective state.

  Spatial separation is defined operationally: two effective regions are
  close if structural variations can be efficiently correlated between
  them.
  In the continuum and weak-constraint regime, this operational notion
  admits a compact description in terms of an effective spatial metric
  summarizing the collective response to relative variations.
  Spacetime curvature emerges as a descriptive manifestation of how
  localized configurations modulate collective ordering, correlation
  efficiency, and the associated self-consistency overhead.

  A more explicit relational construction of the coupling mechanism is
  presented in Appendix~\ref{subsec:collective-coupling}.
