\section{Testable Predictions and Observational Signatures}
  \label{sec:testable-predictions-and-observational-signatures}

  Before detailing specific observational signatures, it is important to clarify the status of the numerical estimates
  provided in this section. Values such as the $\sim 8-10\%$ correction to the Hubble constant or
  the $\sim 10^{-10} \text{ yr}^{-1}$ drift in effective constants are intended as order-of-magnitude consistency tests
  rather than precision measurements.
  These scales are derived from the fundamental geometric coupling between the $\chi$ field and the local density $\Omega_\chi$.
  They demonstrate that the Cosmochrony framework operates within a phenomenologically relevant regime without requiring
  fine-tuning, providing a bridge between the core dynamics and current cosmological tensions.

  \subsection{Hubble Constant from $\chi$ Dynamics}
    \label{subsec:hubble-constant-from-$chi$-dynamics}

    In Cosmochrony, the Hubble parameter is not a free cosmological parameter but follows directly from the relaxation
    dynamics of the $\chi$ field:
    \begin{equation}
      H(t) = \frac{\dot{\chi}}{\chi}.
    \end{equation}

    Assuming a maximal relaxation speed $\dot{\chi} \simeq c$, the present value becomes
    \begin{equation}
      H_0 \simeq \frac{c}{\chi(t_0)}.
    \end{equation}

    This relation predicts a direct correspondence between the observed Hubble constant and the characteristic
    wavelength of $\chi$ at the current cosmic epoch.
    Early-universe probes (e.g.\ CMB-based measurements) and late-time distance ladder measurements are therefore
    expected to yield systematically different values, reflecting different effective $\chi$ scales.

  \subsection{Redshift Drift}
    \label{subsec:redshift-drift}

    The monotonic increase of $\chi$ implies a slow temporal evolution of cosmological redshifts.
    The predicted redshift drift differs quantitatively from that of $\Lambda$
    CDM, particularly at intermediate redshifts.

    Future high-precision spectroscopic observations, such as those planned with extremely large telescopes, may
    distinguish between these predictions.

    The predicted redshift drift $\dot{z} \sim H_0 (1+z) - c / \chi(t)$ implies a secular change of
    $\Delta z \sim 10^{-10} \, \text{yr}^{-1}$ at $z \sim 1$,
    potentially detectable with next-generation spectroscopic surveys (e.g., ELT-HIRES). This differs from
    $\Lambda CDM$ predictions by $\sim 10\%$
    at intermediate redshifts, offering a direct test of the geometric vs. dark energy interpretations of cosmic
    acceleration.

  \subsection{Gravitational Wave Propagation}\label{subsec:gravitational-wave-propagation}

    Gravitational waves correspond to propagating modulations of the $\chi$ field.
    In regions of high excitation density, such as near compact objects, partial absorption or dispersion of these
    modulations is expected.

    This suggests small deviations from general relativistic predictions in the late-time tails of gravitational wave
    signals, potentially observable with next-generation detectors.

    Near compact objects, the absorption of gravitational waves by slowed $\chi$-relaxation is estimated at
    $\sim 10\%$ for waves passing within $10 \, GM/c^2$
    of a black hole horizon. This would manifest as a frequency-dependent attenuation in the ringdown phase of
    binary mergers, potentially detectable in LISA-era observations with signal-to-noise ratios exceeding 100.

  \subsection{Spin and Topological Signatures}\label{subsec:spin-and-topological-signatures}

    If particle spin arises from topological configurations of $\chi$,
    as proposed in this framework, then spin-related phenomena may exhibit subtle geometric signatures.

    In particular, interference experiments sensitive to $4\pi$
    rotational symmetry could probe deviations from standard quantum mechanical descriptions at extreme precision.

  \subsection{Absence of Dark Energy Signatures}\label{subsec:absence-of-dark-energy-signatures}

    Because cosmic acceleration emerges without invoking dark energy, Cosmochrony predicts the absence of dynamical
    dark energy signatures, such as evolving equation-of-state parameters.

    Observations consistent with a strictly geometric origin of acceleration would favor this interpretation.

    \paragraph{Discriminating observational signatures.}
      While a negligible primordial tensor contribution is not in itself discriminating, Cosmochrony predicts that the absence of an inflationary phase should manifest through correlated deviations in the large-scale CMB observables. These include a suppression of power at low multipoles, specific angular correlations in polarization, and the absence of an inflationary tensor imprint at large angular scales. The combination of these features, rather than any single parameter such as $r$, provides a potential observational discriminator with respect to standard inflationary cosmologies.

\subsection{Summary}\label{subsec:summary2}

  Cosmochrony yields testable predictions across cosmology, gravitation, and quantum phenomena.
  While most predictions reproduce existing observations, several offer quantitative differences that may be
  experimentally probed in future high-precision measurements.
