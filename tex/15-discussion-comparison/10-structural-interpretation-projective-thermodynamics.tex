\subsection{Structural Interpretation: Projective Thermodynamics}
  \label{subsec:structural-interpretation-projective-thermodynamics}

  This subsection provides a unifying structural interpretation of several results
  developed throughout this work.
  It does not introduce additional dynamical assumptions, nor modify the underlying
  substratic dynamics of $\chi$.
  Rather, it reformulates temperature, curvature, horizon structure, and related
  effective observables as necessary responses to non-injective projection under
  bounded relaxation.

  Within the Cosmochrony framework, physical observables arise through a projection
  $\Pi : \Omega \rightarrow O$ from the relational configuration space $\Omega$ of the
  $\chi$ substrate.
  As established in previous sections, this projection is generically non-injective
  and subject to spectral admissibility constraints.
  Distinct substratic configurations may therefore be structurally identified at the
  level of effective observables, while the underlying relaxation dynamics remains
  strictly local and bounded.

  In such circumstances, effective descriptions cannot retain full relational
  resolution.
  The resulting loss of distinguishability induces a structural entropy associated
  with the projection itself.
  This \emph{projection entropy} does not quantify epistemic ignorance about
  microscopic degrees of freedom, but reflects an objective compression of relational
  structure enforced by the admissibility criteria.
  Effective thermodynamic and geometric quantities emerge as compensatory parameters
  encoding this unresolved complexity.

  From this perspective, temperature should not be interpreted solely as a measure of
  microscopic agitation.
  Instead, it functions as a Lagrange multiplier absorbing relational information that
  cannot be resolved within the projected description.
  High effective temperatures may therefore arise in regimes where the projection
  becomes strongly non-injective, even when no corresponding increase in local
  substratic energy density is present.

  An analogous interpretation applies to geometric observables.
  Metric curvature, horizon formation, and related spacetime features act as effective
  descriptors compensating for the saturation of admissible relational flux.
  These quantities do not signal a breakdown of the underlying dynamics, but rather
  the progressive loss of validity of local spacetime-based descriptions in strongly
  projective regimes.

  The bounded character of substratic relaxation, enforced by the Born--Infeld-type
  constraint introduced earlier, plays a decisive role in this interpretation.
  It prevents the unbounded growth of effective parameters and ensures that
  projection-induced thermodynamic and geometric quantities remain finite.
  As a result, the framework remains predictive and falsifiable, despite the
  non-injective nature of the projection.

  This structural viewpoint unifies several phenomena often treated separately,
  including quantum correlations, anomalous thermodynamic behavior, and spacetime
  singularities.
  In all cases, apparent anomalies reflect the limits of projectability rather than
  the presence of exotic substratic dynamics.
  They mark transitions between descriptive regimes governed by the same underlying
  relational principles.
