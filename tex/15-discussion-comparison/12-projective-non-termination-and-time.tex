\subsection{Projective Non-Termination and the Condition of Temporal Ordering}
  \label{subsec:projective-non-termination}

  The existence of a finite bound on projective resolvability has implications that go
  beyond the saturation of effective responses.
  It constrains the very possibility of terminal states within the projected description.

  Let $\Omega$ denote the space of underlying relational configurations and
  $\Pi : \Omega \rightarrow \mathcal{U}$ the non-injective projection onto effective
  observable states.
  An effective temporal ordering exists only insofar as successive projected states are
  distinguishable, that is,
  $\Pi(\chi_{n+1}) \neq \Pi(\chi_n)$.

  A hypothetical terminal state $U_0 \in \mathcal{U}$ would correspond to a projective
  fixed point such that, for any admissible relational update $\delta \chi \in \Omega$,
  \begin{equation}
    \Pi(\chi + \delta \chi) = U_0 .
  \end{equation}
  In this regime, further variations of the underlying relational configuration would no
  longer induce any distinguishable update at the effective level.

  Within the present framework, time is not identified with the underlying relaxation of
  $\chi$, but with the ordering of distinguishable projected states.
  Operationally, the effective temporal increment satisfies
  $\Delta \tau \sim \mathrm{dist}(U_n, U_{n+1})$.
  A strict projective fixed point would therefore imply $\Delta \tau \rightarrow 0$ for
  all internal observers, even though the underlying relational dynamics may persist.

  This observation leads to a structural reinterpretation of absolute zero.
  A strictly terminal zero-temperature state would correspond to a projectively absorbing
  configuration, in which the finite projective bandwidth introduced in
  Section~\ref{subsec:projective-resolvability-bound} is entirely exhausted.
  Such a state would preclude the existence of any further projected update and is
  therefore incompatible with the definition of an effective temporal regime.

  From this perspective, the inaccessibility of absolute zero emerges as a projective
  necessity rather than as a purely thermodynamic limitation.
  The so-called zero-point fluctuations are reinterpreted not as residual thermal
  excitations, but as the minimal projective bandwidth required to prevent the collapse
  of the observable description into a stationary state.

  This interpretation departs from entropy-based accounts of the arrow of time.
  Thermodynamic equilibrium may be reached within the effective description, but temporal
  ordering persists as long as the projective state machine admits further updates.
  Time ceases only when the projection becomes stationary, a regime that is structurally
  forbidden by the existence of a finite bound on projective resolvability.
