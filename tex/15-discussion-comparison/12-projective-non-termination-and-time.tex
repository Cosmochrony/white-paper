\subsection{Projective Non-Termination and the Condition of Temporal Ordering}
  \label{subsec:projective-non-termination}

  The projective framework developed throughout this work implies a non-trivial
  constraint on terminal states of the effective description.
  This constraint bears directly on the interpretation of absolute zero and on the
  origin of the arrow of time.

  Let $\Omega$ denote the space of underlying relational configurations and
  $\Pi : \Omega \rightarrow \mathcal{U}$ the non-injective projection onto effective
  observable states.
  An effective temporal ordering is defined only insofar as successive projected
  states remain distinguishable, that is,
  $\Pi(\chi_{n+1}) \neq \Pi(\chi_n)$.

  A hypothetical terminal state $U_0 \in \mathcal{U}$ would correspond to a projective
  fixed point such that for any admissible relational update $\delta \chi \in \Omega$,
  \begin{equation}
    \Pi(\chi + \delta \chi) = U_0 .
  \end{equation}
  In this regime, no further variation of the underlying relational configuration would
  induce a distinguishable update at the effective level.

  Since time is defined operationally as the ordering of distinguishable projected
  states, $\Delta \tau \sim \mathrm{dist}(U_n, U_{n+1})$, a strict projective fixed point
  would imply $\Delta \tau \rightarrow 0$ for all internal observers.
  The effective temporal regime would therefore cease to exist, not because the
  underlying dynamics halts, but because no further state update is projectable.

  From this perspective, the inaccessibility of absolute zero emerges as a projective
  necessity rather than as a purely thermodynamic limitation.
  A strict zero-temperature state would correspond to a projectively absorbing
  configuration, incompatible with the existence of a temporal sequence of effective
  states.

  The so-called zero-point fluctuations are thus reinterpreted as the minimal projective
  bandwidth required to prevent the collapse of the observable description into a
  stationary state.
  They are not residual thermal excitations, but the minimal distinguishability
  required to maintain a non-degenerate state-transition sequence.

  This interpretation departs from entropy-based accounts of the arrow of time.
  While thermodynamic equilibrium may be reached in the effective description, the
  temporal ordering persists as long as the projective state machine admits further
  updates.
  Time ceases only when the projection becomes stationary, a regime that is
  structurally forbidden within the present framework.
