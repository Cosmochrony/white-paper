\subsection{Bounds on Projective Resolvability Across Scales}
  \label{subsec:projective-resolvability-bound}

  The preceding sections have established two central ingredients of the Cosmochrony framework.
  First, physical observables arise through a generally non-injective projection $\Pi$ from an
  underlying relational substrate $\chi$.
  Second, effective dynamical descriptions exhibit saturation phenomena across multiple regimes,
  including gravitational response in low-density environments and the accessibility of quantum
  correlations, without introducing additional ontological degrees of freedom.

  In this subsection, we propose a unifying interpretative perspective.
  We suggest that these saturation phenomena reflect a common limitation on the
  \emph{projective resolvability} of relational structure, rather than independent dynamical
  mechanisms operating at different physical scales.
  This limitation constrains the rate at which global relational structure encoded in $\chi$ can
  be rendered operationally accessible within emergent spacetime descriptions.

  \paragraph{Projective resolvability.}
    The projection $\Pi$ is not a temporal process.
    It is an atemporal structural mapping that defines the effective spacetime description rather
    than unfolding within it.
    However, any effective observable $x_{\mathrm{eff}}$ belongs to the emergent spacetime domain
    and is therefore subject to finite operational resolution.

    As a consequence, the accessibility of a given projected description is constrained by the
    resources of the effective observational framework.
    Temporal resolution does not characterize the formation of correlations or the propagation of
    information.
    It characterizes the minimal observation window required for a non-injective projected
    description to become operationally accessible as a stable element of the observable space
    $\mathcal{O}$.

    This distinction is essential for avoiding any interpretation in terms of hidden nonlocal
    dynamics or superluminal signaling.
    No physical influence propagates during this interval.
    Only the effective description stabilizes.

  \paragraph{Projectability versus observability.}
    It is useful to distinguish projectability from observability.
    Projectability refers to the existence of a regime in which a stable effective description can
    be defined.
    Observability refers to the finite time required for that description to become operationally
    resolvable within a given measurement protocol.

    The Planck time does not bound this observability time.
    It marks instead the scale at which a spacetime-based description ceases to be meaningful.
    Once a projectable regime exists, the timescale associated with observability is controlled by
    effective couplings, environmental interactions, and the degree of non-injectivity of the
    projection, rather than by Planckian considerations.

  \paragraph{Attosecond chronoscopy as a probe of resolvability.}
    Attosecond chronoscopy experiments provide a concrete illustration of this notion.
    Time-resolved delay measurements demonstrate that quantum coherence and entanglement are
    experimentally accessible through ultrafast probes, without implying any dynamical buildup of
    correlations or causal influence between spacelike separated regions~\cite{PhysRevLett.133.163201}.

    Within the Cosmochrony framework, these observations are naturally interpreted as probing the
    temporal scale at which a non-factorizable projected description becomes resolvable by a given
    measurement channel.
    The measured delays do not correspond to propagation times or interaction speeds.
    They reflect the minimal temporal resolution required for the global relational structure
    encoded in $\chi$ to imprint a stable non-factorizable signature on effective observables.

  \paragraph{A bound on projective accessibility.}
    Motivated by these considerations, we introduce the notion of a bound on the resolvable flux of
    relational structure.
    We posit the existence of an upper bound $b$ on the rate at which non-injective relational
    structure can become operationally accessible within emergent spacetime descriptions.

    Schematically, this limitation may be expressed as an effective inequality of the form
    \[
      \left| \frac{\partial \mathcal{O}_{\mathrm{eff}}}{\partial \tau} \right|
      \;\leq\;
      b \, \mathcal{S}_{\Pi},
    \]
    where $\mathcal{O}_{\mathrm{eff}}$ denotes the effective projected observable,
    $\tau$ is the operational time associated with the measurement procedure, and
    $\mathcal{S}_{\Pi}$ characterizes the structural complexity or entropic content of the
    non-injective projection.
    \footnote{Bounds of this form, expressed as inequalities on field derivatives rather than as
    dynamical equations, are characteristic of Born--Infeld-type theories.
    In such frameworks, the bound does not constrain propagation speed or causal structure,
      but limits the maximal admissible field response.
      The present inequality should be understood in the same spirit, as a bound on projective
      resolvability rather than as a dynamical law.}

    This inequality is not introduced as a dynamical law.
    It expresses an effective bound on projective resolvability.
    It limits the rate at which global relational structure can be rendered accessible to effective
    observables, without constraining physical propagation, causal influence, or underlying dynamics.

  \paragraph{Finite projective bandwidth.}
    At the operational level, the bound $b$ manifests as a finite projective bandwidth.
    This bandwidth characterizes the maximal rate at which independent relational constraints can be
    stabilized and resolved within a given effective observational channel.

    At cosmological and gravitational scales, the same bound appears as saturation of effective
    gravitational response in low-density regimes, quantified by the acceleration scale $a_{\star}$
    introduced earlier.
    At the level of quantum observability, it appears as a finite timescale associated with the
    accessibility of non-factorizable projected descriptions.
    These manifestations are not independent.
    They represent different operational expressions of a single underlying constraint on
    projective resolvability.

  \paragraph{Formal relation between bounds.}
    We formalize the correspondence between the fundamental saturation bound $b$ and its
    phenomenological manifestations through a set of scaling relations.
    Let $\eta$ denote a dimensionless efficiency factor characterizing how the non-injective
    projection $\Pi$ is operationally realized within a given physical regime.

    The relation between the fundamental resolvability bound $b$ and the gravitational
    saturation scale $a_{\star}$ may be expressed as
    \begin{equation}
      a_{\star} \approx \eta_{G} \, b \, c,
      \label{eq:a_star_b_relation}
    \end{equation}
    where $c$ appears as a conversion factor between structural flux at the projective level
    and effective kinematic acceleration in spacetime.
    Here $c$ does not play the role of a propagation speed.
    It enters solely as the universal scale relating temporal and spatial projective units
    within emergent spacetime descriptions.

    Similarly, the operational projective bandwidth $B_{\Pi}$ associated with a measurement
    protocol $\mathcal{M}$ is related to the same bound $b$ through the structural complexity
    of the accessible projection channel,
    \begin{equation}
      B_{\Pi}(\mathcal{M}) = \eta_{\mathcal{M}} \, b \, \mathcal{S}_{\Pi},
      \label{eq:bandwidth_b_relation}
    \end{equation}
    where $\mathcal{S}_{\Pi}$ characterizes the effective structural entropy associated with
    the non-injective projection relevant to the protocol.

    These relations imply that $a_{\star}$ and $B_{\Pi}$ are not independent parameters, nor
    merely analogous quantities.
    They are distinct dimensional expressions of the same underlying bound on projective
    resolvability.
    A modification of the fundamental resolvability of the $\chi \rightarrow \Pi$ mapping
    would therefore induce correlated shifts in the saturation scale of galactic dynamics
    and in the observability timescales accessible to quantum chronoscopy.

  \paragraph{Relation to saturation phenomena across scales.}
    From this perspective, saturation phenomena observed in galactic dynamics and the finite temporal
    resolution associated with quantum correlations are not unrelated coincidences.
    They reflect a common structural limitation of effective descriptions derived from a
    non-injective relational substrate.

    Further reduction of density, increase of spatial resolution, or refinement of temporal probing
    does not yield proportionally stronger effective response once the projective bound is reached.
    Instead, effective observables approach a regime of saturation governed by $b$ and its
    scale-dependent manifestations.

  \paragraph{Uniqueness of the saturation bound.}
    The quantities $b$, $a_{\star}$, and the projective bandwidth $B_{\Pi}$ should not be interpreted
    as independent constants.
    They represent distinct manifestations of a single underlying saturation bound, expressed at
    different descriptive and operational levels.

    The parameter $b$ denotes the fundamental bound on projective resolvability at the level of the
    $\chi \rightarrow \Pi$ mapping.
    The acceleration scale $a_{\star}$ is its effective realization in gravitational and cosmological
    phenomenology.
    The projective bandwidth $B_{\Pi}$ characterizes how the same bound is operationally accessed
    within a given observational channel.

    No additional fundamental scale is introduced.
    All three quantities encode the same limitation on effective accessibility, translated into the
    language appropriate to the regime under consideration.

  \paragraph{Interpretative status.}
    We emphasize that the considerations presented here are interpretative in nature.
    They do not modify quantum mechanics, relativistic causality, or gravitational dynamics at the
    fundamental level.
    Their role is to clarify how diverse saturation effects across scales may admit a common
    structural origin within the Cosmochrony framework.

    This perspective suggests that bounds on effective accessibility, rather than limits on physical
    propagation, play a central role in shaping the phenomenology of both quantum correlations and
    large-scale gravitational dynamics.
