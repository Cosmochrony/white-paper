\subsection{Bounds on Projective Resolvability Across Scales}
  \label{subsec:projective-resolvability-bound}

  The preceding sections have established two central ingredients of the Cosmochrony
  framework.
  First, physical observables arise through a generally non-injective projection
  $\Pi$ from an underlying relational substrate $\chi$.
  Second, effective dynamical descriptions exhibit saturation phenomena, both in
  gravitational response and in quantum correlations, without introducing additional
  ontological degrees of freedom.

  In this subsection, we propose a unifying interpretative perspective.
  We suggest that these saturation phenomena reflect a common limitation on the
  \emph{projective resolvability} of relational structure, rather than independent
  dynamical mechanisms operating at different physical scales.

  \paragraph{Projective resolvability.}
    The projection $\Pi$ is not a temporal process.
    It is an atemporal structural mapping that defines the spacetime description rather
    than unfolding within it.
    However, any effective observable $x_{\mathrm{eff}}$ belongs to the emergent
    spacetime domain and is therefore subject to finite operational resolution.
    As a result, the accessibility of a given projected description is constrained by
    the resources of the effective observational framework.

    In this sense, temporal resolution does not characterize the formation of
    correlations or the propagation of information.
    It characterizes the minimal observation window required for a non-injective
    projected description to become operationally accessible as a stable element of
    $\mathcal{O}$.
    This distinction is essential for avoiding any interpretation in terms of hidden
    nonlocal dynamics or superluminal signaling.

  \paragraph{Attosecond chronoscopy as a probe of resolvability.}
    Attosecond chronoscopy experiments provide a concrete illustration of this notion.
    Time-resolved delay measurements demonstrate that quantum coherence and entanglement
    are experimentally accessible through ultrafast probes, without implying any
    dynamical buildup of correlations or causal influence between spacelike separated
    regions~\cite{PhysRevLett.133.163201}.

    Within the Cosmochrony framework, these observations are naturally interpreted as
    probing the temporal scale at which a non-factorizable projected description becomes
    resolvable by a given observable.
    The measured delays do not correspond to propagation times or interaction speeds.
    They instead reflect the minimal temporal resolution required for the global
    relational structure encoded in $\chi$ to imprint a non-factorizable signature on
    effective observables.

  \paragraph{A bound on projective accessibility.}
    Motivated by these considerations, we introduce the notion of a bound on the
    resolvable flux of relational structure.
    We posit the existence of an upper bound $b$ on the rate at which non-injective
    relational structure can become operationally accessible within emergent spacetime
    descriptions.

    Schematically, this limitation may be expressed as an effective inequality of the
    form
    \[
      \left| \frac{\partial \mathcal{O}_{\mathrm{eff}}}{\partial \tau} \right|
      \;\leq\;
      b \, \mathcal{S}_{\Pi},
    \]
    where $\mathcal{O}_{\mathrm{eff}}$ denotes the effective projected observable,
    $\tau$ is the operational time associated with the measurement procedure, and
    $\mathcal{S}_{\Pi}$ characterizes the structural complexity or entropic content of
    the non-injective projection.

    This inequality is not introduced as a dynamical law.
    It expresses an effective bound on projective resolvability.
    It limits the rate at which global relational structure can be rendered accessible
    to effective observables, without constraining physical propagation, causal
    influence, or underlying dynamics.

  \paragraph{Relation to saturation phenomena.}
    At macroscopic scales, an analogous limitation manifests as saturation of effective
    gravitational response in low-density regimes.
    The acceleration scale $a_{\star}$ introduced earlier may be interpreted as the
    gravitational counterpart of the same underlying constraint on projective
    resolvability.
    In both cases, further reduction of density or increase of observational resolution
    does not yield proportionally stronger effective response.

    From this perspective, saturation phenomena observed in galactic dynamics and the
    finite temporal resolution of quantum correlations are not unrelated coincidences.
    They reflect a common structural limitation of effective descriptions derived from
    a non-injective relational substrate.

  \paragraph{Interpretative status.}
    We emphasize that the considerations presented here are interpretative in nature.
    They do not modify quantum mechanics, relativistic causality, or gravitational
    dynamics at the fundamental level.
    Their role is to clarify how diverse saturation effects across scales may admit a
    common structural origin within the Cosmochrony framework.

    This perspective suggests that bounds on effective accessibility, rather than limits
    on physical propagation, may play a central role in shaping the phenomenology of
    both quantum correlations and large-scale gravitational dynamics.
