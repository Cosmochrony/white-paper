\clearpage

\section{Introduction and Motivation}
  \label{sec:introduction-and-motivation}

  Modern fundamental physics is built upon two highly successful yet conceptually
  distinct frameworks: quantum mechanics and general
  relativity~\cite{Dirac1930,Einstein1915}.
  Quantum theory accurately describes microscopic phenomena, while general
  relativity provides a geometric account of gravitation and spacetime dynamics
  at macroscopic and cosmological scales.
  Despite their empirical success, these theories rely on incompatible
  foundational assumptions and resist unification within a single coherent
  conceptual framework~\cite{MisnerThorneWheeler1973,Weinberg1972,Rovelli2004}.

  A central difficulty underlying this tension concerns the status of spacetime
  itself.
  In general relativity, spacetime geometry is dynamical and responds to matter,
  yet it remains the arena within which all physical processes are formulated.
  In quantum theory, time is treated as an external parameter, while spatial
  locality is assumed as a primitive input.
  These roles are difficult to reconcile in regimes where spacetime itself is
  expected to emerge or lose operational meaning, such as near cosmological
  origins or in strongly quantum gravitational settings.

  It is important to emphasize that this tension does not, by itself, require
  spacetime to have a temporal origin.
  Scenarios in which spacetime is eternal, cyclic, or undergoes conformal
  transitions remain logically and physically consistent.
  The difficulty addressed here is therefore not the existence of spacetime, but
  the explanatory sufficiency of spacetime-based descriptions for accounting for
  their own structural properties.

  Any attempt to account for the emergence, persistence, or breakdown of
  spacetime descriptions faces a conceptual constraint: spacetime cannot, on its
  own, provide a complete account of the conditions under which its notions of
  time, causality, locality, and dynamics become meaningful.
  Descriptions that presuppose temporal evolution, spatial propagation, or
  geometric structure are necessarily limited when used to explain how such
  notions arise or cease to apply.
  This suggests that spacetime, even if taken as fundamental or eternal, may not
  be explanatorily self-sufficient.

  A closely related motivation concerns a broad class of phenomena commonly
  described as emerging from the vacuum.
  In contemporary quantum field theories, effects such as zero-point energies,
  vacuum fluctuations, virtual processes, and renormalized self-energies are
  routinely invoked to account for observable phenomena.
  While these constructs are operationally successful, they raise a conceptual
  tension similar to that encountered for spacetime itself.

  If physical observables are defined entirely within spacetime-based descriptions,
  then entities attributed to the vacuum cannot originate from observable degrees
  of freedom alone.
  Their explanatory role therefore implicitly points toward a deeper level of
  description, one that is not itself accessible as a physical observable.

  From this standpoint, the vacuum should not be regarded as a physically populated
  medium, but as a manifestation of structural limitations of the effective
  description.
  Apparent vacuum-related phenomena then signal the presence of unresolved
  degrees of freedom associated with a non-observable, infra-physical, and
  pre-geometric level of description.

  This observation reinforces the need for a framework in which both spacetime
  structure and vacuum-related effects arise as effective concepts, emerging from
  the same underlying constraints.
  The question is therefore not what fills the vacuum, but what structural
  conditions render such effective descriptions necessary in the first place.

  From this perspective, the problem of unification is not merely one of combining
  dynamical laws within spacetime, but of identifying at least one additional
  level of description from which spacetime, causality, and dynamical evolution
  can appear as effective concepts.
  Such a level need not provide an ultimate origin of physical reality.
  Its role is more modest: to supply the minimal structural conditions required
  for spacetime-based descriptions to be applicable at all.

  The framework developed in this work, referred to as \emph{Cosmochrony}\footnote{%
  From \textit{$\kappa\acute{o}\sigma\mu o\varsigma$} and
  \textit{$\chi\rho\acute{o}\nu o\varsigma$}, denoting a framework in which
  cosmic structure and temporal ordering are treated as emergent.},
  is motivated by this requirement.
  It explores whether the emergence of spacetime geometry, gravitation, and
  quantum phenomena can be understood as consequences of deeper structural
  constraints, without assuming spacetime itself to be explanatorily complete.
  At this stage, no specific realization of such a structure is assumed.

  The present introduction is therefore concerned exclusively with the
  \emph{necessity} of a pre-geometric level of description.
  The definition of the underlying entity, its ontological status, and the
  minimal assumptions governing its admissible descriptions are introduced
  systematically in
  Section~\ref{sec:definition-and-fundamental-properties-of-the-chi-field}.

  Cosmochrony does not aim to replace the Standard Model or general relativity in
  their empirically validated domains, nor does it claim to provide a final
  unification of quantum theory and gravitation.
  It offers an exploratory and internally coherent framework designed to clarify
  why notions such as time, geometry, gravitation, and quantum correlations may
  require a common explanatory level beyond spacetime-based descriptions.

  \input{1-foundations/01-introduction-and-motivation/001-sec-unification}
  \input{1-foundations/01-introduction-and-motivation/002-sec-minimalism}
  \input{1-foundations/01-introduction-and-motivation/003-sec-time}
  \subsection{Conceptual Context and Related Approaches}
  \label{subsec:conceptual-context-and-related-approaches}

  The idea that spacetime geometry and gravitation may be emergent has been
  explored in a variety of contemporary theoretical frameworks.
  Several approaches interpret the spacetime metric as an effective description
  arising from deeper geometric, informational, or dynamical structures, and
  recast gravitation as a collective
  phenomenon~\cite{Nye2024,Singh2025}.
  Cosmochrony belongs to this broad conceptual lineage while adopting a
  deliberately minimalist ontological stance: a single pre-geometric relational
  substrate~$\chi$ whose irreversible relaxation governs the emergence of
  physical observables.

  Like Loop Quantum Gravity (LQG), Cosmochrony holds that spacetime geometry is
  not fundamental~\cite{Rovelli2004}.
  However, the two frameworks operate at distinct conceptual levels.
  LQG provides a quantized description of geometry once a spacetime structure is
  already assumed, encoding areas and volumes through spin networks and
  holonomies.
  Related ideas have also appeared in modern holographic approaches formulated in
  asymptotically flat spacetimes, such as celestial holography, where scattering
  amplitudes are reorganized as conformal correlators on the celestial
  sphere~\cite{StromingerInfraredLectures}.

  Beyond approaches that focus on the quantization or reformulation of geometry,
  several recent proposals have emphasized the possibility that certain physical
  entities or processes may admit a description that is effectively
  non-temporal or not fully embedded within spacetime.
  In particular, arguments based on null worldlines in relativistic kinematics
  have motivated interpretations in which photons are described as experiencing
  no proper time between emission and absorption, leading to proposals in which
  light is treated as an atemporal or pre-spacetime entity~\cite{McKinley2025}.
  Related speculative frameworks have also invoked underlying non-dynamical or
  quasi-static substrates, sometimes described as motionless or pre-kinematic,
  from which spacetime structure and dynamical behavior are argued to emerge.

  While such approaches share with Cosmochrony the motivation to question the
  fundamental status of spacetime and time, they differ at a structural level.
  In Cosmochrony, the pre-geometric substrate is neither static nor timeless in
  a kinematic sense.
  Instead, it is characterized by an intrinsic and irreversible relaxation
  process, from which temporal ordering and dynamical notions arise only at the
  level of effective, projected descriptions.
  The emphasis is therefore not placed on the atemporality of specific entities,
  nor on a frozen underlying substrate, but on the finite projectability and
  bounded relaxation of relational structure.

  Cosmochrony thus addresses an earlier and more primitive explanatory level.
  It does not quantize geometry, nor does it posit timeless physical entities as
  fundamental.
  Rather, it seeks to explain how geometric and temporal notions themselves arise
  as effective, coarse-grained descriptions of underlying $\chi$-configurations.
  The emergence of spacetime is mediated by a non-injective projection from the
  pre-geometric substrate to effective observables, allowing geometric,
  dynamical, and quantum features to appear only once specific relational and
  spectral conditions are met.
  From this perspective, Cosmochrony does not compete with LQG or related
  frameworks but conceptually precedes them: it aims to account for the physical
  origin of the geometric and temporal degrees of freedom that may subsequently
  be treated within spacetime-based or quantized descriptions, while remaining
  agnostic about the detailed form of their effective implementation.

  % ----------------------------------------------------------------------------
% Section 1.5 --- Scope and Limitations
% From former §2.5
% ----------------------------------------------------------------------------
\subsection{Scope and Limitations}
\label{subsec:context-and-motivation-scope-and-limitations}

The aim of this work is exploratory rather than definitive.
Cosmochrony does not seek to replace established theories within their
empirically validated domains, but to offer a coherent reinterpretation that
may clarify persistent conceptual difficulties concerning time, geometry, and
quantization.
Emphasis is placed on internal consistency, conceptual clarity, and qualitative
contact with observable phenomena, while openly acknowledging open questions
and limitations.

  \subsection{Structure of the Paper}
\label{subsec:structure-of-the-paper}

The paper is organized in five parts.
Part~I introduces the $\chi$ substrate, its minimal structural properties, and
its ontological status
(Section~\ref{sec:definition-and-fundamental-properties-of-the-chi-field}).
Part~II develops the effective dynamics of~$\chi$
(Section~\ref{sec:dynamical-equation-for-the-chi-field}), the emergence of
particle-like excitations
(Section~\ref{sec:particles-as-localized-excitations-of-the-chi-field}), and
the projection fiber with gauge emergence
(Section~\ref{sec:projection-gauge}).
Part~III addresses gravity as a collective effect
(Section~\ref{sec:gravity-as-a-collective-effect-of-particle-excitations}) and
cosmological implications
(Section~\ref{sec:cosmology}).
Part~IV presents the unified treatment of quantum phenomena, entanglement, and
the relation to quantum formalism
(Section~\ref{sec:quantum-phenomena-and-entanglement}).
Part~V collects radiation and quantization
(Section~\ref{sec:radiation-and-quantization}), the spectral mass spectrum
(Section~\ref{sec:spectral-mass-hierarchy}), testable predictions
(Section~\ref{sec:testable-predictions-and-observational-signatures}), and the
discussion with conclusion
(Section~\ref{sec:discussion-comparison-existing-frameworks}).
Appendices~\ref{sec:appendix-math}--\ref{sec:glossary} provide
mathematical foundations, conceptual extensions, cosmological and numerical
supplements, and a glossary of notation.

