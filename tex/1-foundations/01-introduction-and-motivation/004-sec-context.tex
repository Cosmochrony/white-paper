\subsection{Conceptual Context and Related Approaches}
  \label{subsec:conceptual-context-and-related-approaches}

  The idea that spacetime geometry and gravitation may be emergent has been
  explored in a variety of contemporary theoretical frameworks.
  Several approaches interpret the spacetime metric as an effective description
  arising from deeper geometric, informational, or dynamical structures, and
  recast gravitation as a collective phenomenon~\cite{Nye2024,Singh2025}.
  Cosmochrony belongs to this broad conceptual lineage while adopting a
  deliberately minimalist ontological stance: a single pre-geometric relational
  substrate~$\chi$, whose irreversible relaxation governs the emergence of
  physical observables.

  Like Loop Quantum Gravity (LQG), Cosmochrony holds that spacetime geometry is
  not fundamental~\cite{Rovelli2004}.
  However, the two frameworks operate at distinct explanatory levels.
  LQG provides a quantized description of geometry once a spacetime structure is
  already assumed, encoding areas and volumes through spin networks and
  holonomies.
  Related ideas have also appeared in modern holographic approaches formulated in
  asymptotically flat spacetimes, such as celestial holography, where scattering
  amplitudes are reorganized as conformal correlators on the celestial
  sphere~\cite{StromingerInfraredLectures}.
  These approaches reformulate or quantize geometric degrees of freedom, but do
  not address the origin of spacetime geometry itself.

  Beyond frameworks centered on the quantization or reformulation of geometry,
  several recent proposals have emphasized that specific physical entities or
  processes may not admit a fully spacetime-embedded description.
  In particular, arguments based on null worldlines in relativistic kinematics
  have motivated interpretations in which photons are described as experiencing
  no proper time between emission and absorption, leading to conceptual models in
  which light is treated as effectively atemporal or as residing outside the
  standard spacetime description~\cite{McKinley2025}.
  Other approaches have similarly explored descriptions in which physical objects
  are interpreted as stable modal or vibrational structures of an underlying
  substrate, with spacetime notions emerging only at an effective level.

  While these perspectives share with Cosmochrony the motivation to question the
  fundamental status of spacetime and time, they differ at a structural level.
  In Cosmochrony, the pre-geometric substrate is neither static nor timeless in a
  kinematic sense.
  Instead, it is characterized by an intrinsic and irreversible relaxation
  process, from which temporal ordering and dynamical notions arise only at the
  level of effective, projected descriptions.
  The emphasis is therefore not placed on the atemporality of specific entities,
  nor on the postulation of an underlying field or substrate with prescribed
  dynamics, but on the finite projectability and bounded relaxation of relational
  structure.

  Cosmochrony thus addresses an earlier and more primitive explanatory level.
  It neither quantizes geometry nor posits timeless physical entities as
  fundamental.
  Rather, it seeks to explain how geometric and temporal notions themselves arise
  as effective, coarse-grained descriptions of underlying $\chi$-configurations.
  The emergence of spacetime is mediated by a non-injective projection from the
  pre-geometric substrate to effective observables, allowing geometric,
  dynamical, and quantum features to appear only once specific relational and
  spectral conditions are met.
  From this perspective, Cosmochrony does not compete with LQG or related
  frameworks but conceptually precedes them, aiming to account for the physical
  origin of the geometric and temporal degrees of freedom that may subsequently
  be treated within spacetime-based or quantized descriptions.
