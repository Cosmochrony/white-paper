% ----------------------------------------------------------------------------
% Section 1.6 --- Structure of the Paper
% New section replacing the scattered plan paragraphs
% ----------------------------------------------------------------------------
\subsection{Structure of the Paper}
\label{subsec:structure-of-the-paper}

The paper is organized in five parts.
Part~I introduces the $\chi$ substrate, its minimal structural properties, and
its ontological status
(Section~\ref{sec:definition-and-fundamental-properties-of-the-chi-field}).
Part~II develops the effective dynamics of~$\chi$
(Section~\ref{sec:effective-dynamics-of-the-chi-substrate}), the emergence of
particle-like excitations
(Section~\ref{sec:particles-as-localized-excitations-of-the-chi-field}), and
the projection fiber with gauge emergence
(Section~\ref{sec:projection-fiber-and-gauge-emergence}).
Part~III addresses gravity as a collective effect
(Section~\ref{sec:gravity-as-a-collective-effect-of-particle-excitations}) and
cosmological implications
(Section~\ref{sec:cosmological-implications}).
Part~IV presents the unified treatment of quantum phenomena, entanglement, and
the relation to quantum formalism
(Section~\ref{sec:quantum-phenomena-and-entanglement}).
Part~V collects radiation and quantization
(Section~\ref{sec:radiation-and-quantization}), the spectral mass spectrum
(Section~\ref{sec:spectral-mass-spectrum-and-hierarchy}), testable predictions
(Section~\ref{sec:testable-predictions-and-observational-signatures}), and the
discussion with conclusion
(Section~\ref{sec:discussion-and-comparison-with-existing-frameworks}).
Appendices~\ref{appendix:math}--\ref{appendix:relational-glossary} provide
mathematical foundations, conceptual extensions, cosmological and numerical
supplements, and a glossary of notation.
