\subsection{Ontological Interpretation}
  \label{subsec:chi_ontology}

  \subsubsection*{The \texorpdfstring{$\chi$}{χ} Substrate as a Pre-Temporal
  Relational Reservoir}
    \label{subsubsec:the-chi-substrate-as-a-pre-temporal-relational-reservoir}

    The substrate~$\chi$ admits an ontological interpretation as a pre-temporal
    relational reservoir from which spacetime, matter, and effective physical
    laws emerge.
    It is not a dynamical history, nor a prescriptive blueprint of admissible
    universes, but a relational support supplying a continuous flux of
    configurational possibilities.
    Temporal succession is emergent, corresponding to an oriented ordering of
    projected configurations under irreversible relaxation.

    The term ``pre-temporal'' emphasizes that $\chi$ does not encode time or
    causal succession as primitive notions.
    Rather, time arises as a property of stable projected descriptions that
    remain mutually compatible under successive updates.
    No teleology, determinism, or block-universe ontology is implied: multiple
    effective histories may correspond to the same relational substrate through
    non-injective projection.

  \subsubsection*{Relational Ontology and Conceptual Lineage}
    \label{subsubsec:relational-ontology-and-conceptual-lineage}

    The relational character of~$\chi$ bears a conceptual affinity with
    relational approaches in physics, notably those of
    Rovelli~\cite{Rovelli1996,Rovelli2004}, which trace part of their lineage
    to Aristotelian relational ontology~\cite{AristotleCategories,Shields2016}.
    Cosmochrony shares the rejection of intrinsic, observer-independent
    properties but extends relationalism to a deeper ontological level:
    $\chi$ configurations are not relations \emph{between} fundamental objects
    but relational structures that give rise to objects only upon projection.

    Relativistic causality and spacetime locality emerge without postulating
    spacetime as fundamental~\cite{Rovelli2018}.
    Relationality is therefore not a property of spacetime itself, but of the
    pre-geometric substrate whose projected stabilization yields spacetime as
    an effective description.

  \subsubsection*{Projection, Reality, and Ontological Asymmetry}
    \label{subsubsec:projection-reality-and-ontological-asymmetry}

    The emergence of spacetime and physical law is a \emph{projection} from~$\chi$,
    not a dual or bidirectional description.
    The projected universe is fully real at the level of physical experience,
    but ontologically derivative: spacetime entities and dynamical laws do not
    possess ontological primacy.

    While all physical descriptions depend on the projection of~$\chi$, the
    admissibility of a given projection is constrained by the compatibility of
    the update with the current effective state~$U$.
    This introduces an ontological asymmetry: $\chi$ supplies configurations,
    whereas $U$ restricts which projected descriptions can remain stable under
    successive updates.

    Apparent fine-tuning is thus reinterpreted as a projective selection effect:
    only those updates preserving the coherence of an effective state appear as
    physically realized universes.
    Cosmochrony does not postulate a multiverse; the universe is unique at the
    level of physical reality, even though its relational description in terms
    of~$\chi$ is non-unique.

    Formal developments related to projection, including its fiber-bundle
    formulation and the emergence of gauge interactions, are developed in
    Section~\ref{sec:projection-gauge}.

  \subsubsection*{Configurational State Structure}
    \label{subsubsec:chi_state_structure}

    The substrate~$\chi$ defines a configurational space of relational
    structures and an irreversible flux of candidate configurations.
    It does not prescribe a fixed set of admissible macroscopic states.
    Physical reality corresponds instead to projected realizations that remain
    stable under finite resolution and compatibility constraints imposed by the
    current effective state~$U$.

    What appears as temporal evolution corresponds, at the level of~$\chi$, to
    an ordering of projected configurations required to maintain descriptive
    coherence under continuous relaxation.
    Universal bounds such as~$c_\chi$ and~$\hbar_\chi$ reflect invariant limits
    of the projection and update process, rather than prescriptive constraints
    encoded in $\chi$ itself.

  \subsubsection*{Intrinsic Structural Indeterminacy}
    \label{subsubsec:intrinsic-structural-indeterminacy}

    A perfectly closed and fully symmetric relational substrate would admit a
    static, non-updating projected description.
    Cosmochrony therefore assumes an \emph{intrinsic structural openness} of
    $\chi$: relational configurations are not exhaustively specified by a
    finite, closed set of conditions.

    This indeterminacy is ontological rather than dynamical.
    Observable variability and probabilistic behavior arise only at the level
    of projected descriptions, as a consequence of non-injective projection.
    Randomness is therefore \emph{projective}: it reflects the multiplicity of
    effective realizations compatible with a given relational substrate and a
    given effective state~$U$.

  \subsubsection*{Relation to Holographic Descriptions}
    \label{subsubsec:clarifying-holography}

    Cosmochrony is not a holographic theory in the technical sense.
    It does not posit a lower-dimensional boundary description or a dual
    equivalence between bulk and boundary physics.
    The limitation of physically accessible information within a spacetime
    region reflects the degeneracy of underlying $\chi$ configurations
    corresponding to the same effective projection, as constrained by finite
    resolution and compatibility.

    Scaling behaviors reminiscent of holography are interpreted as emergent
    signatures of projection.
    Similarly, Cosmochrony differs from thermodynamic approaches to gravity in
    that it does not posit entropy or information as primitive quantities.
    Thermodynamic descriptions arise only at the effective level, as secondary
    languages applicable when projected configurations admit coarse-grained
    statistical interpretations.
