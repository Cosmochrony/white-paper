% ----------------------------------------------------------------------------
% Section 2.11 --- Energy, Mass, and Fundamental Constants
% Merges former §4.6--4.10, §4.12
% ----------------------------------------------------------------------------
\subsection{Energy, Mass, and Fundamental Constants}
\label{subsec:energy-as-capacity-for-relaxation}

\subsubsection*{Energy as Capacity for Relaxation}

Energy is the effective capacity of projected $\chi$ configurations to relax
unresolved structural constraints.
It is not a fundamental conserved substance but an emergent quantity
characterizing the degree to which a given projected configuration retains
the ability to undergo further relaxation.
Conservation laws arise only at the effective level, as structural
regularities of projected dynamics.
Without intrinsic structural indeterminacy
(Section~\ref{subsubsec:intrinsic-structural-indeterminacy}), the notion of
energy would be ill-defined: a fully determined substrate would admit no
unresolved tension and therefore no capacity for relaxation.
The quantitative formulation is developed in
Section~\ref{subsec:mass_as_resistance}.

\subsubsection*{Mass as Frozen Information}
\label{subsec:mass-as-frozen-information}

Localized and long-lived configurations of~$\chi$ correspond to regions in
which further relaxation is strongly inhibited.
Such configurations trap a fixed amount of unresolved structural
information, preventing it from participating in the global relaxation
process.
Mass represents \emph{frozen energy}: structural information whose capacity
for further relaxation has been locally
suppressed\footnote{%
  This notion of frozen structural information should be distinguished from
  projection-induced entropy, which quantifies loss of distinguishability
  under non-injective projection.}.
Particle annihilation, decay, or radiation emission correspond to the
partial or complete release of frozen structural information.
The operational definition $m = E/c^2$ and its derivation from resistance
to relaxation ordering are developed in
Section~\ref{subsec:mass_as_resistance}.

\subsubsection*{Quarks as Non-Projectable Internal Modes}
\label{subsec:quarks-non-projectable-modes}

Quarks are not independent localized excitations but internal structural
modes of composite solitonic configurations.
They are required to characterize the internal organization of hadronic
excitations but do not admit an autonomous projection into spacetime.
Confinement reflects a structural constraint: isolated quark-like modes do
not correspond to admissible standalone projections
of~$\chi_{\mathrm{eff}}$.
Only collective configurations in which such internal modes are
topologically and relationally closed admit a stable spacetime
manifestation.
Quark confinement thus appears as a direct consequence of the non-injective
character of the projection.

\subsubsection*{The Role of the Universal Bound
\texorpdfstring{$c_\chi$}{cχ}}
\label{subsec:role-of-cchi}

The universal invariant bound~$c_\chi$ characterizes an absolute structural
limit on the degree to which relational information can be locally confined
within admissible configurations of~$\chi$.
It is non-metric and non-temporal: it is not associated with distances,
durations, or lightcones, since none of these are defined prior to
projection.
The constant~$c_\chi$ expresses a maximal admissible rate of structural
ordering.

The effective causal constraint~$c$ observed in spacetime is the projected
manifestation of~$c_\chi$:
\[
  c \;\equiv\; \Pi(c_\chi),
\]
acquiring operational meaning only once notions of locality and causal
ordering become meaningful.
In strong-gravity or near-deprojection regimes, $c$ may lose its geometric
interpretation while $c_\chi$ remains invariant.

\subsubsection*{The Role of
\texorpdfstring{$\hbar_{\chi}$}{ℏχ} and Reprojection}
\label{subsec:the-role-of-hbar_chi-and-reprojection-from-chi}

The parameter~$\hbar_\chi$ characterizes a fundamental structural scale
of~$\chi$, determined by its intrinsic relational density and constraint
structure.
It specifies a minimal quantum of \emph{reprojection}: intrinsic
structural information encoded in~$\chi$ can enter an effective spacetime
description only in discrete units set by this scale.
As spacetime structure stabilizes, reprojection events become increasingly
localized, manifesting phenomenologically as vacuum fluctuations.

\subsubsection*{The Origin of Planck's Constant}
\label{subsec:h-origin}

Within Cosmochrony, $\hbar$ is a \textbf{spectral invariant} of the
relaxation and projection process.
At the substrate level, the quantum of reprojection is fixed by
\begin{equation}
  \hbar_\chi \;=\; \frac{c^3}{K_0\,\chi_c} ,
\end{equation}
where $K_0$ denotes the coupling density of relational constraints and
$\chi_c$ the characteristic correlation scale.
This relation expresses the \emph{spectral rigidity} of the substrate.

In effective spacetime descriptions, the same scale appears
as~$\hbar_{\mathrm{eff}}$, functioning as the quantum of action in
Hilbert-space formulations.
The apparent universality of~$\hbar$ reflects the fact that $K_0$
and~$\chi_c$ are global invariants of the current relaxation epoch.
The transition from $\hbar_\chi$ to $\hbar_{\mathrm{eff}}$ involves a
change in representation, not in value.

\subsubsection*{Spectral Invariance of Planck's Constant and the
Fine-Structure Constant}
\label{subsec:quantum-invariants}

The fine-structure constant~$\alpha$ emerges as a dimensionless spectral
ratio, determined by the geometry of the projection fiber~$\Pi$ and the
spectral rigidity encoded by~$K_0$:
\begin{equation}
  \alpha \;=\;
    \mathcal{F}\!\left(
      \frac{\text{geometry and topology of }\Pi}{K_0}
    \right),
\end{equation}
where $\mathcal{F}$ is a functional fixed by the structure of admissible
projections.
If the structural parameters $K_0$ or $\chi_c$ were to vary---for example
in primordial high-constraint regimes---both $\hbar$ and $\alpha$ would
scale accordingly, preserving the internal structural coherence of the
framework.
