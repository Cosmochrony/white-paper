\section{Cosmological Implications}
  \label{sec:cosmology}

  \subsection{The Big Bang as a Maximal Constraint Regime of the $\chi$ Field}
    \label{subsec:big_bang_chi}

    In standard cosmology, the Big Bang is often described as an initial singular state of
    diverging density and temperature.
    Within the Cosmochrony framework, this interpretation is replaced by a geometric and
    pre-thermodynamic description rooted in the dynamics of the $\chi$ field.

    The Big Bang corresponds to an initial regime in which the $\chi$ field was subject to
    maximal internal constraints.
    Gradients, curvatures, and topological incompatibilities were saturated throughout the
    configuration, placing the field at the limit of its relaxation capacity.
    This regime does not represent thermal equilibrium or dissipation, but rather a state of
    maximal geometric constraint preceding the emergence of effective spacetime.

    Because time is interpreted as the local rate of $\chi$-field relaxation, this initial
    state is naturally associated with a maximal characteristic frequency.
    Similarly, energy—understood as resistance to the expansive relaxation of $\chi$—reaches
    its maximal effective value in this regime.
    No physical singularity is implied at the fundamental level; the apparent divergences arise
    from extrapolating effective spacetime descriptions beyond their domain of validity.

    As relaxation proceeds, constraints progressively loosen and the $\chi$ field undergoes
    its intrinsic expansive reorganization.
    This process gives rise to the emergence of spacetime structure, the dilution of gradients,
    and the appearance of particle-like excitations as locally stable configurations.
    Thermodynamic notions such as temperature and entropy become meaningful only after this
    structural differentiation has taken place.

    In this view, the Big Bang is not an explosive event in spacetime, but the initial boundary
    condition of a universe governed by the monotonic, extensive relaxation of the $\chi$
    field.
    The cosmological arrow of time is therefore identified with the direction of this
    relaxation process.

    \subsubsection{Pre-Geometric Nature of the Initial State}

      At the Big Bang, the $\chi$ field is not embedded in a pre-existing spacetime geometry.
      Rather, spacetime notions emerge only as effective descriptions once large-scale $\chi$
      configurations admit a metric interpretation.
      The initial state should therefore be understood as pre-geometric, and any spacetime
      singularity arises solely from extrapolating effective geometric descriptions beyond
      their domain of validity.

    \subsubsection{Pre-Thermodynamic Regime}

      The Big Bang does not correspond to a thermodynamic equilibrium state.
      At this stage, notions such as temperature, entropy, or thermal degrees of freedom are not
      yet well-defined.
      The initial high-constraint regime of the $\chi$ field precedes the emergence of
      thermodynamic behavior, which becomes meaningful only after structural differentiation
      and the appearance of effective particle-like excitations.

    \subsubsection{Maximal Relaxation Rate and Fundamental Scales}

      Because time is identified with the local rate of $\chi$-field relaxation, the initial
      state is naturally associated with a maximal characteristic frequency.
      Fundamental scales, such as the Planck scale, are interpreted here as dynamical bounds on
      the relaxation rate rather than as indicators of physical singularities or minimal spatial
      lengths.

    \subsubsection{Initial Homogeneity as a Structural Property}

      The large-scale homogeneity observed in the universe does not require causal equilibration
      within spacetime.
      In the Cosmochrony framework, it reflects a structural property of the initial $\chi$
      configuration, inherited by the emergent spacetime description.
      This removes the need for fine-tuned initial conditions or an inflationary smoothing
      mechanism.

      Together, these considerations frame the Big Bang as a well-defined dynamical boundary
      condition for the relaxation of the $\chi$ field, rather than as a physical singularity in
      spacetime.

      The subsequent cosmological evolution is then governed by the continued relaxation of the
      $\chi$ field, whose expansive dynamics naturally give rise to cosmic expansion.

  \subsection{Cosmic Expansion Without Inflation}
    \label{subsec:expansion_without_inflation}

    In standard cosmology, an early phase of accelerated expansion is commonly invoked to
    address the horizon, flatness, and relic problems.
    Within the Cosmochrony framework, these issues are approached from a different perspective,
    rooted in the intrinsic dynamics of the $\chi$ field.

    Cosmic expansion is interpreted as a direct manifestation of the monotonic, extensive
    relaxation of $\chi$, rather than as a transient inflationary phase.
    This relaxation-driven expansion operates from the earliest stages and does not require a
    separate inflation field or finely tuned potential.
    The Big Bang corresponds to an initial regime of maximal constraint, after which the
    relaxation of $\chi$ naturally drives large-scale expansion.

    Because $\chi$ constitutes a pre-geometric substrate, initial large-scale coherence does
    not rely on spacetime causal contact.
    Homogeneity and isotropy therefore arise as structural properties of the initial $\chi$
    configuration, rather than as consequences of rapid exponential expansion.
    As spacetime emerges as an effective description, these properties are inherited without
    requiring an inflationary smoothing mechanism.

    Similarly, the near-flatness of the observed universe follows from the progressive dilution
    of global gradients during the extensive relaxation of $\chi$.
    Unstable or incompatible configurations are dynamically suppressed, leaving only
    topologically stable structures as the universe evolves.
    In this sense, the observed large-scale regularity reflects a geometric attractor of the
    relaxation process rather than a finely tuned initial condition.

    While Cosmochrony does not presently provide a detailed quantitative substitute for
    inflationary perturbation spectra, it offers a unified conceptual explanation for early
    cosmic expansion without introducing additional ad hoc fields or phases.
    Further work is required to establish precise observational discriminants between
    relaxation-driven expansion and inflationary scenarios.

  \subsection{Cosmic Expansion as $\chi$ Relaxation}
    \label{subsec:cosmic-expansion-as-$chi$-relaxation}

    In Cosmochrony, cosmic expansion is not driven by an initial impulse or by a cosmological constant.
    Instead, it results from the monotonic relaxation of the $\chi$ field toward larger characteristic wavelengths.

    As $\chi$ increases uniformly, spatial separations between comoving points grow proportionally.
    The recession velocity between distant objects thus arises as a cumulative effect of local $\chi$
    relaxation rather than as motion through space.

    \begin{figure}[h]
      \centering
      \begin{tikzpicture}[scale=1]

% Axes
        \draw[->] (0,0) -- (6,0) node[right]{Cosmic time};
        \draw[->] (0,0) -- (0,4) node[above]{Scale / Wavelength};

% LambdaCDM
        \draw[thick, gray, dashed]
        plot[smooth] coordinates {(0.5,0.7) (2,1.3) (4,2.5) (5.5,3.7)};
        \node[gray] at (4.5,3.2) {$\Lambda$CDM};

% Chi relaxation
        \draw[thick, blue]
        plot[smooth] coordinates {(0.5,0.8) (2,1.4) (4,2.2) (5.5,2.9)};
        \node[blue] at (4.7,2.5) {$\chi(t)$};

      \end{tikzpicture}
      \caption{Comparison between standard $\Lambda$CDM cosmological expansion and Cosmochrony. In the latter,
        the observed Hubble law emerges from the monotonic relaxation of the fundamental field $\chi$,
        without invoking dark energy.}
      \label{fig:cosmo_comparison}
    \end{figure}

    Primordial fluctuations in $\chi$ at the recombination epoch ($z \sim 1100$
    ) are imprinted as temperature anisotropies in the CMB. The near scale-invariance of these fluctuations
    reflects the universal relaxation dynamics of $\chi$
    , while their acoustic peaks arise from oscillatory coupling between $\chi$
    and matter excitations. Unlike inflationary models, no superluminal stretching is required: correlations
    extend across the observable universe because they originate from a single connected $\chi$
    -field configuration prior to relaxation.

  \subsection{Emergent Hubble Law}
    \label{subsec:emergent-hubble-law}

    Let $\chi(t)$ denote the spatially averaged value of the field.
    The effective scale factor $a(t)$ scales proportionally to $\chi(t)$:
    \begin{equation}
      a(t) \propto \chi(t).
    \end{equation}

    The Hubble parameter follows directly:
    \begin{equation}
      H(t) = \frac{\dot{a}}{a} = \frac{\dot{\chi}}{\chi}.
    \end{equation}

    Assuming a maximal relaxation speed bounded by the invariant scale $c$, the present value of the Hubble constant becomes
    \begin{equation}
      H_0 \approx \frac{c}{\chi(t_0)},
    \end{equation}
    providing a natural scale for cosmic expansion without introducing dark energy.

  \subsection{Cosmic Acceleration Without Dark Energy}
    \label{subsec:cosmic-acceleration-without-dark-energy}

    Because $\chi$ relaxation accumulates over time, recession velocities increase with distance.
    This leads to an apparent acceleration when interpreted through conventional cosmological models.

    In Cosmochrony, this effect does not reflect a change in the expansion rate but the cumulative nature of $\chi$
    growth.
    Thus, accelerated expansion emerges without requiring a cosmological constant or exotic energy components.

  \subsection{Cosmic Microwave Background}
    \label{subsec:cosmic-microwave-background}

    In this model, the Cosmic Microwave Background (CMB) reflects frozen fluctuations of the $\chi$
    field at the epoch when matter-radiation interactions decoupled.

    Primordial variations in $\chi$
    phase and amplitude imprint temperature anisotropies that persist as large-scale correlations.

    These fluctuations originate from stochastic variations in local $\chi$
    relaxation prior to large-scale structure formation.
    Their near scale invariance reflects the universal relaxation dynamics of the field.

    Unlike inflationary scenarios, no superluminal expansion is required to explain horizon-scale coherence:
    correlations are traced back to the pre-relaxation continuity of $\chi$ in the pre-geometric regime.

  Further details on how $\chi$-field fluctuations reproduce the observed CMB anisotropies---including solutions to the
    horizon and flatness problems without inflation, as well as predicted deviations at large angular scales---are
    provided in Appendices~\ref{subsec:chi_cmb_spectrum} and~\ref{subsec:cosmochrony_horizon_flatness}.

  \subsection{Hubble Tension}
    \label{subsec:hubble-tension}

    Measurements of the Hubble constant derived from early-universe observables~\cite{Planck2020,Riess2019}
    , such as the CMB, probe smaller values of $\chi$.
    In contrast, late-time measurements using local distance ladders correspond to larger accumulated $\chi$ values.

    This difference naturally produces a tension between inferred values of $H_0$
    without invoking systematic errors or new particles.

    The CMB anisotropy spectrum in the $\chi$-framework differs from inflationary predictions in the low-$\ell$
    regime, where the absence of a primordial inflationary phase would suppress large-angle correlations. This
    could be tested by future high-precision CMB experiments like CMB-S4 or LiteBIRD\@.

    The predicted decrease of $H_0$
    with cosmic time provides a natural explanation for the current Hubble tension.
    Early-universe probes (e.g., CMB-based measurements yielding $H_0 \approx 67 \, \text{km/s/Mpc}$) sample smaller $\chi$
    values than late-time distance ladder methods ($H_0 \approx 74 \, \text{km/s/Mpc}$), consistent with the observed $\sim 4.4\sigma$
    discrepancy.
    Although this numerical agreement should not be interpreted as a precision prediction, it is an indication that the
    framework naturally selects the observed cosmological scale. Future measurements of $H(z)$ across redshift ranges may
    distinguish between this geometric interpretation and dark energy models.

    This mechanism suggests that the ``tension'' is not a failure of the standard cosmological model but a
    manifestation of the non-linear coupling between matter density and the field’s relaxation rate.
    A formal derivation of this local correction, yielding an $8.4\%$ increase in $H_0$ within the KBC void, is detailed
    in Appendix~\ref{appendix:hubble_tension}.

  \subsection{Entropy and the Arrow of Time}
    \label{subsec:entropy-and-the-arrow-of-time}

    The monotonic increase of $\chi$ defines a preferred temporal direction that is fundamental within the
    Cosmochrony framework.
    This direction is not introduced through thermodynamic or statistical arguments but follows directly
    from the irreversible relaxation of the underlying field.

    As $\chi$ grows, localized excitations become increasingly separated and their configurations progressively
    lose the ability to reconcentrate relaxation potential.
    This leads to an effective irreversibility at the level of composite systems, even though the underlying
    dynamics of $\chi$ remains deterministic.

    Entropy increase is therefore interpreted as an emergent, coarse-grained description of this relaxation
    process, rather than as a fundamental driver of temporal ordering.
    In this view, thermodynamic entropy does not define the arrow of time but reflects it: the macroscopic
    growth of entropy mirrors the monotonic expenditure of the relaxation capacity of the $\chi$ field.

    \subsection{Large-angle temperature anisotropies.}
      At the largest angular scales, the Cosmic Microwave Background (CMB) provides a direct probe of global
      cosmological structure.
      In the standard $\Lambda$CDM framework, deviations from scale invariance at low
      multipoles are interpreted as statistical fluctuations dominated by cosmic variance.

      In Cosmochrony, the largest-scale modes correspond to global configurations of the $\chi$ field and are
      subject to structural constraints arising from its relaxation dynamics. This leads to a scale-dependent
      suppression of low-$\ell$ temperature anisotropies, without introducing additional primordial degrees of
      freedom.

    \begin{figure}[t]
      \centering
      \includegraphics[width=\linewidth]{09-cosmology/cmb_lowell_lcdm_cosmo}
      \caption{
        Planck 2018 CMB temperature power spectrum at low multipoles (points with $1\sigma$ uncertainties),
        compared with the $\Lambda$CDM best-fit prediction (dashed line) and the Cosmochrony low-$\ell$
        suppression ansatz (solid line). Both model curves are normalized at higher multipoles.
        The comparison is intended as a qualitative illustration in a regime dominated by cosmic variance.
      }
      \label{fig:cmb_lowell_lcdm_cosmo}
    \end{figure}

      Figure~\ref{fig:cmb_lowell_lcdm_cosmo} compares the Planck 2018 temperature power spectrum at low multipoles
      with the $\Lambda$CDM best-fit prediction and the Cosmochrony low-$\ell$ suppression ansatz. Both model curves
      are normalized at higher multipoles, and no parameters are adjusted to individual low-$\ell$ data points.
      Given the dominance of cosmic variance at these scales, this comparison is intended as a qualitative
      consistency test rather than a statistical fit.

\subsection{Summary}
    \label{subsec:summary4}

    Cosmological expansion, apparent acceleration, the Hubble law, the CMB, and the arrow of time all emerge from the
    universal relaxation of the $\chi$ field.
    Cosmochrony reproduces key cosmological observations without introducing dark energy or modifying general
    relativity at large scales.
