% ----------------------------------------------------------------------------
% Section 4.9 --- Neutrinos as Partially Projectable Modes
% From former §6.12, drastically condensed
% ----------------------------------------------------------------------------
\subsection{Neutrinos as Partially Projectable Modes}
\label{subsec:neutrinos-partially-projectable-modes}

Neutrinos correspond to \emph{partially projectable modes} of~$\chi$:
configurations whose relational structure admits a stable projection in
some degrees of freedom while remaining weakly or non-projectable in
others.
This partial projectability explains their extremely small effective
masses, weak interaction strength, and sensitivity to global rather than
local structural properties of the projection.

\paragraph{Dirac vs.\ Majorana Character.}
A Dirac neutrino admits distinct conjugate projected realizations; a
Majorana neutrino corresponds to a configuration whose partial
projectability collapses the distinction between conjugate classes.
The Dirac or Majorana character is not a fundamental choice but a
manifestation of how fully the projection resolves relational
conjugation.
Lepton number conservation emerges only in regimes where conjugate
configurations are distinguishable; when partial projectability erases
this distinction, effective lepton number violation becomes admissible.

\paragraph{Neutrino Oscillations without Fundamental Mass Eigenstates.}
\label{subsec:neutrino-oscillations-without-mass-eigenstates}
Different flavors correspond to closely related partially projectable
configurations whose internal relational structures overlap but are not
identical.
As projected descriptions evolve along the monotonic ordering of~$\chi$,
the relative projectability of these configurations varies, leading to
coherent transitions between flavor labels without invoking mass
eigenstates as fundamental objects.

\paragraph{Stability of the Projection Boundary.}
\label{subsec:neutrinos-and-projection-boundary-stability}
Neutrinos occupy an intermediate regime between fully localized particles
and delocalized radiation.
They act as carriers of marginal structural information, redistributing
relational organization without inducing strong backreaction, thereby
preventing abrupt transitions between projectable and non-projectable
regimes.
In strong-gravity or near-deprojection regimes, neutrinos remain among
the last modes to retain partial projectability.

\paragraph{Why Neutrinos Are the Lightest Fermions.}
\label{subsec:why-neutrinos-are-lightest-fermions}
Neutrino configurations reside closest to the boundary of
projectability.
Their absence of electromagnetic coupling and partial self-conjugacy
prevent the formation of tightly bound projected structures.
The smallness of neutrino masses is a structural necessity imposed by
their role as marginally projectable modes.

\paragraph{Neutrinos as Probes of Pre-Geometric Structure.}
\label{subsec:neutrinos-as-probes-of-pregeometric-structure}
Because neutrinos operate near the projection boundary, they provide a
unique observational window into the pre-geometric structure of~$\chi$.
Their weak localization allows them to sample relational structures
inaccessible to fully projectable modes, and deviations in their
phenomenology may encode signatures of pre-geometric ordering.

\paragraph{Failure of Absolute Localization.}
\label{subsec:neutrinos-failure-of-absolute-localization}
Neutrinos cannot be fully confined within a bounded spacetime region
without loss of admissibility, explaining their absence of sharply
defined position operators and extremely small interaction
cross-sections.

\paragraph{Limits of Effective Quantum Field Theory.}
\label{subsec:neutrinos-limits-of-qft}
Neutrinos systematically probe the limits of local QFT assumptions.
Their weak localization, extended coherence, and oscillatory behavior
signal a breakdown of the strict particle ontology.
Standard QFT constructs (mass eigenstates, flavor mixing matrices) are
understood as effective bookkeeping devices for marginally projectable
configurations.

\paragraph{Experimental Signatures.}
\label{subsec:experimental-signatures-projective-neutrinos}
Potential signatures include small departures from standard oscillation
patterns at ultra-long baselines, coherence and decoherence effects
sensitive to global spacetime properties, constraints from neutrinoless
double beta decay on the projective resolution of conjugate
configurations, and cosmological neutrino backgrounds encoding
information about the approach to the projection boundary.

\paragraph{Synthesis.}
\label{subsec:synthesis-neutrinos-structural-frontier}
Neutrinos mark the transition between physics described by local quantum
fields on spacetime and the pre-geometric relational dynamics of~$\chi$.
They constitute the structural frontier of emergent spacetime, where
effective geometry, quantum description, and pre-geometric ontology
converge.
