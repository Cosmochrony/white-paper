% ----------------------------------------------------------------------------
% Section 4.6 --- Fermions, Bosons, and Spin
% Merges former §6.6 and §6.7
% ----------------------------------------------------------------------------
\subsection{Fermions, Bosons, and Spin}
\label{subsec:fermions-and-bosons}

Particle statistics emerge from the topological structure of admissible
projected configurations.
Configurations requiring a $4\pi$ rotation to return to an equivalent
projected description give rise to fermion-like behavior, while
$2\pi$-periodic configurations correspond to bosons.
This distinction reflects a topological obstruction rather than a symmetry
principle imposed at the fundamental level.

\subsubsection*{Spin as a Topological Property}
\label{subsec:spin_topology}

Spin is not an intrinsic kinematic degree of freedom but a purely
topological property of admissible projected configurations.
Fermionic configurations require a $4\pi$ transformation in configuration
space to return to an equivalent effective description, implying that the
relevant configuration space admits a double covering with fundamental
group
\begin{equation}
  \pi_1(\mathcal{C}_{\mathrm{eff}}) = \mathbb{Z}_2 .
\end{equation}

When an effective quantum description is applicable, a $2\pi$
transformation induces a sign change of the associated wavefunction,
\begin{equation}
  \psi \;\longrightarrow\; -\psi ,
\end{equation}
while a $4\pi$ transformation restores the original state.
This $4\pi$-periodicity directly implies fermionic antisymmetry and the
Pauli exclusion principle, as detailed in
Section~\ref{app:relational_spin_statistics}.

Exchanging two identical fermionic excitations corresponds topologically
to a $2\pi$ loop in the combined configuration space and induces a sign
change, dynamically excluding symmetric
configurations~\cite{Pauli1925}.
The spin--statistics connection thus admits a unified topological origin
within the effective descriptive framework.
