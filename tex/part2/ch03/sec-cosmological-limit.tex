% ----------------------------------------------------------------------------
% Section 3.7 --- Homogeneous Cosmological Limit
% From former §5.7
% ----------------------------------------------------------------------------
\subsection{Homogeneous Cosmological Limit}
\label{subsec:homogeneous-cosmological-limit}

In a homogeneous and isotropic regime, projected configurations exhibit no
effective spatial variations and the admissible relaxation rate attains its
maximal value:
\begin{equation}
  \mathcal{D}_{\mathrm{loc}}\chi_{\mathrm{eff}} = c .
\end{equation}
Expressed in terms of an effective cosmological time parameter~$t$, this
yields
\begin{equation}
  \chi_{\mathrm{eff}}(t) = \chi_{\mathrm{eff},0} + c\, t ,
\end{equation}
where $\chi_{\mathrm{eff},0}$ is a reference value.
This linear regime underlies the emergent Hubble law derived in
Section~\ref{subsec:emergent-hubble-law}.

As shown in
Appendix~\ref{subsec:minimal-kinematic-constraint}, the monotonicity
requirement in an expanding regime implies a minimal residual structural
inhomogeneity in projected configurations, manifesting as a non-vanishing
lower bound on gravitational acceleration.
This bound provides a natural route to MOND-like phenomenology without
postulating dark matter
particles~\cite{Milgrom2002,FamaeyMcGaugh2012}.
