\subsection{Strong gravitational lensing}
  \label{subsec:strong-gravitational-lensing}

  For comparison with standard analyses, it is convenient to decompose the effective
  lensing potential as
  \begin{equation}
    \Phi_{\mathrm{eff}}
    =\Phi_{\mathrm{bar}}+\Phi_{\chi},
  \end{equation}
  where $\Phi_{\mathrm{bar}}$ is the contribution reconstructed from baryonic matter,
  and $\Phi_{\chi}$ encodes the emergent geometric contribution associated with
  collective relaxation constraints of the $\chi$-substrate.

  The lensing convergence is given by
  \begin{equation}
    \kappa(\boldsymbol{\theta})
    =\frac{D_lD_{ls}}{c^2D_s}
    \int \nabla_\perp^2
    \Phi_{\mathrm{eff}}\!\left(D_l\boldsymbol{\theta},z\right)\,dz ,
  \end{equation}
  leading to
  \begin{equation}
    \kappa=\kappa_{\mathrm{bar}}+\kappa_{\chi}.
  \end{equation}

  In Cosmochrony, $\kappa_{\chi}$ does not correspond to an additional mass density,
  but to a densification of admissible null geodesics induced by non-uniform
  spectral rigidity.
  Strong lensing and giant arcs therefore arise from geometric focusing rather than
  from massive dark halos.

  At leading order, the emergent contribution may be parametrized as
  \begin{equation}
    \Phi_{\chi}(\mathbf{x})
    =\frac{c^2}{2}\,
    \ln\!\left(\frac{K(\bar\chi(\mathbf{x}))}{K_\infty}\right),
  \end{equation}
  where $K(\bar\chi)$ is the effective spectral rigidity controlling relational
  distances in the projected regime.
  This form follows directly from the weak-field expansion of the reconstructed
  effective metric.

  This framework leads to several testable signatures:
  \begin{itemize}
    \item partial decorrelation between baryonic mass and strong-lensing strength
    \item enhanced sensitivity of arc formation to cluster morphology and relaxation state
    \item strong lensing without corresponding dark matter substructure
    \item redshift dependence tracing relaxation rather than mass accretion history
  \end{itemize}

  These effects typically require significant fine tuning within standard dark
  matter halo models.

  We now illustrate this framework on a well-studied strong-lensing cluster.

\subsection[Strong gravitational lensing in Abell 1689]{Strong gravitational lensing in Abell~1689}
  \label{subsec:a1689-strong-lensing}

  We illustrate the effective lensing formalism on the cluster Abell~1689, for which
  high-quality strong and weak lensing reconstructions as well as X-ray gas profiles
  are available in the literature.
  In the thin-lens approximation, the convergence field is defined by
  \begin{equation}
    \nabla_{\boldsymbol{\theta}}^2\psi(\boldsymbol{\theta})=2\kappa(\boldsymbol{\theta}),
  \end{equation}
  with $\psi$ the lensing potential.
  We define the effective convergence $\kappa_{\mathrm{eff}}$ from lensing reconstruction
  and a baryonic contribution $\kappa_{\mathrm{bar}}=\Sigma_{\mathrm{bar}}/\Sigma_{\mathrm{crit}}$
  from gas (X-ray) and stellar components, and isolate the emergent contribution
  \begin{equation}
    \kappa_{\chi}(\boldsymbol{\theta})
    =\kappa_{\mathrm{eff}}(\boldsymbol{\theta})
    -\kappa_{\mathrm{bar}}(\boldsymbol{\theta}).
  \end{equation}
  The corresponding emergent lensing potential is obtained by solving the 2D Poisson
  equation
  \begin{equation}
    \nabla_{\boldsymbol{\theta}}^2\psi_{\chi}(\boldsymbol{\theta})
    =2\kappa_{\chi}(\boldsymbol{\theta}),
  \end{equation}
  from which deflection, shear, magnification and critical curves follow in the usual
  way.
  In Cosmochrony, $\kappa_{\chi}$ is not interpreted as an additional dark matter
  surface density, but as an emergent geometric focusing induced by collective
  constraints of the projected $\chi$-substrate.
