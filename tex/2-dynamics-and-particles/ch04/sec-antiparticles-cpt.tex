% ----------------------------------------------------------------------------
% Section 4.8 --- Antiparticles, Creation/Destruction, and CPT
% Merges former §6.9, §6.10, §6.11, heavily condensed
% ----------------------------------------------------------------------------
\subsection{Antiparticles, Creation/Destruction, and CPT}
\label{subsec:antiparticles}

\subsubsection*{Antiparticles as Relationally Conjugate Configurations}

A particle and its antiparticle correspond to projected configurations
belonging to distinct but conjugate topological classes within the space
of admissible projected descriptions, related by an internal reversal of
relational organization.
Annihilation occurs when a particle-like configuration and its conjugate
combine into a composite description that no longer supports localized
structural constraints, redistributing relational structure into
delocalized radiation-like excitations.
This constitutes a process of \emph{structural unknotting}: mass itself
measures the degree of topological obstruction to relaxation.

\paragraph{Why Antimatter Does Not Require Time Reversal.}
\label{subsec:why-antimatter-no-time-reversal}
The monotonic ordering of~$\chi$ defines an absolute arrow of admissible
projection that cannot be inverted.
Antiparticles correspond to topologically conjugate classes, not to
reversed temporal trajectories.
The apparent association between antimatter and time reversal in standard
formalisms is a feature of the effective representation.

\paragraph{Matter--Antimatter Asymmetry without Fundamental CP Violation.}
\label{subsec:matter-antimatter-asymmetry-without-cp}
If the projection from~$\chi$ to effective spacetime is chiral, conjugate
classes need not be realized with equal stability or projectability.
Matter--antimatter asymmetry emerges as a selection effect imposed by
differential projectability, without requiring dynamical CP-breaking
interactions.

\subsubsection*{Particle Creation and Destruction}
\label{subsec:particle-creation-and-destruction}

Particle creation corresponds to a projected configuration acquiring
sufficient structural organization to support a stable topological class.
Particle destruction occurs when a configuration loses its topological
admissibility and admits continuous deformation toward a delocalized
effective description.

\subsubsection*{CPT as a Global Projective Property}
\label{subsec:cpt-as-global-projective-property}

C, P, and T are individually effective and representation-dependent.
Their combined action corresponds to a full relational conjugation of
projected descriptions, mapping any admissible effective configuration to
another admissible configuration representing the same underlying
$\chi$ structure.
CPT invariance is therefore not a microscopic symmetry but a global
projective consistency condition ensuring that the space of admissible
descriptions is closed under full relational conjugation.
Violations of CPT would signal a breakdown of projectability itself.

\subsubsection*{CPT as an Admissibility Consistency Condition}
\label{subsec:antiparticles-cpt}

Certain projected configurations carry orientation-sensitive structural
invariants (chirality, phase winding, relational orientation).
Under admissible factorization, these signed invariants must be
redistributed, possibly requiring paired localized excitations carrying
opposite orientations---interpreted as particle--antiparticle pairs.
CPT symmetry reflects the invariance of admissibility under a combined
reversal of signed structural invariants, effective spatial orientation,
and the effective ordering parameter.
A technical formulation is provided in
Appendix~\ref{subsec:structural-interpretation-of-cpt-symmetry}.

\paragraph{Why CPT Survives Quantum Gravity.}
\label{subsec:why-cpt-survives-quantum-gravity}
CPT survives because it expresses the minimal requirement for a coherent
physical projection.
Quantum gravity may challenge locality, geometry, and the notion of time,
but cannot violate CPT without undermining the possibility of a
consistent emergent universe.
