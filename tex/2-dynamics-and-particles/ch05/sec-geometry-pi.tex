% ----------------------------------------------------------------------------
% Section 5.1 --- The Geometry of the Pi Subspace
% From former §10.1
% ----------------------------------------------------------------------------
\subsection{The Geometry of the
\texorpdfstring{$\Pi$}{Π} Subspace}
\label{sec:geometry-pi}

The relational substrate~$\chi$ is not accessed in its full structural
complexity but through admissible local projections onto a reduced
projection fiber $\Pi \cong S^3$.
Here $\Pi$ denotes the projection \emph{fiber}, i.e.\ the space of
admissible internal representatives associated with a projected
configuration, not the projection map itself.

The identification $\Pi \cong S^3$ is not imposed \emph{a priori} but
follows from the minimal compact geometry (in dimension and connectivity)
required to support non-injective yet stable projections.
Its associated Hopf fibration naturally induces an effective
$SU(2)\times U(1)$ symmetry structure, which emerges as an invariance of
the projection process.

The metric on~$\Pi$ is dynamically induced by the local density of
relational connections encoded in the underlying relational graph~$G$,
used as a numerical and representational support, not as a physical
discretization.
The mapping from the global graph Laplacian~$\Delta_G$ to the effective
projected Laplacian~$\Delta_\Pi$ is
\begin{equation}
  \Delta_\Pi = P^\dagger \Delta_G P ,
\end{equation}
where $P$ denotes the projection operator onto the admissible spectral
subspace.
The smooth three-sphere geometry corresponds to a large-$N$ spectral
coarse-graining limit of the discrete relational structure.
