% ----------------------------------------------------------------------------
% Section 3.3 --- Microscopic Origin of the Coupling Tensor and the
%                 Poisson Equation
% From former §5.3
% ----------------------------------------------------------------------------
\subsection{Microscopic Origin of the Coupling Tensor and the Poisson
Equation}
\label{subsec:microscopic-origin-of-the-coupling-tensor-and-the-poisson-equation}

The effective coupling governing projected $\chi$ configurations depends on
the internal structural state of the projected description.
A convenient phenomenological parametrization is
\begin{equation}
  K_{\mathrm{eff}}
  = K_0 \exp\!\left(
    -\frac{(\Delta \chi_{\mathrm{eff}})^2}{\chi_c^2}
  \right),
  \label{eq:effective_coupling_tensor}
\end{equation}
where $\Delta \chi_{\mathrm{eff}}$ measures effective internal variation,
$K_0$ is the maximal relaxation conductivity in a homogeneous background,
and $\chi_c$ sets the scale beyond which inhomogeneities suppress
relaxation efficiency.

Projected configurations exhibiting strong internal variation reduce the
effective coupling and locally slow the admissible relaxation ordering,
providing the microscopic origin of emergent gravitational phenomenology.

An effective gravitational potential~$\Phi$ may then be introduced through
\begin{equation}
  \frac{\mathcal{D}_{\mathrm{loc}} \chi_{\mathrm{eff}}}
       {\mathcal{D}_0}
  \simeq 1 + \frac{\Phi}{c^2},
  \label{eq:relaxation_potential_relation}
\end{equation}
summarizing the relative slowdown of effective relaxation ordering.
In the weak-structure regime, the spatial distribution of~$\Phi$ admits a
Poisson-type relation:
\begin{equation}
  \nabla^2 \Phi \simeq 4\pi G_{\mathrm{eff}} \rho,
  \label{eq:effective_poisson_equation}
\end{equation}
where $\rho$ is the effective density of relaxation-resistant
configurations and $G_{\mathrm{eff}}$ an emergent coupling parameter.
Gravitation appears as a descriptive manifestation of reduced relaxation
conductivity induced by structured projected configurations.

A fully relational formulation is provided in
Appendix~\ref{app:relational_formulation}.
