\subsection{The Inaccessibility of Absolute Zero as a Projective Necessity}
  \label{subsec:absolute-zero-necessity}

  Within the present framework, the Third Law of Thermodynamics is
  reinterpreted as a structural requirement for temporal projectability.
  Temperature $T$ is not introduced as a primitive measure of kinetic
  agitation, but as an effective indicator of the rate at which distinct
  projected descriptions remain differentiable under successive updates.

  A strict absolute zero ($T \rightarrow 0$) would correspond to a regime of
  projective stasis in which successive effective descriptions become
  indistinguishable.
  Formally, this would imply
  \begin{equation}
    U_{n+1} = U_n ,\label{eq:projection-stasis}
  \end{equation}
  so that no non-trivial ordering of projected states could be defined.
  Since time is introduced operationally as the ordering of distinguishable
  effective states, such a regime would eliminate the very notion of
  temporal succession.

  Absolute zero is therefore not merely energetically inaccessible, but
  structurally excluded: a perfectly stationary projection would terminate
  the temporal description itself.
  The persistence of time requires that projected states remain
  minimally non-degenerate under admissible updates.

  In this context, the so-called zero-point fluctuations
  $(\tfrac{1}{2}\hbar\omega)$ are reinterpreted as the minimal relational
  variance required
