% ----------------------------------------------------------------------------
% Section 4.4 --- Metastability, Projection, and Particle Decay
% From former §6.4, condensed
% ----------------------------------------------------------------------------
\subsection{Metastability, Projection, and Particle Decay}
\label{subsec:metastability-and-decay}

A stable particle corresponds to a deep basin of admissible projected
configurations; an unstable particle occupies a shallow or fragile basin.
Particle decay is interpreted as the structural reorganization of a
metastable configuration: when the concentration of constraints exceeds
what can be sustained by a single projected entity, admissibility is
recovered through factorization into less constrained localized
configurations, possibly accompanied by weakly structured excitations.

Entanglement and decay represent two regimes of the same projection
structure.
Entanglement corresponds to non-factorizability without fragmentation,
while decay corresponds to non-factorizability that forces fragmentation.
The distinction lies in the stability properties of the projected
description under admissible fluctuations.

The finite lifetime of unstable particles reflects the probability of
crossing a structural reorganization threshold under admissible
variations, yielding the observed exponential decay law as a statistical
signature of metastability.

\begin{figure}[t]
  \centering
  \begin{tikzpicture}[
    box/.style={draw, rounded corners, align=center, inner sep=6pt},
    arr/.style={->, thick},
    lab/.style={font=\small, align=center}
  ]
    \node[box] (A)
      {$\chi_{\mathrm{eff},A}$\\[-2pt]\footnotesize metastable\\
       [-2pt]\footnotesize (single knot)};
    \node[box, right=2.7cm of A] (T)
      {Admissible\\[-2pt]\footnotesize factorization\\
       [-2pt]\footnotesize threshold};
    \node[box, right=2.9cm of T, yshift=1.1cm] (B1)
      {$\chi_{\mathrm{eff},1}$\\[-2pt]\footnotesize stable knot};
    \node[box, right=2.9cm of T] (B2)
      {$\chi_{\mathrm{eff},2}$\\[-2pt]\footnotesize stable knot};
    \node[box, right=2.9cm of T, yshift=-1.1cm] (B3)
      {$\chi_{\mathrm{eff},3}$\\[-2pt]\footnotesize stable /\\
       [-2pt]\footnotesize sub-threshold};
    \node[box, below=1.6cm of B2] (R)
      {$\chi_{\mathrm{eff,rad}}$\\[-2pt]\footnotesize light
       excitations\\[-2pt]\footnotesize (photon/$\nu$/...)};
    \draw[arr] (A) -- node[lab, above]
      {projective variability\\
       \footnotesize (threshold crossing)} (T);
    \draw[arr] (T) -- (B1);
    \draw[arr] (T) -- (B2);
    \draw[arr] (T) -- (B3);
    \draw[arr] (T) -- node[lab, right]
      {\footnotesize mismatch evacuation} (R);
    \node[lab, above=0.35cm of B2] (Q)
      {$Q(\chi_A)=\sum_i Q(\chi_i)$\\[-2pt]
       \footnotesize structural invariants redistributed};
  \end{tikzpicture}
  \caption{Structural interpretation of particle decay.
    A metastable localized projected configuration transitions, via an
    admissible factorization threshold, into several more stable
    configurations plus weak excitations that evacuate the residual
    structural mismatch.}
  \label{fig:decay-fragmentation}
\end{figure}

A more technical characterization of metastability, admissible
factorization channels, and decay widths is provided in
Appendix~\ref{subsec:metastability-decay-channels-and-exponential-lifetimes}.
