\subsection{Topological Stability}
  \label{subsec:topological-stability}

  The stability of particle-like excitations does not rely on fundamental
  conserved charges postulated \emph{a priori}.
  Certain projected configurations exhibit non-trivial internal organization
  that prevents them from being continuously deformed into homogeneous
  effective descriptions without violating admissibility conditions.

  This stability is topological in character: it reflects the existence of
  inequivalent classes of admissible projected configurations that cannot be
  smoothly connected through continuous reconfiguration while preserving
  monotonic relaxation ordering.
  The long-lived character of solitonic structures follows from this
  topological incompatibility, not from a dynamical balance of forces.

  Detailed geometric constructions of topological solitons, including
  vortex, skyrmion, and knotted configurations, are provided in
  Appendix~\ref{app:topological_solitons}.
  The fully relational formulation is developed in
  Appendix~\ref{app:relational_topological_stability}.

  \paragraph{Stability through self-consistency.}
    The projection $\Pi$ does not operate independently of the effective
    description already in place, but is constrained by the current state $U$
    through a compatibility condition.
    Among the relational configurations supplied by the $\chi$ substrate,
    only those transitions $\chi_n \rightarrow \chi_{n+1}$ whose projection
    remains compatible with $U$ admit a stable effective realization.
    In this sense, physical laws are not imposed by the substrate $\chi$, but
    emerge as the only update rules supporting projective continuity across
    successive descriptions~\cite{Beau2026a}.
