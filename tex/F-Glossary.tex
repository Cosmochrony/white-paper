\clearpage
\section{Glossary of Core Quantities and Notation}
  \label{appendix:glossary}

  This appendix summarizes the meaning, role, and ontological status of the main
  quantities used throughout the Cosmochrony framework.
  It is intended strictly as a reference guide and does not introduce
  new assumptions, dynamics, or physical postulates.

  \subsection{Fundamental Quantities}

    \paragraph{\texorpdfstring{$\chi$}{χ} (Chi substrate).}
      The unique fundamental entity of the Cosmochrony framework.
      $\chi$ is a pre-geometric, relational substrate not defined on a pre-existing
      spacetime manifold.
      Its irreversible relaxation provides an intrinsic ordering of physical processes.
      Localized, topologically stable configurations of $\chi$ correspond to
      particle-like excitations.

    \paragraph{\texorpdfstring{$\chi_i$}{χᵢ} (Local configuration).}
      Discrete local degrees of freedom of the $\chi$ substrate, associated with
      vertices of the relaxation network.
      They encode the microscopic relational state prior to any geometric projection.

    \paragraph{\texorpdfstring{$\chi_c$}{χc} (Critical relaxation threshold).}
      A fundamental structural bound limiting local variations of $\chi$.
      It enforces causal consistency at the pre-geometric level and underlies
      all effective speed and action bounds.

    \paragraph{\texorpdfstring{$\tau$}{τ} (Relational time).}
      An intrinsic ordering parameter associated with the irreversible relaxation
      of $\chi$.
      $\tau$ precedes spacetime notions and does not correspond to a coordinate time.

    \paragraph{\texorpdfstring{$c_\chi$}{cχ} (Fundamental relaxation speed).}
      The maximal propagation speed of relaxation disturbances within the $\chi$
      substrate.
      It represents the fundamental causal bound of the theory, from which the
      effective speed of light emerges.

\subsection{Effective and Projected Quantities}

  \paragraph{\texorpdfstring{$\chi_{\mathrm{eff}}$}{χeff} (Effective projected field).}
    A coarse-grained scalar field arising from the non-injective projection of $\chi$
    onto an emergent spacetime description.
    $\chi_{\mathrm{eff}}$ provides an effective field-theoretic representation
    without fundamental status.

  \paragraph{Fiber (of the projection).}
    For a given effective configuration $\chi_{\mathrm{eff}}$, the fiber is the set of
    underlying $\chi$ configurations mapped to it by the projection $\pi$.
    Elements of a fiber are operationally indistinguishable at the spacetime level.
    Non-trivial fibers reflect the structural non-injectivity of the projection.

  \paragraph{\texorpdfstring{$\pi$}{π} (Projection map).}
    A structural mapping from configurations of $\chi$ to an effective description
    $\chi_{\mathrm{eff}}$ applicable in projectable regimes.
    The projection is generally non-injective: distinct underlying configurations of
    $\chi$ may correspond to the same effective state, defining equivalence classes
    (fibers) under $\pi$.

  \paragraph{\texorpdfstring{$\pi^{-1}$}{π⁻¹} (Deprojection).}
    The inverse reconstruction problem of identifying classes of $\chi$
    configurations compatible with a given effective state.
    Deprojection is not unique and does not destroy structural information.

  \paragraph{\texorpdfstring{$V(\chi)$}{V(χ)} (Effective potential).}
    An effective, coarse-grained description used to model localization and
    stability properties of $\chi$ configurations.
    $V(\chi)$ is not fundamental and is secondary to the spectral characterization
    of mass and inertia.

  \paragraph{Observable.}
    A stable, projectable quantity defined on $\chi_{\mathrm{eff}}$ through an
    interpretative framework.
    Observables do not correspond to additional ontological entities, but to
    operational readings of the same effective physical reality.

  \paragraph{Physical reality.}
    In the Cosmochrony framework, physical reality is identified with the effective
    level $\chi_{\mathrm{eff}}$.
    The substrate $\chi$ is ontologically real but does not constitute a physical
    universe until projected into a projectable regime.

  \paragraph{\texorpdfstring{$t_{\mathrm{proj}}$}{tproj} (Projected time).}
    Operational time measured within the emergent spacetime description.
    It arises from the local rate of $\chi$ relaxation and reproduces relativistic
    time dilation effects.

  \paragraph{Universe.}
    The physically real domain described at the $\chi_{\mathrm{eff}}$ level, where
    spacetime structure, causality, and physical observables are well-defined.
    The Universe does not refer to the fundamental substrate $\chi$, which is
    ontologically prior to any notion of universe.

\subsection{Relaxation Network and Operators}

\paragraph{\texorpdfstring{$G(V,E)$}{G(V,E)} (Relaxation network).}
  A discrete graph representing the underlying relational structure on which the
  $\chi$ substrate is defined.
  Vertices correspond to elementary degrees of freedom and edges encode relaxation
  couplings.

\paragraph{\texorpdfstring{$K_{ij}$}{Kij} (Relaxation coupling).}
  Edge-dependent coupling coefficients defined on the relaxation network.
  They quantify the resistance to relative variations of $\chi$ between
  neighboring nodes and encode geometric and topological information.
  $K_{ij}$ are structural parameters of the pre-geometric substrate and do not
  represent dynamical interaction constants.

\paragraph{\texorpdfstring{$\Delta_G$}{ΔG} (Graph Laplacian / relaxation operator).}
  The discrete Laplace--Beltrami operator associated with the network $G(V,E)$ and
  the couplings $K_{ij}$.
  Its spectral properties govern the stability, localization, and inertial
  behavior of $\chi$ configurations.

\paragraph{\texorpdfstring{$D_{\mathrm{loc}}\chi$}{Dlocχ} (Local relaxation operator).}
  A local relational operator governing the evolution of $\chi$ at the microscopic
  level.
  It replaces differential operators defined on continuous manifolds.

\subsection{Spectral and Inertial Quantities}

\paragraph{\texorpdfstring{$\lambda_n$}{λn} (Spectral eigenvalues).}
  Eigenvalues of the linearized relaxation or stability operator acting on small
  perturbations of a localized $\chi$ configuration.
  They determine inertial mass scales in the effective description.

\paragraph{\texorpdfstring{$\psi_n$}{ψn} (Spectral modes).}
  Eigenmodes associated with the operator $\Delta_G$.
  They encode the internal structure and stability of particle-like configurations.

\paragraph{\texorpdfstring{$m_{\mathrm{eff}}$}{meff} (Effective mass).}
  An emergent invariant determined by the spectral properties of localized
  $\chi$ configurations.
  Mass is not a fundamental parameter nor a coupling constant.

\paragraph{\texorpdfstring{$Q$}{Q} (Topological charge).}
  An integer-valued invariant characterizing the topology of a stable
  $\chi$ configuration.
  Different values of $Q$ correspond to distinct particle families.

\paragraph{\texorpdfstring{$\Omega^\pm$}{Ω±} (Chiral topological sectors).}
  Opposite chiral configurations of topological $\chi$ structures.
  They are related by orientation reversal and need not be energetically equivalent.

\subsection{Dimensionless Parameters}

\paragraph{\texorpdfstring{$S$}{S} (Gradient saturation parameter).}
  A dimensionless quantity defined as
  \begin{equation}
    S \equiv \frac{1}{c^2}\sum_{j\sim i} K_{ij}(\chi_i-\chi_j)^2 ,
  \end{equation}
  measuring the local density of $\chi$ gradients.
  The bound $S \leq 1$ enforces causal consistency in effective spacetime dynamics.

\paragraph{\texorpdfstring{$\Omega_\chi$}{Ωχ} (Relaxation budget parameter).}
  A dimensionless global quantity characterizing the fraction of total $\chi$
  relaxation stored in spatial gradients.
  In cosmological regimes, it plays a role analogous to a density parameter.

\subsection{Constants and Emergent Limits}

\paragraph{\texorpdfstring{$c$}{c} (Effective speed of light).}
  The maximal signal propagation speed in emergent spacetime.
  $c$ is an effective bound derived from the more fundamental speed $c_\chi$.

\paragraph{\texorpdfstring{$\hbar$}{ℏ} (Effective Planck constant).}
  An emergent quantum of action associated with projection thresholds and spectral
  granularity.
  It is not fundamental at the level of $\chi$.

\paragraph{\texorpdfstring{$G$}{G} (Newtonian gravitational constant).}
  An emergent coupling constant arising from large-scale collective relaxation
  dynamics of $\chi$.
  Its value reflects structural properties rather than fundamental interaction
  strengths.

\paragraph{\texorpdfstring{$\Lambda_{\mathrm{eff}}$}{Λeff} (Effective cosmological constant).}
  A residual large-scale relaxation effect associated with incomplete equilibration
  of the $\chi$ substrate.

\subsection{Key Conceptual Terms}

\paragraph{Energy.}
  Energy measures the resistance of $\chi$ configurations to relaxation-induced
  change.
  Standard conservation laws remain valid at the effective level.

\paragraph{Relaxation (of the \texorpdfstring{$\chi$}{χ} field).}
  The intrinsic dynamical tendency of $\chi$ to reorganize under internal coupling
  constraints.
  Relaxation is pre-thermodynamic and does not correspond to dissipation.

\paragraph{Fluctuations.}
  Local stochastic modulations of $\chi$ configurations that affect event timing
  and localization without altering underlying topological constraints.

\paragraph{Matter.}
  Stable topological configurations of $\chi$ whose persistence gives rise to
  particle-like behavior and inertial properties.

\paragraph{Measurement.}
  A localized interaction that selects a specific manifestation of an underlying
  $\chi$ fluctuation without invoking fundamental wavefunction collapse.

\paragraph{Probability.}
  An emergent descriptor reflecting structural constraints imposed by the topology
  of $\chi$, modulated by stochastic fluctuations.

\paragraph{Spacetime.}
  An emergent relational structure arising from large-scale configurations of
  the $\chi$ substrate.
  Its metric description remains valid within its domain of applicability.

\paragraph{Time.}
  An effective parameter associated with the local rate of $\chi$ relaxation.
  Operational and relativistic notions of time are recovered without modification.

\paragraph{Wavefunction.}
  An effective statistical representation of the dynamics and topology of the
  $\chi$ substrate.
  It has no fundamental ontological status.

\paragraph{Wave--Particle Duality.}
  A manifestation of interaction-induced changes in the local configuration of
  $\chi$, producing localized particle-like behavior from an underlying
  wave-like substrate.
