% ----------------------------------------------------------------------------
% Section 8.4 --- Measurement, Decoherence, and Limits of Entanglement
% Merges former §8.5 and §8.7, condensed
% ----------------------------------------------------------------------------
\subsection{Measurement, Decoherence, and Apparent Collapse}
\label{subsec:measurement-and-decoherence}

Quantum measurement does not involve fundamental wavefunction collapse.
Measurement corresponds to the transition from a non-factorizable
admissible projected description to a set of effectively factorized local
projections.
Decoherence suppresses interference between incompatible descriptive
branches by rendering their relative phase information inaccessible
within spacetime representations~\cite{Zurek2003}.
The underlying relational structure remains globally well defined.
The apparent collapse is the effective manifestation of a non-injective
relational structure becoming only partially projectable into spacetime.

\subsection{Limits of Entanglement and Environmental Effects}
\label{subsec:limits-of-entanglement}

Entanglement arises only within restricted regimes in which a
non-factorizable global description remains jointly projectable.
Environmental coupling progressively restricts the set of admissible
projected descriptions to locally stable, approximately factorizable
regimes.
Entanglement is most robust for effectively isolated systems and becomes
increasingly fragile in macroscopic environments.
Classical behavior emerges when only factorized projected descriptions
remain admissible, without requiring any modification of the underlying
relational structure.
