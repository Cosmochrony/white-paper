% ----------------------------------------------------------------------------
% Section 8.6 --- Entanglement as a Critical Regime of Projective
%                 Compression
% Merges former §8.10 and §8.11, condensed
% ----------------------------------------------------------------------------
\subsection{Entanglement as a Critical Regime of Projective Compression}
\label{subsec:entanglement-critical-compression}

Quantum entanglement arises as a structural consequence of the
non-injective projection
$\Pi:\chi \rightarrow \chi_{\mathrm{eff}}$.
The projection reduces a high-dimensional relational configuration to a
lower-dimensional effective description; the resulting fiber
$\Pi^{-1}(\chi_{\mathrm{eff}})$ constitutes an information-theoretic
channel whose bandwidth depends on unresolved modes of~$\chi$.

Non-factorizable correlations emerge only in an intermediate
\emph{critical} regime of projective compression: the effective
description is sufficiently coarse-grained to permit subsystem
identification, yet retains enough global relational structure to prevent
full factorization.
Increased compression (environmental coupling, decoherence) suppresses
correlations, yielding classical behavior; decreased compression breaks
subsystem separation.
The critical regime may be intermittent, with correlations appearing
during specific spectral reconfiguration events (Appendix~D).

\subsection{Implications for Quantum Computation}
\label{subsec:implications-quantum-computation}

Effective resources exploited in quantum computation may be reinterpreted
as manifestations of structural non-injectivity and global consistency
constraints.
Multiple underlying relational configurations may correspond to the same
effective observable state, providing a natural origin for effective
parallelism without explicit information exchange.
Part of the computational advantage may reflect general properties of
non-factorizable descriptive mappings rather than exclusively quantum
dynamical effects.
