% ----------------------------------------------------------------------------
% Section 8.8 --- Relation to Quantum Formalism
% Merges former section 9 (all subsections), condensed
% ----------------------------------------------------------------------------
\subsection{Relation to Quantum Formalism}
\label{sec:relation-to-quantum-formalism}

The formal apparatus of quantum mechanics arises as an effective
framework organizing admissible projected descriptions in regimes where
localization, linearity, and approximate factorization hold.
Quantum mechanics is not replaced but reinterpreted as an effective
theory whose validity is restricted to regimes admitting a stable,
approximately linear, spacetime-based description.

\paragraph{Status of the Wavefunction.}
\label{subsec:status-of-the-wavefunction}
The wavefunction~$\psi$ is not a fundamental physical entity but a
statistical encoding of admissible local reprojections compatible with a
given non-factorizable projected configuration.
Its complex phase encodes relational constraints between alternative
local descriptions; its modulus determines the statistical weight of
accessible reprojections.

\paragraph{Emergence of Hilbert Space Structure.}
\label{subsec:emergence-of-hilbert-space-structure}
In regimes where relational constraints are weak and approximately
factorizable, distinct admissible descriptions can be combined, giving
rise to approximate linear structure.
Superposition reflects the formal coexistence of compatible descriptive
alternatives.
The inner product encodes mutual compatibility; orthogonality corresponds
to mutually exclusive descriptive regimes.
Unitary evolution is a consistency-preserving transformation valid as long
as projectability and factorizability remain intact.

\paragraph{Emergence of the Schrödinger Equation.}
\label{subsec:schrodinger-emergence}
\label{subsec:kg-to-schrodinger}
In the non-relativistic regime, separating a rapidly varying rest-energy
phase from a slowly varying envelope yields
\begin{equation}
  i\hbar\,\partial_t \psi
  = -\frac{\hbar^2}{2m}\nabla^2 \psi + V(x)\psi.
  \label{eq:schrodinger-effective}
\end{equation}

\paragraph{Operators and Non-Commutativity.}
\label{subsec:operators-from-chi}
The canonical commutation relation
$[\hat{x}, \hat{p}] = i \hbar_{\mathrm{eff}}$
expresses the non-commutativity of geometric measurements induced by
projection from the pre-geometric substrate.
The uncertainty principle emerges as a geometric consistency condition on
admissible effective descriptions.

\paragraph{Origin of Quantization.}
\label{subsec:origin-of-quantization}
Only a restricted class of localized projected configurations admits
long-lived, internally consistent descriptions.
Admissible configurations form discrete equivalence classes, yielding
effective energy quantization.
Planck's constant emerges as a universal conversion factor characterizing
the minimal structural scale at which projected descriptions remain
stable.

\paragraph{Measurement and the Born Rule.}
\label{subsec:measurement-and-the-born-rule}
Measurement outcomes correspond to effective reprojections onto mutually
exclusive descriptive regimes.
The Born rule emerges as the unique stable measure on the space of
admissible projected descriptions that remains invariant under loss of
phase coherence, coarse-graining, and environmental coupling.

\paragraph{Entanglement and Nonlocal Correlations.}
\label{subsec:entanglement-nonlocality}
Entangled systems arise when a unified relational configuration admits an
effective projection onto spatially separated degrees of freedom.
Correlations are fully compatible with relativistic causality
(Appendix~\ref{subsec:non-factorization-entanglement}).

\paragraph{Spin and Statistics.}
\label{subsec:spin-statistics}
Spin emerges as a topological property of admissible projected
configurations.
$4\pi$-periodic configurations correspond to fermionic behavior;
$2\pi$-periodic ones to bosonic behavior.
The spin--statistics connection follows from the same topological
constraints
(Appendix~\ref{subsec:4pi_soliton}).

\paragraph{Orbital Geometry as Probabilistic Visibility.}
\label{sec:orbital_geometry_visibility}
Atomic orbitals correspond to effective probabilistic visibility patterns
of admissible projected descriptions.
Regions of high probability correspond to domains where consistent
reprojection is most robust; nodal regions correspond to incompatible
descriptive regimes.

\paragraph{Scope and Limitations.}
\label{subsec:scope-and-limitations}
All standard quantum-mechanical formalisms remain valid and unchanged
within their domains.
Cosmochrony provides an interpretative and unificatory account of the
structural origin of quantum phenomena without introducing new degrees of
freedom or hidden variables.
A complete formal correspondence between the relational substrate and the
operator-based structures of quantum theory is left for future work.
