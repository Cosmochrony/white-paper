% ----------------------------------------------------------------------------
% Section 8.9 --- Summary
% From former §8.12, condensed
% ----------------------------------------------------------------------------
\subsection{Summary}
\label{subsec:summary-quantum}

Gauge interactions emerge as admissible modes of the projection process.
The photon corresponds to scalar transmission modes; $W^\pm$ and $Z^0$
bosons arise as shear-like spectral modes whose masses reflect the
spectral rigidity of the projection fiber.
Strong interactions and confinement are reinterpreted through topological
stability of knotted solitonic configurations.
Mass emerges from spectral overlap between localized configurations and
the global relaxation flux.
Entanglement and nonlocal correlations reflect the persistence of
non-factorizable admissible projected descriptions.
Quantum mechanics appears as an effective statistical framework
describing the limits of local projectability imposed by a globally
non-injective relational structure.
Classical behavior emerges as the limiting case in which only locally
stable, factorizable projected descriptions remain admissible.
