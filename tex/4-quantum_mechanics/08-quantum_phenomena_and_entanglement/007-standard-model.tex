% ----------------------------------------------------------------------------
% Section 8.7 --- Integration with the Standard Model: A Spectral
%                 Interpretation
% From former §8.8, condensed
% ----------------------------------------------------------------------------
\subsection{Integration with the Standard Model: A Spectral
Interpretation}
\label{sec:standard-model-integration}

\subsubsection*{Weak Boson Masses from Spectral Geometry}
\label{sec:weak_boson_masses}

The masses of $W^\pm$ and $Z^0$ emerge from the spectral properties of
the Hodge Laplacian~$\Delta_1$ acting on 1-forms of the fiber
bundle~$\Pi$.
The space of 1-forms decomposes into gauge-invariant subspaces
$\Omega^1_{SU(2)}$ and $\Omega^1_{U(1)}$.
Effective masses are
\[
  m_W \propto \sqrt{\lambda_{1,SU(2)}}, \qquad
  m_Z \propto \sqrt{\lambda_{1,U(1)}},
\]
with the mass ratio governed by the metric anisotropy and curvature
structure of the $\chi$-induced geometry on~$\Pi$.

\subsubsection*{Emergent Gauge Couplings}
\label{sec:gauge_couplings}

Gauge couplings are defined through normalized heat-kernel traces:
\[
  g^2 = 4\pi \left[
    \widehat{\mathrm{Tr}}_{SU(2)}
      \!\left(e^{-t_0 \Delta_{SU(2)}}\right)
    - \widehat{\mathrm{Tr}}_{U(1)}
      \!\left(e^{-t_0 \Delta_{U(1)}}\right)
  \right],
\]
\[
  g'^2 = 4\pi\,
    \widehat{\mathrm{Tr}}_{U(1)}
      \!\left(e^{-t_0 \Delta_{U(1)}}\right),
\]
with $t_0 = L_{\text{fiber}}^2$.
The Weinberg angle follows from spectral asymmetry:
\[
  \tan^2\theta_W =
  \frac{\widehat{\mathrm{Tr}}_{U(1)}
    (e^{-t_0 \Delta_{U(1)}})}
  {\widehat{\mathrm{Tr}}_{SU(2)}
    (e^{-t_0 \Delta_{SU(2)}})}.
\]

\subsubsection*{Geometric Phase Transition and Mass Generation}
\label{sec:higgs_mechanism}

Mass generation is a geometric phase transition of the $\chi$ substrate.
Below a critical spectral density $\chi_{\text{crit}}$, only massless
modes are supported.
Above it, spectral weight condenses into specific invariant subspaces,
generating discrete non-zero eigenvalues $m_n \propto \sqrt{\lambda_n}$.

\subsubsection*{Strong Sector: Topological Confinement and Color}
\label{subsec:qcd-topology}

Color charge maps to the three fundamental degrees of freedom of the
proton's trefoil topology ($Q=3$).
Topological confinement: separating components of a $Q=3$ soliton
requires linear energy increase exceeding the pair-creation threshold.
Asymptotic freedom: at short distances the global topological constraint
is not yet engaged, rendering the interaction effectively weaker.

\subsubsection*{The Origin of Mass: Spectral Overlap}
\label{subsec:yukawa-overlap}

Fermion masses are emergent quantities determined by the resonance
between internal stability spectra and the global relaxation flux:
\begin{equation}
  m_{\mathrm{eff}} \;\propto\;
    \int_{\mathrm{Fiber}}
    \mathcal{S}(\phi_n) \cdot \mathcal{R}(\chi)
    \, d\Pi .
\end{equation}
Multiple fermion generations correspond to distinct stable topological
classes of solitonic $\chi$-configurations.
The mass hierarchy $m_e \ll m_\mu \ll m_\tau$ arises from the ordered
spectrum of $L_{\mathrm{sol}}$:
\begin{equation}
  m_n \sim \mathrm{Spec}(L_{\mathrm{sol}})_n .
\end{equation}
Mixing matrices (CKM, PMNS) encode the misalignment between the mass
basis (eigenvectors of $L_{\mathrm{sol}}$) and the interaction basis
(principal axes of~$\Pi$).
CP violation originates from non-trivial topological torsion of the
projection fiber.
