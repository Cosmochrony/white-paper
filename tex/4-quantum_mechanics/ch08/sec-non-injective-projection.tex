% ----------------------------------------------------------------------------
% Section 8.1 --- Non-Factorizable Projected Descriptions and Quantum
%                 Correlations
% From former §8.1, condensed
% ----------------------------------------------------------------------------
\subsection{Non-Factorizable Projected Descriptions and Quantum
Correlations}
\label{subsec:non-injective-projection}

A single admissible configuration of~$\chi$ may correspond to multiple
distinct effective degrees of freedom.
What appear as separate particles or subsystems are multiple projective
manifestations of a single underlying ontological configuration.
Effective degrees of freedom cannot be assigned independent states; any
admissible description must be globally consistent with the
underlying~$\chi$ configuration, leading to persistent correlations that
do not rely on signal exchange or spacetime proximity.

\subsubsection*{Ontological Monism and the Shared Projection Hypothesis}
\label{subsec:ontological-monism-shared-projection}

A single connected excitation in~$\chi$ can be represented as several
spatially separated effective excitations.
The observed separation is a property of the projected metric
representation, not a fundamental separation of the underlying entity.

\paragraph{Shared Fiber Phase and Spin Correlations.}
\label{subsec:shared-fiber-phase-spin}
Spin correlations can be read as correlations of a \emph{shared}
internal degree of freedom of the projection fiber.
A measurement at one location selects a locally stable reprojection of
that shared fiber degree of freedom; the correlated statistics at the
distant location reflect the restricted set of reprojections still
compatible with the same underlying configuration.

\begin{figure}[t]
  \centering
  \begin{tikzpicture}[
    font=\small,
    node distance=10mm,
    box/.style={draw, rounded corners, align=center,
      inner sep=6pt},
    arrow/.style={-Latex, thick},
    note/.style={align=left, font=\footnotesize},
    micro/.style={draw, circle, inner sep=1.5pt}
  ]
    \node[micro] (c1) at (-2.4, 1.5) {};
    \node[micro] (c2) at (-1.2, 1.7) {};
    \node[micro] (c3) at ( 0.0, 1.55) {};
    \node[micro] (c4) at ( 1.2, 1.75) {};
    \node[micro] (c5) at ( 2.4, 1.6) {};
    \node[note, above=3mm of c3, align=center]
      {Multiple admissible $\chi$ configurations\\
       (same fiber under $\pi$)};
    \node[box, below=16mm of c3,
      minimum width=6.8cm] (chieff)
      {$\chi_{\mathrm{eff}}$\\
       \footnotesize single effective reality\\
       \footnotesize non-factorisable (entangled)};
    \node[box, below left=12mm and 14mm of chieff] (obsA)
      {Outcomes in context $A$};
    \node[box, below right=12mm and 14mm of chieff] (obsB)
      {Outcomes in context $B$};
    \draw[arrow] (c1) -- (chieff.north west);
    \draw[arrow] (c2) -- (chieff.north);
    \draw[arrow] (c3) -- (chieff.north);
    \draw[arrow] (c4) -- (chieff.north);
    \draw[arrow] (c5) -- (chieff.north east);
    \node[note, left=2mm of chieff, anchor=east]
      {\footnotesize non-injective\\
       \footnotesize projection $\pi$};
    \draw[arrow] (chieff)
      -- node[left=2mm, note] {$\mathcal{O}_A$} (obsA);
    \draw[arrow] (chieff)
      -- node[right=2mm, note] {$\mathcal{O}_B$} (obsB);
  \end{tikzpicture}
  \caption{Entangled (non-factorisable) regime.
    Multiple underlying $\chi$ configurations map to the same
    $\chi_{\mathrm{eff}}$.
    Different measurement contexts define distinct operational
    projections producing correlated outcomes without introducing
    multiple effective realities.}
  \label{fig:entanglement-two-projections}
\end{figure}
