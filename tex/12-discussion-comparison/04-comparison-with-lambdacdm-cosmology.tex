\subsection{Comparison with $\Lambda$CDM Cosmology}
    \label{subsec:comparison-with-lambdacdm-cosmology}

  The $\Lambda$ CDM model successfully accounts for large-scale cosmological observations by postulating dark energy, cold
  dark matter, and an early inflationary phase\cite{peebles1993principles, planck2020results}.
  However, these components are introduced phenomenologically rather than derived from first principles.

  In Cosmochrony, cosmic expansion follows directly from the monotonic increase of the characteristic wavelength
  associated with $\chi$.
  The observed Hubble law emerges as a kinematic consequence of differential relaxation, without invoking a
  cosmological constant.
  The present-day Hubble parameter satisfies
  \[
    H(t) = \frac{\dot{\chi}}{\chi},
  \]
  leading naturally to $H_0 \sim c / \chi(t_0)$.

  Dark energy is thus replaced by a geometric relaxation process, and cosmic acceleration reflects the cumulative
  effect of this dynamics over large scales.
  At the background level, Cosmochrony reproduces the homogeneous and isotropic expansion described by
  Friedmann--Lema\^{\i}tre cosmology, while offering an alternative interpretation of its driving mechanism.

  Unlike $\Lambda CDM$, which requires fine-tuned initial conditions and an unexplained dark energy component, Cosmochrony
  derives cosmic acceleration from the geometric relaxation of $\chi$, naturally predicting a decreasing $H(z)$
  without free parameters.
  This resolves the coincidence problem (why $\Omega_\Lambda \sim \Omega$ today) and explains the Hubble tension as an epoch-dependent
  effect, while maintaining compatibility with large-scale structure observations.

  At large angular scales, $\Lambda$CDM treats departures from scale invariance in the CMB as purely
  statistical fluctuations around an ensemble-averaged spectrum.
  No deterministic structure is associated with individual low-$\ell$ modes beyond cosmic variance.

  By contrast, Cosmochrony allows for a scale-dependent attenuation of global modes, reflecting constraints
  on the largest-scale configurations of the $\chi$ field.
  In this view, the suppression of power at low multipoles is not a statistical accident but a structural consequence of
  the relaxation dynamics of the fundamental field.
