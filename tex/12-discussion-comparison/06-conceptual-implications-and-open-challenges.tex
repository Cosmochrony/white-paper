\subsection{Conceptual Implications and Open Challenges}
  \label{subsec:conceptual-implications-and-open-challenges}

  Cosmochrony offers a unifying geometric narrative in which time, distance, energy, gravitation, and quantization
  originate from a single evolving field.
  This conceptual economy is a strength, but it also imposes stringent consistency requirements.

  Several open questions remain:
  \begin{itemize}
    \item the precise mapping between $\chi$-dynamics and observed CMB anisotropies,
    \item the treatment of non-equilibrium quantum measurements,
    \item the emergence of gauge symmetries and interaction hierarchies,
    \item and the robustness of solitonic particle configurations under extreme conditions.
  \end{itemize}

  Addressing these challenges will require:

  \begin{enumerate}
    \item Numerical simulations of $\chi$-dynamics to quantify structure formation and CMB anisotropies.
    \item Collaborations with loop quantum gravity to explore discretized versions of $\chi$ at Planck scales.
    \item Experimental tests of predicted $\chi$-dependent effects in quantum decoherence and gravitational wave propagation.
  \end{enumerate}

  Progress in these areas may elevate Cosmochrony from a conceptual framework to a predictive theory.
