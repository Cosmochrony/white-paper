\subsection{Conceptual Implications and Open Challenges}
  \label{subsec:conceptual-implications-and-open-challenges}

  Cosmochrony offers a unifying geometric narrative in which time, distance, energy, gravitation, and
  quantization originate from a single evolving field.
  This conceptual economy is a central strength of the framework, but it also requires a careful
  reassessment of the status of several foundational notions traditionally treated as independent.

  In particular, Cosmochrony suggests that time, energy, and irreversibility are not separate physical
  primitives.
  A concrete realization of this separation, including an explicit definition of the underlying
  relaxation operator and its spectral role in mass generation, is outlined in
  Appendix~\ref{subsec:spectral-relaxation}.

  The monotonic relaxation of the $\chi$ field provides the fundamental temporal ordering, while energy
  quantifies the residual capacity of $\chi$ configurations to relax.
  Irreversibility, in turn, reflects the progressive exhaustion of this relaxation capacity.
  From this perspective, temporal flow and energetic processes are two complementary descriptions of
  the same underlying geometric dynamics, rather than independent axioms of nature.

  While this reinterpretation resolves several conceptual tensions---such as the origin of the arrow of
  time or the status of energy conservation---it also raises important open questions.
  Among these are:
  \begin{itemize}
    \item the precise mapping between $\chi$ dynamics and observed CMB anisotropies,
    \item the treatment of non-equilibrium quantum measurements and decoherence,
    \item the emergence of gauge symmetries and interaction hierarchies,
    \item and the robustness of solitonic particle configurations under extreme conditions.
  \end{itemize}

  Addressing these challenges will require a combination of analytical, numerical, and experimental
  approaches, including:
  \begin{enumerate}
    \item large-scale numerical simulations of $\chi$ dynamics to quantify structure formation and
    cosmological signatures,
    \item exploration of discretized or network-based realizations of $\chi$ at microscopic scales,
    \item and experimental tests of predicted $\chi$-dependent effects in quantum coherence,
    gravitation, and radiation processes.
  \end{enumerate}

  Progress along these directions may elevate Cosmochrony from a unifying conceptual framework to a
  quantitatively predictive theory, while preserving its minimal ontological foundation.
