\subsection{Relational Consistency of the Effective Lagrangian}
  \label{sec:born-lagrangian_derivation}

  The effective Lagrangian for the $\chi$ field is \textbf{not postulated as a fundamental principle}
  but introduced as an \textbf{auxiliary variational representation} of the effective dynamics.
  This appendix clarifies how its form is \emph{consistent} with the relational dynamics of $\chi$
  introduced in Section~\ref{sec:dynamical-equation-for-the-chi-field}, without assuming any pre-existing spacetime
  structure or differential operators at the fundamental level.

  \subsubsection{Step 1: Relational Constraint and Bounded Variations}

    At the fundamental level, the dynamics of $\chi$ are discrete and relational.
    Admissible variations are constrained by the relational bound
    introduced in Section~\ref{subsec:locality-causality-and-the-role-of-the-bound-c}:
    \begin{equation}
      \mathcal{C}_i[\chi] \equiv \sum_j K_{ij}(\chi_i-\chi_j)^2 \le \chi_c^2,
      \label{eq:A12-relational-constraint}
    \end{equation}
    where $K_{ij}=K_{ji}$ encodes relational connectivity.
    This constraint enforces bounded relative variations and plays the role of a structural
    causality condition.

    No spacetime derivatives or continuum notions are assumed at this stage.

  \subsubsection{Step 2: Ordered Evolution and Variational Structure}

    The discrete dynamics of $\chi$ admit a variational formulation with inequality constraints,
    as detailed in Section~\ref{subsec:minimal-kinematic-constraint}.
    Introducing an ordering parameter $\lambda$ and non-negative Lagrange multipliers
    $\mu_i(\lambda)\ge0$, the action reads:
    \begin{equation}
      S[\{\chi_i\},\{\mu_i\}]
      =
      \int d\lambda
      \left[
        \sum_i \frac{m_i}{2}\left(\frac{d\chi_i}{d\lambda}\right)^2
      -
      U[\{\chi_i\}]
      -
        \sum_i \mu_i(\lambda)\big(\mathcal{C}_i[\chi]-\chi_c^2\big)
      \right].
    \end{equation}

    Global monotonicity is fixed by the order functional
    introduced in Section~\ref{subsec:parameter-independent-relaxation},
    ensuring an oriented evolution without introducing a fundamental notion of time.

  \subsubsection{Step 3: Projectable Regimes and Continuum Description}

    In projectable regimes, the spectral properties of the connectivity matrix $K_{ij}$
    allow a low-dimensional embedding, as discussed in Section~\ref{subsec:relational-foundation-pointer}.
    Only in this regime can continuum descriptors be introduced \emph{a posteriori}.

    Under this projection, the relational bound
    Eq.~\eqref{eq:A12-relational-constraint} maps to a bounded-gradient condition
    of the schematic form:
    \begin{equation}
      |\nabla \chi|^2 \le c^2,
    \end{equation}
    where $\nabla$ is an emergent operator and $c$ is an effective structural scale.
    This inequality characterizes the effective regime but has no fundamental status.

  \subsubsection{Step 4: Auxiliary Born--Infeld--like Representation}

    The bounded-gradient condition admits a compact auxiliary variational representation.
    A convenient choice is a Born--Infeld--like functional:
    \begin{equation}
      \mathcal{L}_{\mathrm{eff}}
      \sim
      - c^2 \sqrt{1-\frac{|\nabla \chi|^2}{c^2}}
      + \partial_t \chi,
      \label{eq:A12-effective-lagrangian}
    \end{equation}
    where $\partial_t$ and $\nabla$ are emergent operators defined only in the projectable
    continuum description.

    This functional is \textbf{not derived from microscopic $\chi$ dynamics} and is
    \textbf{not unique}.
    Its sole role is to reproduce the effective equations governing the bounded-gradient regime
    in a variationally convenient form, consistent with Eq.~(13).

  \subsubsection{Step 5: Connection to Emergent Geometry}

    Once an effective Lagrangian representation is introduced, an effective metric can be defined
    \emph{a posteriori} through the Hessian:
    \begin{equation}
      g_{\mu\nu}^{\mathrm{eff}}
      \propto
      \frac{\partial^2 \mathcal{L}_{\mathrm{eff}}}
      {\partial(\partial_\mu\chi)\partial(\partial_\nu\chi)},
    \end{equation}
    up to conformal rescalings and field redefinitions.
    This construction is consistent with the emergent geometric description discussed in
    Section~\ref{subsec:collective-gravitational-coupling-and-operational-geometry}.
    Geometry here is a derived descriptor of effective relational dynamics, not a primitive entity.

\subsection{Summary of Key Points}

  \begin{itemize}
    \item The fundamental dynamics of $\chi$ are discrete and relational.
    \item Bounded variations are enforced by inequality constraints on relational differences.
    \item Continuum notions arise only in projectable regimes.
    \item The Born--Infeld--like Lagrangian is an auxiliary effective representation.
    \item No spacetime structure or fundamental action principle is assumed.
  \end{itemize}

\subsection{Scope and Limitations}

  The construction presented here establishes consistency, not derivation.
  Outside projectable regimes, no spacetime description or Lagrangian formulation is expected
  to exist.
  Different auxiliary functionals leading to the same effective equations would be equally
  admissible within the Cosmochrony framework.
