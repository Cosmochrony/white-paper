\subsection{Relational Consistency of the Effective Lagrangian}
  \label{sec:born-lagrangian_derivation}

  The effective Lagrangian for the \(\chi\) field is \textbf{not postulated as a fundamental principle}
  but introduced as an \textbf{auxiliary variational representation}
  of the effective dynamics.
  This appendix clarifies how its form is \emph{consistent} with the relational dynamics of \(\chi\) introduced in
  Section~\ref{sec:dynamical-equation-for-the-chi-field}, without assuming any pre-existing spacetime structure or
  differential operators at the fundamental level.

  \subsubsection*{Step 1: Relational Constraint and Bounded Variations}
    \label{subsec:A12-relational-constraint}

    At the fundamental level, the dynamics of \(\chi\)
    are discrete and relational.
    Admissible variations are constrained by the relational bound introduced in
    Section~\ref{subsec:locality-causality-and-the-role-of-the-bound-c}:
    \begin{equation}
      \mathcal{C}_i[\chi] \equiv \sum_j K_{ij}(\chi_i - \chi_j)^2 \le \chi_c^2,
      \label{eq:A12-relational-constraint}
    \end{equation}
    where \(K_{ij} = K_{ji}\)
    encodes relational connectivity.
    This constraint enforces bounded relative variations and plays the role of a structural causality condition.
    \textbf{No spacetime derivatives or continuum notions are assumed at this stage.}

  \subsubsection*{Step 2: Ordered Evolution and Variational Structure}
    \label{subsec:A12-variational-structure}

    The discrete dynamics of \(\chi\)
    admit a variational formulation with inequality constraints. Introducing an ordering parameter \(\lambda\)
    and non-negative Lagrange multipliers \(\mu_i(\lambda) \ge 0\), the action reads:
    \begin{equation}
      S[\{\chi_i\}, \{\mu_i\}] = \int d\lambda
      \left[ \sum_i \frac{1}{2} \left(\frac{d\chi_i}{d\lambda}\right)^2 - U[\{\chi_i\}] - \sum_i \mu_i(\lambda)
        \left(\mathcal{C}_i[\chi] - \chi_c^2\right) \right].
      \label{eq:A12-action}
    \end{equation}
    \begin{itemize}
      \item The kinetic term \(\frac{1}{2} \left(\frac{d\chi_i}{d\lambda}\right)^2\) is a \textbf{quadratic ansatz}
      for the discrete dynamics, chosen for its simplicity and compatibility with the monotonicity condition.
      \item \(U[\{\chi_i\}]\) is a potential encoding additional relational constraints (e.g., topological terms).
      \item The Lagrange multipliers \(\mu_i(\lambda)\) enforce the inequality constraints
      \(\mathcal{C}_i[\chi] \le \chi_c^2\).
    \end{itemize}

    \paragraph{Global Monotonicity:}
      The ordering parameter \(\lambda\) is chosen such that the \textbf{order functional}
      \[
        \Xi[\chi(\lambda)] \equiv \sum_i \chi_i(\lambda)
      \]
      satisfies \(\frac{d\Xi}{d\lambda} \ge 0\)
      . This ensures a global orientation of evolution without introducing a fundamental notion of time.

  \subsubsection*{Step 3: Projectable Regimes and Continuum Description}
    \label{subsec:A12-projectable-regimes}

    In \textbf{projectable regimes}, the spectral properties of the connectivity matrix \(K_{ij}\)
    allow a low-dimensional embedding (Section~\ref{subsec:relational-foundation-pointer}
    ). Under this embedding, the relational bound Eq.~\eqref{eq:A12-relational-constraint}
    maps to a bounded-gradient condition:
    \begin{equation}
      |\nabla \chi|^2 \le c^2,
      \label{eq:A12-bounded-gradient}
    \end{equation}
    where:
    \begin{itemize}
      \item \(\nabla\) is an \textbf{emergent operator} defined on the effective continuum.
      \item \(c\) is an \textbf{effective structural scale}, not a fundamental constant.
    \end{itemize}
    This inequality characterizes the effective regime but has no fundamental status.

  \subsubsection*{Step 4: Auxiliary Born--Infeld-like Representation}
    \label{subsec:A12-born-infeld}

    The bounded-gradient condition Eq.~\eqref{eq:A12-bounded-gradient}
    admits a compact auxiliary variational representation. A \textbf{convenient choice}
    is a Born--Infeld-like functional:
    \begin{equation}
      \mathcal{L}_{\mathrm{eff}} \sim -c^2 \sqrt{1 - \frac{|\nabla \chi|^2}{c^2}} + \partial_t \chi,
      \label{eq:A12-effective-lagrangian}
    \end{equation}
    where:
    \begin{itemize}
      \item \(\partial_t\) is an emergent time derivative, defined only in projectable regimes.
      \item This functional is \textbf{not derived from microscopic \(\chi\) dynamics} but is \textbf{consistent with}
      the effective equations governing the bounded-gradient regime.
      \item Alternative representations (e.g., polynomial expansions) would be equally valid.
    \end{itemize}

  \subsubsection*{Step 5: Connection to Emergent Geometry}
    \label{subsec:A12-emergent-geometry}

    Once an effective Lagrangian representation is introduced, an effective metric can be defined \emph{a posteriori}
    through the Hessian:
    \begin{equation}
      g_{\mu\nu}^{\mathrm{eff}} \propto
      \frac{\partial^2 \mathcal{L}_{\mathrm{eff}}}{\partial(\partial_\mu\chi)\partial(\partial_\nu\chi)},
      \label{eq:A12-emergent-metric}
    \end{equation}
    up to conformal rescalings and field redefinitions.
    This construction is consistent with the emergent geometric
    description discussed in Section~\ref{subsec:collective-gravitational-coupling-and-operational-geometry}.

  \subsubsection*{Summary of Key Points}
    \begin{itemize}
      \item The fundamental dynamics of \(\chi\) are discrete and relational.
      \item Bounded variations are enforced by inequality constraints on relational differences.
      \item Continuum notions arise only in projectable regimes.
      \item The Born--Infeld-like Lagrangian is an \textbf{auxiliary effective representation}
      , not a fundamental derivation.
      \item No spacetime structure or fundamental action principle is assumed.
    \end{itemize}

  \subsubsection*{Scope and Limitations}
    The construction presented here establishes \textbf{consistency}, not derivation.
    Outside projectable regimes, no spacetime description or Lagrangian formulation is expected to
    exist.
    Different auxiliary functionals leading to the same effective equations would be equally admissible within the
    Cosmochrony framework.
