\subsection{Topological Origin of Fermionic and Bosonic Statistics}
  \label{app:relational_spin_statistics}

  Within the fully relational formulation of Cosmochrony, the distinction between
  fermionic and bosonic behavior does not arise from imposed quantum statistics or
  from a fundamental spinorial ontology.
  Instead, it originates from the internal topological structure of localized
  configurations of the \(\chi\) field in its configuration space.

  \paragraph{Internal rotations and configuration space topology.}
    At the relational level, the notion of rotation is not defined with respect to
    physical space.
    It refers instead to closed paths in the configuration space of admissible
    \(\chi\) configurations.
    Two configurations are considered equivalent if they are related by a continuous
    deformation that preserves all relational relaxation constraints.

    Certain classes of configurations exhibit nontrivial topology in this
    configuration space.
    For these configurations, a closed path corresponding to a \(2\pi\) internal
    reorientation does not return the system to an equivalent configuration.
    Only a full \(4\pi\) cycle restores relational equivalence.
    This topological obstruction is naturally associated with a non-orientable structure in configuration space.

  \paragraph{Topological Mass Ratios and Knot Theory.}
    The following considerations should be understood as heuristic geometric
    interpretations of the spectral hierarchy discussed in Section~\ref{subsec:perspectives_mass_spectrum},
    rather than as a derived volumetric mass formula.

    The distinction between fermionic and bosonic statistics is not the only consequence of the \(\chi\)-field's
    topological structure.
    The \textbf{mass ratios between particles} (e.g., proton-to-electron) also emerge from the
    \textbf{knot-like configurations} of \(\chi\):
    \begin{itemize}
      \item An \textbf{electron} corresponds to a \textbf{twisted unknot} (\(Q_e = 1\)), with a fiber volume under the
      projection
      \(\Pi\) scaling as \(\text{Vol}(\Pi^{-1}(\text{electron})) \propto \chi_c\).
      \item A \textbf{proton} corresponds to a \textbf{trefoil knot} (\(Q_p = 3\)), with a fiber volume scaling as
      \(\text{Vol}(\Pi^{-1}(\text{proton})) \propto \chi_c^3\).
    \end{itemize}
    The observed mass ratio \(m_p/m_e \approx 1836\) is then derived from the ratio of these volumes:
    \[
      \frac{m_p}{m_e} = \frac{\text{Vol}(\Pi^{-1}(\text{proton}))}{\text{Vol}(\Pi^{-1}(\text{electron}))} \approx 27
      \chi_c^2.
    \]
    Assuming \(\chi_c \approx 8.3\) (from \(\pi \chi_c^2 \approx 1836/27\)), this provides a
    \textbf{topological explanation} for the mass ratio, independent of ad hoc parameters.
    This mechanism is consistent with the \textbf{relational-topological framework} described in this appendix, where
    particle properties emerge from the internal structure of \(\chi\).

  \paragraph{Emergence of fermionic behavior.}
    Configurations with intrinsic \(4\pi\)-periodicity belong to topological sectors
    that are double-valued under \(2\pi\) reorientation.
    When such configurations admit an effective geometric projection, this internal
    property manifests as fermion-like behavior:
    \begin{itemize}
      \item a sign change under \(2\pi\) rotations,
      \item restoration of equivalence only after a \(4\pi\) cycle,
      \item and transformation properties characteristic of spin-\(\tfrac{1}{2}\)
      degrees of freedom.
    \end{itemize}

    These features arise without introducing fundamental spinors.
    They reflect the topology of the underlying relational configuration rather than a
    representation of the Lorentz group imposed at the outset.

    This \(4\pi\)-periodicity is not only responsible for fermionic statistics but also contributes to the
    \textbf{topological stability} of the soliton.
    For example, the electron's twisted unknot configuration is protected from decay by its nontrivial phase structure,
    which is directly linked to its \textbf{mass} via the fiber volume under \(\Pi\).
    The same topological protection applies to the proton's trefoil knot, explaining its stability and the observed mass
    ratio.

  \paragraph{Emergence of bosonic behavior.}
    Other classes of relational configurations are topologically orientable.
    For these configurations, a \(2\pi\) internal reorientation is sufficient to return
    to an equivalent state.
    When projected onto an effective geometric description, such configurations
    exhibit boson-like behavior, including integer-spin transformation properties and
    the absence of sign inversion under \(2\pi\) rotations.

    The distinction between fermionic and bosonic excitations is therefore encoded in
    the topology of configuration space rather than in any dynamical or statistical
    assumption.

  \paragraph{Geometric metaphors and their limits.}
    Geometric metaphors—such as Möbius twists, non-orientable loops, or knotted
    structures—may be used heuristically to visualize these internal topological
    features.
    However, such images are meaningful only after projection onto an effective
    geometric description.
    For instance, while a \textbf{M\"obius strip} can heuristically illustrate the \(4\pi\)-periodicity of an
    electron's spin, the \textbf{trefoil knot} provides a geometric metaphor for the proton's topological complexity.
    These metaphors become quantitatively meaningful only when linked to the \textbf{fiber volumes} under the projection
    \(\Pi\), which determine the particle's mass and stability.
    At the relational level, no spatial embedding exists, and these metaphors serve solely as intuitive aids.

  \paragraph{Relation to the spin--statistics connection.}
    The relational-topological distinction between \(4\pi\)- and \(2\pi\)-periodic
    configurations provides a natural qualitative explanation of the
    spin--statistics connection.
    Fermionic and bosonic behavior emerge as consequences of internal topological
    constraints rather than as independent quantum postulates.

    While this construction does not constitute a formal proof of the
    spin--statistics theorem, it demonstrates that the observed dichotomy between
    fermions and bosons can arise consistently from the internal organization of
    \(\chi\), prior to and independently of any effective geometric or quantum
    description.

    The topological distinction between fermions and bosons also underpins the \textbf{mass hierarchy} observed in
    particle physics.
    The proton's trefoil knot structure, with its higher topological complexity, results in a larger fiber volume under
    \(\Pi\) compared to the electron's twisted unknot.
    This difference in fiber volumes directly translates into the mass ratio \(m_p/m_e \approx 1836\), demonstrating
    how \textbf{spin, statistics, and mass} are interconnected through the \(\chi\)-field's topological structure.

  \paragraph{Conceptual role.}
    This subsection completes the relational account of particle properties in
    Cosmochrony by showing how spin and statistics arise from topology alone.
    It reinforces the view that quantum transformation properties are emergent features
    of relational structure, not fundamental ingredients of the theory.
