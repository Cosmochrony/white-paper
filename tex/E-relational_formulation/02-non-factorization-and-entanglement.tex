\subsection{Non-Factorization and Entanglement}
  \label{subsec:non-factorization-entanglement}

  This appendix provides a formal and ontological clarification of the relational
  mechanism underlying quantum entanglement, as introduced at a phenomenological
  level in Section~\ref{subsec:entanglement-nonlocality}.
  Its purpose is not to restate the physical interpretation given there, but to make explicit the non-factorization
  properties of the $\chi$ substrate from which those effective behaviors arise.

  Within the relational formulation of Cosmochrony, configurations of the $\chi$ field
  do not generically decompose into independent subsystems.
  Factorization, understood as a decomposition preserving internal relaxation structure while isolating disjoint
  subsets of relations, is therefore not fundamental.
  It emerges only in restricted regimes where relational couplings are weak, hierarchically organized, or dynamically
  suppressed.

  A relational configuration is said to be \emph{non-factorizable} when no decomposition
  exists that preserves its internal relaxation structure while yielding independent
  subconfigurations.
  In such cases, what appear as multiple subsystems at the effective geometric level correspond, at the relational
  level, to a single indivisible configuration.

  Quantum entanglement arises as a direct manifestation of persistent non-factorization.
  When a non-factorizable relational configuration admits an effective projection onto
  spatially separated degrees of freedom, its components may become geometrically
  distant while remaining relationally inseparable.
  Measurement projections acting on one effective subsystem therefore constrain the set of admissible projections
  associated with others, independently of their spatial separation.

  These constraints do not arise from dynamical updates propagating through spacetime.
  They reflect the incompatibility of certain relational patterns with the selected
  projection and are fully determined by global relational consistency conditions.

  Importantly, non-factorization should not be conflated with dynamical nonlocality.
  As emphasized in Section~9.6, all dynamical evolution of $\chi$ remains governed by
  bounded relaxation constraints that respect the invariant speed $c$ once an effective
  causal structure emerges.
  Entanglement correlations therefore do not enable superluminal signaling and do not violate relativistic causality.

  This appendix thus provides the ontological underpinning of the entanglement
  phenomenology discussed in Section~9.6, framing quantum correlations as structural
  properties of the relational configuration space of $\chi$, rather than as consequences
  of nonlocal dynamics or measurement-induced collapse.

  \paragraph{Projection-induced non-factorization.}
    The non-factorizability of relational configurations discussed above acquires direct
    physical relevance only through the projection
    $\Pi:\mathcal{C}_\chi \rightarrow \mathcal{C}_{\mathrm{eff}}$,
    which maps admissible configurations of the relational substrate $\chi$ to effective
    descriptions.
    Because $\Pi$ is generically non-injective, a single effective configuration
    $y \in \mathcal{C}_{\mathrm{eff}}$ corresponds to an equivalence class of underlying
    relational configurations,
    \begin{equation}
      \Pi^{-1}(y) \subset \mathcal{C}_\chi ,
    \end{equation}
    referred to as the projection fiber.
    Entanglement arises precisely when this fiber contains globally constrained
    configurations that do not admit a decomposition into independent relational
    substructures compatible with the effective subsystem decomposition.

  \paragraph{Failure of ontological factorization under projection.}
    Consider an effective description $y$ that admits a decomposition into spatially
    separated subsystems, $y = (y_A, y_B)$.
    Ontological factorization would require that each admissible pre-image
    $x \in \Pi^{-1}(y)$ decompose into independent relational configurations associated
    with $y_A$ and $y_B$.
    However, for non-injective projections, admissible fibers generically consist of
    globally constrained relational configurations that do not factorize in this manner.
    As a consequence, no conditioning on underlying relational degrees of freedom can
    restore a product structure for joint outcome statistics.

  \paragraph{Compression and the limits of entanglement.}
    The degree of non-injectivity of the projection $\Pi$ controls the structure of
    projection fibers and therefore the persistence of non-factorizable correlations.
    If the projection is effectively injective on the relevant relational degrees of
    freedom, fibers collapse to single elements and ontological factorization is
    recovered.
    Conversely, if the projection is excessively coarse-grained, relational constraints
    are erased and effective descriptions become fully factorized.
    Entanglement thus arises only in an intermediate regime, where projection preserves
    sufficient global relational structure to prevent factorization, while still allowing
    a stable decomposition into effective subsystems.

  \paragraph{Structural origin of quantum correlations.}
    In this framework, entanglement correlations reflect residual global constraints
    within projection fibers rather than dynamical influences or information exchange
    between subsystems.
    They are invariant under spatial separation and compatible with relativistic causal
    structure, as they arise from the admissibility conditions imposed by projection rather
    than from spacetime-mediated interactions.
    This completes the ontological account of entanglement in Cosmochrony, grounding
    quantum correlations in the non-factorizable structure of relational configurations
    and their non-injective projection onto effective descriptions.
