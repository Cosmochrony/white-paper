\subsection{Derivation of $\chi_{\text{eff}}$ from Relational Observables}
\label{subsec:chi-eff-derivation}

The projected field $\chi_{\text{eff}}$ is not fundamental: it is a coarse-grained
description of relational configurations of $\chi$. The construction below introduces
$\chi_{\text{eff}}$ \emph{without} assuming a background manifold, coordinates, or
differential operators.

\subsubsection*{Step 1: Coarse-graining domain defined by relational distance}
  Fix an operational coarse-graining scale $\ell_0$.
  For any node $i$, define the relational neighborhood
  \begin{equation}
    \mathcal{V}_{\ell_0}(i)\;=\;\{\,j \mid d_{ij}\le \ell_0\,\},
    \label{eq:relational-ball}
  \end{equation}
  where $d_{ij}$ is the shortest-path distance defined in
  Eq.~\eqref{eq:relational-distance}.

\subsubsection*{Step 2: Definition of the effective field}
  Define the coarse-grained (projected) field at node $i$ by a local average:
  \begin{equation}
    \chi_{\text{eff}}(i)\;=\;\frac{1}{|\mathcal{V}_{\ell_0}(i)|}
    \sum_{j\in \mathcal{V}_{\ell_0}(i)} \chi_j.
    \label{eq:chi-eff-discrete}
  \end{equation}
  This step is purely relational: it only uses adjacency/connectivity and the induced
  distance functional.

\subsubsection*{Step 3: Emergent differential operators as shorthand}
  Once a projectable regime exists, continuum notation becomes a compact shorthand for
  graph operations. For instance, for a locally regular region of the network, discrete
  finite differences provide an operational meaning to gradients:
  \begin{equation}
    \partial_\mu \chi_{\text{eff}}(x)
    \;\widehat{=}\;
    \frac{\chi_{\text{eff}}(x+\hat{\mu}\,\ell_0)-\chi_{\text{eff}}(x)}{\ell_0},
    \label{eq:chi-eff-gradient-shorthand}
  \end{equation}
  while integrals are operationally replaced by weighted sums over coarse-grained cells.
  In this sense, $\nabla$ and $\int$ are \emph{notations} for relational procedures rather
  than primitive geometric inputs.

  \paragraph{Role of the bound \texorpdfstring{$c$}{c}.}
    The constant \(c\) does not represent a signal velocity in a pre-existing
    spacetime.
    At the relational level, it defines a fundamental upper bound on the rate at which
    the internal structure of a \(\chi\) configuration may change.
    This bound is encoded directly in the relaxation constraints governing the
    evolution of \(\chi\).

    As a consequence, while relational configurations may be globally defined and
    non-factorizable, their evolution remains locally constrained once an effective
    notion of causality emerges.
    No relational reconfiguration can induce arbitrarily rapid changes in projected
    descriptions.

  \paragraph{Emergence of locality.}
    Locality is not assumed at the fundamental level.
    It arises only when relational configurations admit a projectable regime in which
    spatial organization becomes meaningful.
    In such regimes, the bound \(c\) manifests operationally as a maximum propagation
    speed for disturbances in the projected field \(\chi_{\mathrm{eff}}\).

    This emergent locality ensures that effective interactions respect relativistic
    causal ordering.
    Dynamical influences propagate continuously and are limited by the same invariant
    speed that governs relativistic kinematics.

  \paragraph{Compatibility with entanglement.}
    The coexistence of global relational correlations and bounded dynamical evolution
    resolves the apparent tension between quantum entanglement and relativistic
    causality.
    Entangled configurations correspond to non-factorizable relational structures that
    are globally specified.
    Their correlated measurement outcomes do not result from superluminal influences,
    but from the consistency constraints imposed by a shared relational configuration.

    Because no dynamical update propagates between subsystems during measurement,
    entanglement does not violate the causal bound \(c\).
    All observable dynamical effects remain subluminal in the emergent geometric
    description.

  \paragraph{Causality as a projected concept.}
    Causality, in Cosmochrony, is not a primitive feature of the relational ontology.
    It is a property of the effective geometric description that emerges once temporal
    ordering and spatial separation become meaningful.
    The bound \(c\) ensures that this emergent causal structure is well-defined and
    internally consistent.

    Thus, Cosmochrony distinguishes sharply between:
    \begin{itemize}
      \item \emph{relational correlations}, which may be global and nonlocal in a
      geometric sense, and
      \item \emph{dynamical causation}, which is always constrained by the bound \(c\)
      once projected.
    \end{itemize}

  \paragraph{Conceptual role.}
    This distinction clarifies how Cosmochrony accommodates both the holistic features
    of quantum phenomena and the strict causal structure of relativistic physics
    without contradiction.
    The bound \(c\) acts as the bridge between the relational and geometric levels,
    ensuring that emergent spacetime dynamics remain causal even though the underlying
    ontology is non-geometric and nonlocal.

    The present subsection therefore establishes the foundation for reconciling
    entanglement, relativistic causality, and the emergence of spacetime within a
    single coherent framework.
