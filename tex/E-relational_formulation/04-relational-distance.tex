\subsection{Relational Distance as a Minimal Path Functional}
  \label{subsec:relational-distance}

  To prevent circularity, the effective notion of distance used in projected
  descriptions is introduced \emph{only} as an operational summary of relational
  connectivity, not as a primitive geometric input.

  Given a relational network with nodes $i=1,\dots,N$ and symmetric connectivities
  $K_{uv}>0$ between adjacent nodes $(u,v)$, we define the operational distance
  between nodes $i$ and $j$ as the minimal weighted path length:
  \begin{equation}
    d_{ij} \;=\; \min_{\gamma_{ij}} \sum_{(u,v)\in \gamma_{ij}} w_{uv},
    \label{eq:relational-distance}
  \end{equation}
  where $\gamma_{ij}$ ranges over all paths connecting $i$ to $j$, and the edge weights
  $w_{uv}$ encode \emph{resistance to relaxation}.

  A minimal bounded choice consistent with positivity, symmetry, and monotonicity is
  \begin{equation}
    w_{uv} \;=\; \frac{1}{K_{uv}}
    \;=\; \frac{1}{K_0}\left[1+\left(\frac{\chi_u-\chi_v}{\chi_c}\right)^2\right],
    \label{eq:weight-wuv}
  \end{equation}
  with $K_0$ the maximal stiffness scale and $\chi_c$ a characteristic variation scale.

  \paragraph{Metric properties.}
    By construction, $d_{ij}$ is symmetric, positive, and satisfies the triangle inequality,
    since it is defined as a shortest-path functional on a weighted graph.

  \paragraph{Minimality and non-uniqueness.}
    The choice~\eqref{eq:weight-wuv} is \emph{not unique}. Any weight function $w_{uv}$
    preserving (i) positivity, (ii) symmetry, and (iii) monotonic increase with
    $|\chi_u-\chi_v|$ yields a valid operational distance. For instance,
    \begin{equation}
      w_{uv} \;=\; \frac{1}{K_0}\exp\!\left(\frac{|\chi_u-\chi_v|^2}{\chi_c^2}\right)
      \label{eq:weight-wuv-alt}
    \end{equation}
    is equally acceptable.
    The specific form is a phenomenological ansatz to be
    constrained by the requirement that a smooth geometric regime emerges at large scales.
