\subsection{Entanglement and Nonlocal Correlations}
  \label{subsec:entanglement-and-nonlocal-correlations}

  Within the Cosmochrony framework, quantum entanglement does not correspond to a
  physical linkage or interaction between spatially separated entities.
  It reflects the persistence of a shared relational structure within a single,
  non-factorizable configuration of the $\chi$ substrate.

  Entangled systems are described by admissible projected configurations that
  cannot be decomposed into independent subsystems without loss of relational
  consistency.
  Although effective spacetime descriptions assign distinct locations to the
  corresponding subsystems, these locations represent different local projections
  of one and the same underlying relational configuration.

  Nonlocal correlations therefore do not arise from superluminal influences,
  hidden signal exchange, or dynamical nonlocal interactions.
  They follow from the fact that measurement operations act on a globally defined
  relational configuration whose admissible projections must remain mutually
  consistent.
  Once a particular reprojection is realized in one region, the set of admissible
  reprojections elsewhere is correspondingly restricted, without any propagation
  of physical influence.

  In this sense, quantum nonlocality in Cosmochrony is ontological rather than
  dynamical.
  The underlying relational configuration is globally defined, while its
  relaxation and reprojection remain locally governed by the causal and
  projectability constraints of $\chi$.
  As a result, entanglement correlations are fully compatible with relativistic
  causality and do not require the introduction of nonlocal forces, preferred
  reference frames, or violations of relativistic locality.
