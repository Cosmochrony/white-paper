\subsection{Entanglement and Nonlocal Correlations}
  \label{subsec:entanglement-nonlocality}

  Within the Cosmochrony framework, quantum entanglement does not correspond to a
  physical interaction or linkage between spatially separated entities.
  Instead, it reflects the persistence of a shared, non-factorizable relational structure
  within admissible projected descriptions.

  Entangled systems arise when a unified relational configuration admits an effective
  projection onto spatially separated degrees of freedom.
  Although the projected descriptions assign distinct spacetime locations to the corresponding subsystems,
  these locations represent different local projections of a single globally consistent projected description.

  As a consequence, admissible projected descriptions cannot, in general, be decomposed into independent subsystems
  without loss of relational consistency.
  Measurement operations act on a globally defined relational structure whose admissible projections must remain
  mutually compatible.
  Once a particular local reprojection is realized, the set of admissible descriptions elsewhere is correspondingly
  constrained by global consistency, without any physical signal exchange or dynamical influence.

  In this sense, quantum nonlocality in Cosmochrony is structural rather than dynamical.
  The underlying relational configuration is globally defined, while all effective evolution
  and reprojection processes remain governed locally by bounded relaxation and projectability constraints within
  effective descriptions.
  Entanglement correlations are therefore fully compatible with relativistic causality and do not require the
  introduction of nonlocal forces, preferred reference frames, or superluminal influences.

  A formal relational analysis of non-factorization and its role in entanglement is provided
  in Appendix~\ref{subsec:non-factorization-entanglement}.

  \paragraph{Multiplicity of projected excitations.}
    Because the projection $\Pi$ is generically non-injective, a single relational
    configuration of $\chi$ may admit multiple admissible effective realizations.
    These realizations can appear as spatially separated excitations within the emergent
    geometric description, while remaining constrained by a common underlying relational structure.
    Their persistent correlations do not arise from interaction or information exchange,
    but from the shared non-factorizable pre-image in $\chi$.
    This structural multiplicity constitutes the ontological basis of quantum entanglement.
