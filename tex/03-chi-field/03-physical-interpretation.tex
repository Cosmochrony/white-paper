\subsection{Physical Interpretation}
  \label{subsec:physical-interpretation}

  In Cosmochrony, spacetime is not assumed as a pre-existing background structure.
  Instead, it appears as an effective macroscopic description arising from the
  continuous and irreversible ordering intrinsic to the relational substrate $\chi$.
  What are conventionally described as temporal and spatial features are understood
  as distinct, but related, manifestations of this single underlying process, once a
  projectable geometric regime becomes applicable.

  \begin{figure}[t]
    \centering
    \begin{tikzpicture}[
      font=\small,
      node distance=10mm,
      box/.style={draw, rounded corners, align=center, inner sep=6pt},
      arrow/.style={-Latex, thick},
      note/.style={align=left, font=\footnotesize},
      dashedbox/.style={draw, dashed, rounded corners, inner sep=6pt}
    ]

    \node[box] (chi) {$\chi$\\\footnotesize infra-physical substrate};

    \node[box, below=of chi] (chieff) {$\chi_{\mathrm{eff}}$\\\footnotesize factorisable regime};

    \node[dashedbox, below=of chieff, minimum width=6.6cm] (decomp) {
      \begin{tabular}{c}
        \footnotesize $\chi_{\mathrm{eff}} \simeq \chi_{\mathrm{eff}}^{(A)} \otimes \chi_{\mathrm{eff}}^{(B)}$\\
        \footnotesize (independent subsystems)
      \end{tabular}
    };

    \node[box, below left=10mm and 12mm of decomp] (obsA) {Local observables\\in subsystem $A$};
    \node[box, below right=10mm and 12mm of decomp] (obsB) {Local observables\\in subsystem $B$};

    \draw[arrow] (chi) -- node[right=2mm, note] {infra-physical\\projection $\pi$} (chieff);
    \draw[arrow] (chieff) -- (decomp);

    \draw[arrow] (decomp) -- node[left=2mm, note] {operational\\projection $\mathcal{O}_A$} (obsA);
    \draw[arrow] (decomp) -- node[right=2mm, note] {operational\\projection $\mathcal{O}_B$} (obsB);

    \node[note, below=7mm of decomp, align=center] (compat)
    {\footnotesize Compatible operational readings: joint assignment of local observables is well-defined.};

    \end{tikzpicture}
    \caption{Classical (factorisable) regime. After the infra-physical projection $\pi$, the effective reality $\chi_{\mathrm{eff}}$ admits an approximate decomposition into independent subsystems. Operational projections $\mathcal{O}_A$ and $\mathcal{O}_B$ yield compatible local observables, recovering standard classical and relativistic descriptions in stable projectable domains.}
    \label{fig:classical-factorisable}
  \end{figure}

  In regimes where projected $\chi$ configurations exhibit sufficiently stable and
  smooth correlation patterns, variations of the effective scalar descriptor
  $\chi_{\mathrm{eff}}$ give rise to a set of operational observables.
  Because the projection from $\chi$ to $\chi_{\mathrm{eff}}$ is generically
  non-injective, these observables summarize relational structure without exhausting
  the underlying degrees of freedom.
  In particular, an increase in $\chi_{\mathrm{eff}}$ along a given physical process
  is associated with:
  \begin{itemize}
    \item the accumulation of operational proper time along that process,
    \item the progressive decorrelation between effective configurations, summarized
    as an emergent spatial separation,
    \item the large-scale expansion behavior observed when the ordering of projected
    $\chi$ configurations is considered at the cosmological level.
  \end{itemize}

  Within this effective description, temporal duration and spatial separation are not
  independent primitives.
  They represent complementary aspects of the same underlying ordering and relaxation
  structure, captured at different levels of coarse-graining.
  Heuristically, effective distance may be viewed as the persistent imprint of
  relational differentiation that has already occurred, while effective time
  corresponds to the ongoing local ordering of projected $\chi$ configurations.
  These expressions are intended as interpretative guides rather than literal
  definitions, emphasizing their common dynamical origin.

  This unified interpretation is not introduced \emph{ad hoc}.
  It follows directly from identifying temporal ordering, relational separation, and
  cosmological expansion as distinct effective summaries of the same irreversible
  $\chi$ dynamics, once a macroscopic spacetime description becomes appropriate.
  The physical content of the theory therefore resides entirely in the dynamics of the
  fundamental $\chi$ substrate, while spacetime notions serve only as emergent,
  context-dependent descriptive tools.
