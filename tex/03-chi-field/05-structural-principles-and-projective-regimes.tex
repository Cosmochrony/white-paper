\subsection{Structural Principles and Projective Regimes}
  \label{subsec:structural-principles-and-projective-regimes}

  The preceding subsection established that all effective physical observables arise
  through a relational projection $\Pi$ from the underlying configuration space
  $\Omega$ of the $\chi$ substrate.
  Once such a projection is admitted, the emergence of spacetime, dynamics, and
  thermodynamic notions is no longer arbitrary.
  Instead, effective descriptions are constrained by a small set of structural
  principles that recur throughout the framework and govern the admissibility of
  projected regimes.

  These principles are not introduced as independent postulates.
  They are already implicitly operative in the constructions developed in subsequent
  sections, including the emergence of time ordering, causal structure, geometric
  descriptors, quantum correlations, and thermodynamic behavior.
  For clarity, we state them explicitly here.

  \paragraph{Principle I: Substratic Locality and Bounded Relaxation.}
    The fundamental relaxation dynamics of the $\chi$ substrate is strictly local and
    subject to universal bounds.
    In particular, the transport of relational relaxation admits a maximal admissible
    flux, constraining the rate at which structural information can be redistributed.
    This bound does not arise from spacetime geometry or signal propagation, but from
    the intrinsic stability and admissibility conditions of the $\chi$ substrate itself.
    As a result, no fundamental divergence or singular transfer of influence can occur
    at the substratic level.

  \paragraph{Principle II: Non-Injective Projective Realization.}
    The projection $\Pi : \Omega \rightarrow O$ from substratic configurations to
    effective observables is generically non-injective.
    Distinct configurations of $\chi$ may therefore be structurally identified at the
    level of observable descriptions.
    This identification is not an epistemic limitation, but a structural feature of the
    projection itself.
    It implies that effective descriptions need not admit a factorisable or locally
    complete representation, even when the underlying dynamics remains strictly local.
    The emergence of non-classical correlations and contextual behavior is a direct
    consequence of this non-injectivity.

  \paragraph{Principle III: Projective Compensation.}
    Whenever the projection $\Pi$ fails to resolve the full relational complexity of the
    $\chi$ substrate, effective descriptions must compensate for this loss of structural
    information.
    Such compensation occurs through the inflation of effective parameters used to
    summarize the projected regime, including temperature, curvature, horizon structure,
    or other geometric and thermodynamic descriptors.
    These quantities do not correspond to additional fundamental degrees of freedom.
    Rather, they function as Lagrange multipliers encoding unresolved relational structure
    within a reduced descriptive framework.

    Together, these principles delineate distinct projective regimes.
    In regimes where the projection is approximately injective and the relaxation flux
    is far from saturation, standard local and geometric descriptions apply.
    In contrast, in regimes approaching structural saturation or strong non-injectivity,
    effective parameters may grow large, signaling the progressive breakdown of local
    spacetime-based descriptions rather than the onset of exotic substratic dynamics.

    The subsequent sections develop the consequences of these principles.
    Section~\ref{subsec:monotonicity-and-arrow-of-time} shows how intrinsic relaxation
    ordering gives rise to a directed notion of time.
    Later sections demonstrate how geometric structure, quantum correlations, and
    thermodynamic behavior emerge as regime-dependent effective descriptions constrained
    by the same projective logic.
