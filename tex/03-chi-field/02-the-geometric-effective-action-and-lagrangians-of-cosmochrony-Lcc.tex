\subsection{The Geometric Effective Action and Lagrangians of Cosmochrony ($\mathcal{L}_{\text{CC}}$)}
\label{subsec:the-geometric-effective-action-and-lagrangians-of-cosmochrony-Lcc}

\paragraph{Interpretational caution.}
  The action principle presented below employs conventional field-theoretic notation, including a metric tensor
  $g_{\mu\nu}$ and a four-dimensional integration measure.
  \emph{This should not be interpreted as assuming pre-existing spacetime structure.}

  The formalism serves two purposes:
  \begin{enumerate}
    \item To provide a compact representation of $\chi$ dynamics in regimes where an effective spacetime description is valid.
    \item To establish the bridge between the fundamental relational network and the effective manifold description used in standard physics.
  \end{enumerate}

  The fundamental content of the theory is the field $\chi$ and its relaxation dynamics on a discrete graph (see
  Appendix~\ref{subsec:relational_foundation}).
  The metric $g_{\mu\nu}$ appearing in the action is a \emph{statistical emergent structure} representing the
  connectivity and correlation density of the $\chi$ field, not an independent ontological input.

\paragraph{Effective action formulation.}
  In regimes where $\chi$ admits a quasi-stable geometric interpretation, the dynamics may be encoded in an effective action:
  \begin{equation}
    S_{\text{CC}} = \int \mathcal{L}_{\text{CC}} \sqrt{-g} \, d^4x
  \end{equation}
  where the Lagrangian density decomposes as:
  \begin{equation}
    \mathcal{L}_{\text{CC}} = \mathcal{L}_{\text{Gravity/Time}} + \mathcal{L}_{\chi/\text{Soliton}} + \mathcal{L}_{\text{Forces/Matter}}
  \end{equation}

  The symbol $\sqrt{-g}$ represents the invariant volume element.
  In regimes where no spacetime interpretation yet exists (e.g., at the nodes of the fundamental graph), this should be
  understood as an abstract integration measure $d\mu$ on the configuration space of $\chi$.

\paragraph{Status of $g_{\mu\nu}$ in this formulation.}
  The metric $g_{\mu\nu}$ is an effective description of the coupling strengths $K_{ij}$ between $\chi$ nodes.
  It is defined by the requirement that the distance $ds^2$ in the continuum matches the operational distance derived
  from the network's connectivity:
  \begin{equation}
    g_{\mu\nu} dx^\mu dx^\nu \approx \sum_{(uv) \in \text{path}} \frac{1}{K_{uv}}
  \end{equation}
  Consequently, $g_{\mu\nu}$ is a \emph{phenomenological summary} of the underlying relational dynamics, capturing the
  local rate of $\chi$-relaxation and its spatial correlations.
