\subsection{Local Relaxation Speed}
  \label{subsec:local-relaxation-speed}

  A fundamental structural constraint of the Cosmochrony framework is that the effective
  local ordering rate associated with projected $\chi$ configurations is bounded.
  In effective geometric descriptions, this constraint takes the form
  \begin{equation}
    \left| \mathcal{D}_{\mathrm{loc}} \chi_{\mathrm{eff}} \right| \le c ,
  \end{equation}
  where $\mathcal{D}_{\mathrm{loc}} \chi_{\mathrm{eff}}$ denotes an effective local
  relaxation functional characterizing the maximal admissible ordering of projected
  $\chi$ configurations.
  The constant $c$ is the \emph{effective} causal bound observed in spacetime and
  coincides numerically with the speed of light.

  Importantly, this inequality does not define a fundamental propagation speed at the
  level of the $\chi$ substrate.
  Rather, it is the projected manifestation of a more primitive structural bound on
  admissible ordering within $\chi$, which constrains how rapidly causal relations and
  geometric structure may locally emerge in effective descriptions.
  The quantity $c$ therefore characterizes the causal structure of the projected regime,
  not the dynamics of the pre-geometric substrate itself.

  This bound does not represent the propagation speed of particles, fields, or signals,
  nor does it presuppose a pre-existing spacetime manifold.
  Instead, it limits the maximal rate at which effective causal connectivity and local
  geometric relations can be established within projected descriptions compatible with
  the intrinsic ordering structure of $\chi$.

  Apparent superluminal recession velocities at cosmological scales arise naturally from
  cumulative and global effects of projected $\chi$ ordering, and do not violate this
  local causal constraint.
  Local causal relations remain bounded by $c$ in all effective descriptions, while the
  underlying $\chi$ substrate itself is not subject to any spacetime notion of velocity
  or signal propagation.
