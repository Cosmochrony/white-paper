\subsection{Relation to Conventional Fields}
  \label{subsec:relation-to-conventional-fields}

  Although effective descriptions derived from projected $\chi$ configurations may
  exhibit formal similarities with scalar fields employed in cosmology (such as
  inflaton-like fields), the ontological role of $\chi$ is fundamentally different.
  The $\chi$ substrate is not a physical field propagating on spacetime, but a
  pre-geometric relational structure from which spacetime notions themselves emerge.

  Accordingly, $\chi$ does not carry energy in the conventional field-theoretic sense,
  nor is it subject to quantization at the fundamental level.
  Quantization arises only at the effective level, as a consequence of the non-injective
  projection from $\chi$ to emergent observables.
  In particular, only certain stable, localized, and spectrally isolated
  $\chi$ configurations admit a particle-like interpretation and can be consistently
  described using standard quantum field-theoretic tools within an emergent spacetime
  regime.

  Within this framework, matter, radiation, and interactions do not correspond to
  independent fundamental fields coupled to $\chi$.
  They arise instead as effective manifestations of topological obstructions,
  spectral constraints, or long-lived relational patterns of projected $\chi$
  configurations.
  Conventional fields of the Standard Model are thus recovered as effective
  descriptions of these emergent degrees of freedom in regimes where a spacetime
  interpretation is valid and a coarse-grained field-theoretic language provides an
  accurate approximation.

  From this perspective, Cosmochrony does not introduce an additional field beyond
  those of the Standard Model.
  Rather, it provides a deeper ontological account of why effective field descriptions
  are possible at all, and why their quantized excitations, interaction structures,
  and mass scales arise with the specific properties observed in nature.
