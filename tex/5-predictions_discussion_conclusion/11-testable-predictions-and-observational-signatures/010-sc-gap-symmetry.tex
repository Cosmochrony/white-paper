\subsection{Superconducting Gap Symmetry from Fiber Geometry}
  \label{subsec:sc-gap-symmetry}

  The collective coherent regime of the $U(1)$ fiber discussed in
  Section~\ref{subsec:collective-U1-coherence} leads to concrete and
  falsifiable predictions for the symmetry of the superconducting gap.
  In the present framework, gap symmetry is not determined by a specific
  momentum-space interaction kernel, but by a real-space minimization
  problem in the fiber raccordement cost under lattice constraints.

  \subsubsection*{Geometric Selection Principle}

    Let $\mathcal{C}_{\text{tot}}[\theta]$ denote the effective cost functional
    associated with fiber phase mismatch in the presence of lattice
    symmetry group $\mathcal{G}$ and dominant frustration wavevector $\mathbf{Q}$.
    Admissible composite configurations must satisfy simultaneously:

    \begin{enumerate}
      \item minimization of overlap with the dominant frustrated sector,
      \item global continuity of the fiber phase without introducing additional defects.
    \end{enumerate}

    Among the irreducible representations of $\mathcal{G}$ available to a
    $w=2$ composite class, the physically realized superconducting gap
    corresponds to the representation that minimizes $\mathcal{C}_{\text{tot}}$.

    This selection principle is independent of microscopic pairing details.
    It depends only on lattice symmetry and the dominant frustration structure.

  \subsubsection*{Prediction for Cuprate Superconductors}

    In cuprate materials, the relevant lattice symmetry is $D_{4h}$ and
    the dominant frustration vector is approximately $(\pi,\pi)$,
    as measured through the magnetic structure factor $S(\pi,\pi)$.

    Under these constraints, the minimization of the fiber raccordement
    cost selects the $B_{1g}$ irreducible representation.
    The resulting gap symmetry is therefore
    \begin{equation}
      \Delta(\mathbf{k}) \propto \cos k_x - \cos k_y,
    \end{equation}
    corresponding to $d_{x^2-y^2}$ symmetry.

    This prediction does not rely on a specific spin-fluctuation mechanism.
    It follows from geometric frustration in the projected fiber sector.
    The nodal lines arise from the required sign alternation that reduces
    overlap with the $(\pi,\pi)$ frustration sector while preserving
    global fiber continuity.

  \subsubsection*{Prediction for Nickelate Superconductors}

    In infinite-layer nickelates, the antiferromagnetic frustration sector
    is reduced and partially isotropized relative to cuprates.
    Let $r_F$ denote a dimensionless proxy for the relative frustration
    amplitude, operationally linked to the normalized structure factor.

    For sufficiently reduced staggered frustration ($r_F < r_F^{\text{crit}}$),
    the minimization of $\mathcal{C}_{\text{tot}}$ favors an extended
    $A_{1g}$ representation with sign-changing structure across Fermi
    surface sheets,
    \begin{equation}
      \Delta(\mathbf{k}) \sim s^{\pm}.
    \end{equation}

    The framework therefore predicts a symmetry transition between
    $d_{x^2-y^2}$ and extended $s^{\pm}$ depending on the frustration ratio $r_F$.
    This constitutes a direct and falsifiable structural prediction.

  \subsubsection*{Disorder Sensitivity}

    Because sign-changing order parameters are sensitive to non-magnetic
    disorder, the predicted $s^{\pm}$ regime in nickelates should display
    intermediate sensitivity to impurity scattering, weaker than pure
    $d$-wave cuprates but stronger than conventional isotropic $s$-wave
    superconductors.

    Systematic impurity studies therefore provide a discriminating test
    of the fiber-geometry selection mechanism.

  \subsubsection*{Scaling of the Critical Temperature}

    To leading order, the phase stiffness scales as
    \begin{equation}
      \rho_s(0) \propto \delta J f(r_F),
    \end{equation}
    where $\delta$ is the carrier concentration, $J$ the dominant
    frustration scale, and $f(r_F)$ a dimensionless function encoding
    the geometric reduction of frustration.

    In quasi-two-dimensional systems,
    \begin{equation}
      k_B T_c \simeq \frac{\pi}{2} \rho_s(T_c^-),
    \end{equation}
    yielding
    \begin{equation}
      T_c \sim \frac{\pi}{2k_B} \delta J f(r_F).
    \end{equation}

    All quantities entering this relation are independently measurable
    through spectroscopy, transport, and neutron scattering.
    No free fitting parameter specific to superconductivity is introduced.

  \subsubsection*{Falsifiability}

    The present mechanism would be falsified if:

    \begin{itemize}
      \item cuprates exhibited a robust fully isotropic $s$-wave gap
      at optimal doping despite strong $(\pi,\pi)$ frustration,
      \item nickelates with reduced staggered frustration showed
      stable $d_{x^2-y^2}$ symmetry across all doping levels,
      \item the critical temperature displayed no correlation with
      independently measured exchange or frustration scales.
    \end{itemize}

    The symmetry channel is therefore not an adjustable ingredient.
    It is fixed by lattice group and frustration geometry,
    providing a sharp structural test of the framework.
