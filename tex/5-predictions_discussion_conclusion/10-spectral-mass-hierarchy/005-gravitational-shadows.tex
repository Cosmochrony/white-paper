% ----------------------------------------------------------------------------
% Section 10.3 --- Gravitational Shadows and the Spectral Wake
% From former §11.3, condensed
% ----------------------------------------------------------------------------
\subsection{Gravitational Shadows and the Spectral Wake}
\label{sec:gravitational-shadows}

The torsional constraint~$\Omega_w$ induces an extended spectral
deformation of the surrounding substrate---a persistent modification of
the local spectral density of relaxation modes, referred to as a
\emph{spectral wake} or gravitational shadow.
This shadow does not correspond to additional matter or propagating
degrees of freedom, but to a long-lived redistribution of relaxation
capacity induced by the topological obstruction.
It provides an ontological basis for phenomena usually attributed to dark
matter without introducing new particles.

Two characteristic features emerge.
\emph{Elastic remanence}: the deformation persists under displacement of
the baryonic configuration, accounting for the observed offsets in systems
such as the Bullet Cluster.
\emph{Non-local susceptibility}: when the local gradient falls below
$a_0 \sim c H_0$, the substrate response transitions from linear to
non-linear, recovering MOND-like phenomenology as an emergent phase of
spectral response.
Dark matter effects are thus reinterpreted as manifestations of
persistent spectral memory in the relational substrate.
