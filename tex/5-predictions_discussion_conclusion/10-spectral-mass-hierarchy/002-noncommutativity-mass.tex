\subsection{Non-Commutativity as a Source of Mass}
\label{sec:noncommutativity-mass}

Torsion acts as a dynamical constraint that competes with relaxation
transport through the projection fiber.

\subsubsection*{Inhibition of Relaxation}
\label{sec:inhibition-relaxation}

Let $\Omega_w$ be the effective torsion operator associated with winding
number~$w$.
For the fundamental lepton configuration ($w=1$), relaxation and torsion
are spectrally compatible:
\begin{equation}
  [\Delta^{(0)}_G,\Omega_{1}] = 0 .
\end{equation}
For higher-winding configurations ($w\ge 2$), torsional constraints
become spectrally frustrated:
\begin{equation}
  [\Delta^{(0)}_G,\Omega_{w}] \neq 0
    \qquad (w\ge 2).
\end{equation}
This non-commutativity inhibits uniform relaxation across the fiber and
induces irreducible spectral compression, manifesting as amplified
inertial mass.
The torsional action is a purely spectral invariant:
\begin{equation}
  \mathcal{A}(w) \;\equiv\;
  \frac12 \ln\!\left(
    \frac{\det(\Delta^{(0)}_G+\Omega_{w})}
         {\det(\Delta^{(0)}_G)}
  \right),
  \label{eq:Aw-fredholm}
\end{equation}
where the determinant is understood in the zeta-regularized sense.

\subsubsection*{The Pisano Ratio as a Stability Fixed Point}
\label{sec:pisano-ratio}

For $w=2$, the projection fiber ceases to be spectrally isotropic and
splits into competing sectors
$\Pi = \Pi_{\parallel} \oplus \Pi_{\perp}$.
Dynamical stability selects the most irrational admissible ratio between
their spectral frequencies, in analogy with KAM-type criteria:
\begin{equation}
  \frac{\lambda_{\parallel}}{\lambda_{\perp}}
    \;=\; \varphi
  \qquad\Longrightarrow\qquad
  \beta \;\equiv\; \frac{1}{\varphi},
  \label{eq:beta-golden}
\end{equation}
where $\varphi=(1+\sqrt5)/2$ is the golden ratio.

\subsubsection*{Leptonic Spectrum Synthesis}
\label{sec:leptonic-synthesis}

For the muon, non-commutative torsion yields the parameter-free
prediction
\begin{equation}
  \boxed{
    \frac{m_\mu}{m_e}
    = \sqrt{\frac{\lambda_{2}}{\lambda_{1}}}
    \cdot \frac{3}{2\alpha}
    \cdot \frac{1}{\varphi}
  }
  \qquad\text{with}\qquad
  \frac{\lambda_2}{\lambda_1} = \frac{8}{3}.
  \label{eq:muon-ratio-final}
\end{equation}
The robustness of the ratio $\lambda_2/\lambda_1 = 8/3$ is demonstrated
independently in Appendix~\ref{sec:spectral_ratio_derivation}.
