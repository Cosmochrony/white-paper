\subsection{Gravitational Wave Propagation}
  \label{subsec:gravitational-wave-propagation}

  In regions of high excitation density near compact objects, local
  suppression of~$\chi$ relaxation modifies the persistence and coherence
  of gravitational-wave projective descriptions.
  This effect induces frequency-dependent phase shifts or amplitude
  suppression, rather than purely dissipative losses.

  We predict a relative amplitude deviation from general relativistic
  templates scaling as
  \[
    \frac{\Delta A}{A}
    \sim \left(\frac{r_s}{r}\right)^2 ,
  \]
  where $r_s = 2GM/c^2$ is the Schwarzschild radius and $r$ the effective
  propagation radius of the dominant ringdown mode.

  For propagation at $r \approx 10\,r_s$, this yields
  \[
    \frac{\Delta A}{A} \sim 10^{-2} ,
  \]
  corresponding to a percent-level suppression of coherent gravitational-wave
  amplitudes relative to general relativity.

  This deviation should manifest most clearly during the late-time
  ringdown phase of binary black hole mergers as a systematic,
  frequency-dependent mismatch with general relativistic ringdown templates.
  The effect preserves luminal propagation speed but alters coherence
  structure.

  \textbf{Falsifiability condition.}
  If future high signal-to-noise detections ($\mathrm{SNR} \gtrsim 100$)
  by space-based observatories such as LISA constrain
  $\Delta A/A < 10^{-3}$ at $r \sim 10\,r_s$,
  the present scaling law would be excluded at the predicted amplitude level.
