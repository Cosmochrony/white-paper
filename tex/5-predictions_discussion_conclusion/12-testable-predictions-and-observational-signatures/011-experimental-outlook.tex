% ----------------------------------------------------------------------------
% Section 11.10 --- Experimental Outlook and Discriminating Signatures
% From former §14.11, condensed
% ----------------------------------------------------------------------------
\subsection{Experimental Outlook and Discriminating Signatures}
\label{subsec:experimental-outlook-and-discriminating-signatures}

Particle creation is reinterpreted as a universal dissipation channel
activated whenever directional relaxation approaches its maximal
transport capacity.
This unifies several high-energy phenomena:

\paragraph{Astrophysical Jets.}
Relativistic jets are dynamically selected channels through which excess
substrate tension is discharged by continuous structure creation.
A predicted \emph{saturation clamping} near the horizon---a non-linear
relation between near-horizon stress and plasma density---is testable
with Event Horizon Telescope observations.

\paragraph{Primordial Cosmology.}
Matter production emerges as a byproduct of relaxation-driven expansion.
Cosmochrony predicts that primordial fluctuations may carry imprints of
discrete nucleation events, suggesting specific non-Gaussianity and phase
correlations at large angular scales.

\paragraph{Ultra-High-Energy Cosmic Rays.}
UHECRs exceeding the GZK threshold may be locally produced during
transient saturation spikes near compact objects.
A predicted strong anisotropy correlated with nearby compact objects and
multimessenger correlations would provide a decisive signature.

\paragraph{Multi-Scale Falsification Program.}
The framework suggests a hierarchy of tests: precision atomic physics
(Section~\ref{subsec:lamb-shift}), Schwinger threshold in ultra-intense
lasers, saturation clamping in relativistic jets, and large-scale CMB
anomalies as relics of discrete primordial nucleation.
