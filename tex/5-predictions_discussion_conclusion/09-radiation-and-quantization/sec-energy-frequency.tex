% ----------------------------------------------------------------------------
% Section 9.3 --- Geometric Origin of E = hν
% From former §13.3, condensed
% ----------------------------------------------------------------------------
\subsection{Geometric Origin of
\texorpdfstring{$E = h\nu$}{E = hν}}
\label{subsec:energy-frequency-radiation}

Building on
Section~\ref{subsec:energy-frequency-solitons}, the Planck relation
\begin{equation}
  E = h \nu
\end{equation}
acquires a geometric interpretation in the context of radiative
processes.
The energy~$E$ measures the relaxation potential redistributed during a
reprojection event; the frequency~$\nu$ characterizes the minimal
temporal resolution required for a coherent effective description of this
redistribution.
The constant~$h$ is the effective projection of the fundamental
substrate invariant
$\hbar_\chi \equiv c^3 / (K_{0,\text{bare}}\,
\chi_{c,\text{bare}})$
(Section~\ref{subsec:renormalization-hbar}), acting as a universal
conversion factor between structural relaxation capacity and temporal
projective resolution.

In the photoelectric effect, the threshold frequency~$\nu_0$ corresponds
to the minimal projective resolution required to destabilize a bound
electronic soliton.
Quantization of radiative energy arises not from discretized propagation
but from the discrete nature of local reprojection events, which impose
a minimal unit of effective relaxation transfer determined by the
spectral graininess~$\hbar_\chi$.
