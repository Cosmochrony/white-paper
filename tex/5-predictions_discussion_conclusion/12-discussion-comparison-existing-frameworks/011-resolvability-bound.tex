\subsection{Bounds on Projective Resolvability Across Scales}
  \label{subsec:projective-resolvability-bound}

  Saturation phenomena across scales---bounded gravitational response in low-density environments and the operational
  accessibility of quantum correlations---reflect a common limitation on projective resolvability.
  In Cosmochrony, the projected description $U$ does not expose the full relational content of the substrate~$\chi$.
  It exposes only what can be rendered operationally accessible through the non-injective mapping~$\Pi$ under admissible
  update transport.
  The relevant limitation is therefore not a postulated geometric cutoff, but a throughput bound on the rate at which
  relational structure can be updated while maintaining operational closure of the projected state.

  Attosecond chronoscopy experiments illustrate this principle.
  Measured delays can be interpreted as the minimal temporal resolution required for a non-factorizable projected
  description to become operationally resolvable, rather than as a dynamical buildup of correlations~\cite{PhysRevLett.133.163201}.

  We summarize this limitation by an effective inequality on projected update rates,
  \begin{equation}
    \left| \frac{\partial \mathcal{O}_{\mathrm{eff}}}{\partial \tau} \right|
    \;\leq\;
    b_{\chi}\,\mathcal{S}_{\Pi},
    \label{eq:resolvability_bound}
  \end{equation}
  where $b_{\chi}$ is the invariant bound on admissible relaxation transport in the substrate and $\mathcal{S}_{\Pi}$
  parametrizes the effective structural complexity of the projection for the class of observables under consideration.
  The factor $\mathcal{S}_{\Pi}$ is not a new dynamical field.
  It encodes how many independent relational channels can be synchronized within one operational update step.

  Distinct phenomenological manifestations correspond to distinct choices of $\mathcal{O}_{\mathrm{eff}}$ and the associated
  resolution scale.
  In macroscopic weak-field regimes, the same bound induces an operational saturation threshold that can be expressed as an
  effective acceleration proxy,
  \begin{equation}
    a_{\star} \sim \kappa \frac{b_{\chi}}{\ell},
    \label{eq:a_star_b_relation}
  \end{equation}
  where $\ell$ is the local descriptive resolution of the projected state and $\kappa=\mathcal{O}(1)$ encodes the
  normalization convention of~$\Pi$.
  In measurement-limited quantum protocols, the bound similarly appears as a bandwidth constraint,
  \begin{equation}
    B_{\Pi}(\mathcal{M})
    = \eta_{\mathcal{M}}\,b_{\chi}\,\mathcal{S}_{\Pi},
    \label{eq:bandwidth_b_relation}
  \end{equation}
  for a measurement procedure~$\mathcal{M}$, with $\eta_{\mathcal{M}}$ absorbing protocol-dependent conventions.
  These expressions represent different dimensional projections of the same underlying transport bound.
  A modification of the effective resolvability of the $\chi \rightarrow \Pi$ mapping would therefore induce correlated
  shifts in both galactic saturation scales and quantum chronoscopy thresholds.

  A further generic consequence is the divergence of projective stress under sufficiently deep refinement.
  Let $D$ denote an operational update demand, defined as a characteristic rate at which projected relations must be
  recomputed to preserve admissibility.
  Let $C$ denote the admissible projective throughput under the bound~$b_{\chi}$ for the same class of updates.
  The dimensionless stress ratio
  \begin{equation}
    \Xi \equiv \frac{D}{C}
  \end{equation}
  provides a regime indicator.
  Whenever a process increases $D$ faster than $C$ can be redistributed by the available synchronized channels, $\Xi$ grows
  and a saturated reconfiguration becomes unavoidable.
  This statement is independent of any particular emergent geometry and follows solely from bounded transport and finite
  projective resolvability.
