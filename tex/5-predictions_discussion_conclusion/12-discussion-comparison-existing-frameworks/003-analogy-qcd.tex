\subsection{Analogy with Collective Phenomena in QCD}
  \label{subsec:analogy-with-collective-phenomena-in-qcd}

  A useful structural analogy may be drawn with low-energy
  QCD~\cite{Shifman2007QCDVacuum}, where fundamental degrees of freedom
  (quarks, gluons) do not appear as isolated entities in the infrared.
  The observable spectrum consists instead of collective bound states,
  whose properties are not transparently reducible to perturbative
  constituents.

  Similarly, Cosmochrony formulates fundamental dynamics solely in terms
  of~$\chi$ and its relaxation ordering.
  Observable quantities arise only after projection into regimes
  admitting stable effective descriptions.
  What appear as elementary constituents at a given effective level may
  represent stable, regime-dependent invariants of the underlying
  dynamics, in direct analogy with QCD confinement.

  The analogy extends beyond particle confinement.
  In QCD, the vacuum itself admits multiple collective phases,
  including chiral symmetry breaking and color superconductivity,
  which emerge from non-perturbative organization of the same
  underlying degrees of freedom.
  In Cosmochrony, different physical phenomena likewise correspond
  to distinct collective regimes of a single substrate.

  Gravitational curvature arises as a collective slowdown of relaxation
  ordering.
  Schwinger pair production corresponds to saturation of directed
  relaxation flux.
  Superconductivity, as discussed in
  Section~\ref{subsec:collective-U1-coherence},
  represents a coherent collective phase of the $U(1)$ fiber sector,
  in which composite winding classes stabilize and lock globally.

  In all these cases, the effective phenomena are not introduced
  as additional dynamical postulates.
  They correspond to regime-dependent reorganizations of the same
  projective structure.
  The underlying ontology remains unchanged, while the effective
  description undergoes phase-like transitions analogous to those
  observed in strongly coupled gauge theories.

  This analogy is structural rather than dynamical.
  Cosmochrony does not import the specific field content of QCD.
  The comparison serves to clarify how radically different effective
  laws may arise from a single underlying relational dynamics,
  without multiplying fundamental entities.
