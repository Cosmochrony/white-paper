\subsection{Relation to the Higgs Mechanism: Emergence from
\texorpdfstring{$\chi$}{χ} Dynamics}
  \label{subsec:relation-to-the-higgs-mechanism}

  The Higgs field and its VEV are reinterpreted as effective low-energy
  descriptors of a specific projective regime of~$\chi$.
  In this framework, mass generation does not originate from an
  independent scalar sector, but from the stabilization of
  relaxation-resistant configurations of the underlying substrate.

  \subsubsection*{Structural Transition}
    \label{subsec:emergence-higgs-vev}
    \label{subsec:chi_c-electroweak-scale}

    Below a critical scale~$\chi_c$ (homogeneous regime), only massless
    globally coherent configurations are admissible.
    Above~$\chi_c$ (structured regime), localized relaxation-resistant
    configurations stabilize as massive excitations.
    The electroweak scale is related through
    \begin{equation}
      \langle \phi_H \rangle
      \;\propto\;
      \frac{\hbar_{\mathrm{eff}}\, c}{\chi_c},
    \end{equation}
    naturally recovering the observed scale for
    $\chi_c \sim 10^{-18}\,\mathrm{m}$.

  \subsubsection*{Mass Generation as Solitonic Stabilization}
    \label{subsec:mass-generation-solitons}

    Fermion masses scale as
    $m_f \propto y_f\,\hbar_{\mathrm{eff}}/\chi_c$;
    gauge boson masses as
    $m_W \propto g\,\hbar_{\mathrm{eff}}/\chi_c$.
    These relations reproduce standard Higgs-generated mass terms at the effective level.

    Within the spectral interpretation developed in
    Section~\ref{subsec:flavor-structure-cp},
    the Yukawa parameters encode the effective projection of underlying
    relaxation eigenvalues.
    The observed flavor hierarchies and associated CP structure therefore
    reflect the same spectral organization,
    rather than introducing independent mass-generating dynamics.

  \subsubsection*{Distinction from Superconducting Mass Generation}

    A potential source of confusion concerns the appearance of an
    effective mass term for gauge fields in superconductivity,
    discussed in Section~\ref{subsec:collective-U1-coherence}.
    In that regime, global phase locking of $w=2$ composites produces
    a London term proportional to $A^2$, leading to Meissner screening
    and an effective photon mass inside the medium.

    This phenomenon is not interpreted here as fundamental symmetry breaking.
    It corresponds to emergent projective coherence in a condensed phase
    of the $U(1)$ fiber sector.
    The underlying gauge symmetry of the $\chi$ substrate remains intact.
    The apparent mass arises from collective reorganization of degrees
    of freedom within a bounded projective regime.

    By contrast, electroweak mass generation operates in the vacuum
    structure of the effective theory and is associated with a distinct
    projective transition at the scale~$\chi_c$.
    Although both mechanisms yield quadratic gauge-field terms at the
    effective level, their ontological status differs.
    The superconducting London mass is medium-dependent and
    disappears above the coherence temperature,
    whereas electroweak gauge boson masses persist in the vacuum.

    The formal similarity between the two mechanisms reflects a deeper
    structural principle:
    mass terms emerge whenever a collective projective configuration
    restricts admissible gauge variations.
    However, no additional fundamental Higgs-like degree of freedom
    is introduced in the superconducting case.
    The effective mass acquisition of the gauge field in
    superconductivity is therefore a manifestation of emergent
    projective coherence, not a reinvention of the Higgs mechanism.

  \subsubsection*{Phenomenological Status}
    \label{subsec:phenomenological-implications}
    \label{subsec:higgs-summary}

    No deviation from established collider results is implied at accessible
    energies.
    Departures may arise only in extreme regimes where the mapping between
    solitonic stabilization scales and effective parameters becomes nonlinear.
    Open challenges include deriving the detailed correspondence between
    $\chi$ soliton spectra and the full Standard Model mass hierarchy.
