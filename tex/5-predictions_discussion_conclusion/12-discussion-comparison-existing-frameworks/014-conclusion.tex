% ----------------------------------------------------------------------------
% Section 12.13 --- Conclusion and Outlook
% From former section 16, condensed
% ----------------------------------------------------------------------------
\clearpage

\section{Conclusion and Outlook}
\label{sec:conclusion-and-outlook}

Cosmochrony reduces the fundamental assumptions of physics to a single
dynamical origin: the irreversible relaxation of a pre-geometric
relational substrate~$\chi$.
From this substrate, time, spacetime geometry, and a wide spectrum of
physical phenomena emerge as effective descriptions.

A central result is the \emph{ab initio} derivation of the effective
dynamical laws: the Born--Infeld-like Lagrangian is the unique functional
compatible with causal saturation of relaxation fluxes at~$c_\chi$.
The speed limit~$c_\chi$ and Planck's constant~$h$ appear as
complementary bounds on projectability, respectively limiting the maximal
propagation rate and the minimal resolvable granularity.
General Relativity is recovered as a thermodynamic limit of the
underlying relational dynamics.

The framework provides a unified geometric origin for the Standard Model:
gauge interactions emerge as projection dynamics; matter and mass arise
from topological obstructions and spectral frustration; the dark sector
is reinterpreted as non-projected spectral density (dark matter) and
global relaxation flux (dark energy).
Quantum entanglement arises when a single relational configuration
admits multiple admissible effective realizations under non-injective
projection, persisting only within a finite critical regime.

At cosmological scales, expansion and the arrow of time follow from the
diminishing tempo of relaxation.
At galactic scales, saturation of the relaxation constraint produces
flat rotation curves without dark matter halos.
In strong-gravity regimes, black hole evaporation is reinterpreted as
discrete reprojection ensuring information preservation.

The space of admissible projected configurations evolves as relaxation
proceeds: the particle content of the Universe is a historically
conditioned outcome, not a timeless input.

\subsection*{Conceptual Shift and Outlook}

Cosmochrony represents a shift from a ``matter-on-spacetime'' paradigm
to a ``relaxation-of-substrate'' ontology.
Immediate falsifiable research directions include:
cosmological signatures (low-$\ell$ CMB anomalies from finite relaxation
capacity), galactic phenomenology (dark matter effects correlated with
local relaxation gradients), and fundamental-scale effects
(environment-dependent variations in decay rates or couplings in extreme
regimes).
Future work will focus on the systematic derivation of the Standard
Model spectrum as a hierarchy of topological frustration modes within the
relaxation dynamics.
