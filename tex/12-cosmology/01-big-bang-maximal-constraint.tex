\subsection{The Big Bang as a Maximal Constraint Regime of the \texorpdfstring{$\chi$}{χ} Substrate}
  \label{subsec:big-bang-maximal-constraint}

  In the Cosmochrony framework, the Big Bang is interpreted as a limiting cosmological
  manifestation of the maximal constraint regime of the $\chi$ substrate, rather than
  as a spacetime singularity or a physical event occurring at a definite moment.
  It corresponds to the boundary of applicability of effective spacetime descriptions,
  as defined by the admissibility constraints introduced in Section~\ref{subsec:initial-conditions-and-global-structure}.

  In this regime, the density of structural and topological constraints within $\chi$
  exceeds the threshold required for stable geometric projection.
  As a result, effective notions of spatial distance, temporal duration, curvature, and
  causal ordering cease to be well-defined.
  The apparent singular behavior encountered in standard cosmological models reflects
  the extrapolation of geometric descriptions beyond their domain of validity, rather
  than a fundamental divergence of physical quantities.

  Cosmological evolution is therefore described as the progressive relaxation of this
  maximal constraint regime.
  As the admissible space of projected configurations expands, increasingly stable
  effective spacetime descriptions emerge, allowing geometric and causal structures to
  become well-defined.
  The Big Bang thus marks not the origin of spacetime, but the transition beyond which
  spacetime becomes an appropriate effective framework.

  As established in Section~\ref{subsec:monotonicity-and-arrow-of-time}, the arrow of
  time arises from the intrinsic monotonic ordering of $\chi$ configurations.
  In the cosmological context, this ordering manifests as a directed evolution away
  from the maximal constraint regime, without invoking special initial conditions or
  entropy postulates at the fundamental level.
