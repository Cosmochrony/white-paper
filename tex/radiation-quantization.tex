\subsection{Radiation as $\chi$–Matter Interaction}\label{subsec:radiation-as-$chi$matter-interaction}

  In cosmochrony, radiation does not correspond to the emission of pre-existing particles.
  Instead, it arises from the interaction between localized excitations (matter) and the surrounding $\chi$ field.

  When an excited configuration interacts with $\chi$, part of the excitation may detach as a propagating crest.
  This process is stochastic, reflecting local fluctuations of $\chi$, and gives rise to radiation.

\subsection{Emergence of Photons}\label{subsec:emergence-of-photons}

  Photons are not fundamental entities in this framework.
  They correspond to transient, propagating disturbances of $\chi$ generated during interactions with matter.

  Prior to detection or emission, no localized photon exists.
  Quantization appears only at the moment of interaction, when continuous $\chi$
  dynamics produces discrete energy transfer.

  Although propagating electromagnetic waves correspond to continuous $\chi$
  -disturbances, localized photon-like excitations only emerge during interactions with matter. In a double-slit
  experiment, the interference pattern arises from the continuous wave nature of $\chi$
  , while individual detection events correspond to interaction-induced localizations. This duality explains why
  photons exhibit both wave-like and particle-like behavior depending on the measurement context, without
  invoking wavefunction collapse as a fundamental process.

\subsection{Geometric Origin of $E = h\nu$}
  \label{subsec:energy-frequency-radiation}

  This section develops, in the context of radiation processes, the energy--frequency relation
  introduced earlier in Section~\ref{subsec:energy-frequency-solitons} for localized excitations
  of the $\chi$ field.

  The energy of a radiative event is proportional to the local curvature and frequency of the $\chi$ disturbance.
  Higher-frequency disturbances correspond to tighter curvature of the field and thus greater energy concentration.

  The Planck relation
  \begin{equation}
    E = h \nu
  \end{equation}
  emerges as a geometric proportionality between excitation frequency and curvature energy within $\chi$.

  In this interpretation, Planck's constant $h$ encodes a structural property of the $\chi$
  field rather than a fundamental quantum postulate.

  The proportionality constant $h$ in $E = h\nu$
  reflects the geometric conversion factor between the oscillation frequency of a $\chi$
  -disturbance and its curvature energy. In the photoelectric effect, the threshold frequency $\nu_0$
  corresponds to the minimal curvature required to eject an electron soliton from a material binding potential,
  while the linear dependence on $\nu$ arises from the energy stored in the oscillatory structure of $\chi$
  . This geometric interpretation preserves the empirical success of quantum mechanics while deriving
  quantization from interaction dynamics rather than postulating it.

\subsection{Vacuum Fluctuations and the Casimir Effect}\label{subsec:vacuum-fluctuations-and-the-casimir-effect}

  Vacuum fluctuations correspond to stochastic variations of $\chi$ in the absence of localized excitations.
  Boundary conditions imposed by matter constrain these fluctuations, altering the local spectrum of allowed modes.

  The Casimir effect arises naturally as a pressure difference resulting from modified $\chi$
  dynamics between closely spaced boundaries.

\subsection{Weakly Interacting Radiation}\label{subsec:weakly-interacting-radiation}

  Disturbances with minimal curvature, such as low-frequency electromagnetic waves or neutrino-like excitations,
  interact weakly with matter.
  Their near-planar structure reduces the probability of producing localized energy transfer.

  This explains the transparency of the vacuum to radiation and the weak interaction cross sections of certain
  particles.

\subsection{Summary}
  \label{subsec:summary3}

  Radiation and quantization arise from the interaction between matter excitations and the $\chi$ field.
  Photons emerge during interactions rather than existing as independent entities, and quantization reflects
  geometric constraints of $\chi$ dynamics.
