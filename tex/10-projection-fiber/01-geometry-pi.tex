\subsection{The Geometry of the $\Pi$ Subspace}
  \label{sec:geometry-pi}

  The relational substrate $\chi$ is not accessed in its full structural
  complexity, but through admissible local projections onto a reduced projection
  fiber $\Pi \cong S^3$.
  This projection acts as a spectral filter, retaining only those modes of the
  relaxation flow that remain jointly projectable and stable under local
  re-identification.

  The identification $\Pi \cong S^3$ is not imposed \emph{a priori}, but follows
  from the minimal compact geometry required to support non-injective yet stable
  projections.
  Its associated Hopf fibration naturally induces an effective
  $SU(2)\times U(1)$ symmetry structure, which emerges as an invariance of the
  projection process rather than as a fundamental symmetry of the substrate.

  The metric on $\Pi$ is not fixed but dynamically induced by the local density of
  relational connections encoded in the underlying graph $G$.
  The mapping from the global graph Laplacian $\Delta_G$ to the effective projected
  Laplacian $\Delta_\Pi$ is given by
  \begin{equation}
    \Delta_\Pi = P^\dagger \Delta_G P ,
  \end{equation}
  where $P$ denotes the projection operator selecting admissible modes.
  The appearance of a smooth three-sphere geometry thus corresponds to a
  large-$N$ limit of the discrete relational structure, in which coarse-grained
  spectral properties dominate.
