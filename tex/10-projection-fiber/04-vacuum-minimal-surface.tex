\subsection{The Vacuum State as a Minimal Surface}
  \label{sec:vacuum-minimal-surface}

  In the absence of localized excitations, the projection fiber $\Pi$ relaxes toward
  a configuration of minimal spectral tension, understood as minimal global
  frustration of admissible relaxation modes.
  This vacuum state is not an empty background, nor the ground state of a quantized
  field, but the smoothest admissible projected configuration compatible with global
  relaxation constraints.

  Geometrically, the vacuum corresponds to a minimal surface in a purely spectral sense, not a spatial or spacetime
  embedding:
  a configuration for which curvature, torsion, and winding modes are simultaneously
  minimized within $\Pi$.
  In this state, the relaxation flow encounters no topological obstruction and
  propagates uniformly across the fiber.
  Any deviation from this minimal configuration—whether through localized curvature,
  torsional distortion, or non-trivial winding—manifests as the presence of effective
  fields or particles.

  From this perspective, the usual notion of field quantization is replaced by the
  quantization of admissible topological modes within a finite-volume projection fiber.
  Only a discrete set of deviations from the minimal surface is compatible with
  stability and projectability constraints, leading naturally to a discrete spectrum
  of excitations.

  As a consequence, vacuum energy is neither divergent nor arbitrary.
  It is intrinsically bounded by the spectral cutoff imposed by the relational graph underlying $\Pi$, understood as a
  representational support rather than a physical discretization.
  The finiteness of the vacuum energy reflects the finiteness of admissible spectral
  deformations of the minimal configuration, rather than the summation of zero-point
  energies of independent field modes.
