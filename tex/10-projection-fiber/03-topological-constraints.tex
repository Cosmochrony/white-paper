\subsection{Topological Constraints and Invariants}
  \label{sec:topological-constraints}

  The stability of localized physical descriptions within the projection fiber
  $\Pi$ is governed by the conservation of topological invariants.
  When an excitation of the relational substrate $\chi$ admits a closed,
  non-contractible configuration within $\Pi$, it forms a persistent topological
  obstruction to complete relaxation.
  Such configurations cannot be eliminated by smooth deformation of the
  relaxation flow without violating admissibility constraints.

  These topological obstructions are naturally described in terms of knot-like (i.e.\ non-contractible loop and
  self-linking) topological structures within the projection fiber.
  Once formed, they impose global constraints on the relaxation process and give
  rise to long-lived, localized projected configurations.
  In this sense, particles correspond to stable topological defects of the
  projection, rather than to elementary excitations of a fundamental field.

  The winding number $w$ constitutes the primary (though not necessarily unique) invariant characterizing these
  topological obstructions, arising from the admissibility constraints of the projection fiber.
  It labels distinct equivalence classes of admissible projected configurations
  and plays a central role in the organization of the mass spectrum discussed in
  the following chapter.
  The energetic cost required to maintain a non-trivial winding against the global
  pressure of relaxation is perceived, at the effective level, as rest energy
  ($mc^2$).

  Rest energy therefore does not measure an intrinsic substance or inertia, but
  quantifies the degree to which a topologically constrained configuration inhibits
  the relaxation of the substrate.
  Mass emerges as a spectral and topological consequence of persistent
  non-contractible structures within the projection fiber.
