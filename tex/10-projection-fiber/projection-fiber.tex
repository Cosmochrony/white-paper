\section{The Projection Fiber and Gauge Emergence}
  \label{sec:projection-gauge}

  \textit{This section introduces the geometric structure of the projection fiber
    $\Pi$ and shows how gauge interactions emerge as effective symmetries of the
    relaxation flow of the $\chi$ substrate when constrained by projectability and
    topological admissibility.}

  In the Cosmochrony framework, the projection fiber $\Pi$ is not an auxiliary
  mathematical space but a central structural element linking the pre-geometric
  substrate $\chi$ to effective spacetime descriptions.
  It encodes the admissible modes through which relational configurations of
  $\chi$ can be consistently projected into geometric and field-like
  representations.

  Gauge interactions do not arise from fundamental interaction fields defined on
  spacetime.
  They emerge instead as symmetry structures of the projection process itself,
  reflecting the invariance properties of admissible relaxation flows within the
  restricted topology of $\Pi$.
  In this sense, gauge symmetries are not imposed \emph{a priori}, but arise as
  stability conditions of projected descriptions under local re-identification.

  This section develops the geometry of the projection fiber and clarifies how
  distinct classes of gauge phenomena—transmittance, shear, and topological
  constraints—naturally emerge from its structure.
  These constructions provide the geometric basis for the effective gauge sectors
  discussed later in the context of the Standard Model, without introducing
  fundamental gauge fields or additional dynamical degrees of freedom.

  \subsection{The Geometry of the \texorpdfstring{$\Pi$}{Π} Subspace}
  \label{sec:geometry-pi}

  The relational substrate $\chi$ is not accessed in its full structural
  complexity, but through admissible local projections onto a reduced projection fiber $\Pi \cong S^3$.
  This projection acts as a spectral filter, retaining only those modes of the
  relaxation flow that remain jointly projectable and stable under local re-identification.
  Here, $\Pi$ denotes the projection \emph{fiber}, i.e. the space of admissible internal
  representatives associated with a projected configuration, not the projection map itself.

  The identification $\Pi \cong S^3$ is not imposed \emph{a priori}, but follows from the minimal compact geometry
  (in dimension and connectivity) required to support non-injective yet stable projections.
  It is the minimal compact manifold admitting nontrivial fibrations while preserving
  global stability and boundedness of the projection dynamics.
  Its associated Hopf fibration naturally induces an effective
  $SU(2)\times U(1)$ symmetry structure, which emerges as an invariance of the
  projection process rather than as a fundamental symmetry of the substrate.

  The metric on $\Pi$ is not fixed but dynamically induced by the local density of
  relational connections encoded in the underlying relational graph $G$ used as a numerical and representational
  support, not as a physical discretization.
  The mapping from the global graph Laplacian $\Delta_G$ to the effective projected Laplacian $\Delta_\Pi$ is given by
  \begin{equation}
    \Delta_\Pi = P^\dagger \Delta_G P ,
  \end{equation}
  where $P$ denotes the projection operator onto the admissible spectral subspace.
  The appearance of a smooth three-sphere geometry thus corresponds to a a large-$N$ spectral coarse-graining limit of
  the discrete relational structure, in which coarse-grained spectral properties dominate.

  \subsection{Gauges as Relaxation Transmittance}
  \label{sec:gauges-transmittance}

  Within the Cosmochrony framework, gauge interactions are not fundamental forces
  mediated by independent fields.
  They are reinterpreted as degrees of freedom of the relaxation flow within the
  projection fiber $\Pi$, encoding how locally admissible projections of the
  substrate $\chi$ are coherently related despite the intrinsic non-injectivity of
  the projection.

  \begin{itemize}

    \item \textbf{Electromagnetism ($U(1)$):}
    Electromagnetism corresponds to the phase degree of freedom of the relaxation
    flow along the fibers of the Hopf fibration of $\Pi \cong S^3$.
    This phase reflects the freedom to locally re-identify projected descriptions
    without altering global relational consistency.
    The fine-structure constant $\alpha$ characterizes the effective
    \emph{transmittance} of the relaxation flow through the fiber, quantifying the
    robustness with which phase coherence is maintained across projections.

    \item \textbf{Weak Interaction ($SU(2)$):}
    The weak interaction emerges from the rotational degrees of freedom of the
    $S^3$ projection fiber itself.
    These degrees of freedom encode non-abelian modes of relaxation transmittance,
    corresponding to shear-like distortions of admissible projections.
    The massive character of the $W^\pm$ and $Z^0$ bosons follows from the presence
    of a non-zero spectral gap associated with these modes, which can be interpreted
    as a form of spectral drag or torsion intrinsic to the fiber geometry.

    \item \textbf{Strong Interaction ($SU(3)$):}
    The strong interaction is associated with topological constraints arising from
    non-trivial winding structures of projected configurations.
    In particular, the effective $SU(3)$ symmetry reflects the threefold (triality)
    structure of the minimal self-intersecting stable soliton, identified with the
    trefoil knot ($w=3$).
    Color symmetry thus emerges as a geometric consequence of topological stability
    classes within the projection fiber, rather than as a fundamental internal
    degree of freedom.

  \end{itemize}

  \subsection{Topological Constraints and Invariants}
  \label{sec:topological-constraints}

  The stability of localized physical descriptions within the projection fiber
  $\Pi$ is governed by the conservation of topological invariants.
  When an excitation of the relational substrate $\chi$ admits a closed,
  non-contractible configuration within $\Pi$, it forms a persistent topological
  obstruction to complete relaxation.
  Such configurations cannot be eliminated by smooth deformation of the
  relaxation flow without violating admissibility constraints.

  These topological obstructions are naturally described in terms of knot-like (i.e.\ non-contractible loop and
  self-linking) topological structures within the projection fiber.
  Once formed, they impose global constraints on the relaxation process and give
  rise to long-lived, localized projected configurations.
  In this sense, particles correspond to stable topological defects of the
  projection, rather than to elementary excitations of a fundamental field.

  The winding number $w$ constitutes the primary (though not necessarily unique) invariant characterizing these
  topological obstructions, arising from the admissibility constraints of the projection fiber.
  It labels distinct equivalence classes of admissible projected configurations
  and plays a central role in the organization of the mass spectrum discussed in
  the following chapter.
  The energetic cost required to maintain a non-trivial winding against the global
  pressure of relaxation is perceived, at the effective level, as rest energy
  ($mc^2$).

  Rest energy therefore does not measure an intrinsic substance or inertia, but
  quantifies the degree to which a topologically constrained configuration inhibits
  the relaxation of the substrate.
  Mass emerges as a spectral and topological consequence of persistent
  non-contractible structures within the projection fiber.

  \subsection{The Vacuum State as a Minimal Surface}
  \label{sec:vacuum-minimal-surface}

  In the absence of localized excitations, the projection fiber $\Pi$ relaxes toward
  a configuration of minimal spectral tension, understood as minimal global
  frustration of admissible relaxation modes.
  This vacuum state is not an empty background, nor the ground state of a quantized
  field, but the smoothest admissible projected configuration compatible with global
  relaxation constraints.

  Geometrically, the vacuum corresponds to a minimal surface in a purely spectral sense, not a spatial or spacetime
  embedding:
  a configuration for which curvature, torsion, and winding modes are simultaneously
  minimized within $\Pi$.
  In this state, the relaxation flow encounters no topological obstruction and
  propagates uniformly across the fiber.
  Any deviation from this minimal configuration—whether through localized curvature,
  torsional distortion, or non-trivial winding—manifests as the presence of effective
  fields or particles.

  From this perspective, the usual notion of field quantization is replaced by the
  quantization of admissible topological modes within a finite-volume projection fiber.
  Only a discrete set of deviations from the minimal surface is compatible with
  stability and projectability constraints, leading naturally to a discrete spectrum
  of excitations.

  As a consequence, vacuum energy is neither divergent nor arbitrary.
  It is intrinsically bounded by the spectral cutoff imposed by the relational graph underlying $\Pi$, understood as a
  representational support rather than a physical discretization.
  The finiteness of the vacuum energy reflects the finiteness of admissible spectral
  deformations of the minimal configuration, rather than the summation of zero-point
  energies of independent field modes.

