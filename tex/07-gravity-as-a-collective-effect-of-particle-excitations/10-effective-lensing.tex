\subsection{Effective gravitational lensing}
  \label{subsec:effective-lensing}

  In the weak-field and thin-lens regimes, gravitational lensing depends only on the
  effective spacetime geometry experienced by null geodesics.
  Independently of its microscopic origin, the lens equation can be written as
  \begin{equation}
    \boldsymbol{\beta}
    =\boldsymbol{\theta}
    -\frac{D_{ls}}{D_s}\,
    \nabla_{\boldsymbol{\theta}}\psi(\boldsymbol{\theta}),
  \end{equation}
  where the lensing potential is given by
  \begin{equation}
    \psi(\boldsymbol{\theta})
    =\frac{2D_{ls}}{c^2D_lD_s}
    \int \Phi_{\mathrm{eff}}\!\left(D_l\boldsymbol{\theta},z\right)\,dz .
  \end{equation}

  Within the Cosmochrony framework, the effective potential $\Phi_{\mathrm{eff}}$
  is not interpreted as the Newtonian potential sourced by a local mass density,
  but as a geometric descriptor encoding collective constraints of the projected
  $\chi$-substrate.
  It arises from the reconstruction of the effective metric from spectral distances,
  and remains fully compatible with the standard lensing formalism.

  All observable lensing quantities (convergence, shear, magnification and caustics)
  follow from $\Phi_{\mathrm{eff}}$ in the usual way.
  No additional dark matter component is required: lensing directly probes the
  emergent geometry rather than a hidden mass distribution.
