\subsection{Summary}
  \label{subsec:summary-gravity}

  Within the Cosmochrony framework, gravity does not arise as a fundamental interaction
  or as an independent geometric degree of freedom.
  It emerges at the level of effective descriptions as a macroscopic consequence of
  localized projected configurations collectively constraining admissible relaxation
  ordering.

  Classical gravitational phenomena—including gravitational time dilation, effective
  spacetime curvature, gravitational waves, and black holes—are recovered as distinct
  descriptive regimes of this collective constraint.
  They reflect variations in the projectability and correlation structure of admissible
  descriptions, rather than the dynamics of a fundamental spacetime or gravitational
  field.

  In regimes of extreme constraint, such as horizons, gravitation does not merely act
  as an emergent organizing principle, but also as a diagnostic interface.
  The spectral structure of horizon-associated phenomena encodes direct information
  about the relaxation dynamics and projection topology of the underlying $\chi$
  substrate, providing, in principle, observational access to its micro-structural
  organization.

  In this perspective, gravitation appears as an emergent and operational phenomenon,
  summarizing how localized projected configurations collectively constrain admissible
  ordering and correlation across extended regions, without introducing gravity as a
  primitive force or a fundamental geometric entity.
